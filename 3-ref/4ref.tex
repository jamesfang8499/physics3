\chapter{电磁振荡和电磁波}\minitoc[n]
\section{教学要求}

这一章在电场、磁场、电磁感应等知识的基础上,讲述电
磁振荡和电磁波,以及无线电发射和接收的初步知识,是电学
知识的继续,也是振动和波的知识的发展.这一章的知识又
为以后认识光的电磁本质作准备.

本章内容既讲述基本知识又讲解实际应用,全章知识多
是介绍性的,这是本章的特点,讲述基本知识时,应着重介
绍主要之点.这一章的核心内容是麦克斯韦电磁场的理论.
电磁波的实际应用中,需要解决许多实际技术问题,教材对解
决发射与接收的主要问题——开放电路、调制、调谐和检波
作了简要介绍,讲述时应着重介绍原理和解决问题的思路,
不宜过多地涉及技术细节.

本章教材可分为三个单元.第一单元包括第一节和第二
节,讲述电磁振荡的产生和$LC$振荡电路的周期和频率公式.
第二单元包括第三节至第五节,讲述麦克斯韦电磁理论要点
和电磁波的知识.第三单元包括第六节至第十二节,讲述电
磁波应用技术初步知识和发展简况.

$LC$回路可以产生振荡电流,要通过演示实验来使学生确
信.电磁振荡的产生过程,要突出电场能和磁场能的相互转
化.由于在中学不容易搞清楚$LC$回路中为什么电流最大时电
压最小这类问题,而进一步深究会加重学生负担,在教学中不
要过多探讨,对$LC$振荡电路的周期公式,应使学生掌握它
的物理意义.由于在后面学习谐时要用到改变电磁振荡周
期的内容,要使学生切实理解电磁振荡的周期怎样随电容和
感的改变而改变.

麦克斯韦电磁场的理论把电场和磁场统一起来,要让学
生了解麦克斯韦理论的两个要点,这两个要点都是用场的观
点分析得出的,虽然学生已经学过电场和磁场,但不熟悉用
变化的场分析现象.要使学生明确由变化磁场产生电场跟是
否有闭合电路无关,由变化的电场产生的磁场跟空间里存在
电流而产生的磁场是一样的,但无需提出位移电流的概念.教
材还指出了均匀变化的场产生稳定的场,非均匀变化的场产
变化的场,这是为讲电磁波作准备的.在分析时,要向学生
说明什么是均匀变化的场和非均匀变化的场,以帮助学生理
解怎样用变化的场分析有关现象.

在讲解电磁波的性质时,要突出它可以脱离电荷而独立
存在、具有能量、不需要别的物质做媒质等特点.正是电磁波
的这些特点,表明电磁场是客观存在的物质.赫兹实验证实
了电磁波的存在,确立了光的电磁说.有条件的学校应尽量
演示这个实验,使学生确信麦克斯韦理论的正确,也有助于认
识电磁场的物质性.

麦克斯韦电磁理论在教学中之所以重要,还在于它在思
想方法上给人们以重大的启示.许多重大发现的提出往往不
是在其理论系统完成之时,而是在人们根据种种物理现象定
性分析、深入思考并进行猜测之时,这是创造性思维的特点.
这一章无论是麦克斯韦理论的提出,还是$LC$电路产生电磁
振荡的物理过程,以及电磁波发送与接收过程的教学都要充
分重视从人们是怎样提出和思考问题的角度进行教学,避免
仅作一般知识介绍,这样会有利于进行创造能力的培养,对于
掌握教材内容也是有益的.

本章安排的学生实验是“安装简单的收音机”,其目的是
通过安装、调试,使学生确信接收原理,培养他们的动手能力,
而不在于熟悉具体的线路.因此,只要求了解实验线路图中
的可变电容器、二极管、三极管、耳机的作用,对线路中的其他
元件的作用不要求讲解.

本章的教学要求是:
\begin{enumerate}
\item 了解电磁振荡产生的过程,掌握电磁振荡的周期和频
率的公式.
\item 了解电磁场理论的要点,了解电磁波的产生和特点.
知道电磁场是一种物质形态.知道赫兹实验.
\item 了解无线电发射和接收的基本原理,了解无线电传播
的特性.
\end{enumerate}

\section{教学建议}
\subsection{电磁振荡}
这一单元的教材中,首先安排了$LC$振荡电路的演示
实验,以使学生获得$LC$振荡电路中产生振荡电流的初步感性
知识,在此基础上运用电容器充放电现象与电感线圈的自感
现象对$LC$电路产生振荡电流的物理原因作了定性分析,
并进一步从能量转化角度突出指出了$LC$电路产生振荡电流
的物理实质,由此引出了电磁振荡概念以及阻尼、无阻尼振荡
概念,说明了在$LC$振荡电路中维持电磁振荡的条件,可以看
出,教材的基本思路是:
\begin{center}
\begin{tikzpicture}[>=latex]
\node (A) at (0,0){物理现象};
\node (B) at (3,0){物理原因};
\node (C) at (6,0){物理实质};
\node (D) at (9,0){有关概念};
\draw[->](A)--(B);\draw[->](B)--(C);\draw[->](C)--(D);
\end{tikzpicture}
\end{center}
教材的结构是:
\begin{center}
\begin{tikzpicture}[>=latex, align=center]
\node (A) at (0,0){$LC$电路的\\[-1ex]振荡电流};
\node (B) at (3,0){电容与电感\\[-1ex]基本特性};
\node (C) at (6.5,0){电场能与磁场能\\[-1ex]的互相转化};
\node (D) at (9.5,0){电磁振荡};
\node (D1) at (11.5,0.5){无阻尼振荡};
\node (D2) at (11.5,-0.5){阻尼振荡};
\draw[->](A)--(B);\draw[->](B)--(C);\draw[->](C)--(D);
%\draw[->] (D)-- (D1);\draw[->] (D)-- (D2);

\draw[decorate, decoration=brace] (10.5,-.5)--(10.5,.5);

\end{tikzpicture}
\end{center}


在具体进行教学时,有多种教学方案可供选择,关键
是要针对所教班级学生接受与理解能力.对于理解能力稍差
的班级,可以突出物理现象的演示观察和着重从能量转化角
度定性分析现象,同时适当复习电容与电感线圈的物理特性.
还可与弹簧振子或单摆作些类比(参看练习答案),对于理解能
力较强的班级,可以在复习电感线圈与电容器物理特性基础
上,直接讨论一个电容器充电后向一个电感线圈放电可能产
生的物理现象,例如可以先将一个充电电容器向一个电阻放
电与一个充电电容器向一个电感线圈放电可能产生的现象作
对比,着重讨论能量的转化情况.也就是采用设想电路——猜
测现象——定性分析结论——演示观察这样一种教学结构.

在分析$LC$电路时,要复习电容器与电感线圈基本物
理特性,要着重使学生领会:
\begin{enumerate}
    \item 对电容器而言,两板板之间电压不能突变,即充放电
都需要时间,这一点可以通过一些演示实验说明,也可以从放
电电流$i=\Delta Q/\Delta t=c\cdot \Delta U/\Delta t$分析说明.式中,若$\Delta t=0$, 而
$\Delta Q$与$\Delta U$是有限量,则$i$将无限大,这是不可能的,通过反面
论证说明电压不能突变、充放电需要时间这一物理特性.
\item 对电感线圈而言,通过电感线圈的电流不能突变.这
一点可以通过演示实验复习说明,也可从自感电动势$\mathcal{E}=
L\cdot \Delta i/\Delta t$分析说明.若$\Delta t=0$, 因$\Delta i$为有限量,那么$\mathcal{E}$将
无限大,这也是不可能的:因此在接通电路时,电流只能逐步
增大;而当电路中存在电流时,也不能突然消失,只能逐步
减小.换言之,电流变化需要时间.理解以上两点无论对于
$LC$ 振荡电路的分析,还是对将来进一步学习都是重要的.
\end{enumerate}

$LC$振荡电路周期(频率)公式的教学,既可以先提出
问题进行定性讨论并配合演示实验而后给出公式;也可以先
进行演示实验,在对现象进行讨论的基础上给出公式.这一
部分知识教学,应根据班级情况注意以下几个问题:
\begin{enumerate}
\item 实际使用的$LC$振荡电路常是通过改变电容 器电容
或改变线圈电感来改变振荡周期的,因此需要复习一下决定
电容器电容的因素和决定线圈电感的因素.教学时可适当出
示可变电容器实物或作有关演示实验.
\item 通常实用的$LC$振荡电路周期很短,为了表达方便,
计算时多使用频率公式,这一点与单摆不同.同时由于公式
中各量单位实际上都带有构成十进倍数成分数的词头,例如
频率常以千赫、兆赫为单位,电容常以微法、皮法为单位,电感
常以毫亨、微亨为单位,因此可以将书末附录三介绍给学生,
要求学生在计算中能正确将上述单位换算为主单位,相应的
算结果能用常用的单位表示;例如
\[10^5{\rm Hz}=100{\rm kHz},\qquad 10^{-5}{\rm F}
=10{\rm mF},\qquad 10^{-6}{\rm H}=100{\rm mH}\]
使学生逐步熟悉我国法定
计量制中有关单位的正确使用方法.
\item 有些学生指数运算易出错,应通过例题指出运算方
法,例如可将$\sqrt{10^{-11}}$化为$\sqrt{10\x10^{-12}}$进行计算.
\end{enumerate}

\subsection{电磁波}
这一单元是本章的核心.教材包括麦克斯韦关于电磁场
的两个重要理论推测,电磁波激发过程、电磁波的特点和电磁
场的物质性、赫兹实验等内容.这一单元在知识理解上难点
较多,最关键的是要理解“均匀变化磁场(或电场)产生稳定电
场(或磁场),非均匀变化的磁场(或电场)产生变化的电场(或
磁场)”这两句话的意义.理解了这两句话,对电磁波的产生
及其特点就能较好理解和掌握了.

\subsubsection{电磁场的教学}

可以分两步进行.第一步着重从问题
的提出、考虑问题的方法角度重点进行麦克斯韦电磁场理论
的两个要点的教学,对第一个理论要点即变化的磁场能够在
周围空间产生电场,应突出麦克斯韦对穿过闭合线圈中磁通
量变化而产生感生电流现象的思考,即在闭合线圈中驱使电
荷作定向运动的原因是什么?在指出他认为是一种电场力作
用,而且这种电场的存在与有无闭合线圈无关后,还应指出麦
克斯韦的创造性更表现在他认为这种电场不同于静电场,是
一种脱离电荷存在的涡旋电场,以使学生初步了解两类电场
的区别,并为后一节讲电磁场可脱离电荷而存在作出铺垫,对
第二个理论要点,可通过电和磁的对称提出来.在物理学发
展史上,许多问题都是从对称角度提出的,这一点可以启发学
生回忆学过的知识.例如奥斯特发现电生磁后,法拉第正是
从对称角度出发提出磁生电的问题,经过十年努力终于发现
了电磁感应现象(见教材第51页阅读材料).在教材第65页
阅读材料“寻找磁单极子”中,人们也是把电磁类比,从两者的
对称性出发希望找到磁单极子.又如人们从永磁体出发,希
望对称地找到永电体,在本世纪五十年代终于找到了这种永
电体(现在称为驻极体),并广泛运用于技术中(如驻极体话筒
等).因此对称、类比地提出某种推测,是人们常用的一种思
考方法.这一点,应通过教学使学生有所体会.

以上两个观点,学生是易于接受的,这一部分教学的困
难在于第二步骤,即使学生充分领会和理解教材上关于均匀
变化的磁场(或电场)产生稳定电场(或磁场),非均匀变化的磁
场(或电场)产生变化电场(或磁场)的含义.为使学生易于理
解,可以通过分析法拉第电磁感应定律表达式:$\mathcal{E}=\Delta \phi/\Delta t$加
以说明.如果所教班级学生已在数学课中学过导数知识,这
个问题比较好处理.所谓均匀变化就是指按一次函数规律变
化,设磁通量$\phi=kt+b$, 因此感生电动势$\mathcal{E}=\dd\phi/\dd t=k$, 为一与时间$t$无关的恒量,换言之,驱使电荷作定向运动的涡
旋电场一定,这就是均匀变化磁场产生稳定电场的含义.若
磁通量$\phi$不是随时间均匀变化的,例如按正弦规律变化,
$\phi=\Phi_m \sin\omega t$, 那么感生电动势:
\[\mathcal{E}=\frac{\dd\phi}{\dd t}=\Phi_m \omega \cos \omega t\]
说明涡旋电场场强将随时间作余弦变化,这就是非均匀变化磁场
产生变化电场的含义.

\begin{figure}[htp]
    \centering
\includegraphics[scale=.7]{fig/4-1.png}
    \caption{}
\end{figure}

如果所教班级学生尚未学过导数知识,可以通过作图线
的方法加以说明(图4.1).若$\phi$是均匀变化的,即按一次函
数变化,那么 $\Delta \phi/\Delta t$是定值,从图线上看斜率处处相同
(图4.1甲),因而$\mathcal{E}$是定值,不随时间变化.而当$\phi$作非均
匀变化时,如图4.1乙所示,那么在不同时刻,$\Delta \phi/\Delta t$不同,
图线表示出不同时刻图线斜率不同,说明$\mathcal{E}$将随时间而变化,
也就是说产生的涡旋电场场强将随时间而变化.

在讨论清这个问题后,除了说明电磁场概念外,可以进一
步引导学生思考,讨论和分析这个问题究竟有什么物理意义?
鼓励学生去思考想象,或阅读教材发表意见.一般地可把这
作为本节课教学的小结并为下一节课讲述电磁波做好准备.

\subsubsection{电磁波}

这一节的教学,有三个主要内容:一是在上
一节麦克斯韦两个理论要点的基础上讲述电磁波的产生过
程,二是使学生初步形成电磁波的图景,三是使学生初步了解
电磁波的物质性.

对第一点的教学,可在上面所提问题的基础上展开.
使学生能够在头脑中形成如下分析推测的思路:
\begin{center}
\begin{tikzpicture}[>=latex, align=center]
\node (A) at (0,0){非均匀变\\[-1ex]化的磁场};
\node (B) at (2.75,0){变化\\[-1ex]电场};
\node (C1) at (4.5,1){若均匀\\[-1ex]变化};
\node (C2) at (4.5,-1){若非均匀\\[-1ex]变化};
\node (D1) at (7.5,1){稳定\\[-1ex]磁场};
\node (D2) at (7.5,-1){变化\\[-1ex]磁场};
\node (E) at (10,1){不再激发};
\node (E1) at (10,-0.5){若均匀变化};
\node (E2) at (10,-1.5){若非均匀变化};
\node (F) at (12.5,-0.5){稳定\\[-1ex]电场};
\draw[->](A)--node[above]{激发}(B); \draw[->](C1)--node[above]{激发}(D1);\draw[->](C2)--node[above]{激发}(D2);
\draw[->] (D1)-- (E);\draw[->] (E1)--node[above]{激发} (F);
\draw[->] (E2) to [bend left=30] (A);
\draw[decorate, decoration=brace] (3.5,-1.2)--(3.5,1.2);
\draw[decorate, decoration=brace] (8.5,-1.8)--(8.5,-.2);
\end{tikzpicture}
\end{center}

由此可推测得出,若过程中产生的始终是非均匀变化的
磁场与电场,那么某处的这种非均匀变化磁场(或电场)将在
周围空间由近及远激发起一系列电场和磁场,这就形成了电
磁波的初步图景.在这个基础上可以指出:
\begin{enumerate}
\item 周期性变化的磁
场(或电场)可以激发电磁波,在利用课本上图4.6分析电磁
波产生过程时,应当注意指出,这个图只是说明电磁波传播的
示意图,并不是空间电磁场某一时刻的电力线与磁力线真实
分布情况.
\item 这种周期性变化的磁场(或电场)可以利用$LC$
振荡电路在振荡时电容或线圈中的电场和磁场获得.
\item 振荡
电路中电荷作快速振动是产生周期性变化电场(或磁场)的根
本原因.因此一切电荷作快速振动时都会激发电磁波.$LC$
振荡电路仅是激发电磁波的一种方法.
\end{enumerate}

对于第二点的教学,应通过对课本上图4.7的分析
明确以下三个特点,从而使学生获得正确的电磁波图景:
\begin{enumerate}
\item 电磁波在空间传播时,在任一位置上(或任意时刻)电场强度方
向、磁感应强度方向和传播方向三者互相垂直,即$E$、$B$、$v$矢
量在空间中互相垂直.
\item 在电磁场强度为最大值的空间某
一位置上,相应的磁感应强度也为最大值(这一点的理由不必
作解释,只告诉学生是被理论和实验所证实的).
\item 电磁波波
速与光速相同,这是麦克斯韦从电磁理论中所预言的.通过这一点,教师应指出科学理论的重大意义在于可以预见人类
尚未认识的事实.关于波长、波速与频率(周期)的关系可从
跟机械波公式的类比中得出.
\end{enumerate}

关于第三点电磁波物质性的教学,根据教材可从电
磁波可脱离电荷独立存在、不需借助媒质传播、具有能量三个
方面,简单说明电磁波与实物粒子一样是自然客体.这是人
们对于物质认识的一个重大发展,这一点使学生有初步认识
即可.

\subsubsection{赫兹实验}

在这一册教材中,安排了赫兹实验这一节教学内容,
目的是通过历史上这个重要实验说明理论正确与否最终需要
由实验来检验.在讲述这个实验时,应注意从赫兹对这个实
验现象的分析判断的角度进行介绍,使学生能够逐步体会
到,对于物理现象的敏锐的观察力来自对基本理论的深入理
解,赫兹在实验室里检查仪器时偶然观察到感应圈放电时发
生的电火花,立即想到电火花的跳跃实质上是电荷在作间断
的快速振动引起的.根据麦克斯韦的理论,快速振动的电荷,
必定在周围空间激起电磁波,在这个思想引导下,赫兹设计了
如课本图4.9所示的实验装置,其中圆形开口状接收器显然
是根据电磁感应现象设计出来的.结果正如赫兹所设想的那
样,实验获得了成功.在进行本节教学时,最好在分析讨论之
后再做实验,以加深学生的感性认识和对电磁理论的理解.


\subsection{电磁波的发射与接收}

这一单元主要是联系实际的知识,教学中只要求简单介
绍原理,使学生有一个大致的了解即可,不宜分析太细.对无
线电技术感兴趣的同学,可以引导他们阅读有关书籍,组织课
外讲座与实验小组发展他们的兴趣与特长.这一单元演示实
验较多,在进行演示时,应力求装置简单、效果明显.教学时,
要突出说明主要装置及其作用,需观察的实验现象,使学生把
精力集中在所讨论的问题上.

\subsubsection{电磁波的发送}

在讲述这一问题时,应当着重向学生
说明,我们讨论电磁波的发送是要决如何运用电磁波传递
人们需要的信息,从而引入解决电磁波发送(与接收)中的一
系列理论与实际技术问题.在发送时需解决的问题有两个:
第一个是如何向外有效地发射电磁波(传得足够远).这就要
利用开放电路与高频电磁波.第二个问题是如何有效地传递
人们所需要的信息,这要通过调制来解决.

关于开放电路的教学,应注意对比$LC$振荡电路(闭合
振荡电路)与开放电路的不同特点.要让学生了解$LC$振荡电
路,是利用电感与电容的不同储能(磁场能和电场能)特性,通
过能量转化完成电磁振荡的.在这种情况下,能量被集中在
电感线圈内部和电容两个极板之间.而开放电路是要在广阔
空间激发电磁波,希望能量以电磁辐射形式向外传播.利用
这两种电路的目的不同,它们的特点不同.开放电路是闭合
电路的变形.

在讲解调制过程时,学生对调幅过程的理解有困
难.这是因为,调制过程不是载波和调制信号的简单叠
加,而是通过调制过程使载波的振幅随调制信号的变化而
变化.因此,需结合课本上图4.13调制电路示意图的分析并
配合示波器演示波形加以说明,在说明中,应突出传递的信
号频率较低而发送电磁波却需要高频振荡这个矛盾.课本上
图4.13调制电路的话筒是一种碳粒话筒,它本身兼有两种
作用:一方面将语言(声音)信号转变为电信号,另一方面由于
碳粒话筒的电阻随碳粒松紧程度改变,兼起调制幅度的作用.
而其他话筒,例如常用的动圈电磁话筒是不具备这种特性的.
分析时可以指出以下思路:
\begin{center}
\begin{tikzpicture}[>=latex]
\node (A) at (0,0){语言信号};
\node (B) at (3,0){碳粒松紧};
\node (C1) at (6,0){引起电阻变化};
\node (C2) at (9,0){高};
\draw[->](A)--node[below]{控制}(B); 
\draw[->](B)--(C1); 
\draw[->](C1)--node[below]{控制}(C2);

\node (D1) at (0,-1){频振荡电流幅度大小};
\node (D2) at (5,-1){调幅高频电流};
\draw[->](D1)--(D2); 
\end{tikzpicture}
\end{center}

一般地应将调制过程画成以下所示方框图:
\begin{center}
    \begin{tikzpicture}[>=latex]
    \node (A) at (0,0)[shape=rectangle, draw]{调制信号};
    \node (B) at (3,0)[shape=rectangle, draw]{调制器};
    \node (C1) at (6,0)[shape=rectangle, draw]{放大器};
    \node (C2) at (9,0)[shape=rectangle, draw]{开放电路};
    \node (D) at (3,-1.5)[shape=rectangle, draw]{高频振荡器};
    \draw[->](A)--(B); 
    \draw[->](B)--(C1); 
    \draw[->](C1)--(C2);
    \draw[->](D)--(B); 
    \end{tikzpicture}
    \end{center}

在说明调制的两种方法,即调幅、调频时,可举一些实
例,如一般中、短波无线电广播都是用调幅波传送的;电视节
目的图象信号是利用调幅波传送的;而伴音信号都是用调频
波传送的,这样可使学生获得一些实用知识,也可提高他们
的学习兴趣.

应将教材最后所介绍的电磁波发送的方框图给学生讲
讲,使他们大致了解这一过程.

\subsubsection{电磁波的接收知识}

电磁波的接收知识与学生实际生活联系密切,因此尽
可能结合实际进行讲解.应从接收需解决的问题开始讨论.
接收电路解决获得电磁波信号和从中选取电磁波信号两个问
题,解调解决从被调制的高频电流中取出调制信号的问题.

在介绍获得电磁波信号的方法时,可以从电磁场的电
场分量与磁场分量两个方面讲述.任何一个导体位于电磁波
传播的空间中,导体中的自由电子受电场分量驱动作用,就会
产生相同频率的高频振荡电流.这就是课本上讲的天地线接
收.实际上,一般无线电广播电场分量是在竖直方向上的(见
课本图4.6),故一般收音机天线是竖直的.而电视广播电场
方向一般是水平的,所以电视天线都是水平放置的,这一点观
察一下就会知道.利用磁场分量接收也是广播收音机常采用
的方法,半导体收音机中有一个磁棒(叫磁性天线),其上绕有
线圈,当磁棒水平放置并有磁力线穿过磁棒内时,由电磁感应
现象可知,磁棒绕组上将产生感生高频电流.这一点,打开几
个半导体收音机后盖就可清楚地观察到.如果将磁棒直立放
置,收音机接收信号将明显变弱.这样讲述,既使学生获得了
实际知识,同时也运用学过的知识分析了实际问题,可以取得
较好的教学效果.

关于电谐振的教学.可以类比机械振动中的共振现
象.教材安排的电谐振的演示实验,实际上就是赫兹实验.
如果无此设备,也可用其他实验代替(见实验部分).应当向
学生具体介绍一下实际调谐的方法,告诉学生收音机选台调
谐使用可变电容器,而更换波段采用接入不同线圈的方法.课
本上没有明确给出调谐电路频率公式,教师应附带说明一下,
调谐电路固有频率的计算与$LC$振荡电路是相同的.

关于检波的教学,可以先向学生补充说明,由于调制
有两种方式:调幅与调频,因而其反过程即解调也有两种方
式:检波与鉴频.在实际中有调幅收音机与调频收音机两种.
课本上只粗略地介绍了调幅波的检波,用了交流电一章学过
的半导体二极管单向导电性和脉动电流分解的知识.教学时
可配合演示实验利用示波器对照课本图4.19进行讲解.同时
可让学生做安装简单收音机的分组实验.这个实验也可以改
为按课本图4.18的电路进行.这样安排,实验内容简单、所
用器材少、说服力强,学生容易完成.有关数据见后面实验
部分.课本最后介绍了一般收音机的方框图,学生作一了解
即可.

教材第十节介绍传真、电视、雷达等常识,是选学教
材.建议将本节与第十二节电子技术一瞥合在一起,作课外
讲座,可配合图片或教学电影进行,比在课堂上讲授效果更好
一些.教学时,要尽可能向学生介绍我国在四化建设中不断
取得的新成就,例如通信卫星等.

在讲解传真、电视、雷达这部分内容时,一方面应注意
突出基本原理,不过多涉及技术细节,另一方面也应指出,无
线电电子技术给人们在传递和处理信息方面开辟了广阔的领
域,其中重要的一个环节是如何将各种信息转化为电信号.传
真和电视在将图像的光信号转变为电信号上原理是相同的,
即:
\begin{center}
    \begin{tikzpicture}[>=latex, align=left]
    \node (A) at (0,0){图像上每点\\[-1ex]光的强弱};
    \node (B) at (4,0){光电流某一\\[-1ex]时刻的大小};
    \node (C1) at (8,0){光电流变化\\[-1ex]$+$同步信号};
    \node (C2) at (11,0){发送};
    \draw[->](A)--node[above]{光电效应}(B); 
    \draw[->](B)--node[above]{扫描}(C1); 
    \draw[->](C1)--(C2);
    \end{tikzpicture}
    \end{center}
这里要着重讲解扫描和同
步的作用:扫描是将一幅图像通过逐点逐行逐幅的有顺序的
扫描形成某一段时间内的光电流变化;而同步的作用则是需
要接收部分逐点逐行逐幅与发送端步调一致地将电信号转变
为光信号,才能重现图像.为便于理解,可通过下述例子说明
扫描与同步的作用.利用两块坐标黑板,一块上以每一小格
为一个象素,画上图案或字,通过顺序叙述每行每格的“黑”
“白”,可以将这幅图像通过声音传递出去,由别人在另一处
的坐标黑板上重现出来,讲完电视工作原理以后还可利用黑
白电视机演示一下行同步与桢同步的实际情况.

由于电视传递的是活动图面,需要解决一些特殊问题,教
学中可说明这样几点:
\begin{enumerate}
\item 与电影相仿,活动画面分解为若干静
止画面,利用人眼视觉暂留(约0.1秒),连续放映就可看到活
动图景.为使眼睛不感觉闪烁,每秒钟需传递10幅以上画面
(电影为24幅,电视为25幅,但通过隔行扫描,实际感觉为50
幅).
\item 画面光信号的转换不能只靠一个光电管进行,扫描也
不宜用机械方式,在电视中,摄像管中嵌镶板上是许许多多的
小光电管,形成一幅电的“图像”,通过电子束扫描形成光电
流.
\item 扫描时,电子束偏转由偏转线圈的磁场控制.同步作
用也要通过接收机显像管偏转线圈控制电子束完成.对有兴
趣的学生可以介绍一些课外读物,供他们选择阅读,课堂上作
些简要介绍与引导即可.
\end{enumerate}


第十一节电磁波的传播的内容在教学时,可归纳成
下表,使学生明确天波、地波及直线传播的基本特点和适用
波段.这一节的教学可在学生阅读课文的基础上作出简单
归纳.
\begin{center}
\begin{tabular}{lll}
\hline
传播形式&适用波段&传播特点\\
\hline
地波&长波、中波&易衍射、稳定、地面吸收大\\
天波&短波&由电离层反射、不稳定、损耗小\\
直线传播&微波&与光传播性质相近,可穿越电
离层\\
\hline
\end{tabular}
\end{center}

“电子技术一瞥”的教学,主要是使学生了解电子技
术的各个领域在国民经济中的重大意义和发展简况,在教学
时,可以历史地叙述电子学的发展及其涉及的各个领域,介绍
我国电子工业的发展和应用情况,也可以概要地从电子技术
涉及的主要领域介绍有关发展与应用情况.

近代电子技术是从无线电技术发展起来的,而无线电技
术只是近代电子技术的一个部分.近代电子技术从电子三极
管问世迄今只有79年(至1986年)的历史,但它已经衍生了许
多新兴学科,可从这样几个方面进行简单的介绍:
\begin{enumerate}
\item 电子器件及其材料;
\item 通信、广播、电视技术;
\item 自动化、遥控遥测技术;
\item 电子计算机;
\item 边缘学科如仿生电子学、电子显微镜、射电
天文学、医学电子学、光电子学、空间电子学等等.
\end{enumerate}

\section{实验指导}
\subsection{演示实验}
\subsubsection{用示波器观察振荡电路中的电磁振荡}
利用示波器可以方便地观察$LC$振荡电路中产生的振荡
电流波形(实际观察的是$L$或$C$两端电压波形).下面分阻
尼振荡和无阻尼振荡两种情况介绍具体演示方法.

(1)$LC$振荡电路的阻尼振荡

实验的电路如图4.2所示.示波器用J2458型教学示波
器(或其他型号示波器,用长余辉超低频示波器演示效果会更
好),电感线圈$L$可用J2423型可拆变压器红色线圈0—1400
匝绕组(也可用电子管收音机电源变压器初级线圈等电感大、
电阻小的线圈),电容器$C$可选用电容在1至50微法之间的
(也可用电解电容),图中$K$为单刀双掷开关,应尽量选用拨动
速度快的开关,如钮子开关或拨动开关.电源用6伏电池组.
实验前先调整好示波器,并将Y衰减置于10, Y增益旋至中
间,扫描频率拨至10—100赫挡,并将扫描微调钮逆时针旋到
底,接好电路后,将$K$拨向$a$, 给电容器充电,然后将$K$拨向
$b$, 示波器上将显示阻尼振荡波形,根据振荡衰减情况,适当
调节扫描微调与Y增益,使示波器荧光屏上显示至少有3个
明显周期的波形,必要时可适当调整电容器$C$的电容大小.
演示时,先较慢地拨动开关两三次,引起学生注意.然后可以
来回迅速拨动开关,示波器上将连续出现阻尼振荡波形.

\begin{figure}[htp]\centering
    \begin{minipage}[t]{0.48\textwidth}
    \centering
\includegraphics[scale=.7]{fig/4-2.png}
    \caption{}
    \end{minipage}
    \begin{minipage}[t]{0.48\textwidth}
    \centering
\includegraphics[scale=.7]{fig/4-3.png}
    \caption{}
    \end{minipage}
    \end{figure}

如果想在屏上连续得到稳定而不闪烁的阻尼振荡波形.
可以按下述方法制作一个多谐振荡器控制继电器.用继电器
一组常闭常开接点作为单刀双掷开关的装置.其电路如图4.
3所示.三极管为3DK型,其$\beta$值选40—70间的.各电阻
均可用1/8W碳膜或金属膜电阻.2微法电容也可用两个5
微法电解电容器负极相接后将两正极接入电路.电源用12伏
积层电池,也可用其他电池或低压电源.继电器尽可能选用
线圈电阻500欧以上,吸合电流20毫安以下的小型继电器,
如JRX-4型等,也可将J2414型学生实验用电磁继电器接入
电路使用.继电器常闭接点应接$LC$电路,常开接点接电源.
此电路只要安装与元件无问题,无需调整即可工作.可将整
个装置装在一个小盒中,演示时,最好先按前面方法用手拨
动开关演示两次,再用此装置说明它的作用,使演示的说服力
强一些.

更简单一些的自动开关,可以用一个2CP(或2CZ)二极
管与一继电器串联后接入低压交流源,利用二极管只有半
个周期导电使继电器吸合,另半个周期截止使继电器释放起
到上述自动开关作用,这样可以在荧光屏上显示稳定的阻尼
振荡波形.但由于继电器吸放频率为50赫,相应的$LC$振荡
电路固有周期较小.而一般在振荡电路频率为50赫开关频
率的10倍左右时,才能看到几个周期的衰减波形,如果$L$
利用J2423型可拆变压器0—1400匝绕组,则相应$C$可取
0.1微法或0.2微法.制作这个装置要选择好继电器,最好选
用吸放动作时间短的小型或超小型继电器、干簧继电器等.电
源电压可从继电器直流电阻与吸合电流求出,取两者乘积的
1.5倍即可.继电器常闭接点与$LC$电路相接,常开接点与电
源相接.

(2)$LC$电路的等幅振荡

\begin{figure}[htp]
    \centering
\includegraphics[scale=.7]{fig/4-4.png}
    \caption{}
\end{figure}

$LC$电路的等幅振荡需要利用$LC$振荡器进行演示,振荡
器电路如图4.4所示.这是一个典型的变压器反馈振荡电
路.由$L_1$与$C_1$(或$C_2$)组成$LC$振荡电路,经$L_1$抽头与$C_3$
将振荡信号输入三极管基极,放大后由集电极经$L_2$正反馈,
从而形成$L_1C_1$电路的等幅振荡.图中$R_2$为负反馈电阻,用
以改善振荡波形;$R_1$为三极管偏置电阻,用以调整三极管工
作点.实验时,$C_1$取0.1微法.$L_1$与$L_2$可以使用J2423型
可拆变压器上的线圈.$L_1$用红色线圈0—1400匝绕组的
800 匝抽头.$L_2$用绿色线圈0—400匝绕组.实验前应先
进行调试.首先应先调整好示波器,利用示波器检查电路起
振情况.再将$R_1$调至最大阻值位置,接通12V电源(最好用电
池组).当逐渐减小$R_1$阻值时,就可在示波器上观察到等幅
振荡波形.如果观察不到,将绿色线圈(反馈线圈)头尾接线
颠倒一下,重新调整$R_1$, 就可起
振.适当调节$R_1$和示波器上
衰减钮、Y增益钮、扫描微调钮,
使荧光屏上出现3—5个稳定
波形(一般扫描频率可置10—100, Y衰减可置10处).

这个实验中,$L_1$、$L_2$还可用半导体收音机用的低频放大
推挽输入变压器,$L_1$用推挽初级绕组,$L_2$用次级绕组的一半
(即一端与中心抽头端).也可以自制一个电感线圈:铁心用
GEIB18型硅钢片,叠厚18毫米,其铁心截面积(中央舌宽$\x$
叠厚)为18毫米$\x$18毫米.铁心一般地可从旧的有线广播15
瓦或25瓦变压器中拆出,也可利用学生实验用的可拆变
压器铁心.用0.2毫米直径的漆包线在自制框架上按图4.5
所示匝数绕制成$L_1$与$L_2$(图中抽头多些,是为演示振荡频率
与电感电容关系准备的).实验时,可将0匝与400匝两抽头
取出信号送入三极管放大.相应电容器$C_1$仍可取0.1微法.

\begin{figure}[htp]
    \centering
\includegraphics[scale=.7]{fig/4-5.png}
    \caption{}
\end{figure}

\subsubsection{用示波器观察电磁振荡的周期跟电容和自感系数的
关系}

这个实验与等幅振荡演示实验的装置相同.$L_1$可以用
自制的电感线圈,或者J2423型可拆变压器,接法同前.另外
准备0.2微法或0.5微法的电容器.实验时,只需先后接入不
同电容器,即可看到电容越大周期越长的现象.注意此时扫
描微调只能略微变动,以稳定波形.在扫描频率相同时从两
次出现波形的个数不同比较周期长短.

在演示电感与振荡周期的关系时,若用自制电感线圈,应
先后将0—800匝与0—600匝绕组接入电路进行观察比较.
若用J2423型变压器红色绕组,就先后将0—1400匝绕组
50—800匝绕组(至三极管信号仍从800匝抽头接出)接入电
路,进行对比演示,可看出匝数越多,电感越大,振荡电路周期
越长的现象.

如果有条件,可以从三极管集电极处通过一个10微法电
解电容输出信号到一低频放大器(小扩音机)中,通过喇叭发
声的音调高低来同时了解振荡电路频率(周期)变化情况.

\subsubsection{电磁波的发射与接收(赫兹实验)}

演示这个实验是配合本章第五节赫兹实验的教学进行
的,因而实验应尽可能重现赫兹当年的实验装置.发射天线
与感应圈与课本上的相同,接收装置改为振子天线,并将一
试电笔用氖管接在振子之间,如图4.6甲所示,以提高实验可
见度.实验所用两根发射天线与接收天线都可以用电视机或
收音机用的拉杆天线,将两根天线分别固定在有机玻璃板上,
另用长绝缘棒(塑料棒等)固定在有机玻璃板上作支架,底座
可用木制.注意拉杆天线的固定方法:将有机玻璃板弯成U
型(可用40瓦电烙铁铁管发热部位与折弯处接触烘烤弯制)
在安装天线的部位各钻一个孔.天线固定螺丝用稍长一些的
铜制M3螺丝.将拉杆天线插入孔中,一端装一个静电仪器
用的光滑小铜球(也可用焊锡熔化制成),两铜球之间的缝隙
在5毫米左右,如图4.6乙所示.支座尺寸可根据实际制作
条件确定.

\begin{figure}[htp]
    \centering
\includegraphics[scale=.7]{fig/4-6.png}
    \caption{}
\end{figure}

实验时,将感应圈(如J1206型感应圈)高压线圈两端放
电针卸下,用导线分别将这两端与两根发射天线接好,把感应
圈电源接线端与低压电源(J1201型低压电源)的6—8伏直
流输出端接好,调节好断续器的调节螺丝,使发射天线两球间
产生连续稳定的火花放电.然后将接收天线的拉杆都拉出
来,使天线长度与发射天线相同,平行地靠近发射天线,就可
以看到接在接收天线间的氖管在感应高电压作用下发光.特
别需要注意的是,人体任何部位都不要靠近或接触发射天线,
以免受到电击.

\subsubsection{调制}

演示调制的实验实际上只演示调幅,所用仪器为J2458
型教学示波器、J2464型教学信号源及一台半导体收音机.
演示步骤如下:
\begin{enumerate}
\item 将教学信号源置于中间,半导体收音机置
于信号源高频输出接线柱一侧,示波器置于另一侧.
\item 将示
波器Y输入端与信号源高频输出端相接,Y衰减置于1,扫描
频率置于100—1k,预热示波器,并调出细亮扫描线.
\item 将
信号源“调幅—等幅”开关扳向“等幅”,高频频率范围“频率I
—频率II”开关扳向“频率I”,高频增幅钮顺时针旋到底,频率调节钮应调至当地无线电台信号的中波某一频率,低频频
率选择钮拨至“1k”,低频增幅钮旋至中间位置.
\item 开启信号
源,调节示波器,可以看到等幅波波形,由于频率很高,示波器
荧光屏上将出现宽亮带,这时若将示波器扫描频率拨至“10k—
100k”,调整扫描微调可以看到高频等幅振荡的清晰波形,然
后仍将扫描频率拨回“100—1k”.
\item 将信号源“调幅一等
幅”开关扳向“调幅”,打开半导体收音机.调整调谐钮即可听
到1k赫声音,调整示波器扫描微调,即可看到蝶状调幅波形
(其上下轮廓线就是1k赫低频信号的波形).
\item 调节信号源
低频增幅钮,就可以看到调幅波形轮廓线幅度变大或变小,听
到声音变大或变小,低频增幅逐渐为零,调幅波波形轮廓线
(也是正弦曲线)逐渐变直,就是等幅波了.从而说明低频信
号的强弱反映在调幅波上就是轮廓线幅度的大小.
\item 使信
号源低频频率增高或降低(拨至500, 1.5k, 2k挡),一方面听到
声音音调变化,另一方面看到荧光屏上调幅波轮廓线正弦波
形个数相应变化,从而说明轮廓线的频率反映低频信号的频
率高低.
\end{enumerate}


如果学校中有J2459型学生示波器与J2465型学生信号
源,只要求学生自带一台半导体收音机(小型),就可让学生自
已按上述步骤去观察实验了.这样做也增加了学生使用示波
器与信号源的机会,有助于培养学生使用实验仪器的
能力.

\begin{figure}[htp]
    \centering
\includegraphics[scale=.7]{fig/4-7.png}
    \caption{}
\end{figure}

如果想自制一台信号源,可按图4.7所示电路进行制作.
图中低频振荡与高频振荡均为共振$LC$振荡器,低频两挡,
由$K_{11}$与$K_{12}$(双刀双掷)转换,在图示开关位置约为400赫,另
一挡约为1000赫,$L_1$、$L_2$分别为半导体收音机用输出变
压器,次级$L_2$不用.高频振荡输出频率约为550千赫至
1700千赫,$L_3$、$L_4$与$L_5$为绕制在磁棒上的自制线圈.绕制时
磁棒可选用MX-400型中波磁性天线棒,长100毫米左
右.线圈用七股纱包线绕制在小纸筒上,$L_3$绕76匝,继续绕
4匝为$L_4$, $L_5$绕15匝.然后将小纸筒套在磁棒上.将$L_3$一
端置于磁棒一端约2厘米左右处,$L_5$置于磁棒中央附近,用
蜡将它们固定好.线圈也可以直接用废旧半导体收音机磁性
天线与其上线圈直接改制.图中打*号的电阻是用于调整三
极管工作点的,整个装置可装在一小金属盒中.

\subsubsection{电谐振和调谐}
电谐振和调谐,可以做两个演示实验.第一个是课本上
图4.16所示的电谐振装置,如无成品,可以用静电起电机(感
应起电机)上的两个莱顿瓶,将自制的矩形固定线框和可动矩
形线框固定在两个木底座上.制线框时要用较粗铜线,线框
尺寸约15厘米$\x$15厘米,可动(谐振)线框固定边稍长一些,
氖管可用试电笔中氖管.实验时将高压感应圈(J1206型)高
压线圈两端与放电线圈两端(莱顿瓶内外金属板)接好,先调
节好放电火花,再调节谐振线框可动边.可以看到,在线框几
何尺寸一样即电感一样时(电容也相同),发生电谐振,氖管发
光.实验过程中,两线框平面距离一般控制在几十厘米至1
米左右,并注意人体不要靠近感应圈高压引线或放电线
圈.另外也可用演示实验3中介绍赫兹振子进行电谐振演
示,实验时只需改变接收天线长度(伸缩拉杆天线),当发射天
线与接收天线几何尺寸相近时,即可观察到电谐振现象(氖管
发光).

\begin{figure}[htp]
    \centering
\includegraphics[scale=.7]{fig/4-8.png}
    \caption{}
\end{figure}

第二个实验是演示实际应用中的电谐振与调谐装置.先
自制一个$LC$谐振电路示教板(图4.8中间图).$L$可在MX-400型中波磁棒上用多股纱包线绕70 匝左右制成,$C$可用
收音机用大型空气单连(改变电容时,可以看到动片旋出),最
大容量270皮法或360皮法均可,注意动片接地.另外将
J2464型教学信号源(J2465型学生信号源或其他信号源)的
高频输出端接一根用约半米长铜丝弯制的环状天线.调整好
示波器(同上一个演示实验,只是将Y增益调至最大),将$LC$
电路电容两端通过屏蔽线接入示波器.实验步骤如下:
\begin{enumerate}
\item 由信号源产生1千赫调幅信号,此时,高频增幅钮顺时旋到低,
信号源高频载波频率可置于较低频率.
\item 将$LC$调谐电路 靠
近天线,调整电容大小,可以看到,只有电容动片旋到某一位置
时,示波器上显示幅度较大的1千赫调幅波形,说明此时发生
电谐振.
\item 改变信号源高频载波频率,那么必须随之调整调
谐电路电容,才能重新发生电谐振,观察到较大幅度1千赫调
幅波形,从而说明从多个电台中选台的道理.
\end{enumerate}

\subsubsection{检波}


\begin{figure}[htp]
    \centering
\includegraphics[scale=.7]{fig/4-9.png}
    \caption{}
\end{figure}

为了较好地演示检波原理,建议用自制的示教板配合信
号源与示波管进行演示,在图4.9中,$LC$调谐回路可用上
面实验所用$LC_1$调谐回路示教板,检波部分示教板电路如图
4.9 左边虚线所示.其中,$D$为2AP型二极管,负载电阻
$R$为10千欧,$C_2$是高频旁路电容,可选0.01微法电容器,$C_3$
为输出耦合电容,可用10微法10伏的电解电容器,演示时
首先上面实验的方法调整好信号源与示波器,并调好$C_1$, 使
$LC_1$电路产生电谐振,用示波器观察$A$点波形,可在荧光屏上
出现幅度较大的调幅波形(蝶状波形)(图4.10甲).观察$B$点波
形,可以看到经检波后的单向脉动高频振荡电流波形,其轮廓
线为低频信号波形.注意,由于二极管本身非线性,不是理想
二极管,显示出的波形下方不是严格的一条直线(图4.10
乙).然后闭合$K$可以看到经$C_2$旁路后,$R$两端波形为低频
信号波形(图4.10丙).如果再观察$C$点波形,可以看到刚
才低频波形向下移动一段距离,这表明刚才$B$点波形中有直
流成分,经$C_3$的隔直流通交流作用后,$C$点与地间输出的只
是低频交流信号(图4.10丁).

\begin{figure}[htp]
    \centering
\includegraphics[scale=.7]{fig/4-10.png}
    \caption{}
\end{figure}

\subsubsection{简单收音机的构造和工作原理}
在实验6中介绍的示教电路,实际上已是最简单的收音
机了,与课本上图4.18所示的电路是一致的.若将该电路的
$C$与地点间输出信号接一小扩音机(或自制一低放电路),就
可以听到1千赫调制信号的声音.而改变$C_1$就听不到声音,
这说明了选台的作用.若用导线代替二极管也听不到声音,
说明了二极管的检波作用,如果信号源是自制的,还可以将
电唱机(或小录音机)输出信号用屏蔽线接在高频振荡器中振
荡三极管基极与地间作为发射台,用这个简单收音机就可收
到信号,听到广播了.当然也可将成品信号源输出的信号用
屏蔽线接入三极管基极与地间进行实验,如果当地中波广播
电台信号很强,还可以直接接收电台广播.接收时,要架设室
外天线,使天线通过一个几十皮法的小电容接在调谐回路电
容定片上,动片接地线(如接自来水管等).

\begin{figure}[htp]
    \centering
\includegraphics[scale=.7]{fig/4-11.png}
    \caption{}
\end{figure}

具有放大器的简单收音机的演示实验最好配合学生实验
六“安装简单收音机”的教学进行,图4.11所示电路与课本
图10.6所示简单收音机电路基本一致,略有改动,照图做一
大型示教板,并在示教板上方画出结构方框图与波形图.演
示时,可将信号源与环状天线靠近示教板$L_1C_1$电路的磁棒,
要依次讲解每一部分的作用,并观察$A$、$B$、$C$、$D$各点波
形.制作时,图中的$L_1$在MX-400型长100毫米以上的磁棒
上用多股纱包线绕70匝左右制成,置于距磁棒一端2厘米左
右;$L_2$用同种导线绕8匝左右,置于磁棒中央附近.实际收听
当地电台广播时,可适当调节$L_1$、$L_2$位置,有最佳效果时再固
定.GZL为高频扼流圈.自制时可用50千欧以上,1/2瓦
碳膜电阻当骨架,用直径0.08毫米左右的高强度漆包线在其
上绕400匝左右,两端与电阻引线焊好即成.$T$为半导体收
音机用输出变压器,如果是推挽式的,可只用初级两端,中心
抽头不用,次级阻抗与喇叭阻抗应匹配,一般多是8欧的.电
阻$R_1$与$R_4$均为偏流电阻,制作时,可用20千欧1/8瓦电阻
与330千欧电位器串联接入电路,以便于调整.三极管3AG
与3AX的$\beta$值应尽量选高一些的为好,一般可选60—100
范围内的,穿透电流$I_{ceo}$要尽可能小些,电源尽量用电池,也
可用低压电源的稳压输出.

电路的工作原理,一般不作介绍.但是,由于高中学生求
知欲较强,对于有兴趣的学生,可以根据情况作简要介绍.一
般可以先介绍低频放大电路.学生学习过三极管的放大作用,
这里可以先说明低频信号通过$C_6$加在基极与地(发射极)之
间,引起三极管基极电流变化,造成通过集电极电流的较大
变化,通过变压器将放大后的低频信号输入到扬声器发声,偏
流电阻$R_4$的作用,可以通过不接入,阻值较大或较小时示波
器波形的变化,说明要使三极管完整地放大有正负半周变化
的信号电流,必须使三极管有一定工作电流,使正负半周变化
反映在集电极电流的大小变化上,然后再介绍高频放大器,
$R_1$的作用同上,这里主要说明高频放大的道路.由于$C_3$有
隔直流通交流作用,因此$L_2$上感应高频信号加在${\rm BG}_1$的基
极与发射极之间,由基极电流变化引起集电极电流较大变化.
由于高频扼流圈$GZL$有阻交流通直流的作用,对高频信号
呈现较大阻抗(可由$X_L=2\pi fL$估算出阻抗约10千欧以上),
而$C_4$有隔直流通交流作用,对高频信号阻抗很小(也可由$X_c=\frac{1}{2}\pi fC$ 估算出约几百欧),
因此放大的高频信号通过$C_4$与$R_2$与发射极形成回路,可以画出如图4.12所示高频放大的等效电路,说明在$R_2$两端得到放大的高频信号,并配合示波器观察该处波形($B$与地间波形)加以印证.

\begin{figure}[htp]
    \centering
\includegraphics[scale=.7]{fig/4-12.png}
    \caption{}
\end{figure}

由于该电路很简单、接收灵敏度较低,用信号发生器与环
状天线靠近$L_1C_1$电路磁棒时,才能够听到较大声音.如果听
当地强电台广播,由于声音较小,只能靠近才能听到,加室外
天线后,效果稍好.学生在自己实验时能听到声音就达到实
验要求了.

\subsection{学生实验}
\subsubsection{安装简单的收音机}
这个实验的电路、元件数据及工作原理在演示实验7中
都已作了说明,如果有配套的高阻耳机也可以代替扬声器与
输出变压器直接接入电路.如果当地电台信号太弱可以考虑
加接室外天线或用信号源进行实验.如有条件还可配备信号
源与示波器由学生自行观察各点波形.这里只说明一下器材、
实验步骤及有关调试问题.

\begin{figure}[htp]
    \centering
\includegraphics[scale=.7]{fig/4-13.png}
    \caption{}
\end{figure}

课本上所说的实验器材是指课本图10.5所示的J2467
型学生电子实验箱中的器材,该实验箱外形与器材情况如图
4.13所示.如果无此实验箱,也可以自制,自制时可以将单
面敷铜板裁成长5厘米宽2厘米左右的长条,两端留1.5厘
米铜箔,中间部分刻去,两端铜箔可各焊一小弹簧(可将吉他
的低音弦中的细铜丝,先烧热退火,绕成螺线状后,再烧红投
入油中淬火即可)作接线柱,元件接在其中,并用漆标明元件
符号、规格、正负.三极管可用稍宽铜箔板制成三端接线板使
用.由于需要调整三极管工作点,实验时还须给每一组配备
一个万用表.电源可用电池组或学生电源6伏稳压输出.实
验步骤可按课本上的说明进行,但应向学生提出接线的先后
顺序,一般习惯是由后至前(即低放先接,依次接检波、高放与
调谐回路),并要接一级检查一级电路是否正确,不要全部接
好后再检查,以使学生养成良好电路接线习惯.在调整三极管
工作电流时,也最好接一级调一级,并特别注意万用表正负端
的接法.在调试时,与万用表相连的引线要稍长一些,以免表
针与电路其他部分相碰(特别是与基极相碰)造成元件损坏.

观察有关各点波形,可参照上面演示实验7进行,学生信
号源高频输出端可以将一根1米左右长的导线两端接好,代
替环状天线.

如果要按前面教法建议部分所述进行只有调谐电路与检
波电路的实验,即按课本上图4.18所示收音机电路进行实
验,$C_1$最好用空气介质的大型单连,$L_1$与$L_2$绕制方法同演示
实验7中所介绍的,只是$L_2$应增至20匝左右,二极管用2AP
型,$C_2$用0.01微法电容,耳机一定要用高阻柱的(1.5千欧
以上).$L_1,L_2$的制作,也可以在直径4厘米的一个浸蜡纸筒
上用直径0.35毫米漆包线绕80匝(可在60匝、70匝处抽头)
与30匝,两线圈相距5毫米左右.实验时要接室外天线(通
过一个50皮法电容接入$L_1$)与地线,$L_1$抽头用以取得较好调
谐效果.实验时,可以分别接入试听.如当地电台信号较弱,
可以改用信号源进行实验.


\section{习题解答}
\subsection{练习一}
\begin{enumerate}
	\item 画一条按正弦规律变化的振荡电流的曲线,并在这条曲线上标出对应于课本图4.2甲、乙、丙、丁、戊各个时刻电流值的点.

    \begin{solution}
        所求图线如图4.14所示.
\begin{figure}[htp]
    \centering
\begin{tikzpicture}[>=latex]
\draw[->](-1,0)--(7,0)node[right]{$t$};
\draw[->](0,-2)--(0,2.5)node[right]{$i$};
\draw[domain=0:6, samples=500, very thick] plot (\x, {1.5*sin(\x*pi/2 r)});
\draw[dashed](0,-1.5)node[left]{$-I_m$}--(3,-1.5)node[below]{丁};
\draw[dashed](0,1.5)node[left]{$I_m$}--(1,1.5)node[above]{乙};
\foreach \x/\xtext in {1/\dfrac{T}{4},2/\dfrac{T}{2},3/\dfrac{3T}{4},4/T,5/\dfrac{5T}{4},6/\dfrac{3T}{2}}
{
    \draw(\x,0)node[below]{$\xtext$}--(\x,0.1);
}
\node at (0,0)[above left]{甲};
\node at (2,0)[above right]{丙};
\node at (4,0)[above left]{戊};
\end{tikzpicture}
    \caption{}
\end{figure}
    \end{solution}
    
	\item 在课本图4.2所示的电磁振荡中,何时电容器里的电场最强?何时线圈里的磁场最强?电场能和磁场能是怎样相互转化的?

    \begin{solution}
在$t=0, T/2, T$时(图中的甲、丙、戊),电容器里的电
场最强;$t=T/4, 3/4T$(图中的乙、丁)时,线圈里的磁场最强.
电容器充电后,两极板间的电场里储存着电场能.电容器通
过线圈放电,形成放电电流.两极板储存电荷电量减少,电压
降低,电容器中电场减弱,电场能减少;但电路中放电电流增
大,线圈里磁场逐步增强,磁场能逐渐增大.在这个过程中,电
路的电场能逐渐转化为磁场能,当电容器放电完毕时,电路里
的电流最大,这时电路里的电场能完全转化为磁场能.由于线
圈的自感作用,该电流形成反向充电电流,磁场能又逐渐转化
为电场能.反向充电完毕时,电路中的电流强度减小到零,电
容器两极板间的反向电场达最大,电路中的磁场能完全转
化为电场能.然后又开始反方向的放电过程,电场能又转化为
磁场能,这种过程反复进行下去,电场能与磁场能通过电路里
的充、放电电流,不断实现互相间的转化.
    \end{solution}
    
	\item 把$LC$回路中产生的自由振荡跟单摆的简谐振动相对比,说明它们类似的地方.

    \begin{solution}
将$LC$回路中产生的自由振荡跟单摆的简谐振动类
似的地方列出下表进行对比.
\begin{center}
\begin{tabular}{p{.4\textwidth}p{.4\textwidth}}
\hline

\multicolumn{1}{c}{$LC$振荡电路}& \multicolumn{1}{c}{单摆}\\
\hline
用电源给电容器完电,电容器获得电场能,$LC$电路具有了初始能量.
 &用外力使摆球偏离平衡位置,摆球获得重力势能,单摆具有了初始能量.\\\hline
 电容器放电,两极板间场强不断减小,放电电流不断增大,电场能逐渐转化为磁场能.放电完了时,电场能完全转化为磁场能.
&摆球摆向平衡位置,高度不断增大,重力势能逐渐转化为动能,到达平衡位置时,重力势能最小,动能最大.\\\hline
电容器反方向充电,充电电流不断减少,两极板间的场强不断增大,磁场能逐渐转化为电场能.充电完了时,磁场能完全转化为电场能.
&摆球从平衡位置上升速度不断减小,高度不断增高,动能逐渐转化为重力势能.达到最高点时,动能完全转化为重力势能.\\\hline
电容器反方向放电,两极板间的场强不断减小,放电电流不断增大,电场能逐渐转化为磁场能.反方向充电完了时,电场能完全转化为磁场能.
& 摆球沿着与原来相反的方向摆向平衡位置,高度不断降低,速度不断增大,重力势能逐渐转化为动能,到达平衡位置时,重力势能最小,动能最大.\\\hline
电容器重新开始充电,磁场能转化为电场能.电容器充电完了时,$LC$回路回复到初始状态.
&摆球从平衡位置沿着与原来相反的方向上升,动能转化为重力势能.
摆球上升到最高点,单摆回复到初始状态.\\
\hline
\end{tabular}
\end{center}
    \end{solution}
\end{enumerate}



\subsection{练习二}
\begin{enumerate}
	\item 一个$LC$回路能够产生535千赫到1605千赫的电磁振荡,已知线圈的自感系数是300微亨,可变电容器的最大电容和最小电容各是多少?

    \begin{solution}
        当线圈自感系数一定时,$LC$回路中电容器电容越大,振荡频率越低.根据$LC$振荡电路频率公式
\[f=\frac{1}{2\pi\sqrt{LC}}\]
可得:
\[\begin{split}
    C_{\text{大}}=\frac{1}{4\pi^2\cdot f^2_{\text{低}}\cdot L}&=\frac{1}{4\pi^2\x (535\x 10^3)^2\x 300\x 10^{-6}}\\
    &=0.295\x 10^{-9}{\rm F}=295{\rm pF}\\
\end{split}\]
    \[\begin{split}
    C_{\text{小}}=\frac{1}{4\pi^2\cdot f^2_{\text{高}}\cdot L}&=\frac{1}{4\pi^2\x (1605\x 10^3)^2\x 300\x 10^{-6}}\\
    &=32.8\x 10^{-12}{\rm F}=32.8{\rm pF}\\
\end{split}\]
此题求$C_{\text{小}}$还可利用比例求解:
因为$f\propto\dfrac{1}{\sqrt{C}}$,所以
\[\frac{f_{\text{高}}}{f_{\text{低}}}=\frac{\sqrt{C_{\text{大}}}}{\sqrt{C_{\text{小}}}}\]
因此,
\[C_{\text{小}}=\frac{f^2_{\text{低}}}{f^2_{\text{高}}}\cdot C_{\text{大}}=\left(\frac{535\x 10^3}{1605\x 10^3}\right)^2\x 295=32.8{\rm pF}\]
    \end{solution}
    
	\item $LC$回路中的可变电容器的电容可从30皮法变到15皮法.要使这个回路的最低固有频率为1000千赫,线圈的自感系数应为多大?用这个线圈,回路的最高固有频率是多大?

    \begin{solution}
        当可变电容器电容最大时,$LC$回路的振荡频率最
        低.由频率公式
        \[f_{\text{低}}=\frac{1}{2\pi\sqrt{L\cdot C_{\text{大}}}}\]
可得:
\[\begin{split}
    L=\frac{1}{4\pi^2\cdot f^2_{\text{低}}\cdot C_{\text{大}}}&=\frac{1}{4\pi^2\x (10^6)^2\x 30\x 10^{-12}}\\
    &=0.84\x 10^{-3}{\rm H}=0.84{\rm mH}
\end{split}\]
因为$f\propto\dfrac{1}{\sqrt{C}}$,所以
\[\frac{f_{\text{高}}}{f_{\text{低}}}=\frac{\sqrt{C_{\text{大}}}}{\sqrt{C_{\text{小}}}}\]
因此,
\[f_{\text{高}}=\sqrt{\frac{C_{\text{大}}}{C_{\text{小}}}}\cdot f_{\text{低}}=\sqrt{\frac{30\x 10^{-12}}{15\x 10^{-12}}}\x 10^6=\sqrt{2}\x 10^6{\rm Hz}=1.4{\rm MHz}\]
    \end{solution}

	\item 如果把$LC$回路中电容为$C$的电容器用两个电容也为$C$的电容器串联起来代替,回路的固有周期怎样变化?如果不是串联而是并联,固有周期又怎样变化?

    \begin{solution}
设变化了的固有周期分别为$T'$和$T''$.两个电容为
$C$的电容器串联后的电容
\[C'=\frac{1}{2}C\]
根据$LC$回路有周期$T\propto \sqrt{C}$,
\[T:T'=\sqrt{C}:\sqrt{C'},\qquad T'=\sqrt{\frac{C'}{C}}\cdot T\]
所以
\[T=\sqrt{\frac{C/2}{C}}\cdot T=0.707T\]
即周期变短.

当两电容器并联后的电容为$2C$. 这时的固有周期
\[T''=\sqrt{\frac{2C}{C}}\cdot T=1.414T\]
即周期变长.
    \end{solution}
    
\end{enumerate}



\subsection{练习三}
\begin{enumerate}
	\item 从地球向月球发射电磁波,经过多长时间才能在地球上接收到反射回来的电磁波?地球到月球的距离为$3.84\x10^5$千米.

    \begin{solution}
由$v=s/t$得$t=s/v$,  代入数值可得
\[t=\frac{s}{v}=\frac{2\x 3.84\x10^5\x 10^3}{3.00\x 10^8}=2.56{\rm s}\]
    \end{solution}
    
	\item 我国第一颗人造地球卫星采用20.009兆赫和19.995兆赫的频率发送无线电信号,这两种频率的电磁波的波长各是多少?

    \begin{solution}
        本题波长应计算到5位有效数字,书末附表中真空
        中光速只给出三位,因此需查教材第186页所给光速值
        $c=299792458\ms$,计算时应取$c=299792\x10^3\ms$.

由$\lambda=c/f$知:
\[\begin{split}
\lambda_1&=\frac{c}{f_1}=\frac{299792\x10^3}{20.009\x 10^6}=14.983{\rm m}\\
\lambda_2&=\frac{c}{f_2}=\frac{299792\x10^3}{19.995\x 10^6}=14.993{\rm m}\\
\end{split}\]
    \end{solution}
    
	\item 有一台收音机,它接收的波长范围由560.7米到186.9米,它接收的频率范围是多少?

    \begin{solution}
该接收机接收的最低频率
\[f_1=\frac{c}{\lambda_1}=\frac{3.00\x10^8}{560.7}=535\x 10^3{\rm Hz}=535{\rm kHz}\]
该接收机接收的最高频率
\[f_2=\frac{c}{\lambda_2}=\frac{3.00\x10^8}{186.9}=1605\x 10^3{\rm Hz}=1605{\rm kHz}\]
因而接收机接收的频率范围为535千赫至1605千赫.
(这个频率范围是我国规定的中波广播频率范围.)
    \end{solution}
    
	\item 一个振荡电路辐射出的电磁波的波长是300米,这个振荡电路的周期是多大?

    \begin{solution}
由$c=\lambda/T$知:
\[T=\frac{\lambda}{c}=\frac{300}{3.0\x 10^8}=100\x 10^{-8}=1.00\x 10^{-6}{\rm s}\]
    \end{solution}
    
\end{enumerate}



\subsection{练习四}
\begin{enumerate}
	\item 有一台收音机,把它的调谐电路中的可变电容器的动片从完全旋入到完全旋出,仍然接收不到某一较高频率电
	台的信号,要想接收到该电台的信号,应该增加谐振线圈的匝
	数,还是减小谐振线圈的匝数?为什么?

    \begin{solution}
应该减小谐振线圈的匝数.当可变电容器动片完全
旋出时,其电容最小,谐振电路固有频率最高.如果要接收
较高频率电台的信号必须提高谐振电路固有频率.根据
$f\propto 1/\sqrt{L}$, 应适当减小线圈自感系数值.因为线圈自感系数
与匝数有关,匝数越少,$L$值越小,因此可用减小谐振线圈匝数
的方法来提高谐振电路的频率,以接收较高频率电台的信号.
    \end{solution}
    
	\item 收音机由收听某一较高频率的电台改为收听某一较低频率的电台,调谐电路中可变电容器的动片应该旋进一些还是旋出一些?为什么?

    \begin{solution}
    应旋进一些.当可变电容器动片旋进时,电容变大.
在电感线圈自感系数一定时,谐振回路的频率与电容器电容
平方根成反比.电容增大,谐振频率降低.
    \end{solution}
    
	\item 某收音机调谐电路的可变电容器的动片完全旋入时,电容是390皮法,这时能接收到520千赫的无线电波.动片完全旋出,电容变为39皮法,这时能接收到的无线电波的频率是多大?在这两种情形下,接收到的电磁波的波长分别是多长?

    \begin{solution}
调谐回路电感线圈自感系数一定时,其谐振频率与
电容平方根成反比,即:$f\propto 1/\sqrt{C}$. 因此,
\[\frac{f_1}{f_2}=\frac{\sqrt{C_2}}{\sqrt{C_1}}\]
\[f_2=\sqrt{\frac{C_1}{C_2}}\cdot f_1=\sqrt{\frac{390\x 10^{-12}}{39\x 10^{-12}}}\x 520=1644{\rm kHz}\]
这两种情形下所接收的电磁波波长分别为:
\[\begin{split}
\lambda_1&=\frac{c}{f_1}=\frac{3.00\x 10^8}{520\x 10^3}=577{\rm m}\\
\lambda_2&=\frac{c}{f_2}=\frac{3.00\x 10^8}{1644\x 10^3}=182{\rm m}\\
\end{split}\]
    \end{solution}    
\end{enumerate}


\subsection{习题}
\begin{enumerate}
	\item 要使$LC$回路的固有频率增大,应该采用下述哪种方法?
	\begin{enumerate}
		\item 增大电容器两极板间的距离;
		\item 增大电容器两极板的正对面积;
		\item 在线圈中插入铁心;
		\item 减小线圈的匝数.
	\end{enumerate}

    \begin{solution}
根据$f=\dfrac{1}{2\pi\sqrt{LC}}$
可知,要使$LC$回路固有频率增
大,可以减小$L$或$C$的值.根据决定$L$与$C$值的条件可知,
增大电容器两极板距离时,电容$C$减小;减小线圈匝数时,自
感$L$减小,这两种方法都可使$LC$回路固有频率增大,因而
应采用(a)(d)所说的方法.(b)(c)所说的方法都将使固有频率
减小.
    \end{solution}
    
	\item 赫兹实验中的发射器为什么用课本图4.9那种形式,面不用该实验中接收器那种形式?赫兹实验中需要把接收器的两金属球调节到合适的距离,才有明显的接收效果,为什么?

    \begin{solution}
        赫兹实验中发射器用课本图4.9这种形式是为了使
        发射回路形成开放电路,以使电磁辐射能力增大.把接收器
        两金属球调节到合适距离时,接收回路的固有频率将与发射
        频率相同而发生电谐振,使接收效果明显.
    \end{solution}
    
	\item 回旋加速器中的磁感应强度为$B$,被加速的粒子的电量为$q$,质量为$m$.用$LC$振荡器作为高频电源,电感$L$和电容$C$的数值应该满足什么条件?

    \begin{solution}
在回旋加速器中,造成两个$D$型盒窄缝间交变电场的
高频电源频率应满足的条件是:
\[f_1=\frac{1}{T}=\frac{qB}{2\pi m}\]
$LC$振荡电路的频率由自感$L$和电容$C$决定,即:
\[f_2=\frac{1}{2\pi\sqrt{LC}}\]
因此如果采用$LC$振荡器作为回旋加速器的电源,应满足条
件$f_1=f_2$, 即
\[\frac{1}{2\pi\sqrt{LC}}=\frac{qB}{2\pi m}\]
所以
\[LC=\frac{m^2}{q^2B^2}\]

由上式知振荡器电感$L$和电容$C$应满足的条件是两者
乘积等于粒子质量平方除以粒子电量与磁感应强度乘积的
平方.
    \end{solution}
    
	\item 附近有电焊机进行电焊,或者有汽车通过,收音机和电视机会受到干扰,解释这个现象.

    \begin{solution}
    电焊机工作时产生电弧,与赫兹实验中电火花一样,
都会产生电磁波,汽车发动机利用电火花点燃压缩燃料与空
气混合物完成做功冲程,电火花也会激发电磁波,所以收音机
和电视机会受到干扰.

注:实际上这种火花放电产生电磁波频谱极宽,波形也很
复杂,因此不宜过多解释.
    \end{solution}
\end{enumerate}


\section{参考资料}
\subsection{电子学的发展和电子技术有关资料}

电子学来源于电磁学,并在无线电通信技术基础上
发展起来,1864年麦克斯韦提出的电磁理论预见了电磁波
的存在,23年后,即1887年赫兹成功地进行了用人工方法产
生电磁波的实验.7年后,意大利的马可尼(1874—1937年)
和俄国的波波夫(1859—1906年)分别独立地进行了距离几
百米到千米的无线电报通信试验(马可尼实验距离为一千
五百米,波波夫为二百五十米,马可尼得到了1909年诺贝尔
物理学奖).1901年,马可尼横跨大西洋的无线电报通信试
验成功.1906年德法雷斯特(美国、1873—1961年)发明真空
三极管以后,电子学和电子技术迅速发展,并在基础理论研究
推动下向高频(微波)和电子器件研究两个方向发展,并迅速
应用于各种领域,第二次世界大战期间,为了研制雷达的需
要,出现了与电子管工作原理本不同的速调管、行波管.四
十年代后期,实现了黑白电视广播.1946年出现了第一台用
电子管制成的“电子数字计算机”.1948年美国人肖克莱、巴
丁和布拉登发明了晶体管,为此,他们获得了1956年诺贝尔
物理学奖,晶体管的发明使电子学和电子技术进入新的一代.
1958年出现小规模集成电路,60年代出现第三代计算机,50
年代后期适应空间技术发展产生了“量子振荡器”和“微波激
射器”,在此基础上,60年代出现了“光量子放大器”或称“受
激辐射光放大器”,现在称为“激光”,使光导通信成为可能,电
子学和电子技术的展,使自动化控制从代替人们的体力劳
动发展到代替部分脑力劳动.由于电子技术已渗透到社会各
个领域,因此电子技术的水平已成为现代化的个重要标志.

电子学和电子技术的内容丰富,应用十分广泛,发
展非常迅速.电子学是研究电子和电磁场运动、电路理论及
信息传输系统一般规律及其应用的科学,研究掌握这些方面
的客观规律是电子学基础研究的内容,运用这些规律研究制
造各种电子元件、器件、设备和系统,并广泛地应用于国民经
济、国防以及科学技术的各个领域,是电子技术的任务.

电子学基础理论研究大致有这样几个方面:
\begin{enumerate}

\item 电磁场理论.电磁场理论研究从经典的宏观方面已
发展到量子的微观方面,从自由空间的传播理论发展到特殊
介质里的传播理论,研究电磁场理论,掌握电磁波传播规律,
具有重要现实意义,例如,远距离短波信是在初步掌握电
磁波在高空电离层反射的规律的基础上发展起来的,由于弄
清了超高频电磁波在对流层散射的机理才发展起来的远距离
散射通信的,现在正进一步研究电磁波在地层中传播的规律,
以适应国防、遥感勘探、地震预报技术发展的需要.进一步
研究电磁波在等离子体中的传播规律,以适应可控热核反应
及空间科学中高空等离子体研究的需要.
\item 电子运动理论研究,电子真空三极管、速调管、行波
管、磁控管、电子束管(如示波管、显像管等)、摄像管、晶体管、
集成电路等电子器件,都是研究电子运动规律后的产物.研究
电子从物质表面发射的机理、电子在磁场电场中的运动规律,
是“阴极电子学”、“电子光学”、“微波管电子学”的主要课题.
值得注意的是,这些理论以及器件的发明,都是以物理学基础
理论研究成果为先导的.在对电磁场和电子注能量交换的理
论研究之后,发明了速调管及其它微波管;近代量子力学的建
立以及固体物理中能带理论和导电理论的发展,致使晶体管
和其他半导体器件的发明;微波波谱学的基础研究,导致了受
激辐射微波放大器的发明,并且诞生了激光器.
\item 电路理论和电路技术.电路理论主要是网络理论,网
络理论和电路技术解决的是将电子元、器件有效组合成为电
子设备和电子系统的问题.与网络理论密切结合的一些应用
数学分支如矩阵代数、网络拓扑学也相应发展了.
\item 电磁波新波段的开辟与应用.电磁波新波段的开辟
与应用是电子学研究的重要方向,新波段开辟应用是以能产
生新波段电磁波的电子器件为先导的,依靠真空三极管开辟
了长波无线电通信,真空管的改进又开辟了短波通信领域,速
调管、行波管与磁控管的发明开拓了微波波段,而微波技术在
现代仍是电子学重要而活跃的方面,它广泛适用于微波通信、
微波雷达、微波遥感、射线天文、基本粒子研究(加速器)等领
域,微波的生物生理化学作用也被应用于生物、医学领域,当前
正在研究开辟毫米波与亚毫米波段的电子器件.光波段应用
由于激光器的发明已被广泛运用于光导通信等方面了.可以
说,不远的将来,电磁波谱中全部波段都将被人类开拓应用.
\item 电子学另一基础研究方面是信息论研究.信息论是在
第二次世界大战结束后,由一部分数学工作者和多年从事通
信、雷达工作的电子学工作者,总结实践经验并提高到理论高
度而创立的一门新学科,它研究各种信息传输系统的共同规
律.这里所指的信息,意义广泛,可以是电话、电报、电视、雷
达、声纳、计算机及全息照相系统中的信息;也可以是各类生
物神经系统中所传输的信息;还可以是遗传学中遗传因子的
信息,等等.信息论主要研究如何有效、可靠、快速、经济地组
成信息传输系统传输信号,例如现在迅速发展的编码理论、信
号检测估值理论、模式识别理论和随机过程理论将在许多
方面得到应用.

\end{enumerate}

电子学和电子技术在应用过程中,与其他学科和技术相
结合产生了一系列新的研究与应用领域,丰富了电子学的研
究内容,大大促进了电子技术的发展.其中一些重要方面是:
\begin{enumerate}
\item 材料和工艺:大规模集成电路和其他新电子器件所需
的新材料、新工艺的突破.例如超低温、超纯度、超高压材料.
集成电路光刻工艺将发展到电子束光刻、X光光刻工艺,人们
已经在考虑分子或原子级的加工工艺了.
\item 电子计算机向着巨、微两个方面发展.巨型是指存贮
容量和运算速度将是现在的几百倍的计算机.微型机则将大
量进入人们的家庭,直接为人们起居饮食、学习娱乐、通信联
系服务.电子计算机技术将带来工业上的一次革命.
\item 电子学与生物、医学科学相结合的设备,如电子医疗
设备(医用加速器、医用无损性扫描设备、人工心肺、人工肾、
心脏起博器等),生理生物基础研究设备(如人体参数无损
测量、遥感诊断、听觉视觉机理研究等).值得特别提出的是
仿生电子学开始发展为仿生工程,它研究生命体的各种复杂
功能,发现它的规律,并用电子学方法摸拟其功能.创造新
的仿生电子设备.
\item 光学和电子学结合形成光信息处理、光电子学,这是
从光通信技术发展起来的.预计光纤通信将逐步代替电缆通
信.这也是通信技术上的一个重大变革.
\item 空间电子学.空间科学的发展大大促进了电子技术
的发展,它不但要求电子技术为其提供高质量的、微型化的电
子设备,还要求提供超远程、高精度、大容量的信息传输系统.
控制系统和存贮系统,同时也给电子学带来新的研究领域,如
研究地球外层高空等离子体、地磁变化规律、内辐射带与外辐
射带强度位置变化规律及其对电磁波传播与空间通信的影响
等等.

空间电子技术的发展,使人类实现了空间通信-同步
通讯卫星通信也使人类获得了遥感遥测这样先进的测量方
法,同时也带来许多间接成果.例如,高精度的时间系统促使
量子振荡器的诞生,不仅解决了频率稳定问题,也给人们提供
了时间基准,目前量子用振荡器制成的原子钟已经代替天文
钟作为时间基准了.
\item 自动化技术.现在自动化技术逐渐地发展为“智能”
自动化,出现了机器人、机械手等电子计算机技术和工业自动
化相结合的产物.在此基础上发展起一门新学科:控制论.
\item 一些重要的边缘电子技术,如超导材料与器件、电子
显微镜、射电天文和气象、能源技术等等.
\end{enumerate}

\subsubsection{调制、调幅、调频与调频立体声广播}
调制信号对载波的调制本质上是产生新频率的过程.以
调幅为例,当调制信号和载波电压同时加在非线性元件时,产
生调幅电流,这个调幅波包含着围绕载波频率分布的若干新
频率,分别称为上、下边带,理论上可以证明,若载波频率为
$f_0$, 调制信号频率为$f$, 则经调制后产生调幅(或调频)波频率
范围是$f_0-f$至$f_0+f$, 这个范围称频带宽度,中波范围为535
千赫至1605千赫,由于频率范围较窄,按国际规定,每个中
波(短波)电台只能占有10千赫的频带宽度,否则电台拥挤,
将互相干扰.因此,调幅广播音频信号最高频率不能超过5
千赫.在广播音频信号中,高于5千赫的音频成分无法发送,
又由于调幅波抗干扰能力差,致使中波(短波)广播的质量不
是很好.

调频电台工作在88—108兆赫频率超短波范围内,频率
范围比中波宽得多,每个电台占用频带宽度为200千赫.作为
音频广播,音频信号频率范围为30—15000赫,基本上包括
了人耳能感受到的声波频率.又由于调频波抗干扰能力强,
所以调频广播音质要比调幅广播好得多.但由于工作在超短
波波段,电波基本上以直线方式传播,传播距离较小(一般只
是几十千米范围).

由于调制方式不同,解调方法也不同,因此收音机也有调
幅收音机与调频收音机之分.

调频立体声广播是在调频单声道和音频立体声技术基础
上发展起来的,为了使调频立体声广播与普通单声道调频广
播都能被同一接收电路接收(称为“兼容”,现在电视广播黑白
电视与彩色电视广播就是“兼容”的),采取了许多技术措施,
形成不同的制式,目前我国采用的是导频制,美国、日本、欧
洲国家也采用此制式,苏联和东欧国家采用的是极化调制制.
制式不同,接收电路不同.

导频制调频立体声广播的调制信号由三部分组成,这三
部分信号频带互相分离,其主信号为左右两声道音频信号之
和(频率范围50—15000赫),这个信号可被一般调频收音机
接收到,但无法区分左右声道.第二个信号为导频信号(频率
为19千赫),这个信号是解出左右声道而用的基准频率(实际
上需倍频,成为38千赫基准频率).第三个信号是利用调幅
方法,将左右两声道音频信号之差调制在38千赫频率上,并
只取其上下边带成的.这三个信号一起再调制88—108兆
赫调频广播波段的某一频率载波发射出去.因此调频立体
声广播的调制信号可用下式表示:
\[A(t)=(L+R)+(L-R)\sin\omega t+p\sin\frac{\omega}{2}
t\]
式中,$L+R$为主信号,$(L-R)\sin\omega t$为付信道信号,$p\sin\frac{\omega}{2}t$为导频信号,其中$\omega$为38千赫的角频率.

在接收机中,先要通过鉴频取出这三个调制信号,然后
分别取出$L+R$与$L-R$信号,再还原为左右声道信号分别
送入放大电路放大后输送到左右扬声器,产生立体声音响
效果.

\subsubsection{麦克斯韦}

麦克斯韦(1831—1879),英国物理学家、数学家
麦克斯韦生于英国爱丁堡.他父亲知识广博,使他从小
就受到科学的熏陶.读中学时就对数学和物理发生了兴趣.
1847年考入爱丁堡大学学习数学、物理,1850年秋,转入剑
桥大学,专攻数学.1854年留校工作,并于同年开始电磁学
的研究.1860年任英王学院自然哲学和天文学教授,1871年
任剑桥大学物理教授,领导筹建卡文迪许实验室,并任第一任
主任.麦克斯韦是英国伦敦皇家学会会员.

麦克斯韦在物理学中的最大贡献是建立了统一的经典电
磁场理论和光的电磁理论,预言了电磁波的存在,他在法拉
第工作的基础上,总结了十九世纪中叶以前对电磁现象的研
究成果,建立了电磁场的基本方程,得出了光的电磁本质以及
预言了电磁波的传播速度.1873年,他的著作《电学和磁学论》
出版,这部著作包括了他在电磁学研究方面的所有成果,它的
出版标志着经典电磁理论的建立.

麦克斯韦对统计物理学的建立也作出了重要的贡献.他
运用数学统计方法,导出了分子运动论的麦克斯韦速度分布
律.他在热力学、光学、分子物理学和液体性质的理论等方面
都取得了成就.

1879年11月5日,麦克斯韦在剑桥去世,年仅49岁.后
人为纪念他,把电磁系单位制中磁通量的单位命名为麦克斯
韦,简称“麦”.

\subsubsection{赫兹}
赫兹(1857—1894),德国物理学家.
青年时代的赫兹,在数学、科学和手工艺方面才能出众.
1877年进入慕尼黑大学学习.1878年转到柏林大学学习.
1879年在物理竞赛中成绩突出,荣获金质奖章.1880年获博
士学位,1883年任物理学副教授,1885年任物理学教授.

赫兹在物理学上最主要的贡献是用实验成功地证明了电
磁波的存在,完善了麦克斯韦的电磁场理论.

他通过实验证明了电磁波以与光速相同的速度直线传
播,并具有反射、折射、偏振等性质.赫兹的实验证实了麦克
斯韦关于光是一种电磁波的理论.

赫兹还首先发现了光电效应现象,这一发现,成了爱因
斯坦建立光量子理论的实验基础.

赫兹关于电磁波的实验为无线电技术的发展开辟了道
路,他被誉为无线电通信的先驱.后人为纪念他,把他的名字
命名为频率的单位,简称“赫”.