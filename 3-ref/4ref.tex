\chapter{电磁振荡和电磁波}\minitoc[n]
\section{教学要求}

这一章在电场、磁场、电磁感应等知识的基础上,讲述电
磁振荡和电磁波,以及无线电发射和接收的初步知识,是电学
知识的继续,也是振动和波的知识的发展。这一章的知识又
为以后认识光的电磁本质作准备。

本章内容既讲述基本知识又讲解实际应用,全章知识多
是介绍性的,这是本章的特点,讲述基本知识时,应着重介
绍主要之点。这一章的核心内容是麦克斯韦电磁场的理论。
电磁波的实际应用中,需要解决许多实际技术问题,教材对解
决发射与接收的主要问题——开放电路、调制、调谐和检波
作了简要介绍,讲述时应着重介绍原理和解决问题的思路,
不宜过多地涉及技术细节。

本章教材可分为三个单元。第一单元包括第一节和第二
节,讲述电磁振荡的产生和$LC$振荡电路的周期和频率公式。
第二单元包括第三节至第五节,讲述麦克斯韦电磁理论要点
和电磁波的知识。第三单元包括第六节至第十二节,讲述电
磁波应用技术初步知识和发展简况。

$LC$回路可以产生振荡电流,要通过演示实验来使学生确
信。电磁振荡的产生过程,要突出电场能和磁场能的相互转
化。由于在中学不容易搞清楚$LC$回路中为什么电流最大时电
压最小这类问题,而进一步深究会加重学生负担,在教学中不
要过多探讨,对$LC$振荡电路的周期公式,应使学生掌握它
的物理意义。由于在后面学习谐时要用到改变电磁振荡周
期的内容,要使学生切实理解电磁振荡的周期怎样随电容和
感的改变而改变。

麦克斯韦电磁场的理论把电场和磁场统一起来,要让学
生了解麦克斯韦理论的两个要点,这两个要点都是用场的观
点分析得出的,虽然学生已经学过电场和磁场,但不熟悉用
变化的场分析现象。要使学生明确由变化磁场产生电场跟是
否有闭合电路无关,由变化的电场产生的磁场跟空间里存在
电流而产生的磁场是一样的,但无需提出位移电流的概念。教
材还指出了均匀变化的场产生稳定的场,非均匀变化的场产
变化的场,这是为讲电磁波作准备的。在分析时,要向学生
说明什么是均匀变化的场和非均匀变化的场,以帮助学生理
解怎样用变化的场分析有关现象。

在讲解电磁波的性质时,要突出它可以脱离电荷而独立
存在、具有能量、不需要别的物质做媒质等特点。正是电磁波
的这些特点,表明电磁场是客观存在的物质。赫兹实验证实
了电磁波的存在,确立了光的电磁说。有条件的学校应尽量
演示这个实验,使学生确信麦克斯韦理论的正确,也有助于认
识电磁场的物质性。

麦克斯韦电磁理论在教学中之所以重要,还在于它在思
想方法上给人们以重大的启示。许多重大发现的提出往往不
是在其理论系统完成之时,而是在人们根据种种物理现象定
性分析、深入思考并进行猜测之时,这是创造性思维的特点。
这一章无论是麦克斯韦理论的提出,还是$LC$电路产生电磁
振荡的物理过程,以及电磁波发送与接收过程的教学都要充
分重视从人们是怎样提出和思考问题的角度进行教学,避免
仅作一般知识介绍,这样会有利于进行创造能力的培养,对于
掌握教材内容也是有益的。

本章安排的学生实验是“安装简单的收音机”,其目的是
通过安装、调试,使学生确信接收原理,培养他们的动手能力,
而不在于熟悉具体的线路。因此,只要求了解实验线路图中
的可变电容器、二极管、三极管、耳机的作用,对线路中的其他
元件的作用不要求讲解。

本章的教学要求是:
\begin{enumerate}
\item 了解电磁振荡产生的过程,掌握电磁振荡的周期和频
率的公式。
\item 了解电磁场理论的要点,了解电磁波的产生和特点.
知道电磁场是一种物质形态。知道赫兹实验。
\item 了解无线电发射和接收的基本原理,了解无线电传播
的特性。
\end{enumerate}

\section{教学建议}
\subsection{第一单元~~ 电磁振荡}
这一单元的教材中,首先安排了$LC$振荡电路的演示
实验,以使学生获得$LC$振荡电路中产生振荡电流的初步感性
知识,在此基础上运用电容器充放电现象与电感线圈的自感
现象对$LC$电路产生振荡电流的物理原因作了定性分析,
并进一步从能量转化角度突出指出了$LC$电路产生振荡电流
的物理实质,由此引出了电磁振荡概念以及阻尼、无阻尼振荡
概念,说明了在$LC$振荡电路中维持电磁振荡的条件,可以看
出,教材的基本思路是:
\begin{center}
\begin{tikzpicture}[>=latex]
\node (A) at (0,0){物理现象};
\node (B) at (3,0){物理原因};
\node (C) at (6,0){物理实质};
\node (D) at (9,0){有关概念};
\draw[->](A)--(B);\draw[->](B)--(C);\draw[->](C)--(D);
\end{tikzpicture}
\end{center}
教材的结构是:
\begin{center}
\begin{tikzpicture}[>=latex, align=center]
\node (A) at (0,0){$LC$电路的\\[-1ex]振荡电流};
\node (B) at (3,0){电容与电感\\[-1ex]基本特性};
\node (C) at (6.5,0){电场能与磁场能\\[-1ex]的互相转化};
\node (D) at (9.5,0){电磁振荡};
\node (D1) at (12.5,0.5){无阻尼振荡};
\node (D2) at (12.5,-0.5){阻尼振荡};
\draw[->](A)--(B);\draw[->](B)--(C);\draw[->](C)--(D);
\draw[->] (D)-- (D1);\draw[->] (D)-- (D2);
\end{tikzpicture}
\end{center}



































































































































































































