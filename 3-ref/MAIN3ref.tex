\documentclass[a4paper, openany]{ctexbook}

%\usepackage{anysize}
%\papersize{26cm}{18.5cm}
%\marginsize{2.25cm}{2.25cm}{2cm}{2cm}
\usepackage[margin=2.5cm]{geometry}

\newcommand{\atom}[3]{{}^{#2}_{#3}{\rm #1}}
% 修改脚注的编号为加圈样式,并且各页单独编号

\usepackage{pifont}
\usepackage[perpage,symbol*]{footmisc}
\DefineFNsymbols{circled}{{\ding{192}}{\ding{193}}{\ding{194}}
{\ding{195}}{\ding{196}}{\ding{197}}{\ding{198}}{\ding{199}}{\ding{200}}{\ding{201}}}
\setfnsymbol{circled}



\usepackage{amsmath,amsfonts,mathrsfs,amssymb}
\usepackage{graphicx}

\usepackage[font=bf,labelfont=bf,labelsep=quad]{caption}

\usepackage{tikz}

\usepackage{multirow}

\usepackage{physics}
%\usepackage{amsthm}
\usepackage{ntheorem}
\theoremseparator{\;}



\usepackage{blkarray}
\usepackage{bm}
\usepackage[colorlinks=true, linkcolor=black]{hyperref}



\theoremstyle{plain}
\theoremheaderfont{\normalfont\bfseries} 
\theorembodyfont{\normalfont}


\newtheorem{example}{\bf 例题}[chapter]
\newenvironment{solution}{\noindent {\bf 解答:}}{}%{\hfill $\blacksquare$\par}


\renewcommand{\proofname}{\bf 证明:}
\newenvironment{proof}{{\noindent \bf 证明:}}{}%{\hfill $\square$\par}

\newcommand{\E}{\mathbb{E}}
\renewcommand{\Pr}{\mathbb{P}}
\newcommand{\EP}{\mathbb{E}^{\mathbb{P}}}
\newcommand{\EQ}{\mathbb{E}^{\mathbb{Q}}}
\newcommand{\dif}{\,{\rm d}}
\newcommand{\Var}{{\rm Var}}
\newcommand{\Cov}{{\rm Cov}}



 \usepackage{tcolorbox}
 \tcbuselibrary{breakable}
 \tcbuselibrary{most}

\setcounter{tocdepth}{1}

\setcounter{secnumdepth}{3}



\ctexset {
section = {
	name = {第,节},
	number = \chinese{section}},
subsection = {
	name = {\hspace{2em},、\hspace{-1em}},
	number = \chinese{subsection}
},
subsubsection = {
	name = {\hspace{2em}(,)\hspace{-1em}},
	number = \chinese{subsubsection},
}
}


\renewcommand{\contentsname}{目~~录}

\usepackage{paralist}
\let\itemize\compactitem
%\let\enditemize\endcompactitem
\let\enumerate\compactenum
%\let\endenumerate\endcompactenum
\let\description\compactdesc
%\let\enddescription\endcompactdesc


\usepackage{titlesec}
\titlespacing{\chapter}{0pt}{*1}{*1}
\titlespacing{\section}{0pt}{*1}{*1}
\titlespacing{\subsection}{0pt}{*1}{*1}

\titlespacing{\subsubsection}{0pt}{*1}{*1}

\renewcommand{\le}{\leqslant}
\renewcommand{\ge}{\geqslant}
\usepackage{mathtools}

\setlength{\abovecaptionskip}{0.cm}
\setlength{\belowcaptionskip}{-0.cm}

%\setmainfont{Times New Roman}
\usetikzlibrary{decorations.pathmorphing, patterns}
\usetikzlibrary{calc, patterns, decorations.markings}
\usetikzlibrary{positioning, snakes}

%\cover{cover.pdf}

\usepackage{yhmath}
\newcommand{\ms}{\text{m}/\text{s}}
\newcommand{\cms}{\text{cm}/\text{s}}
\newcommand{\msq}{\text{m}/\text{s}^2}
\newcommand{\cmsq}{\text{cm}/\text{s}^2}
\newcommand{\kmh}{\text{km}/\text{h}}

\newcommand{\x}{\times }

\usepackage{circuitikz}



\usepackage{minitoc}
\setcounter{minitocdepth}{3}  



\begin{document}
\fontsize{11}{14}\selectfont



\title{\Huge\bfseries 高级中学\\物理(甲种本)第三册\\教学参考书\vspace*{2cm} }



\author{\Large 人民教育出版社物理室~~编}
\date{\Large 1987年1月}



\maketitle
\dominitoc
\tableofcontents


\frontmatter

\chapter{前言}
为了帮助教师使用好高中物理(甲种本)第三册教材,我
们编写了这本教学参考书.内容包括全书的说明,以及各章
的教学说明和资料.

“高中物理(甲种本)第三册说明”对这教材的内容安排
及编写这册课本的一些主要想法,作了概括的说明.

各章的教学说明和资料,包括教学要求、教学建议、实验
指导、习题解答、参考资料五项内容,在“教学要求”中,对教
学内容提出了具体的要求和说明,在“教学建议”中,对怎样
进行教学提了参考性意见.在“实验指导”中,提出了演示实
验,学生实验以及课外实验活动中应当注意的问题,还提供了
简单仪器的自制方法和不同的实验方法,补充了一些实验内
容,供教师选用.在“习题解答”中,给出了课本中全部练习和
习题的解答,供教师参考.在“参考资料”中提供了一些可供
教学参考的材料,有些材料也可在教学中引用.

全书的“教学建议”、“习题解答”、“参考资料”分别由北京
工业学院附中王杏村(一、二章)、北京海淀区教师进修学校蒋
宏涵(三、四章)、北京和平街一中王天谡(五—七章)、北京朝
阳中学邵醒凌(八、九章)编写,“实验指导”分别由蒋宏涵(一—
四章)、王天谡(五—七章)、邵醒凌(八、九章)编写.人民教
育出版社杜敏编写其余部分并统稿,刘克桓复审.

欢迎教师对本书提出宝贵意见.

\begin{flushright}
    编者
\end{flushright}


\chapter{高中物理甲种本第三册的说明}

高中物理课本(甲种本)第三册,同前两册一样,是按
照高中物理教学纲要(草案)的较高要求编写的.这册共有九
章教材,第一章至第四章讲电磁学知识,第五章至第七章
讲光学知识,第八章和第九章讲原子和原子核物理方面的
知识.

这册教材的内容,同试用本比较,增加了几何光学和三相
交流电的知识,调整了“电子技术基础”一章的内容,其他内容
与试用本基本相同.

为使本册教材内容要求更适应学生的接受能力,使
他们把基础知识掌握得更好些,本册教材对一些非主干知识
放低了教学要求.

关于直线电流的磁场,只要求学生对这个磁场的磁感应
强度的大小由哪些因素决定有所了解,而不要求用直线电流
的磁场公式$B=kI/r$
进行计算,以免增大习题难度,加重学生
负担.

关于电感和电容对交流电相位的影响,现象虽然容易演
示,但道理不易说清.为便于教学,将这两项内容改为选讲.
教学中只要求通过实验介绍现象,而不要求说明道理,使学生
对这方面的知识有个一般的了解即可.

光的衍射.在中学阶段,很难讲清衍射条纹是如何产生
的,这是因为多数学生理解不了为什么在干涉中把狭缝看作
是单一光源,而在衍射中却把狭缝看作是无数光源的集合.本
册教材只介绍了衍射现象及其产生的条件,而没有用波的叠
加来分析衍射现象.

为便于教学和有助于学生能力的培养,本册教材在
讲法和叙述上努力做到循序渐进,思路清楚.概念和规律的
得出尽可能比较平易,易于为学生接受.例如,用位移电流的
概念讲述电磁场,教学上困难较多,本册教材不再提出位移电
流和传导电流的概念,力求使讲法深入浅出一些,变压器的
输入电流随着输出电流而增大的道理,在中学很难讲清楚,而
中学阶段讲变压器,主要是讲它的基本原理,没有必要深究这
个问题,本册教材将这个内容删去.

在讲述电磁感应的知识时,紧紧把握磁通量变化这个线
索.讲述感生电流产生的条件,注意启发和引导学生一步一
步地分析问题.得出结论为了突出磁通量变化这个主要线索,
楞次定律不再用能量守恒定律推导的办法,直接用磁通量变
化的观点分析现象.对法拉第电磁感应定律先用磁通量
变化率的观点表达一般结论,然后再讲切割磁力线的特殊
情况.

用能量的观点分析现象,是研究物理问题时常用的方法
之一.为了培养学生较深入地体会能量守恒定律在电磁现象
中的应用,本册教材改变了过去将能量守恒问题夹在各节中
分析的做法,单独设立一节专门分析电磁感应现象中能量的
转化和守恒.

本册教材在讲解知识的同时,注意介绍认识客观事
物的方法,以开阔学生的眼界,提高他们分析问题、解决问题
的能力.

人类认识客观世界总是从片面到全面,从现象到本质,从
宏观世界到微观世界,经过实践,认识,再实践,再认识而
步步深入的.教材在介绍光的本性时,介绍了在认识上经历
的辩证发展过程,使学生对光的波动性和粒子性都有明确的
印象.原子结构和原子核内容的介绍,也都按历史线索叙述,
使学生一步步体会人类认识微观世界的方法和途径,理解所
学的知识.

物理实验对物理学的发展起着很重要的作用.本册教材
对在物理学史上起了重要作用的一些著名实验作了介绍,介
绍的实验有:罗兰实验、赫兹实验、$\alpha$粒子散射实验.还介绍
了人们测量光速的几种方法.希望通过这些介绍,使学生了
解实验在物理学中地位的重要,并通过学习前人设计实验
时的巧妙构思,培养他们灵活运用知识,发展他们的思维
能力.

本册教材的内容涉及近代物理学中的一些重大发
现,讲述这些内容,使学生了解近代物理学中的科学观点,可
以增长他们的见识,有助于培养辩证唯物主义的世界观.

“场”是物理学的基本概念.引入场的概念,使物理学获
得很大的成就.教材中注意了培养学生用场的观点分析电磁
现象,知道场的客观存在和这个概念的重要性,教材从稳定的
电场和磁场到变化的电磁场,再到电磁波,强调了电磁波可以
脱离电荷而独立存在,不需要别的物质做媒介而传播,并且具
有能量,通过这样的讲解,逐步扩展学生对场的认识,使他们
获得场是客观存在的具体印象.

客观世界是相互联系的统一整体.将表面上不同的现象
联系起来,揭示出它们的共同本质,是物理学的伟大成果,物
理学发展中已完成了几次大的统一.教材中介绍了电现象和
磁现象的统一,光现象与电磁现象的统一,还介绍了把微观世
界统一起来的波粒二象性,学生对这些内容有所认识,将有
助于他们把知识联系起来,培养他们的综合能力.

质量守恒、电荷守恒、动量守恒和能量守恒是自然界普遍
遵守的规律,对微观世界和宏观世界都适用.教材注意介绍
在宏观、微观领域中这些规律的应用.人们对微观粒子相互
作用的认识的发展,很大一部分归功于人们能自觉地用守恒
定律来分析物理现象.通过教学,要使学生逐步体会在千变
万化的物理世界中,在现象背后存在着规律.守恒定律就体
现出变化的规律,要使学生体会到这一点,并培养学生重视
用守恒定律来处理问题.

这册教材介绍的波粒二象性、能量量子化等,反映了微观
世界的特有规律,体现了微观世界特有规律与宏观世界的不
同.教学中要注意告诉学生,观念必须与认识对象相适应,不
能用宏观世界得出的一般观念来看微观世界.

本册教材注意理论联系实际,联系现代科学技术.
讲好联系实际的内容,可以巩固基本知识,开阔思路和眼
界,提高运用知识的志趣和能力.根据实际情况,这册教材联
系实际的内容有三种情况:
\begin{enumerate}
    \item 基本知识的实际运用,对这部分内容,要讲清原理,弄清把基本知识应用于实际中去的思路.这类内容有:电流
    表工作原理,荷质比的测定和质谱仪,回旋加速器,自感现象
    的应用,变压器,电能的输送,感应电动机,光电效应等.教学
    中要注意不深究技术细节和枝叉问题.
    \item 一般性说明原理(不宜过细分析).这部分内容有:涡
    流,无线电波的发送和接收原理,光导纤维,显微镜和望远镜,
    薄膜干涉及其应用,红外线、紫外线、伦琴射线的应用,光谱分
    析,激光,放射性同位素及其应用,核反应堆,可控热核反
    应等.
    \item 介绍性,这部分内容,大部分是阅读材料,如寻找磁
    单极子,直流输电,直线电机和悬浮列车,传真、电视和雷
    达,电子显微镜和射电望远镜,全息照相,偏振光和立体电
    影等.
\end{enumerate}

这册教材的习题,难度同试用本大体相当,根据这
册内容的特点,加强了联系实际的题目.为加强能力培养,安
排了少量的改进或设计实际装置的题目,还安排了一些综合
运用知识的题目.

本册教材同前两册一样,安排了演示实验、学生实验
和课外实验.高中物理教学仍然以实验为基础,对实验教学
应予以足够的重视.

对演示实验,要尽量做给学生看.在实验设备不足的情
况下,教师应尽可能地自制一些仪器设备,以保证教学效果.
在学校的仪器设备允许的情况下,可以将一些演示实验让学
生自己动手在课堂上做,以增加学生动手的机会.

对课外实验,虽不作要求,但应该鼓励学生做.有的课外
实验活动,需要教师给予必要的指导和帮助.

高中物理甲种本第三册的教学内容可按每周4课
时,全年共112课时讲授完.其中,第一章13课时,第二章10
课时,第三章16课时,第四章12课时,第五章17课时,第六
章11课时,第七章4课时,第八章5课时,第九章9课时;平
时复习和机动时间15课时.














\mainmatter


\chapter{磁场}
\minitoc[n]
\section{教学要求}
关于电学的知识,在本书第二册中讲了电场和稳恒电流
的知识,在这册书中讲解磁场、电磁感应以及电磁场的知识。
本章研究磁场以及磁场对电流的作用、磁场对带电粒子的作
用。这些内容是电学的重要组成部分,也是学习电学后几章
内容的基础。

本章教材分三个单元,第一单元包括第一节到第六节,
是在复习初中的磁场知识基础上进一步阐明磁现象和电现象
的统一性,介绍描述磁场的基本物理量--磁感应强度B.第
二单元包括第七节到第九节,讲述磁场对电流的作用力及其
在电表上的应用,第三单元包括第十节到第十三节,讲述洛
仑兹力及其在科学技术中的一些应用。

教材在复习初中学过的有关知识的基础上,要求学生了
解磁极和磁极之间,磁极和电流之间,电流和电流之间的相互
作用都是通过磁场来传递的,从而使学生认识磁场是客观存
在的物质。还要使学生了解,磁场可以用磁力线形象地描
述。在教学中应加强安培定则的练习,使学生能够利用这个
定则判断磁场的方向。

介绍磁现象的电本质,是为了使学生了解电流的磁场和
磁铁的磁场有着共同的起源。为了使学生确信运动电能够
产生磁场,教材介绍了罗兰实验。但在教学中不要求做这个实
验,也不要求对实验的细节加以讨论。教材介绍了分子电流,
但不要求对分子电流是如何形成的作深人的探讨。

磁感应强度描述了磁场的性质,是本章教材的重点内容。
由于这个概念比较抽象,它也是教学上的一个难点。教学中,
为使学生对磁感应强度有个基本认识,可在演示磁场对电流
的作用力的基础上,引出$F$与$I\ell$成正比的关系,再给出磁感
应强度的概念。

介绍直线电流磁场的公式,是为了使学生知道直线电流
磁场的磁感应强度大小跟什么因素有关系,使学生对这种磁
场有具体的了解,不要求学生用公式进行定量的计算。
安培力和洛仑兹力表示磁场对电流和运动电荷的作用,
是电学中的重要规律,因此它们是教学的重点。为了使学生
顺利学好后续课程,应该做好左手定则的练习,使学生掌握好
安培力和洛仑兹力方向的判定。

带电粒子在匀强磁场中的运动在实际中有广泛的应用,
是学习荷质比、质谱仪、回旋加速器等知识的基础。为使学生
理解这一知识,复习好有关的力学知识是个关键。带电粒子
做匀速圆周运动的轨道半径公式和周期公式,要求学生能够
理解公式的物理意义,而不要求学生记忆这两个公式(由于这
两个公式推导过程所用的知识都是他们已经学过的,只要理
解推导过程,推出公式并不困难)。

荷质比的测定和质谱仪、回旋加速是洛仑兹力的具体
应用。要求学生了解测定荷质比的原理及其意义。在质谱仪
和回旋加速器的教学中,主要介绍它们的原理。

由于高中阶段所介绍的有关磁场知识,如磁感应强度、磁
通量、安培力、洛仑兹力等都是通过分析、推理和定量推导才
得出的,因此,教材具有一定的难度。这就要求教师要尽量做
好演示实验,尽量增加学生的感性知识。还要注意讲清分析
问题的思路,使学生能够理解学的内容。

本章的教学要求是:
\begin{enumerate}
\item 理解磁场和磁力线的概念,能运用安培定则确定直线
电流、环形电流以及通电螺线管的磁场的方向。
\item 了解磁现象的电本质和磁性材料的应用.
\item 理解磁感应强度和磁通量的概念.了解匀强磁场的
特点。了解直线电流磁场的磁感应强度的计算公式。
\item 掌握计算安培力的公式和判断安培力方向的左手定
则,了解匀强磁场对平面通电线圈的作用和磁电式仪表的工
作原理。
\item 掌握洛仑兹力的计算公式,理解带电粒子在磁场中做
匀速圆周运动的道理,了解质谱仪和回旋加速器的工作原理。
\end{enumerate}

\section{教学建议}
本章教材的中心是研究磁场的特性及有关的规律,因此
磁感应强度$B$是本章的中心概念,教学时,首先研究磁场的
形象化描述,引出磁力线的概念。然后给出磁感应强度$B$的
概念,并以此概念来展开全章教材,依次讲述磁场对电流和运
动电荷的作用力的规律及其在实际中的种种应用。这样围绕
着磁感应强度$B$这个中心概念,形成了本章的知识结构。

\subsection{磁场及描述磁场的物理量}

从第一节到第三节教材,大部分内容学生在初中均已
学习过,可以在课堂上或在课前要求学生自己阅读课文。教师
在课堂上应做好以下几个演示实验:
\begin{enumerate}
\item 磁极对磁极的作用;
\item 电流对磁极的作用(即奥斯特实验);    \item 磁极对电流的作用;
\item 电流对电流的作用。磁力线的演示,可根据学生情况决定
是否需要做。
\end{enumerate}
在观察实验现象和自己阅读课文的基础上,引
导学生思考和讨论以下几个问题:
\begin{enumerate}
\item 在什么条件下可以在空
间出现磁场?    \item 在四个实验中,各种互相作用的实质是什么?
\item 磁场的方向是怎样规定的?    \item 安培定则的内容是什么?   
 \item 
磁场的起源是什么?其根据是什么?
\end{enumerate}
总之,要用问题来激发
学生思维,尽量调动学生的学习积极性,在上述问题讨论基础
上,根据学生所反映的问题,教师的讲授应着重明确以下
几点:

\paragraph{磁场的起源}
教师通过对罗兰实验的分析和讲解,
明确运动电荷是磁场的起源,从而使学生了解电流周围空间
的磁场和磁体周围空间磁场具有相同的起源。

讲授这个内容的顺序可以这样安排:首先根据奥斯特实
验明确电流是磁场的起源,然后提出磁体的磁场是否也是由
电流引起的问题,教师对安培提出的分子电流的假说进行
介绍,并且进一步说明这个假说完全符合近代原子的电结构
学说(顺便对学生解释一下“假说”是一种推动科学发展的重
要思维方法)。其次根据电流是由电荷的运动形成的,那么能
否直接证明静止电荷一旦运动起来就会产生磁场呢?这时教
师再提出罗兰实验,进行介绍和分析。罗兰实验在中学很难
做成,不要求演示,但必须把这个实验的原理和结果给学生讲
清楚。

\paragraph{磁场力问题}
磁场对电流运动电荷要产生磁场力的
作用,是一个实验事实,必须使学生深刻认识它的重要意义。
磁极之间、磁极与电流之间、电流与电流之间的相互作用都是
通过磁场来传递的。也就是说都是通过磁场对电流产生磁场
力来实现的。教师可以重点讲授电流之间的相互作用,使学
生理解上述观点.如果给平行的两根直导线分别通以电流$I_1$,
和$I_2$, 由于电流$I_2$处于电流$I_1$形成的磁场中,因而受到$I_1$
的磁场的磁场力作用.同样,电流$I_1$处于电流$I_2$的磁场中,
因而受到$I_2$的磁场力作用.所以电流$I_1$和$I_2$之间表现出
的互相吸引或排斥,是通过它们的磁场而间接发生相互作
用的。

\paragraph{磁场的方向}
所谓磁场方向,实质上是指磁感应强
度的方向,这和电场方向是指的电场强度方向相类似。在磁
感应强度概念没有提出之前,应介绍根据磁场力的特性对磁
场方向的规定:在磁场中,探测磁针的$N$极受的磁场力方向
即磁场中该点的磁场方向。这种规定比利用磁场对电流或运
动电荷的作用判定磁场方向要具体和简便得多,有利于磁力
线概念的引入。

要明确向学生指出磁力线是一条闭合的曲线,能够形象
地描述磁场的强弱和方向,磁力线并不是一种客观存在的线。
磁力线上所画的箭头,表示的是磁力线的流向,而不代表磁场
方向。磁力线上任一点的切线方向才是该点的磁场方向。如果
已知磁力线的流向,根据上述方法可确定磁场方向。电流方
向和磁力线方向之间有确定的关系,电流方向变化其磁力
线方向也随之改变,安培定则是记忆和判断这种关系的方
法。安培定则是初中学习过的知识,但仍然要通过一些例题
的练习,使学生熟练掌握,这对于后续知识的学习是十分
必要的。

\subsubsection{磁感应强度和磁通量}
磁感应强度概念的建立是本章的重点.可用类比方
法讲述,即从电场强度概念的建立引入建立磁感应强度概念
的必要性。使学生明确磁场中某点磁感应强度的强弱可通
过一个检验电流在该点受的磁场力的大小来描述。建议按以
下步骤来安排教学:

通过演示实验使学生认识磁场对电流的作用力的零
值条件和最大值条件,即当电流方向和磁场方向平行时(夹
角为$0^{\circ}$或$180^{\circ}$),磁场力为零;当电流方向和磁场方向垂直
时,磁场力具有最大值;当电流方向和磁场方向斜交时,磁场
力介于零值与最大值之间。进而引导学生了解在研究磁场的
强弱时,只有用磁场力的最大值进行比较,才能有确定的
意义。

引导学生分析磁场力的最大值的决定条件。首先使
学生认识在同一个磁场中某处的两根长度相同的直导线上通
以相同电流,因条件完全相同,两根导线受的磁场力的最大值
必然相等。如果将此两直导线并联起来,并保证每根线通过的
电流不变,则两根直导线受的磁场力最大值是一根直导线受
力的两倍,由此可得:$F_{\text{最大}}\propto I$, 如果将此两根直导线串联起
来,并将它们全部置于磁场中,则整根导线受力也是一根导
线受力的两倍,由此可得$F_{\text{最大}}\propto \ell$. 将上述两个结论合并可得:
\[F_{\text{最大}}\propto I\ell\]
可见,对磁场中某处$\dfrac{F_{\text{最大}}}{I\ell}$
是一个恒量,使学生认识到这个恒量可用来描述磁场的强弱,因此给这个比值取名
叫磁感应强度。在这里可向学生指出,电场强度的定义也是
利用电场力与电荷电量的比值。用比值定义物理量是物理学
经常采用的方法。使学生了解这一点对于掌握物理概念是十
分有用的。根据磁感应强度的定义
式$B=\dfrac{F_{\text{最大}}}{I\ell}$,
使学生了解$B$
的单位,并向学生明确$B$是矢量,要指出$B$的方向就是磁场方
向,即磁针的$N$极受的磁场力方向,还要指出$B$的方向和$F_{\text{最大}}$
的方向是不同的,对这一点可让学生讨论练习二2来加深认
识。还应诉学生,$B$矢量完全遵从平行四边形法则,可以进
行合成或分解。在这里还应对匀强磁场的特点作一些介绍。

磁通量是一个重要物理量.在下一章讲电磁感应现
象时,它是描述电磁感应规律的基本概念。首先要根据教材
要求将磁通量的定义呈现给学生。使学生正确理解定义式
$\phi=BS$. 对物理概念要抓住定义式来理解是十分重要的,其
次要从磁力线角度来形象说明磁通量的物理意义。第三,要
讲清$\phi=BS\cos\theta$这一公式,可提出“如果所取的平面与$B$
方向平行时则通过此平面的磁通量为多少?”的问题来研究,
这个问题用磁力线能够很形象地说明,因为没有磁力线通过
该平面,所以磁通量为零。然后再讨论“如果平面面积$S$与$B$
方向既不平行又不垂直时磁通量应如何计算”的问题。因为通
过面积$S$的磁力线条数等于通过面积$S$在垂直于磁力线方向
的投影平面$S_n$上的磁力线的条数,又根据磁通量的定义式可
得$\phi=B\cdot S$, 设平面$S$与$S_n$的夹角为$\theta$, 即$S_n=S\cos\theta$, 则
$\phi=B\cdot S\cdot\cos\theta$. 最后介绍磁通量的单位及磁通密度的概念.

\subsubsection{直线电流的磁场}

这一节教材,是研究直线电流周围磁
场的磁感应强度大小的决定条件。在定性演示实验的基础
上,给出定量公式$B=kI/r$。
可以告诉学生,公式中各量采用国
际单位制时,$k=2\x 10^{-7}{\rm N/A^2}$。
在学过下一节教材后,可指导学
生学习阅读材料,就可以了解$k$值的来源。虽在教学中不要
求学生利用$B=kI/r$
进行计算,但用这个公式来对一些问题
作定性分析,还是必要的,使用公式的条件必须给学生明确
讲清楚。最后,应使学生了解虽然直线电流磁场、环形电流磁
场、通电螺线管磁场的磁感应强度大小的计算公式不同,但均
与电流强度成正比。

\subsection{安培力及其应用}
\subsubsection{安培力的大小}

根据磁感应强度的定义式可得安培
力最大值的计算式为$F=I\ell B$. 本节重点讨论电流方向与磁场
方向不垂直时安培力的计算式.设$I$与$B$的夹角为$\theta$, 我们
以电流方向为基准,把$B$分解为平行于电流方向的分量
$B_{\parallel}=B\cdot\cos\theta$和垂直于电流方向的分量$B_{\bot}=B\cdot\sin\theta$. 由于分
量$B_{\parallel}$对安培力无贡献,即对电流不产生作用力,所以电流受
到的力完全由$B_{\bot}$来决定.则可得$F=I\ell B_{\bot}$, 因而$F=I\ell B\sin\theta$. 
要使学生了解这一公式的适用条件,以及各量的单位。

\subsubsection{安培力的方向}
教学时应通过演示实验,使学生在了
解安培力方向的基础上介绍左手定则。通过实验使学生认识
安培力方向有如下特点:第一,安培力$F$的方向总是垂直于$I$
与$B$所构成的平面,即$F\bot I$和$F\bot B$. 第二,安培力$F$的方向
由$I$和$B$两个量的方向共同决定,如果$I$和$B$其中一个量的
方向变为原来方向的反方向,$F$的方向也就变为原来方向的
反方向。如果$I$和$B$两个量的方都变为原来方向的反方向,
则$F$的方向不变,第三,$I$、$B$、$F$三个量的方向关系如图1.1所
示,可构成三维空间中正交坐标系。在了解上述三个特点之
后提出如何记忆$I$、$B$、$F$三个量的方向关系的问题,可以让学
生思考提出自己的记忆方法,然后介绍左手定则。教学时,
应使学生注意当通电导线与磁场方向斜交时,如何判断安培
力的方向。

\begin{figure}[htp]\centering
    \begin{minipage}[t]{0.48\textwidth}
    \centering
\begin{tikzpicture}[>=latex, scale=1]
\draw[<->](0,3)node[right]{$z$}--(0,0)--(3,0)node[right]{$y$};
\draw[->](0,0)--(-2,-2)node[right]{$x$};
\draw[thick,->](0,0)--(0,1.5)node[right]{$F$};
\draw[thick,->](0,0)--(1.5,0)node[above]{$B_{\bot}$};
\draw[thick,->](0,0)--(-1.5,-1.5)node[right]{$I$};
    \end{tikzpicture}
    \caption{}
    \end{minipage}
    \begin{minipage}[t]{0.48\textwidth}
    \centering
    \begin{tikzpicture}[>=latex, scale=1]

\draw[<->](-1.06,-1.3)--node[fill=white]{$d$}(1.06,-1.3);
\foreach \x in {-.5,.5,1.5,-1.5}
{
    \draw[->](-3,\x)--(3,\x);
}      
\node at (3,0){$B$};
\draw[rotate=45](-1.5,-.1) rectangle (1.5,.1);

\draw[dashed](-1.06,-2.5)node[below]{$N$}--(-1.06,2.5)node[above]{$M$};
\draw[dashed](1.06,-2.5)node[below]{$N'$}--(1.06,2.5)node[above]{$M'$};

\node at (45:1.5) [right=3pt]{$d$};\node at (45+180:1.5) [left=3pt]{$a$};

\draw[->, thick](45:1.5)--+(0,1)node[right]{$F_{cd}$};
\draw[->, thick](45+180:1.5)--+(0,-1)node[right]{$F_{ab}$};
\draw(45:1.5)[fill=white] circle (.15);
\draw(45+180:1.5)[fill=white]  circle (.15);
\node at (45:1.5){$\cdot$};
\node at (45+180:1.5){$\x$};
    \end{tikzpicture}
    \caption{}
    \end{minipage}
    \end{figure}

电流天平一节为选讲教材,可在讲完安培力大小和方向
后,把电流天平作为一个实际例子进行介绍。


\subsubsection{磁场对通电线的作用}

可先让学生观察通电线圈
在磁场中偏转或旋转的现象,引导学生分析线圈各边所受安
培力的方向。然后启发学生回忆力偶及力偶矩概念。在此基础
上引导学生定量研究安培力偶矩的大小,再进行分析。在图
1.2中,因$F_{ab}$和$F_{cd}$大小相等、方向相反,又不在一条直线
上,所以$F_{ab}$和$F_{cd}$形成一对力偶。两力作用线$MN$和$M'N'$
之间的垂直距离d为力偶矩的臂。根据力偶矩的定义可得安
培力矩为$M=F\cdot d$, 如是可推导得
\[M=BIS\cos\theta\]

引导学生理解公式时应注意以下几点:
\begin{enumerate}
\item $\theta$角为线圈平
面与磁力线的夹角。如果取线圈平面与中性面(垂直于磁场
方向的平面)的夹角$\alpha$来进行计算,则力偶的臂$d=ad\cdot \sin\alpha$.
因而安培力偶矩为$M=BIS\cdot \sin\alpha$.
\item 如果线圈由$N$匝线圈串
联组成,则$M=NBIS\cos\theta=nBIS\sin\alpha$.
\item 上述安培力偶矩
公式只适用于匀强磁场。
\end{enumerate}

\subsubsection{电流表的工作原理}

首先要使学生明确利用永久磁
铁的磁场使通电线圈偏转的现象制成的仪表叫磁电式仪表,
并通过实物使学生了解电流表的构造。然后重点讲授磁铁与
铁心之间的磁场特点,即磁场是均匀地辐向分布的,就是说所
有磁力线的延长线都通过铁心的中心。不管线圈处于什么位
置,线圈平面与磁力线之间的夹角都是零度。还要明确该磁
场不是匀强磁场,但在以铁心中心为圆心的圆周上,各点的磁
感应强度$B$的大小是相等的。这样,通电线圈受的安培力偶
矩就由下式决定:即$M_1=NBIS$. 电流表内的弹簧产生一个
阻碍线圈偏转的力矩$M_2$. 当$M_1=M_2$时,线圈就停止在某一
偏角$\theta$上,固定在转轴上的指针也转过同样的偏角$\theta$, 并指示
刻度盘上的某一刻度。从刻度的指示数就可以测得电流
强度。

下面再作一些定量的讨论。已知弹簧(游丝)产生的弹性
力矩$M_2$与指针的偏转角$\theta$成正比,即$M_2=k_2\theta$, 其中$k_2$是由
弹簧决定的恒量。由此可得$NBIS=k_2\theta$, 则
\[\theta=\frac{NBS}{k_2}\cdot I\]
从公式得出如下几点结论:
\begin{enumerate}
\item 对同一个电表,$N$、$B$、$S$和$k_2$为不
变的量,则$\theta\propto I$. 可见$\theta$与$I$一一对应,从而可以用指针偏转
角度来测量电流强度$I$的值.
    \item 因为$\theta\propto I$, $\theta$随$I$的变化是
线性的,所以表盘的刻度是均匀的.    
\item 如果$NBS>k_2$, 只要有
很小的电流值,偏转角$\theta$的值就比较大.因此这种电流表的灵
敏度较高. 
   \item 指针最大偏角为$\theta_{\text{最大}}$时,所测得的电流为满偏
电流$I_g$,可得:
\[I_g=\frac{k_2}{NBS}\cdot \theta_{\text{最大}}\]
$I_g$越小,要求$N$、$S$、$B$几个量
都较大。而当线圈的匝数增大时,绕制线圈的导线长度也增
大,则线圈的电阻即电流表的内阻$R_g$增大,可见,灵敏度大
(即$I_g$小)的电流表其内阻$R_g$较大。
\end{enumerate}

\subsection{洛仑兹力及其应用}
\subsubsection{洛仑兹力的大小和方向}

从产生安培力的微观本质
的设想出发,再通过阴极射线管的演示实验来验证这个设想,
使学生获得关于洛仑兹力的感性认识。在此基础上引导学生
推导洛仑兹力公式.推导过程中应抓住以下两点:
\begin{enumerate}
\item 磁场
对电流的作用力(安培力),可看作是作用在每个运动电荷上
的洛仑兹力的合力.这个观点是推导公式的出发点.
\item 引
导学生复习电流公式$I=nqvS$的物理意义。抓住上述两点,按教材思路引导学生得到洛仑兹力大小的公式$f=qvB\sin \theta$.
然后应强调指出$\sin\theta$对$f$大小的影响,即当$\theta=0^{\circ}$时$f=0$, 
$\theta=90^{\circ}$时$f=qvB$为最大值。这些结论对分析运动电荷在磁
场中运动规律是重要的,洛仑兹力的方向仍然用左手定则判
定,要注意讲清洛仑兹力的方向特点。
\end{enumerate}

\subsubsection{带电粒子在磁场中的运动}

没有受到其他力作用的
带电粒子,在磁场中的运动性质是由洛仑兹力来决定的。本
节主要是研究一个带电粒子在匀强磁场中的运动。首先引导
学生思考如果$v$与$B$的夹角$\theta=0^{\circ}$时,带电粒子作何种性质
的运动。然后引导学生重点讨论当$v_0\bot B$时,带电粒子在匀
强磁场中的运动。研究时应抓住以下几点:
\begin{enumerate}
\item 洛仑兹力的方
向永远垂直于含有$v$与$B$的平面,而且由于$v\bot B$, 带电粒子
只能在垂直于磁场的平面内运动。    
\item 由于洛仑兹力的方向垂
直于速度$v$, 所以洛仑兹力不改变速度的大小,只改变速度的
方向。从$f=qvB$公式中可知,如果$q$、$v$、$B$都是恒量,则洛仑
兹力的大小也不变,从中引导学生判断粒子的运动性质,并作
演示实验验证。    
\item 根据带电粒子作圆周运动时洛仑兹力起着
向心力的作用,可得 
\[qvB=\frac{mv^2}{r}\]
进而得出半径公式$r=\dfrac{mv}{qB}$
和周期公式$T=\dfrac{2\pi m}{qB}$
\item 洛仑兹力对运动电荷不做功。这是洛
仑兹力的重要特征.在讲解时还应结合练习七6的讨论,使
学生能利用运动合成、分解的方法,定性地分析带电粒子的运
动方向不和磁感应强度的方向垂直时粒子的运动情况。
\end{enumerate}



\subsubsection{洛仑兹力在近代技术中的应用}

教材中主要介绍质
谱仪和回旋加速器。

\paragraph{质谱仪} 
首先根据课本中质谱仪的装置示意图,介
绍质谱仪的几个主要部件(电离室、加速电场、偏转磁场和显
示器)及其作用,然后推导出荷质比公式
\[\frac{q}{m}=\frac{2U}{B^2r^2}\]

\begin{enumerate}
\item 由于$U$、$B$、$r$均可通过实验测定,因而$q/m$
是微观粒子的基本参量。这
就使学生认识到可运用物理规律设计实验装置,通过宏观量
的测定来测定微观量这样一种重要的物理方法。    
\item 当被测粒
子的电量$q$相同时,在$U$、$B$一定的条件下,不同质量的粒子
在显示器上显示不同的谱线。如果已知电量,则可计算出它
们的质量,质谱仪是十分精密的仪器,是测定带电粒子质量和
分析同位素的重要工具。
\end{enumerate}

\paragraph{回旋加速器}
回旋加速器是利用电场对电荷的加速作用和磁场对
运动电荷的偏转作用来获得高能粒子的装置。可在指导学生
阅读课文后,重点讲清以下几点:
\begin{enumerate}
    \item 使学生了解带电粒子在磁
场中作匀速圆周运动和在电场中作匀加速直线运动的过程。
\item 使学生了解怎样使粒子每次通过电场时速度方向都和电场
力方向相同。主要说明由于带电粒子在匀强磁场中作匀速圆
周运动的周期$T=\dfrac{2\pi m}{qB}$
与速率和半径无关,所以只要交变电
场的变化周期等于粒子的运动周期,就可以使粒子每次通过
电场时都能得到加速。
\item 使学生了解粒子通过$D$形金属扁盒
时,由于金属盒的静电屏蔽作用,盒内空间的电场极弱,所以
运动粒子只受到洛仑兹力的作用而作匀速圆周运动。
\item 设$D$形盒的半径为$R$, 则粒子获得的最大动能为
\[E_k=\frac{1}{2}mv^2_{\text{最大}}=\frac{1}{2}\frac{q^2B^2}{m}R^2\]
可见由于装置的限制,带电粒子在这种加速器中获得的能量也
是有限制的。还要进一步说明,从相对论力学观点来看,粒子
的质量随着速度的增大而增大,这样粒子运动的周期就不再
是恒定的了。因而交变电场的变化周期与粒子运动的周期不
能步,这就使加速器工作条件受到了破坏,进一步提高粒子
的速率也就不可能了。
\end{enumerate}

\section{实验指导}
\subsection{演示实验}
\subsubsection{电流对磁针的作用(奥斯特实验)}



进行这个实验最好自制一个框架,如图1.3所示.底板
可用一长40厘米宽15厘米的厚木板制成,通电导线可用从
旧输电线中拆出的粗单股铜线或铝线,弯成图中所示的形状
固定在底板上,水平长直导线长度约30厘米,高度视所用大
磁针高度而定。接线柱$A$、$B$分别与导线两端相接,接线柱$C$
和$B$相距4厘米,两者之间可接3A保险丝,导线两端经接
线柱$A$、$C$与低压电源直流输出端相连,低压直流电源应选输
出电流大于5A的,例如J1201型教学用低压电源.如无低
压电源,可用4节一号干电池并联后作电源.使用低压电源
时,电压可选2—4V, 此时$B$、$C$间保险丝起限流保护电源作
用.该实验只宜瞬时通电.如果能制作一个如图1.4所示的
大容量电容器充放电装置,利用电容器放电的瞬时大电流,效
果更理想.其中10000 $\mu{\rm F}$ 16V电解电容可用多个电容并联
组成,这个装置还可用于电流间相互作用等实验。

\begin{figure}[htp]\centering
    \begin{minipage}[t]{0.48\textwidth}
    \centering
\includegraphics[scale=.7]{fig/1-3.png}
    \caption{}
    \end{minipage}
    \begin{minipage}[t]{0.48\textwidth}
    \centering
\includegraphics[scale=.7]{fig/1-4.png}
    \caption{}
    \end{minipage}
    \end{figure}

\subsubsection{磁场对电流的作用}
这个实验在初中阶段演示过,若需重新演示,可按以下几
种方法进行。


方法一:按图1.5制作一带有平行金属导轨的底座,导轨
两端与接线柱$A$、$B$相接,用教学演示用的大蹄形磁铁置于导
轨中间,直裸铜线跨搭在两根导轨上,电源可使用供电电流
大于5A的低压直流电源,或用由几节干电池组成的并联电
池组,也可用上一实验介绍的电容瞬时放电装置。同样,这个
实验最好在导线中串有一段3A保险丝以保护电源。如果金
属导轨与直裸铜线所需粗铜线不易找到。可以用从大容量电
解电容器或油浸纸介质电容器中拆出的铝箔制做,把铝箔卷
在细竹棍(毛笔笔杆)上制做导轨。直裸导线则可将铝箔卷在
细竹棍上以后,将竹棍抽出制成。由于铝筒状导线质量极小,
一般用一节干电池即可获得满意效果。

\begin{figure}[htp]\centering
    \begin{minipage}[t]{0.48\textwidth}
    \centering
\includegraphics[scale=.7]{fig/1-5.png}
    \caption{}
    \end{minipage}
    \begin{minipage}[t]{0.48\textwidth}
    \centering
\includegraphics[scale=.7]{fig/1-6.png}
    \caption{}
    \end{minipage}
    \end{figure}

方法二:如想按课本图1.2所示装置进行实验,最好按图
1.6所示将装置改进.图中上方是一根细木条,可固定在铁架
台上。细木条上有两个接线柱,接线柱下固定着铜片。AB是
用铝箔卷成的铝筒,长15厘米左右.与铝筒连接的是两根铝
箔带(宽1.5厘米),使铝筒、铝箔带接触良好,用胶粘牢.铝箔
带上端绕在铜片上两三圈,以保证良好地接触。演示时,用一
节干电池作电源,便可看见通电铝筒在磁场中发生明显偏转。
最好制作两个这样的装置,便于以后演示电流间的相互作用。

若用铜棍与铜导线演示,则应选择能提供5A以上大电流的
低压直流电源,且只宜瞬时接通。

\subsubsection{电流间的相互作用}
由于电流间相互作用力很小,实验所需的大电流(10A以
上)的低压电源又较难找到,在一般学校演示这个实验最好按
下述方法进行。
\begin{figure}[htp]
    \centering
\includegraphics[scale=.6]{fig/1-7.png}
    \caption{}
\end{figure}

方法一:用两条宽2厘米左右的铝箔按图1.7甲所示方
法压制成两条长30厘米的弯曲铝带作平行导线,固定在木架
上,导线间相距1厘米左右为宜(图1.7乙).电源可用6V蓄
电池,通过瞬时短接来获得大电流。注意,只能瞬时短接。如
果利用演示实验1中所述大电容放电装置,效果更好,且不
易损坏电源。

方法二:用演示实验2中所述的铝带与铝筒,将两个这样
的装置平行放置.演示所用电源同上,演示相斥以相距1厘米
为宜,相吸以相距3厘米为宜.以上两个实验若想用低压直
流电源,应选10A输出电流的.实验时,要串一根长4厘米左
右的5A保险丝以保护电源.实验时,仍要瞬时点接。

\subsubsection{磁感应强度的方向}
用十余个小磁针分布在条形磁铁或通电螺线管周围,即
可观察到不同位置的小磁针北极取向不同,表示各点磁感应
强度不同,这个方法简单,但由于只能放在桌面上,因此观察
很不方便。


如果通过投影仪将指南针投影可提高可见度。这需要制
作十余个指向透明的小磁针(图1.8)(可以发动学生在课外
小组活动时制作)。制作方法如下:将磁化的缝衣针(长3厘
米左右)两枚分别对称平行地穿过按扣上的小孔(方法见课本
第364页课外实验活动一“自制指南针”),用透明胶片剪成
箭头与箭尾形状,并分别涂上红绿两色(不能用水彩色或广告
色,可用彩色水笔或照相用的透明颜料),用胶粘在针的两端。
小磁针底座用透明有机玻璃,中间支针可用较粗缝衣针。截取
针尖部分(长1.5厘米),将其烧热后插在有机玻璃底座上,实
验时,将磁针放在投影仪玻璃上,即可在屏幕上看到磁场中透
明磁针南北极的指向。

\begin{figure}[htp]\centering
    \begin{minipage}[t]{0.48\textwidth}
    \centering
\includegraphics[scale=.7]{fig/1-8.png}
    \caption{}
    \end{minipage}
    \begin{minipage}[t]{0.48\textwidth}
    \centering
\includegraphics[scale=.7]{fig/1-9.png}
    \caption{}
    \end{minipage}
    \end{figure}


\subsubsection{通电直导线、环形电流、螺线管磁场的磁力线}
这三种电流磁场的磁力线演示,可以用现成演示仪器
演示,其装置如图1.9、图1.10、图1.11所示.实验电源可用
低压直流电源或6V蓄电池,通过滑线变阻器将电路电流控
制在3—5A间.铁粉最好选用化学实验室中的还原铁粉,也
可用经筛过的细铁屑。实验前,应将铁粉(屑)均匀撒在导线
周围。通电后,轻敲板面,便出现磁力线形状。呈现磁力线
形状后即应断电。

\begin{figure}[htp]\centering
    \begin{minipage}[t]{0.48\textwidth}
    \centering
\includegraphics[scale=.7]{fig/1-10.png}
    \caption{}
    \end{minipage}
    \begin{minipage}[t]{0.48\textwidth}
    \centering
\includegraphics[scale=.7]{fig/1-11.png}
    \caption{}
    \end{minipage}
    \end{figure}

由于环形电流与螺线管磁力线演示器的板面是透明的,
因而可以放在投影仪上进行投影。

以上装置自己制作亦不困难,直线电流磁场磁力线演示
器主要是绕制一个矩形线圈,该线圈可用直径0.3毫米的漆
包线绕制,尺寸为20厘米$\x$20厘米左右,共绕60匝.环形
电流磁场磁力线演示器也用相同直径的漆包线绕60匝左右,
环的直径5厘米左右即可。安放线圈的底座可用有机玻璃制
成(也可用普通窗玻璃条拼接后用胶粘成)。螺线管磁场磁力
线演示器中,螺线管线圈仍用上述导线绕制,每20匝捆扎在
一起,共五组,管直径4—5厘米,间距1.5厘米左右.底座仍
用有机玻璃制作。

如果将这三个装置配以小磁针即可用来验证右手螺旋法
则(安培定则)。


\subsubsection{电流天平}
\begin{figure}[htp]
    \centering
\includegraphics[scale=.6]{fig/1-12.png}
    \caption{}
\end{figure}

J2408型电流天平如课本上图1.28所示,它主要由
激磁线圈(管内产生被测匀强磁场)和横梁系统两部分组成
(图1.12)。激磁线圈内径5.5厘米,匝数约500匝,两端引
线与靠近线圈一侧的接线柱连。横梁系统由铜箔板制成,
中央两侧有黄铜制的刀口并兼作导电接点,铜箔板中央有一
“7”形螺丝,水平部分螺母为横梁平衡螺母.固定在板上竖
直部分下端螺母为配重螺母,用于调节天平灵敏度(越往下,
横梁重心越低,平衡越好,但灵敏度越低)。左侧有一挂钩,
用于悬吊砝码,右侧伸入线圈内部部分铜箔上形成“彐”字形
导体,通过转换开关改变导体在磁场中受力部分(产生力矩部
分)长度为4厘米或2厘米.

演示方法:电流天平主要用来演示一种测量磁感应强度
的方法,即根据定义式$B=\dfrac{F}{I\ell}$测量磁感应强度。演示前
应使天平底座置于水平位置,并调节横梁平衡螺母使横梁平
衡(指针指示中间零刻度),演示时应先接激磁电路。由于激
磁绕组额定电流为3A, 电路中串有3A量程电流表用于监
视.然后再接测量电路.电路连接如图1.13所示,通电导体
长度可根据需要通过转换开关在4厘米或2厘米中选择一个,
通电电流可由串联在电路中的安培计读出,受力大小可通过
悬挂砝码使天平重新平衡的砝码质量求出,需要注意的是,
测量之前应先通电观察一下,不悬挂砝码时指针偏转方向是
否右端(伸入磁场部分)向下,否则应更换测量电路电源极性,
以使电流受力方向改变.由于测量部分额定电流也是3A,因
而实验中电流取值也应限制在3A以下,其次,由于电流天
平配置的砝码只有20毫克、40毫克、60毫克三种,所以在演
示时,调好平衡后,悬挂20毫克砝码,再接通激磁电路,调节
激磁电流,使天平恢复平衡,而后就不需再调,测量对象就是
该激磁电流所产生的磁感应强度。
\begin{figure}[htp]
    \centering
\includegraphics[scale=.6]{fig/1-13.png}
    \caption{}
\end{figure}

这个电流天平还可演示$F\propto I$及$F\propto\ell$的关系,也可演示
激磁线圈内磁感应强度$B$与激磁电流成正比,不再赘述。使
用电流天平前最好用酒精清洗刀口与刀口支承处,以使通电
接触良好,使用后要把横梁从支承架上取下放好,以保护
刀口。

\subsubsection{磁电式仪表的工作原理}
这个实验主要是演示通电线圈在磁场中所受力偶矩和这
个力偶矩与游丝反力偶矩的平衡问题。磁电式仪表的幅向磁
场及完整结构仍需用大幅教学挂图或模型说明。

线圈在磁场中受力偶矩的演示可以用J2416型电机原理
说明器进行,实验装置如图1.14所示.激磁线圈所用电源可
用6V蓄电池或低压直流电源(6—8V),通电线圈电路中串有
变阻器,接在6—8V低压电源或蓄电池上,电路中电流控制
在3A左右为好.
\begin{figure}[htp]
    \centering
\includegraphics[scale=.6]{fig/1-14.png}
    \caption{}
\end{figure}

演示游丝反力矩的作用,需要自制一个教具,其结构包括
磁铁轭、动圈绕组与底座三部分,见图1.15. 制作方法如下:
磁铁轭需要从旧的扬声器(口径4英寸或5英寸,小一些也
可)上拆下钡恒瓷(钡铁氧体)磁体,根据它的大小用2毫米
厚的铁板弯成“凵”形,并用胶把磁体按1.15甲粘在其上,磁
体另一端面粘一块矩形铁片,并使两个磁体$N$、$S$极相对,间
距6—8厘米,动圈绕组用硬泡沫塑料(可从仪器包装中找到)
制成框架,其尺寸根据磁铁轭大小而定,其宽度应尽可能接
近磁铁两极的距离.在框架上用直径0.3毫米漆包线绕20至
30匝,在框架两端中央插入两根粗缝衣针作为转轴,用胶将
线圈与转轴粘牢,线圈头尾部分先分别在前后转轴上紧绕
4—5匝固定后,再弯制成大型游丝状,最后可用按钉将线头固
定在底座木板上。在线框中央立一根用红色油漆涂过的塑料
管作指针,磁铁轭两极也应分别涂上红、蓝两色油漆来表示
$N$、$S$极。底座可用木板制,动圈支架可用薄铁皮或曲别针铁
丝弯制成,底座宽度应恰好卡紧在磁铁轭下部,底座上动圈支
架兼作导电接点,安装两个接线柱便于通电。装好后的仪器
如图1.15乙所示.
\begin{figure}[htp]
    \centering
\includegraphics[scale=.6]{fig/1-15.png}
    \caption{}
\end{figure}

实验时,可先将固定游丝的按钉拔下,使游丝自由。通电
后可以发现,动圈将转至中性面,再固定游丝,通电后可以看
到只转过一定角度。电流越大,偏转角度越大,从而说明了动
圈式磁电仪表的原理。

\subsubsection{带电粒子在磁场中运动时的偏转}
这个实验可以用多种方法进行演示。
\begin{figure}[htp]
    \centering
\includegraphics[scale=.6]{fig/1-16.png}
    \caption{}
\end{figure}

方法一:用JYSI型阴极射线管演示,将一台感应圈与
低压直流电源(或蓄电池)产生高压加在阴极射线管两端。可
以看到在荧光屏上呈现绿色光带,这是电子流撞击荧光屏而
发出的,它表示电子流径迹。当用磁铁靠近射线中部时,阴极
射线将发生偏转,如图1.16所示.反转磁场方向,射线也向
相反方向偏转。这个实验也可以用一般的示磁效应阴极射线
管进行,实验时,如果仅在阴极板附近有绿光而无光带出现,
说明感应圈极性接反,应通过更换极性开关改变极性。

方法二:利用教学示波器荧光屏演示,先将Y增益逆时
针旋到底,Y轴衰减拨至100或1000, 扫描范围拨至外X, X增
益也逆时针旋到底。开启电源后,调节亮度与聚焦,使光点圆
而亮度适中并居于中心,再用磁铁靠近光点,就会发现光点发
生位置变化,根据位置变化与磁极位置就可验证洛仑兹力的
方向与左手定则。
\begin{figure}[htp]
    \centering
\includegraphics[scale=.6]{fig/1-17.png}
    \caption{}
\end{figure}

方法三:利用通电溶液中正负离子受洛仑兹力偏离原来
径迹而带动液体转动来演示.实验装置如图1.17所示,所需
器材在图中均已注明,置于扬声器磁铁上的玻璃培养皿(生
物实验室中常用)的中央部分,置一旧的铜质砝码作为中心电
极,紧靠玻璃培养皿器壁内侧,用紫铜片弯成一个环形电极。
培养皿中倒入比重约为$1.04{\rm g/cm^3}$的硫酸铜溶液,深度为
0.5厘米至1厘米,电流强度由变阻器控制,电流用安培计监
视,一般控制在0.5至0.6安之间,电源可使用低压电源或6V
蓄电池,实验时,在环形电极与中央电极间形成幅向电流。形
成电流的正负离子在洛仑兹力作用下(由于正负离子运动方
向相反,故受洛仑兹力方向一致),带动液体围绕中央电极旋
转,其旋转方向可由左手定则判定,为观察方便,可将培养皿
与磁铁置于投影仪玻璃板上,在液面上撒一些碎纸屑,投影在
屏幕上可使学生清楚看到旋转的液流。若改变电流方向或磁
铁方向就可看到液流反转,从而说明带电粒子在磁场中的受
力及其规律。

\subsubsection{电子束在匀强磁场中作圆周运动}
实验要用磁场、电场作用力演示仪。该仪器目前有两种
规格,即J9283-1型与J9283型.J9283型是改进型,内部附
有电源.J9283-1型内部无电源,需要与J1201型低压电源
(或有9V稳压输出的低压直流电源)及J1205型高压电源
(或有300V高压直流及6.3V交流电压输出的其他高压电
源)配合使用。

此仪器主要由演示球形电子射线管(威尔尼特电子管)与
激磁线圈(亥姆霍兹线圈)组成.J9283型磁场、电场演示仪
外形如图1.18所示.电子管阴极在加热后经加速电压作用
形成电子流射出,充满管内的低压惰性气体在电子流撞击下
将放出紫色辉光,从而观察到电子流的径迹。两亥姆霍兹线
圈互相平行,在两者中央平行平面附近将形成一个匀强磁场
区。电子流在磁场作用下将受洛仑兹力作用,若电子流运动
方向与线圈平面平行,将观察到电子流作匀速圆周运动的径
迹。改变加速电压可改变电子运动速率,改变激磁电流可改变
电子流所处磁场磁感应强度,从而可观察到电子圆运动半径
与电子速度和磁感应强度的关系。
\begin{figure}[htp]
    \centering
\includegraphics[scale=.6]{fig/1-18.png}
    \caption{}
\end{figure}

实验操作步骤如下:
\begin{enumerate}
\item 先将加速极电压旋钮旋至0(逆时
针到底),励磁电流方向旋钮拨至“断路”,幅值旋钮拨至最小
值“0.6A”,偏转板电压钮拨至“断路”(这是用于演示电场偏
转的,此实验中不用偏转电压)。
    \item 接好电源后开启电源,
预热5分钟.
    \item 缓慢调节加速电压旋钮,可以看到电子射线
的紫色直线径迹。用手转动电子管,使径迹与线圈平面平
行.
    \item 拨动励磁旋钮至“顺时”位置,此时,线圈中通有0.6A
励磁电流,其方向由线圈上符号标明为顺时针方向电流(正面
看),此时可看到电子射线径迹发生弯曲。
\item 逐渐加大励磁电
流,可以看到径迹逐渐变为圆形,由此清楚表明运动电子在
磁场力作用下作匀速圆周运动。
\item 继续加大励磁电流(注意,
使用J1201低压电源时,其稳压电流输出不要超过1.2A),由
于磁感应强度增大,电子射线圆径迹半径将减小,说明磁场越
强,圆运动半径越小。
\item 固定励磁电流不变,改变加速电压大
小,将看到电压越高,电子射线圆径迹半径越大,说明磁场不
变时,带电粒子速度越大,其圆运动半径越大,从而定性地说
明带电粒子在垂直磁场方向上运动时,其圆运动半径
$r=\dfrac{mv}{qB}$.
\end{enumerate}

实验时必须注意以下两点:一定要在将加速电压钮逆时
针旋到底再开启电源对灯丝预热5分钟,而后进行实验.实验
完毕后仍应将加速电压钮逆时针旋到底,再关闭电源。其次
在使用励磁线圈电流方向钮时,一定要将幅值钮旋回“0.6A”
处再使用,防止大电流换切造成开关点电弧烧毁接点。

这个仪器还可定量研究验证洛仑兹力规律,亦可配合加
速电压测定电子荷质比,请参阅本仪器有关说明。

实验中若需观察粒子速度方向与磁力线不垂直时出现的
螺旋状径迹,只需用手慢慢旋动电子管即可。

\subsection{学生实验}
\subsubsection{观察磁铁对电流的作用}
这个实验比较简单,最好在课本图10.1所示实验电
路中串入变阻器和安培计。因为现在许多学校已改用低压学
生电源,不再使用蓄电池作为学生实验电源,若采用课本上
的电路,则实际上处于短路状态,往往将保险丝烧断。实验
前,将电路电流调节至2A左右(输出电压4~6V),断开电
键,放好磁铁再进行实验。

在学生进行实验前,教师应作如下要求:首先是弄清
实验原理,即将按左手定则判断导线受力方向的结果与实际
通电后导线受力方向进行比较,这就需要先作出表格(可画
图),其次要求学生观察清楚线圈绕制情况,确定出受力边框
上导线电流流向。

在实验时,可以安排两个思考问题供学生思考:①这
个实验为什么要用多匝线圈代替单根导线进行实验?②有人
说,导线框通电后受磁场力才运动起来,可见力是物体运动的
原因。这种说法对吗?为什么?

\subsection{课外实验活动}
\subsubsection{自制指南针}
最好按前面演示实验4中所介绍的方法制做透明磁针.
磁针底座可改为小木块。安装支持钢针时,可先在木块上用
直径类似的小钉钉一个小孔,再将钢针底部插入固定,并用一
点乳胶粘牢,底座也可用大橡皮,将钢针插在其上便可。

\subsubsection{验证环形电流的磁场}
应让学生弄清线圈绕制方向与电流关系,电流从哪端流
人,怎样在线圈中流动,弄清这一点对后面电磁感应有关线圈
问题和实验很有好处。同时还要告诉学生用砂纸将漆包线两
端漆皮打去,露出铜金属光泽后才能通电,这一点许多学生并
不懂得,实验中所采用的漆包线直径可选用直径0.3毫米左右
的,最好绕30匝左右,通电时间亦不应过长,用低压直流电
源要串一变阻器。

\subsubsection{验证通电螺线管的南北极}
在实验前要指导学生制做螺线管。螺线管直径应大一些,
为使螺线管有一定的形状,需要先用硬卡片纸(或画报纸)制
作一个稍粗的纸质圆筒.绕制时,可用直径0.2—0.3毫米的
漆包线.固定漆包线头尾的方法如图1.19所示,图中的$A$、$B$
是两根粗棉线,待漆包线绕紧后抽紧。螺线管绕制匝数在
30—40匝之间即可.实验时,线头处漆皮也应用砂纸打去,
并且不要长时间通电,以防止损坏电源。用低压直流电源,一
定要加变阻器控制电流强度。
\begin{figure}[htp]
    \centering
\includegraphics[scale=.6]{fig/1-19.png}
    \caption{}
\end{figure}

\subsubsection{观察磁化现象}
注意在做这个实验时,是要把磁铁靠近铁钉。对于实验
现象,可作这样的解释:第一次用磁极一端(如$N$极)靠近时,
铁钉被磁化,靠近磁铁端为异名磁极(如$S$极),吸引铁屑端为
同名磁极(如$N$极)。小铁屑也被磁化,靠近铁钉一侧为异名磁
极(如$S$极)。当移去磁铁时,少量铁屑粘在铁钉上是由于这
部分铁屑都残留有剩磁,其极性不变。当用磁铁另一端(如$S$
极)靠近钉子头的瞬间,由于铁钉磁化靠近铁屑端为同名磁
极($S$极),而铁屑这一端由于原来剩磁磁极未变(还是$S$极),
因而同名相斥而掉落下来了。在解释时,可以向学生说明,铁
钉一般都是熟铁,是软磁性材料,而铁屑成分复杂,即使原
来是软磁性的,由于切削过程的高温,将使内部碳成分集中在
表面,表现出碳钢性质(硬磁性质),这就是铁屑磁化后有些剩
磁较强的原因。










































\chapter{电磁感应}
\minitoc[n]
\section{教学要求}
这一章教材是以初中学过的电磁现象以及高中学过的电
场、磁场等知识为基础,阐述电磁感应现象和电磁感应的基本
规律。这些内容是电磁学的基础知识,也是学习交流电、电磁
振荡和电磁波的基础。

本章教材可分为四个单元。课文的第一节为第一单元,
讲述电磁感应现象。第二、三节为第二单元,讲述楞次定律。第
四、五节为第三单元,讲述法拉第电磁感应定律和电磁感应现
象中的能量转化问题。第六节至第九节为第四单元,讲述电磁
感应现象中的几种特殊情况。

本章教材以法拉第电磁感应定律为中心,进一步揭示了
电与磁的内在联系,因而法拉第电磁感应定律是这一章的重
点。楞次定律及其应用,法拉第电磁感应定律的应用,是教
学上的难点。

用磁通量的概念来概括和表达电磁感应现象的规律,是
这一章的特点。教材是用磁通量的变化来叙述感生电流的产
生条件和楞次定律,用磁通量的变化率来叙述法拉第电磁感
应定律的。因此,理解磁通量的变化和磁通量的变化率的意
义,了解它们之间的区别与联系,对于学好全章具有重要的
意义。

研究感生电流产生的条件、判定感生电流的方向、确定感
生电动势的大小,都是通过实验分析得出规律的。因此在教
学中,做好实验是非常重要的。

用磁通量的变化来表述楞次定律,可适用于电磁感应现
象的各种情况,因而具有普遍性。在教学中,要注意强调这一
点。但是,对于由相对运动而产生感生电流的情形,用阻碍相
对运动来表述楞次定律,用起来比较方便,特别是在分析电磁
感应现象中的能量转化时要用到。因此,在教材中也注意了这
种情况下楞次定律的表述。学习楞次定律要着重使学生学会
运用这个定律来判断感生电流及其磁场的方向。学生已经学
过用右手定则判断感生电流的方向,教学中应该使学生理解
在由相对运动(导体切割磁力线)而引起的电磁感应现象中,
用右手定则和用楞次定律来判断感生电流的方向,结果是一
致的。

感生电动势是表示电磁感应现象的重要物理量,是电磁
学的基本概念,应使学生掌握。教材不要求区别感生电动势
和动生电动势,不要求学生掌握内电路中各点电势的高低。感
生电动势的大小,教材表述为$\mathcal{E}=k\dfrac{\Delta\phi}{\Delta t}$,
,而没有在等式的右边
加上“$-$”号,这是因为,对一般学生,理解负号的意义比较
困难。

能的转化和守恒定律,是普遍适用的客观规律。楞次定
律和法拉第电磁感应定律也符合这一规律。教材只就由相对
运动产生的电磁感应现象来分析能的转化和守恒问题,对于
没有相对运动而磁场变化的情形,由于能的转化涉及场能,而
没有加以分析。

直流电动机的反电动势是选讲教材,反电动势的概念在
讲变压器时要用到。这节内容,可以使学生知道电磁感应现
象不是孤立发生的,通电线圈在磁场中受力而运动,同时就切
割磁力线,产生电磁感应现象,出现反电动势。这个反电动势
又反过来影响线圈中的电流和线圈受力情况、运动情况。这
节内容对培养学生综合分析问题的能力是有好处的。如果课
时允许,希望讲一下这节教材。

自感现象是电磁感应现象的特殊情况,这个内容是学习
交流电、电磁振荡等内容的基础。教学中,要注意讲清自感系
数的物理意义。在自感现象的应用中,要讲清日光灯的两个
主要元件-起动器和镇流器的工作原理和作用。还可结合
实际情况,补充讲解一些日光灯的使用维修知识。

涡流一节是选讲教材,使学生知道产生涡流的条件,对
涡流的有利、有害两方面有所了解就可以了,不要求作进一步
的讨论。

本章的教学要求是:
\begin{enumerate}
\item 理解电磁感应现象,掌握产生感生电流的条件.
\item 掌握楞次定律和右手定则,并会应用它们判断感生电
流的方向。
\item 掌握法拉第电磁感应定律,并能用来计算有关感生电
动势的问题。
\item 理解自感现象及其在实际中的应用,理解自感电动势
的概念和自感系数的物理意义。
\end{enumerate}

\section{教学建议}
\subsection{电磁感应现象}
这一单元研究感生电流产生的条件,是本章教材的起始
课。教学时要注意从复习电生磁以及磁场对电流的作用这样
一些电与磁相互联系的已有知识中提出问题,即引导学生研
究如何利用磁场来获得电流的问题。同时可以简要介绍法拉
第其人及其在物理学上的贡献,以激发学生的学习积极性。

\subsubsection{感生电流产生的条件}
课本中安排了三个演示实验,它是本单元中研究感
生电流产生条件的依据,也是后面研究楞次定律、法拉第电磁
感应定律的基础。因此,认真做好三个演示实验,力争有较好
的观察效果,是教学的关键。同时,要引导学生逐步从实验
现象中总结出正确的结论,这不但是学生掌握概念和规律的
需要,而且也有利于培养学生的抽象概括能力。

三个实验的观察重点不同.实验一是观察闭合电路
的一部分导体在磁场中运动时产生感生电流的条件。这里要
引导学生注意观察导体向上或向下运动和向左或向右运动的
不同结果,从而了解“导体做切割磁力线运动”的含义。可以
明确告诉学生,所谓切割磁力线的运动,就是导体运动速度的
方向和磁感应强度的方向不平行,这也为推导$\mathcal{E}=B\ell v\sin\theta$
这一公式打下基础。实验二是进一步观察导体不动而磁铁运
动时是否产生感生电流。从而使学生了解,只要导体和磁场
之间发生切割磁力线的相对运动,闭合电路中就会产生感生
电流,这里要强调相对运动必须是导体切割磁力线的,否则
闭合电路中是没有感生电流的。实验三是研究导体静止在
磁场中时能否获得感生电流的问题。在这个实验里,学生能
观察到不论是线圈$A$电路的接通或断开还是变阻器滑动片的
移动,线圈$B$电路中都出现感生电流。在观察实验现象的基
础上,要引导学生分析上述现象的物理过程,在上一章的学习
中,学生已经知道由电流所激发的磁场的磁感应强度$B$总是
正比于电流强度$I$的,即$B\propto I$是一个有普遍意义的关系。
电路的闭合或断开控制了电流从无到有的变化,变阻器则是
通过改变电阻来改变电流。而电流的变化必将引起磁场的
变化,线圈$B$中出现感生电流就是由穿过它所围面积的磁场
变化所引起的。

\subsubsection{结论}

教材是在每个演示实验的后面,对实验现象加
以分析,通过对三个实验的分析,逐步得出产生感生电流的
条件,这就是:“不管是闭合电路的一部分导体做切割磁力线
的运动,还是闭合电路中的磁场发生变化,穿过闭合电路的
磁力线条数都发生变化,这时闭合电路中就有感生电流
产生。”然后,再用磁通量的概念总结出只要穿过闭合电路的
磁通量发生变化,闭合电路中就会产生感生电流这个 一般
条件。

处理这段教材时应注意以下几点:

通过提问学生来复习磁通量的概念,明确磁通量的
定义式为$\phi=BS\cos\theta$, $\theta$为面积$S$与垂直于磁感应强度方向
平面的夹角。当$S$与$B$垂直时,$\phi=BS$. 复习磁通量的概念之
后,可向学生提出“根据磁通量的定义式来分析,使一个面积
为$S$的闭合线圈中的磁通量发生变化,有哪几种方法”的问题,
以使学生进一步明确通过一闭合面积的磁通量是由哪些因
素决定的,同时为总结概括产生感生电流的条件打下基础。

利用磁通量概念概括产生感生电流的条件时,要注
意强调磁通量的变化。教学时,可画出上述三个实验的示意
图,并将它们的磁力线的分布在图上表示出来,如图2.1中
甲、乙、丙所示。
\begin{figure}[htp]
    \centering
\includegraphics[scale=.6]{fig/2-1.png}    
    \caption{}
\end{figure}

在图2.1甲中,闭合电路的一部分导体$AB$向右或向左
运动时都作切割磁力线运动,通过闭合电路的磁力线条数发
生了或增或减的变化.磁通量的变化量$\Delta\phi=B\Delta S$. 在图2.1
乙中,当$N$极向下插入线圈时,由于离$N$极近处磁感应强度较
大,使得线圈内部空间的磁感应强度变大。而线圈的面积不
变,故通过线圈的磁通量增加,图2.1丙中,接通开关$K$的瞬
间,线圈$A$中的电流由无到有,故磁感应强度由零开始增大,
而线圈$A$的面积$S$不变,所以磁通量也由零开始增大。上面
两个实验中,磁通量的变化$\Delta\phi=\Delta B\cdot S$. 通过上述分析,应使学生认识到:不论是导体作切割磁力线运动,还是磁场发生
变化,实质上都是引起穿过闭合电路的磁通量发生变化。在
此基础上给学生明确指出,产生感生电流的条件可以归结为
“穿过闭合电路的磁通量发生变化”。

因为初中教材中只讲了切割磁力线的情况,给学生
留下了一个先人为主的印象。又由于在某些情况下利用切割
磁力线来分析感生电流比较方便,因此,学生习惯于利用切割
磁力线来判断电磁感应现象,为了使学生对于用磁通量观点
来判断电磁感应现象加深认识,可通过一些典型例题让学
生进行讨论.例如课本中练习一4就可以放在课堂上处理。
通过例题分析还应使学生认识到,掌握磁体和通电导体所产
生的磁场的特点及磁力线在空间的分布情况,在判断感生电
流时也是十分重要的。


\subsubsection{培养学生用实验研究问题的能力}

在学校实验器
材较全且学生水平比较整齐的情况下,可采取在课堂上学生
随老师一起做实验的办法来组织这一节的教学,这样可对感
生电流产生的条件获得深刻的印象。

\subsection{楞次定律}
楞次定律是确定感生电流方向的一般规律。由于它的内
容抽象,涉及到电与磁间复杂的相互关系,因此它是本章教材
的一个难点。

试用本是从能量守恒定律出发,通过推理得出楞次定律
的。但是从能量守恒定律得出的是阻碍相对运动,要得出阻
碍磁通量变化的表达,还需要进一步的讲解。这就使楞次
定律的得出显得不够轻快。为了解决这个问题,本单元教材
抓住了磁通量变化这个线索,尽可能简捷地引出楞次定律。至
于电磁感应现象中的能量守恒问题,放在第三单元中去讨论。

本单元的重点是使学生明确引起感生电流的磁通量
的变化和感生电流所激发的磁场之间的关系。为了解决这个
问题,可先复习学生所学过的知识:
\begin{enumerate}
\item 磁通量的变化是产生感
生电流的条件。  
  \item 根据电流的磁效应,感生电流一定会激发
磁场,感生电流的方向与它所激发的磁场的方向间的关系可
由右手定则来判定.
\end{enumerate}
在上述知识基础上,再向学生提出“磁通
量的变化与感生电流的磁场之间有什么关系”的问题,并通过
实验引导学生来研究。

在利用课本图2.2所示的装置做实验时,要注意:
\begin{enumerate}
\item 
明确线圈导线的绕向(实验用的线圈最好是实验者自己绕制
的).   
 \item 实验前,要用旧干电池来确定电流的方向与电流表
指针偏转方向的关系。
\end{enumerate}

实验时,最好能将磁铁磁场的磁通量的变化以及感生电
流所激发的磁场的磁力线方向简明示出,使学生能顺利地找
出二者之间的关系。可将实验中观察到的现象填入下表。
\begin{center}
\begin{tabular}{c|p{.08\textwidth}p{.08\textwidth}p{.08\textwidth}p{.13\textwidth}p{.08\textwidth}c}
    \hline
   & 原磁场方向  & 原$\phi$变化情况& 感生电流方向& 感生电流磁场$B$方向& $B$与$\phi$的关系& 结论\\
   \hline
   $N$极向上\\
   $N$极向下\\
   $S$极向上\\
   $S$极向下\\
   \hline
\end{tabular}
\end{center}

为使学生更加形象地了解感生电流的磁场B和原磁场磁
通中的关系,分析实验结果时要注意强调:原磁场的磁通量
减少时,感生电流磁场与原磁场方向相同;原磁场的磁通量
增加时,感生电流磁场与原磁场方向相反,然后使学生进一
步认识,感生电流是通过其磁场与原磁场同方向或反方向来
起到阻碍磁通量变化的作用的。

\subsubsection{楞次定律的表述}

楞次定律表述为:“感生电流具有这样的方向,就是
感生电流的磁场总要阻碍引起感生电流的磁通量的变化。”虽
然这一表述是建立在实验的基础上,但为避免学生产生误解,
要针对学生容易产生的误解作一些讲解。①学生往往把“阻碍
原磁场的变化”理解为“阻碍原磁场”,从而得出“感生电流的
磁场必与原磁场的方向相反”,或者“感生电流的流向必与原
来电流的流向相反”等错误结论。②学生往往把“阻碍原磁场
的变化”理解为“阻止原磁场的变化”,从而得出“有了感生电
流,原磁场就不会变化了”或“感生磁场加原磁场等于稳恒磁
场”等错误结论。在实际教学时,应根据学生的实际问题,有
针对性的讲解,使学生真正理解楞次定律上述表述的意义。

教材通过分析磁铁和通电螺线管之间的磁极相互作
用,提出了楞次定律的另一种表述,即导体和磁体发生相对运
动时,感生电流总要阻碍相对运动。应该告诉学生,这是从不
同角度表述同一个定律,而不是两个定律。

\subsubsection{楞次定律的应用}
这一节的教学,要使学生通过实例的分析,加深对楞
次定律的理解。总结出用楞次定律判断感生电流方向的思路。
应用楞次定律来判断感生电流方向的步骤是:
\begin{enumerate}
\item 明确原来磁
场的方向。
\item 穿过闭合电路的磁通量是增还是减。
\item 根据楞次
定律判定感生电流的磁场方向。
\item 利用安培定则来判定感生
电流的方向。
\end{enumerate}
教学时,切忌由教师将这几个步骤直接讲给学
生,然后用这四个步骤来套例题。这种注入式的方法不利于
学生思维的发展,也不利于调动学生的积极思维。思路应该
让学生自己总结出来,在学生独立总结有困难时,教师可组
织学生讨论”应用之一”得出解题思路。

“应用之二”的分析难度较大.关键是引导学生把第
一步分析好。可分为:电键闭合时、电键打开时、变阻器滑动
片向左移动时、向右移动时四种情况下感生电流方向的判断。
让学生根据上述步骤进行判断后,再用实验演示进行验证。

“应用之三”的教学可先通过复习初中学过的右手定
则来判断,然后用楞次定律的第一种表述来判断。引导学生
进行比较,了解以下两点:
\begin{enumerate}
 \item 用楞次定律和用右手定则来判定
感生电流方向的结果是一致的。
\item 导体切割磁力线时,用右
手定则判感生电流方向更为简便。   
\end{enumerate}
学生条件容许时,还可
以引导学生用楞次定律的第二种表述来进行判断,这里的关
键是根据导体的运动方向来判定感生电流受到的安培力的方
向。然后根据安培力的方向和原磁场方向利用左手定则来判
断感生电流方向。最后告诉学生,究竟用什么方法来判断感
生电流方向,应根据题目的情况来决定,但以简便为原则,为
了加深这种认识,可将练习二中的题3提到课堂作巩固练
习.练习二中题6的实验应在课堂进行演示,或将仪器交给
学生在课后进行观察。

\subsection{法拉第电磁感应定律}
本单元的教学,首先讲述法拉第电磁感应定律,然后对电
磁感应现象中能量转化进行分析,使学生认识到电磁感应现
象的规律不但有实验基础,而且能用能量转化和守恒定律这
一普遍规律来认识,处理本单元教材应注意以下几点:
\begin{figure}[htp]
    \centering
\includegraphics[scale=.6]{fig/2-2.png}
    \caption{}
\end{figure}

法拉第电磁感应定律是研究感生电动势大小的规律,
感生电动势是反映电磁感应现象的重要物理量,要使学生理
解感生电动势的概念。可以从图2.2两个电路的对比中使学
生认识感生电动势,根据闭合电路的欧姆定律可知,闭合电路
的电流是由电源的电动势决定的.图2.2甲电路中的线圈相
当于电源,其电动势为感生电动势,线圈的导线电阻为电源内
阻。$a$端电势高为正极,$b$端电势低为负极,可以画出图
2.2乙的电路跟图2.2甲的情况类比,告诉学生对含有电磁
感应现象的闭合电路分析时,可首先画出如图2.2乙所示的
电路。还应通过分析使学生了解到,电磁感应现象中感生电
动势比感生电流更有本质意义。至于电路中出现的感生电
流,只是在闭合电路中有感生电动势存在的必然结果,当电
路不闭合时,也会产生电磁感应现象,这时并没有感生电流,
感生电动势却仍然存在。

要向学生说明,感生电动势方向的判定要借助于感生电
流方向的判定。即感生电动势的方向与闭合电路中感生电流
的方向相同。如果电路没有闭合,也要将它想象为闭合的。

关于电磁感应定律,课本是按如下线索来安排教材
的:
\begin{enumerate}
\item 重新演示第一节中的三个实验,使学生了解到由于磁通
量的变化快慢不同,直接观察到产生的感生电流大小不同。磁
通量变化越快,感生电流越强,感生电动势越大。    \item 提出用单
位时间内磁通量的变化来定量描述磁通量的变化快慢。从而
引出磁通量的变化率的概念。    \item 由上述思路自然引出感生电
动势大小由磁通量的变化率来决定的结论。即磁通量的变化
率越大,感生电动势就越大。
\end{enumerate}
按上述线索处理教材时,一是要
使学生对演示实验的现象观察清楚。二是要讲清变化率的
概念。教学时,可以列举速度由位移的变化率决定,加速度是
速度的变化率,等等,通过用这些已有知识的类比来加深对
变化率的理解。

讲解法拉第电磁感应定律的定量表达式时,要注意讲
清比例常数$k$, 可引导学生推导出$1{\rm Wb/s}=1{\rm V}$,从而得出
$k=1$的结论,因此法拉第电磁感应定律的表达式为$\mathcal{E}=\Delta\phi/\Delta t$
要使学生了解如果闭合电路是由$n$匝线圈串联组成,整个线
圈的总电动势是单匝线圈的$n$倍,即$\mathcal{E}=n\Delta\phi/\Delta t$。
这里要注意说
明,穿过每匝线圈的磁通量变化率是相同的。应明确指出,利
用$\mathcal{E}=\Delta\phi/\Delta t$
进行定量计算,所得结果是$\Delta t$时间内的平均电动
/
势。因为$\Delta t$是一个有限的时间间隔,在这个时间内磁通量
的变化可以是不均匀的。如果在$\Delta t$时间内磁通量的变化是
均匀的,则其变化率是恒定的,这时平均电动势和即时电动
势相等。

利用$\mathcal{E}=\Delta\phi/\Delta t$
来研究导体做切割磁力线运动,可
推导出计算感生电动势大小的公式为$\mathcal{E}=B\ell v\sin\theta$, 注意条
件是$v$与$\ell$垂直.其中,$\theta$是$B$与$v$的夹角.不论从分解速度
角度还是从分解磁感应强度角度来理解这个公式的意义,都
是等价的.因为不论以速度$v$为基准把$B$分解为$B_{\parallel}$和$B_{\bot}$, 还
是以$B$为基准把$v$分解为$v_{\parallel}$和$v_{\bot}$, 都会得出相同的结果.上
式中如果速度$v$为即时速度,则求得的$\mathcal{E}$为即时电动势,如果
速度$v$为平均速度,则电动势$\mathcal{E}$为平均电动势。切割磁力线运
动产生的感生电动势的计算式虽然是$\mathcal{E}=\Delta\phi/\Delta t$
的一个特例,
但在计算感生电动势时是一个十分重要的公式,必须使学生
熟练掌握。当闭合电路所包围的面积$S$不变时,由于磁场的
变化而引起磁通量发生变化,其感生电动势大小的计算式为
\[\mathcal{E}=S\cdot \frac{\Delta B}{\Delta t}\]
这种将磁感应强度的变化率
$\Delta B/\Delta t$
和感生电动势
相联系的公式,也应要求学生有所了解。

讲完法拉第电磁感应定律后,可引导学生从$\mathcal{E}=\Delta\phi/\Delta t$
这一公式出发,对产生感生电动势的条件、大小和方向作一总
结性分析,使学生理解公式中磁通量的变化量$\Delta\phi$是产生感
生电动势的条件,而感生电动势的方向是由磁通量的变化情
况来决定,即是增加还是减少来决定的。感生电动势大小则
是由$\Delta\phi/\Delta t$
来决定。

电磁感应现象中的能量转化问题在课本中是作为独
立的一节来安排的,教师可通过复习提问,明确能量的转化可
通过做功来实现。感生电动势的存在,意味着有其他形式的能
转变成了电能,然后引导学生来分析课本中第一节的实验一
和实验二,分析时,应注意:
\begin{enumerate}
\item 使学生定性了解机械能转化为
电能的物理过程,要讲清使导体和磁场之间维持相对运动克
服磁场力作功的过程是机械能转化为电能的过程;闭合电路
中的感生电流作功是电能转化为电路的内能等其他形式能量
的过程。
\item 要说明楞次定律是与能量转化和守恒定律相符的。
可以引导学生从反面来分析:假设感生电流方向与用楞次定
律判断的方向正好相反,所得的结果是什么?引导学生来讨
论这个问题。
\end{enumerate}


在上述定性讨论基础上,利用课本中图2.20来定量研究
电磁感应现象中的能量转化与守恒。定量讨论时,可按照物理
过程提出一系列问题进行引导。例如,$ab$导体作匀速运动时
受到的安培力大小是多少?方向如何?导体受到的外力多大?
外力克服安培力做功的表达式是什么?如果已知$\mathcal{E}$, 在$\Delta t$时
间内感生电流做功的计算式是什么?这两个功有什么关系?为
什么?使学生在回答或讨论这些问题中得到结论。

发生电磁感应现象时常伴随着其他现象发生。当闭
合电路(电阻已知)中产生感生电动势时,电路中出现感生电
流,而感生电流的强弱又由欧姆定律所决定。感生电流在磁
场中必将受到磁场力作用,可见通过感生电流可将电磁感应
与电路、力学等知识联系起来,因此本单元要安排习题课,但
题目不要选得过难,课本中作复习用的习题中的题4和题
5可作习题课的例题处理。根据学生学习中存在的问题,教
师也可以自选一些例题。

\subsection{直流电动机的反电动势~~自感和涡流}
本单元内容是在电磁感应定律的基础上研究电磁感应现
象的几种特殊情况。

\subsubsection{直流电动机的反电动势}
“直流电动机的反电动势”一节是选讲教材.处理这节
教材时要注意以下几点:

让学生观察演示实验和直流电动机模型,了解其结
构后,分析其转动原理。使学生了解通电线圈在磁场中受到的
安培力矩是使线圈产生转动的动力矩,由负载和摩擦引起的
力矩为阻力矩,当动力矩等于阻力矩时,线圈作匀速转动。

重点引导学生讨论直流电动机的线圈转动时所产生
的感生电动势及其特点,可从导线作切割磁力线运动的角度,
使学生了解在转动的线圈切割磁力线的边上产生感生电动
势。这个电动势的大小与线圈转动的快慢有关(不需作出定
量的分析)。感生电动势的方向,根据楞次定律来判断,方向
与线圈中的电流方向相反。而电流方向和外加电压方向相
同,故其感生电动势方向与电压方向相反。讲清线圈中感生电动势的上述特点后,引导学生得出电流公式
\[I=\frac{V-\mathcal{E}}{R}\]
将此式变形为$V=\mathcal{E}+IR$, 使学生了解电路中有反电动势存在
时,加在电路两端的电压$V$等于反电势$\mathcal{E}$跟线圈电阻上损失
的电压$IR$之和。

利用能量观点分析直流电动机时,应讲清以下两点:
\begin{enumerate}
    \item 明确公式$VI=\mathcal{E}I+I^2R$中各项的物理意义,说明它是一个
体现能量守恒的方程,使学生进一步理解电功和电热不一定
相等,只有在纯电阻电路中电功才等于电热,而象直流电动机
这种存在反电动势的电路中电功是大于电热的。
\item 分析直流
电动机的输入功率($VI$)随负载的增大而增大时,应按如下思
路进行:在外加电压$V$一定、直流电动机作匀速转动时,安培
力矩等于阻力矩;当负载加大时阻力矩增大,引起转速减小,
反电势$\mathcal{E}$随之减小,根据公式$I=\dfrac{V-\mathcal{E}}{R}$
可知电流$I$增大。再根据$VI=I\mathcal{E}+I^2R$可得输入功率增大.从而得出直流电动
机的输入功率随负载的增大而增大。
\end{enumerate}

为了加深学生对本节知
识的理解,可将课本85页习题中的题10作课堂例题引导学
生进行讨论。

\subsubsection{自感}
自感是电磁感应现象中的一种重要特例,在交流电路
中起着重要作用。处理这段教材时,要注意以下几点:

做好课本中两个演示实验,使学生认识到自感电动
势的存在及其对电流的作用。处理这个问题可以有两种不同
的思路。一种思路是先让学生观察实验现象。然后根据电磁
感应规律分析自感电动势的产生及其对电流的作用。另一思
路是先不让学生观察灯的亮度变化这一现象,而是让学生根
据实验装置利用电磁感应规律进行分析,引导学生预测现象,
然后进行实验验证,这种办法有利于将新知识较自然地纳入
学生原有知识结构中去,不管用哪一种思路,都应紧扣产生电
磁感应现象的一般规律——磁通量发生变化上,突出产生
自感现象的特殊条件。处理时还可将课文中的实验图画成如
图2.3的示意图来进行分析,使学生明确了解自感电动势是
由电路本身电流变化引起的磁通量变化而产生的。自感电动
势的作用是阻碍电流的变化,要强调“变化”二字。

\begin{figure}[htp]
    \centering
\includegraphics[scale=.6]{fig/2-3.png}
    \caption{}
\end{figure}

讲授自感电动势的计算公式时要注意:根据电流磁
场的磁感应强度$B$与电流强度$I$成正比,推导出$\Delta\phi\propto \Delta I$的关
系。由于自感电动势符合法拉第电磁感应定律,可得$\mathcal{E}\propto \dfrac{\Delta I}{\Delta t}$,
因而得出$\mathcal{E}=L\dfrac{\Delta I}{\Delta t}$。

要讲清自感系数$L$的物理意义及其决定因素与单
位。要向学生说明,自感系数$L$是线圈等电路元件自身的属
性,它的大小完全由自感线圈本身的形状、大小、匝数来决定。
当线圈中插入铁心后,线圈的$L$值将大大提高,由于受铁的磁
化特性的影响,$L$值不再是个恒量,而是随着电流的变化而变
化的。

\subsubsection{自感现象的应用}

主要是介绍日光灯的知识.日
光灯是根据辉光放电原理制成的,这一点在高中物理甲种本
第二册中已经学习过,在讲授日光灯的电路时,要通过实物
观察使学生了解电路结构和各部件的名称和作用。重点要讲
清起动器和镇流器的结构和作用。讲起动器时可作一个双金
属片,演示其受热和冷却时的弯曲现象,来帮助学生理解它的
开关作用,讲镇流器时应从自感现象来讲授它的作用。在起动
过程中,镇流器产生瞬时高电压(可达500—600伏).产生的
原因是由于双金属片断开,电路中的电流变化率$\Delta I/\Delta t$
很大,电
路中产生了一个很大的自感电动势。当日光灯正常发光时,
镇流器起降压限流作用,以维持灯管所需的一个较低的工作
电压,至于镇流器的限流作用,很难从自感电动势的角度讲
清,待讲交流电路时,可用感抗概念再作一些补充分析。

日光灯是学生很熟悉的灯具。可在演示实验和观察实物
后,让学生自学课文。然后分预热、起动、正常发光三个过程
来讲授。有条件的学校,可组织学生进行日光灯安装的课外
活动。

教学中,还可介绍些日光灯的使用常识。日光灯常见故
障大体有:管脚或开关接触不良、起动器接触不良或失效、灯
管老化等。镇流器较耐用,如果其铁心松动,则会发出“嗡嗡”
声,日光灯一般可使用3000小时以上,但每通断一次对使用
寿命有较大影响,所以日光灯不宜频繁开关。

\subsubsection{涡流}

这一节是选讲教材,要使学生了解块状金属在
变化的磁场中或在磁场中运动时产生感生电流的现象叫涡
流。然后通过实例讲清如何避免涡流的有害作用及利用涡流
获得大热量的应用.结合练习六题1的演示实验及电流表表
头中铝框的阻尼作用,讲讲电磁阻尼的原理。讲述时可将练
习六2题的实验在课堂上进行演示,然后让学生课后完成此
作业。

\section{实验指导}
\subsection{演示实验}
\subsubsection{电磁感应现象}
电磁感应现象应演示三个实验,这就是单根导线切割磁
力线、磁铁插入闭合线圈内、原线圈通断电时副线圈内产生感
生电流的现象。现分别说明如下:

单根闭合直导线切割磁力线产生感生电流.由于单
根直导线切割磁力线时产生感生电动势太微小,无法直接用
演示灵敏电流计演示出来。因此,通常采用一个高倍数放大
的直流放大器加接在灵敏电流计前。将微弱电流放大后,再
通过直流放大器输出端接到灵敏电流计进行演示。
\begin{figure}[htp]
    \centering
\includegraphics[scale=.6]{fig/2-4.png}
    \caption{}
\end{figure}

直流放大器的制作方法介绍如下。图2.4所示分差放大
电路制作较简单,但放大倍数较小,图2.5所示电路采用集
成运放电路,放大倍数较高,可根据学校条件进行制作,两种
电路均可安装焊接在一块铜箔板上,并和示教电表的其他附
件一样,安几个铜片接头。使用时直接插接在电表附件接头
上.在按图2.4电路装制放大器时,第一级的两个三极管和第
二级的两个三极管应尽量配对使用,其$\beta$值选40—70的.图
中10k电位器用以调节平衡,即将输出端接J0401型演示电
表200微安检流计挡时,若指针偏离零位,可调节该电位器。
此电路制成后不需调整即可使用,按图2.5制作放大器时,
最好用集成电路块的管座,以免因焊接不当造成集成块损坏。
CA741型集成块管脚排列已在图中画出,该电路放大倍数由$R_3$与$R_1$比值$R_3/R_1$决定,安装完毕后亦无需调整,接通电
源后就能工作.$R_4$亦为调零电位器.电源可以用两个6伏
积层电池。
\begin{figure}[htp]
    \centering
\includegraphics[scale=.6]{fig/2-5.png}
    \caption{}
\end{figure}

演示时,可将单根导线接在输入端,输出端与J0401型演
示电表检流计挡接好(将检流计指针调成中心零位式)。接通
电源后,再调节一下调零电位器,然后就可以将导线在蹄形磁
铁中作切割运动,电表指针的偏转说明切割时导线中产生了
感生电流,从指针偏转方向与导线切割方向的对应,可以说明
右手定则。

磁铁相对于闭合线圈运动时产生感生电流的实验装
置如课本图2.2所示,演示电表为J0401型检流计挡,所用
线圈为原副线圈的副线圈。应分别演示条形磁铁插入、静止
不动与拔出线圈时检流计指针摆动情况。在得出磁通量变化
时线圈中会产生感生电流的结论后,还可让学生观察磁铁垂
直于线圈平面在线圈外部附近运动时的情况,以启发学生思
考如何研究通过线圈的磁通量有无变化的问题。

穿过闭合电路的磁场发生变化产生感生电流的演示
装置如课本图2.3所示.所用仪器为J0401型演示电表200
微安检流计挡,演示用原副线圈,J2354型10欧滑动变阻器,
6伏蓄电池(或四节一号干电池,尽量不用低压电源,因其输
出直流事实上是脉动电流,而稳压输出允许电流又太小)。演
示时,应先将变阻器滑动端置于中间位置,然后依次演示闭合
电键瞬间、闭合电键后、打开电键瞬间以及通过变阻器滑动端
滑动时、停止时是否造成线圈内磁场变化而引起感生电流。再
归纳出产生感生电流的条件。

\subsubsection{楞次定律}
这个定律的演示包括两个方面,一是得出楞次定律的结
论,二是从物理实质的角度理解感生电流总是使自己产生的
磁场反抗引起它的磁通量的变化原因,第一个方面的演示,
按课本实验装置(课本图2.2)。演示时,分别将条形磁铁$N$
极插入、拔出与$S$极插入、拔出时情况加以分析。进行实验前,
必须向学生说明线圈中电流流动方向与检流计指针偏转方向
之间关系,必要时可用一节电池串一几十千欧的电阻接在检
流计上,说明指针偏转方向与通入电表接线柱电流的关系。按
照有关规定,当电流从正接线柱流入从负接线柱流出时电表
指针为顺时针偏转(观察者面对电表)。

第二个方面的演示,这里介绍两个方法。
\begin{enumerate}
\item 用楞次定律演示器演示,其装置如课本上图2.19所
示。实验时,为使现象显著,可将两根条形磁铁的同名端捆在
一起,迅速插入闭环或拔出闭环。用楞次定律分析闭环远离
或跟随磁铁移动的原因:闭环中磁通量若增加,感生电流阻碍
它的增加只有使闭环远离磁铁;若闭环中磁通量减少,感生
电流阻碍它的减少只有使闭环跟随磁铁移动。不论把磁铁
插入开口环还是从开口环中拔出,开口圆环总是不动,是因为
电路不闭合,环中只有感生电动势产生而无感生电流。
\item 用图2.6所示装置演示,其中线圈用J2423型可拆变
压器0—400匝绕组,铁心用其封闭铁心的条形铁心,用粗铜
丝焊成一个圆环并如图示悬挂起来,电源可用6伏蓄电池或
低压电源6V直流输出.实验时连好电路,在闭合电键瞬间铜
环将被推出,而在打开电键瞬间,铜环将被吸入。对这个现象,用楞次定律可作很好的分析与说明。
\end{enumerate}

\begin{figure}[htp]
    \centering
\includegraphics[scale=.6]{fig/2-6.png}
    \caption{}
\end{figure}

\subsubsection{法拉第电磁感应定律的定性演示}
这个定性实验分两个步骤进行。
\begin{enumerate}
\item 演示感生电动势大小
与磁通量变化快慢有关。实验时用J2423型可拆变压器红色
线圈0—800匝绕组,将其两端与J0401型演示电表200微安
检流挡相接,把条形磁铁分别以较慢与较快的速度插入线圈
之中。从所产生的感生电流的情况可以说明,磁通量变化越
快感生电动势越大这个定性关系。
\item 演示感生电动势的大小与线圈匝数有关,实验所用
仪器如1所述.演示时,磁铁以一定速度分别插入0—800匝
绕组线圈和0—1400绕组线圈中.通过检流计指针偏转的
情况,可定性说明在穿过每匝线圈的磁通量变化率相同的情
况下,线圈匝数越多产生感生电动势越大。分析这个现象时
需要指出的是,从感生电流的大小判断感生电动势的大小,是
以检流计为研究对象的。检流计内阻一定,两端电压越高,通
过的电流越大,如果不接检流计,直接将线圈头尾接为闭合
电路,这时,感生电动势虽随匝数增加而增大,但因每匝导线
的电阻相同,匝数增加内阻也相应增大,可以证明,在线圈匝
数多与少的两种情况下,感生电流的大小是一样的,在演示
时,为了保证两次线圈中人磁铁的速度一样,可以在线圈底
部垫一块泡沫塑料,使条形磁铁两次都从相同高度自由下落
进入线圈之中。此种情况下,由于感生电流产生磁场对条形
磁铁的阻力很小,可以略不计,而认为两种情况下磁铁运动
快慢是一样的。
\end{enumerate}

\subsubsection{直流电动机的反电动势}
这个实验可以用J2418型小型直流电动机(供学生组装
用的)或任一种玩具电动机来演示.实验电路如图2.7所示.
电源可用两节1号干电池串联.演示电流表用J0401型演示
电表(直流1安挡),串入的变压器为10欧滑动变阻器.演示
前应做好调整。先使变阻器串入电路部分的电阻为最大。闭
合电键,调整变阻器阻值,使电动机电枢固定时电路总电流接
近1安.演示时,先将电枢固定住不使它转动,这时电路中无
反电动势,电流表中电流较大(接近1安).然后再使电枢转
动起来,可以看到电流表示数变小,这表明电路中出现了反电
动势。在说明电动机输入功率、输出功率(转化为机械能)及
电机线圈上的热功率时,可以用伏特计测出固定电枢不动时
两端电压,用电流表测出电流值,再算出其线圈电阻(近似
值).在转动过程中再测出电压、电流值,分别计算$UI$值与$I^2R$
值并进行比较,从而说明电动机工作时能量转化情况。

\begin{figure}[htp]
    \centering
\includegraphics[scale=.6]{fig/2-7.png}
    \caption{}
\end{figure}

若要演示外加电压一定时电动机输人功率随阻力矩变化
的实验,可以用J1201型教学用低压电源的稳压输出(6伏,
1安),相应电动机应选用6伏的玩具电动机或者录音机用小
直流电动机。注意,电路中不能再串联变阻器。如果仍用上面
J2418型小型直流电动机模型,可以用4V蓄电池进行实验.
实验时要接好伏特表与安培表,在接通电源后,在电枢转动正
常时读出读数,然后用手指去阻碍电动机转轴转动,可以看
到读数增大。分析此时输入功率、热功率及输出功率情况,可
以说明电动机输入功率随负载变化而变化的情况。

\subsubsection{自感现象}
自感现象要演示两个方面的现象:通电瞬间的自感现象
和断电瞬间的自感现象。分别介绍如下:

\begin{figure}[htp]\centering
    \begin{circuitikz}[>=latex,european]
    \draw (2,0) to [battery] (0,0)--(0,3)--(.5,3);
    \draw (2,0) to [cute open switch](3,0)--(4,0) to [R] (5.1,0);
    
    \node at (2.5,-.5){$K$};
    \node at (4.5,-.5){$R_1$};
    \draw [american] (.5,3)--(.5,3.7) to [cute choke,l=$L$] (2.5,3.7) to [lamp] (4.5,3.7);
    \draw [->](.5,3)--(.5,3-.7)--(1.5,2.3)--(1.5,1.8);
    
    \draw  (1,1.6) to [R] (2.1,1.6)  to [lamp] (4.5,1.6)--(4.5,3.7);
    \draw [->](4.5,5.3/2)--(5.5,5.3/2)--(5.5, 1)--(4.5,1)--(4.5,.2);
    
    \node at (3.3,2.3){$A_2$};
    \node at (3.5,4.3){$A_1$};  \node at (1.5,1.2){$R$};
    \draw [fill=black](4.5,5.3/2) circle(1.5pt);
    \draw [fill=black](.5,3)circle(1.5pt);
    \draw[dashed](-.5,.65) rectangle (6,4.6);
    \end{circuitikz}
    \caption{}
    \end{figure}

通电自感现象演示的电路如图2.8所示(课本图2.25)。所用器材可有三种方案:
\begin{enumerate}
    \item 用J2425型变压器原理说明器成套装置中的示教板,实验时,将示教板紧固在该变压器上紧
    固铁心用的螺丝上即可,该示教板上有几段导线是可换接
    的铜片,换接后可进行断电自感演示。$R$是该说明器的一个
    附件,为一线绕电位器,实验时,可将电阻$R$及两个灯座(也
    是附件)按图示电路接在示教板上,两个灯泡可选用6—8伏
    指示灯,$L$为该变压器红色线圈0—1600匝绕组,实验时需将
    它们用导线接在示教板相应接线柱上,电源可用J1201型教
    学电源直流12伏输出,$R_1$用50欧1.5安滑动变阻器(J2355
    型).
    \item 选用J2423型可拆变压器0—1400匝绕组封闭铁心
    作为$L$(直流电阻为40欧左右),$R$可选用J2355型50欧1.5
    安滑线变阻器(或50欧线绕电位器),灯泡仍可选用6—8伏
    指示泡,此时最好自制一个示教板.
    \item 选用20瓦或40瓦日
    光灯镇流器,其电阻为25欧左右,故$R$仍可用J2355型50欧
    变阻器,灯泡可选用2.5伏或3.8伏手电筒用小电珠.2、3方
    案中的电源与$R_1$同1.
\end{enumerate}

 在演示前,先应闭合电键,调节$R_1$, 使与$L$串联的灯$A_1$
    灯丝由暗红转变为较白亮(因为实验时,小灯在这种情况下亮
    度变化对其电流变化的反应最显著)。然后调节$R$, 使与$R$串
    联的灯$A_2$亮度与$A_1$相同.演示时,可以鲜明地看到,在闭合
    电键瞬间,$A_2$是立即亮起来,而$A_1$是逐渐亮起来的,从而说
    明了线圈由于自感作用,其电流是逐渐增大的。

    \begin{figure}[htp]\centering
        \begin{circuitikz}[>=latex, yscale=.7]
        \draw (4,3) -- (5,3) -- (5,0) to [cute open switch] (2.5,0) to [battery] (0,0)--(0,3)--(1,3);
        \node at (7.5/2,.5){$K$};
        \draw (1,3)--(1,3.75) to [lamp] (4,3.75)--(4,3);
        \draw (1,3)--(1,2.25) to [L] (4,2.25)--(4,3);
        \node at (2.5,1.6){$L$};
        \node at (2.5,4.8){$A$};
        \draw [<-](2,3.75)--(1.5,3.75)node [above]{$I$};
        \draw [->](2,2.25)--(1.5,2.25);
        \end{circuitikz}
        \caption{}
        \end{figure}

    断电自感现象演示的电路如图2.9所示(课本图2.26)。
    图中电感线圈可选用J2423型可拆变压器绿色线圈0—400
    匝绕组,灯$A$可用6—8伏指示灯,电源用J1201型教学低压
    电源直流6伏输出,若选用J2425型变压器原理说明器,则可
用示教板上已备的电路,此时$L$用该变压绿色线圈0—400
匝绕组.由于绕组电阻只有4欧左右,通过电流$I_2$将远大于
小灯通过电流$I_1$, 因此当断开电键瞬间,由于自感现象,在由
$L$与灯$A$组成的电路中,$I_2$通过灯$A$形成回路而逐渐衰减,
由于$I_2$比$I_1$大得多,灯$A$将比原来亮许多而再熄灭,注意此
时通过灯$A$电流方向与原来$I_1$方向相反.这个实验也可以
利用扩音机所用25瓦线间变压器0—8欧输出端绕组作
为$L$, 此时小灯改用1.5伏小电珠,电源电压降至2伏进行
实验。

\subsubsection{日光灯电路中起动器与镇流器的作用}
演示日光灯起动器作用的电路比较简单,将拆去外罩的
起动器与一只220伏40瓦白炽灯串联,把起动器氖管放在投
影幻灯玻璃板上并固定好(或用其他幻灯机),将电路接入220
伏照明电路,通过投影和氖管、电灯交替发光可以明显看出
双金属片起的开关作用。为加强演示效果,可在起动器两端
接一个交流250伏电压表,说明起动电压。

日光灯镇流器所起的两个作用可以分别作以下两个演
示,首先演示断电瞬间,由于自感现象电感线圈两端产生高
电压,实验电路如图2.10所示,镇流器可选用8瓦日光灯镇
流器,氖管$A$用试电笔中的氖管(起辉电压60—70伏,日光灯
起动器中氖管起辉电压在100伏左右),电源只用1节干电
池,这样说服力强一些。在断开$K$瞬间,可以观察到氖管亮一
下,表明氖管两端电压超过60伏以上,引起管中辉光放电.这
一点与日光灯管内气体击穿导电道理是相同的。其次是演示
镇流器降压限流作用.实验电路如图2.11所示,最好制成示
教板,日光灯用20瓦的,灯$A$用220伏60瓦白炽灯.实验
时,先闭合$K_1$, 使日光灯正常发光,然后再闭合$K_2$, 灯管仍
发光,但亮度增加,表明镇流器与灯并联后,降压减小,再断
开$K_1$, 灯管还发光,说明此时灯$A$电阻作用代替了镇流器的
降压作用,如果断开$K_2$, 切断电源后,重新闭合$K_2$, 则灯管不
能正常发光,说明电阻(灯$A$)不能代替镇流器所起的产生瞬
时高压作用。

\begin{figure}[htp]\centering
    \begin{minipage}[t]{0.48\textwidth}
    \centering
\includegraphics[scale=.7]{fig/2-10.png}
    \caption{}
    \end{minipage}
    \begin{minipage}[t]{0.48\textwidth}
    \centering
\includegraphics[scale=.7]{fig/2-11.png}
    \caption{}
    \end{minipage}
    \end{figure}

\subsubsection{涡流现象}
涡流现象实验主要是演示涡流的机械效应即电磁阻尼现
象,课本上本章练习六第1题所述的就是这个实验所演示的
现象,因而可以先做这个演示实验,再要求学生自己对现象
作出分析,实验使用J2425型变压器原理说明器,其附件中
有专为演示电磁阻尼准备的强弱两种阻尼摆。实验时,将绿
色线圈(用0—400匝绕组)套在U型铁心上,装好两块极靴
并使两者间相距2厘米左右,再将两个强、弱阻尼摆连同支架
一起在固定条形铁心的卡板上安装好,并使阻尼摆摆动灵活
且不与极靴相碰,如图2.12所示,演示时,先不通电,将两摆
同时偏离平衡位置到相同角度,释放后可以看出两摆振动情
况相同,且衰减很慢.然后再给线圈通电(电源可用J1201型
低压电源直流12伏挡),这样在极靴间隙里存在一个较强的
磁场。再将两摆拉到偏离平衡位置相同角度,释放两摆后可
以明显看到,整铝片制的强阻尼摆很快停止下来,而梳状铝片
制的弱阻尼摆却能摆动较长一段时间。实验时因线圈中电流
较大,通电时间不宜太长。

\begin{figure}[htp]\centering
    \begin{minipage}[t]{0.48\textwidth}
    \centering
\includegraphics[scale=.7]{fig/2-12.png}
    \caption{}
    \end{minipage}
    \begin{minipage}[t]{0.48\textwidth}
    \centering
\includegraphics[scale=.7]{fig/2-13.png}
    \caption{}
    \end{minipage}
    \end{figure}

    这个实验也可以将两块大型演示用蹄形磁铁异名磁极相
对放置,在一对异名磁极间放置两片封闭磁铁磁路用的衔铁,
另一对异名磁极间的空隙就形成一个较强的磁场(图2.13)。
用直径0.3毫米左右的漆包线绕制两个软弹簧,下端分别焊上
同完整铜片制成的矩形铜片和梳状铜片,悬挂在物理支架上。
实验时,先后使它们在磁铁异名磁极间隙中上下振动,比较它
们所受阻尼的情况。


\subsection{学生实验}
\subsubsection{研究电磁感应现象}
实验电路如课本图2.3所示,实验器材除原副线圈外,
检流计可用J0408型灵敏电流计,其量程为士300微安(注意:
早期生产的一些灵敏电流计规格很多,而且表盘刻度并不代
表测量真实读数,只用于观察偏转多少).电源可使用J1202
型学生电源直流4伏挡,也可用2节干电池串联作为电源.滑
线变阻器用J2354型10欧2安的.在教学中要注意以下几
个问题:
\begin{enumerate}
\item 从课本上这段教材写法上看,是一个探索性实验,安
排在讲完第一节感生电流产生条件之后进行,通过实验归纳
出判断感生电流方向的规律。如果这个实验在教学时这样安
排的话,建议实验时先做条形磁铁插入与拔出的实验,然后再
做原线圈通断电时的实验,使学生首先把精力集中在规律的
归纳上,可不做课本上将通电线圈当作磁铁插入与拔出的
实验。
\item 这个实验需要实际认识线圈的绕向和检流计偏转方
向与实际线圈中电流方向间关系。这使一些学生实验时感到
困难,因而要加强实验辅导。与检流计指针偏转方向有关的
问题在前面演示实验2中已作说明.
\item 实验时注意引导学生掌握好四个环节。一是弄清实
验两个内容的区别,二是弄清如何从电表偏转判断副线圈中
感生电流方向,三是弄清通电原线圈中磁场方向,四是记录好
实验现象,这个实验的现象尽可能要求学生用画图方法记录。
对于他们来说,画图本身就是一种能力训练,而且对学生理解
课本上的一些示意图有好处。
\end{enumerate}

\subsection{课外实验活动}
\subsubsection{判断指南针的偏转方向}
这个实验是利用感生电流的直线电流部分产生的磁场使
小磁针偏转,要求学生先作判断,然后用实验验证,实验所用
漆包线尽可能粗一些(如直径0.4—0.5毫米).无漆包线时,
也可用普通照明电路用的塑料皮多股导线。原副线圈匝数
分别绕40匝左右,实验时需注意使铁心(小刀或铁钉)远离
小磁针,以免通电时铁心磁场(比直线电流磁场强得多)引起
小磁针偏转造成假象,还需注意通电时间要很短,否则电池
会由于长时间处于短路状态而损坏。实验时,先想好电源接
通与断开时应该产生的现象,再很快地接通电路,2—3秒钟
就断开,以看出现象为准,实验时也可以自制一个小检流计:
做一个小纸盒(以放下小磁针为准),在盒上用导线绕30匝左
右,将两端按上一章课外实验指导中介绍的方法固定好。使
用时,先将线圈平面调整到小磁针静止时所在的南北竖直面
上,也按习惯给线头标上正负,使电流由正端流入时,小磁针
顺时针偏转,就可接入电路进行实验了(图2.14)。
\begin{figure}[htp]
    \centering
\includegraphics[scale=.6]{fig/2-14.png}
    \caption{}
\end{figure}


\section{习题解答}

\subsection{练习一}
\begin{enumerate}
    \item 如图2.15所示,在磁场中有一个闭合的弹簧线圈,当图甲中人的双手离开后,线圈收缩(图乙),线圈收缩时,其中是否有感生电流?为什么?
\begin{figure}[htp]\centering
\includegraphics[scale=.6]{fig/2-15.png}
\caption{}
\end{figure}


\begin{solution}
    如图所示,线圈置于匀强磁场中。虽然磁感应强度$B$
    是不变的,但是线圈在收缩过程中它的面积是减小的。所以
    穿过线圈的磁通量减少,故线圈中有感生电流。
\end{solution}

    \item 在课本图2.6所示的匀强磁场中有一个线圈板,线圈平面垂直于磁力线,当线圈框在磁场中上下通动对,是否会在线圈框中引起感生电流?当线圈框在磁场中左右运动时,是否会在线圈框中引起感生电流?为什么?


\begin{solution}
    穿过线圈框的磁通量$\phi=B\cdot S_{\bot}$, 只要线圈在匀强磁
    场中作平动,则$\phi$不变,故线圈框上下运动或左右运动时,不
    会在线圈框中引起感生电流。
\end{solution}

    \item 如课本图2.7所示,线圈在匀强磁场中绕$OO'$轴转动时,线圈里是否有感生电流?为什么?

    \begin{solution}
        穿过线圈的磁通量为$\phi=B\cdot S_{\bot}$, $S_{\bot}$为线圈面积在
        垂直于磁力线方向上的投影.当线圈转动时,$S_{\bot}$是变化的,
        因此穿过线圈面积的磁通量是变化的,所以线圈里有感生电
        流产生.
    \end{solution}
    
    \item 如图2.16所示,让闭合线圈由位置1通过一个匀强磁场运动到位置2.线圈在运动过程中,什么时候有感生电流,什么时候没有感生电流?为什么?

    \begin{solution}
        从闭合线圈的右边刚刚进入磁场开始,到线圈的左
        边刚好进入磁场为止的这一段运动过程中,由于磁通量不断
        增加,故线圈中有感生电流。从线圈的右边刚好离开磁场开
        始,到线圈的左边刚好要离开磁场为止,在这一段运动过程
        中,磁通量不断减少,线圈中也有感生电流。线圈左边开始
        进入磁场到线圈右边开始离开磁场的过程中,因为穿过线圈
        的磁通量不变,线圈中无感生电流。
    \end{solution}
    
\begin{figure}[htp]
\centering
\includegraphics{fig/2-8.pdf}
\caption{}
\end{figure}


    \item 矩形线圈$ABCD$位于通电直长导线附近(课本图2.9),线圈跟导线同在一个平面内,且线圈的两个边与导线平行.在这个平面内,线圈远离导线平动时,线圈中有没有感生电流?线圈和导线都不动,当导线中的电流$I$逐渐增大或减小时,线圈中有没有感生电流?为什么?

    \begin{solution}
    线圈远离导线平动时,线圈中有感生电流。因为通
电长导线周围磁场的磁感应强度$B\propto \dfrac{1}{r}$, 
即磁场的分布是距
导线近处$B$大,距导线远处$B$小,所以线圈远离导线时,穿过
线圈的磁通量减少。线圈和导线都不动,当导线中的电流$I$
变化时,线圈中有感生电流,因为$B\propto I$, $I$变化要引起穿过
线圈的磁通量变化。
    \end{solution}
    
    \item 把一个铜环放在匀强磁场中,使环的平面与磁场的方向垂直(图2.17甲).如果使环沿着磁场的方向移动,铜环
中是否产生感生电流?为什么?如果磁场是不均匀的(图2.17乙),是否产生感生电流?为什么?
\begin{figure}[htp]\centering
\includegraphics{fig/2-10.pdf}
\caption{}
\end{figure}

\begin{solution}
    甲图中铜环上没有感生电流,因为铜环的运动不引
    起穿过的磁通量变化。乙图中的环中有感生电流,因为环向
    右运动时,由于磁感应强度$B$分布变小,引起穿过回路的磁
    通量减少。
\end{solution}

\end{enumerate}




\subsection{练习二}

\begin{enumerate}
    \item 在课本图2.9中,当线圈远离通电导线而去时,线圈中感生电流的方向如何?

    \begin{solution}
        线圈中感生电流方向为顺时针方向,即$B\to A\to D\to C
        \to B$的流向。
    \end{solution}
    
    \item 如课本图2.15所示,导线$AB$和$CD$互相平行.试确定
    在闭合和断开开关时导线$CD$中感生电流的方向.

    \begin{solution}
        开关闭合时,导线$CD$中感生电流方向由$D$流向$C$. 
        开关断开时,导线$CD$中感生电流由$C$流向$D$.
    \end{solution}
    
  \item 在图2.18中$CDEF$是金属框,当导体$AB$向右移动时,试确定$ABCD$和$ABFE$ 两个电路中感生电流的方向.应用楞次定律,我们能不能用这两个电路中的任意一个来判定导体$AB$中感生电流的方向?
  \begin{figure}[htp]
\centering
\begin{circuitikz}[>=latex, yscale=.8]
\draw (0,0)--(5,0) to [rmeter, t=G] (5,4)--(0,4) to [rmeter, t=G] (0,0);

\foreach \x in {.5,1.5,...,4.5}
\foreach \y in {.5,1.5,2.5,3.5}
{
   \node at  (\x,\y) {$\times$};
}

\node at (0,-.5){$E$};
\node at (2.7,-.5){$A$};
\node at (5,-.5){$D$};
\node at (0,4.5){$F$};
\node at (2.7,4.5){$B$};
\node at (5,4.5){$C$};
\draw [fill=white](2.7,4.25) rectangle (2.9,-.25);
\draw [->](2.9, 2)--(3.8,2)node[right]{$v$};

\end{circuitikz}
\caption{}
\end{figure}

\begin{solution}
    导体$AB$向右移动时,$ABCD$闭合电路中的磁通量减
    小,而$ABFE$闭合电路中的磁通量增大。根据楞次定律可知,
    $ABCD$闭合电路中的感生电流是顺时针方向流动的,而$ABFE$
    闭合电路中的感生电流是逆时针方向流动的。

    也可以根据右手定则判断出$AB$导体上的感生电流是由
    $A$流向$B$, 把$AB$视为电源,则$CD$和$FE$为两并联支路,可得
    两个闭合电路中的感生电流方向和上述结果相同,可以用两
    个电路中的任意一个来判定导体$AB$中的感生电流方向。
\end{solution}

  \item 在课本图2.17所示的电路中把滑动变阻器$R$的滑动片向左移动使电流减弱.试确定这时线圈$A$和$B$中感生电流的方向.

  \begin{solution}
    A和B线圈中的感生电流的方向如图2.19所示。
\begin{figure}[htp]\centering
\includegraphics[scale=.6]{fig/2-19.png}
\caption{}
\end{figure}  
  \end{solution}
  
  \item 如课本图2.18所示,把一个备形磁铁从闭合螺线管的右端插入,由左端抽出,在整个过程中,螺线管里产生的感生电流的方向是否发生改变?

\begin{solution}
    感生电流方向发生改变。磁铁从右端插入时和由左端抽出时,感生电流方向相反。
\end{solution}



  \item 课本图2.19中的$A$和$B$都是很轻的铝环,环$A$是闭合的,环$B$是断开的.用磁铁的任一极来接近$A$环,会产生什么现象?把磁铁从$A$环移开,会产生什么现象?磁极移近或远离
  $B$环时,又会产生什么现象?用所学的知识解释这些现象.

  \begin{solution}
用磁铁的任一极来接近$A$环时,$A$环均被推斥,运动
方向和磁铁移动方向相同;把磁铁从$A$环移开时,$A$环被磁铁
吸引而随磁铁运动。磁铁移近或远离$B$环时,则$B$环静止不
动。因为$A$环是闭合的,当磁铁接近或移开时,$A$环中产生感
生电流。当磁铁接近$A$环时,$A$环对着磁铁的一侧形成与磁
铁同名的磁极,因而$A$环被推开。当磁铁离开$A$环时,$A$环
对着磁铁的一侧形成与磁铁异名的磁极,因而$A$环被吸引,$B$
环不闭合,不产生感生电流,也就没有感生电流的磁场与磁铁
发生相互作用,因此$B$环不因磁铁靠近或远离而产生运动。
\end{solution}
\end{enumerate}



\subsection{练习三}

\begin{enumerate}
    \item 下列说法哪个正确?
    \begin{enumerate}
        \item 电路中感生电动势的大小,跟穿过这一电路的磁通量成正比;
        \item 电路中感生电动势的大小,跟穿过这一电路的磁通量的变化量成正比;
        \item 电路中感生电动势的大小,跟穿过这一电路的磁通量的变化率成正比;
        \item 电路中感生电动势的大小,跟单位时间内穿过这一电路的磁通量的变化量成正比.
    \end{enumerate}

    \begin{solution}
    (c)、(d)的说法是正确的。
    \end{solution}
    
\item 试证明:$1{\rm V}=1{\rm Wb}/{\rm s}$;$1{\rm V}=1{\rm T}\cdot 1{\rm m}\cdot 1\ms$.

\begin{solution}
\begin{enumerate}
    \item  因为$\phi=B\cdot S$, $B=\dfrac{F}{I\ell}$, $U=\dfrac{W}{q}$,可得:
   \[ 1{\rm Wb}=1{\rm T\cdot m^2}=1\frac{{\rm N}}{{\rm A\cdot m}}\cdot {\rm m^2}=1\frac{{\rm N\cdot s}}{{\rm C}}\cdot {\rm m}=1\frac{\rm J\cdot s}{\rm C}=1{\rm V\cdot s}\]
    所以: $1{\rm V}=1{\rm Wb}/{\rm s}$
    \item  \[1{\rm V}=1{\rm Wb/s}=1{\rm T}\cdot \frac{\rm m^2}{\rm s}=1{\rm T}\x 1{\rm m}\x 1\ms\]
\end{enumerate}
\end{solution}

\item  长5cm的导线在0.02T的匀强磁场中运动,运动的方向跟磁力线垂直,运动的速率$v=0.1\ms$,求感生电动势.

\begin{solution}
\[\mathcal{E}=B\ell v=0.02\x 0.05\x 0.1=10^{-4}{\rm V}\]
\end{solution}

\item 在0.4T的匀强磁场中,长度为25cm的导线以6$\ms$的速率做切割磁力线的运动,运动方向跟磁力线成$30^{\circ}$角,并跟导线本身垂直.求感生电动势.

\begin{solution}
    \[\mathcal{E}=B\ell v\sin\theta=0.4\x 0.25\x 6\x \sin 30^{\circ}=0.3{\rm V}\]
\end{solution}

\item 50匝的线圈,穿过它的磁通量的变化率为0.5${\rm Wb}/{\rm s}$,求感生电动势.

\begin{solution}
    \[\mathcal{E}=n\frac{\Delta \phi}{\Delta t}=50\x 0.5=25{\rm V}\]
\end{solution}

\item 有一个1000匝的线圈,在0.4秒内穿过它的磁通量从0.02韦增加到0.09韦,求线圈中的感生电动势,如果线圈的电阻是10欧,把它跟一个电阻为990欧的电热器串联组成闭合电路时,通过电热器的电流是多大?

\begin{solution}
\[\begin{split}
    \mathcal{E}&=n\frac{\Delta \phi}{\Delta t}=1000\x \frac{0.09-0.02}{0.4}=175{\rm V}  \\
    I&=\frac{\mathcal{E}}{R+r}=\frac{175}{990+10}=0.175{\rm A}
\end{split}\]
\end{solution}

\item 课本图2.23是法拉第做成的世界上第一个发电机模型
的原理图.把一个铜盘放在磁场里,使磁力线垂直穿过铜盘;转动铜盘,就可以获得持续的电流.试解释其作用原理.

\begin{solution}
把铜盘视为从盘中心到盘边缘很多相同的条形导体
并联组成。当铜盘转动时,每一根条形导体均切割磁力线而
产生电动势。如果铜盘顺时针转动,根据右手定则可知电动
势方向由盘中心指向边缘,即盘中心为负极,盘边缘为正极。
因此可把转动的铜盘视作很多电动势相同的并联电源。
\end{solution}

\end{enumerate}




\subsection{练习四}
\begin{enumerate}
    \item 在课本图2.20中,以一定速度向右拉动导体$ab$,如果导体$ab$的电阻增大,在相同的时间里,外力做的功是增大还是减小?为什么?

    \begin{solution}
        以一定速度向右拉动导体$ab$, 则外力跟安培力大小
        相等,故外力做功为$W=I\ell Bv\Delta t$, 因为$v$、$\ell$、$B$、$\Delta t$均为不变
        量,所以当导体$ab$的电阻增大时,闭合电路中的电流$I$减小,
        故外力做的功要减小。
    \end{solution}
    
    \item 在课本图2.20中,是快拉还是慢拉所需的功率多?为什么?

    \begin{solution}
        根据$P=I\mathcal{E}=\dfrac{\mathcal{E}^2}{R}$,       $\mathcal{E}=B\ell v$可知,当快拉时导体$ab$切
        割磁力线的速度大,因而产生感生电动势大。故快拉时所需
        功率多。
    \end{solution}
    
    \item 在课本图2.20中,导体$ab$刚开始受到恒力拉动时,它做什么运动?随后它的运动情况将如何变化?感生电动势和感生电流将如何变化?外力做的功产生了哪些效果?如果在导体$ab$还没拉出磁场时停止外力,情况又将怎样?(不考虑摩擦和空气阻力)

    \begin{solution}
        导体$ab$在外力作用下开始运动后,受到两个力,一个
        是恒外力$F$, 一个是安培力$F_{\text{安}}=I\ell B$, 两个力方向相反。根
        据牛顿第二定律可得
\[F-F_{\text{安}}=ma\]
因为
\[F_{\text{安}}=I\ell B=B\cdot \dfrac{\mathcal{E}}{R}\ell =B\frac{B\ell v}{R}\ell =\frac{B^2\ell^2 v}{R}\]
所以
\[F-\frac{B^2\ell^2 v}{R}=ma\]
即:
\[a=\frac{F}{m}-\frac{B^2\ell^2 v}{Rm}\]

由此式可知:当起动时导体$ab$的速度$v=0$, 这时加速度
$a=F/m$
为最大值,故$ab$导体做加速运动。随着$v$的增大,
加速度逐渐减小,导体$ab$作变加速度运动,在这个过程中,
感生电动势不断增大,感生电流也随之增大,这时外力做的
功,一方面使电路获得电能,另一方面使导体$ab$的动能不断
增加。如果导体$ab$始终在磁场中运动,当$F_{\text{安}}=F$时,加速度
$a=0$, 导体$ab$将在磁场中作匀速直线运动。如果在导体$ab$
还没有被拉出磁场时就停止了外力的作用,这时作用在导
体$ab$上的只有安培力,而且安培力的方向和$v$方向相反,因
而加速度方向也与$v$方向相反,故导体$ab$将做减速运动,这
时感生电动势和感电流都开始减小,安培力也不断减小,因
而加速度也不断减小。只要导体$ab$始终在磁场中运动,最后
它将达到静止状态。外力停止作用时导体$ab$所具有的动能,
在这个过程中全部转化为电路的电能,而电能最终转化为电
路的内能,使电路温度升高。
    \end{solution}
    
\end{enumerate}



\subsection{练习五}
\begin{enumerate}
    \item 制造电阻箱时,要用双线绕法,如课本图2.29所示,这样就可以使自感现象的影响减弱到可以略去的程度,为什么?


\begin{solution}
    双线绕法使得线圈中两组绕线电流方向相反。因而
    电流所激发的磁场的合磁感应强度基本为零,故线圈中基本
    无变化的磁通量,可把自感现象的影响减弱到可以略去的
    程度。
\end{solution}

    \item 有一个线圈,它的自感系数是1.2亨,当通过它的电流在$1/200$秒内由零增加到5.0安时,线圈中产生的自感电动势多大?

    \begin{solution}
\[\mathcal{E}_{\text{自}}=L\cdot \frac{\Delta I}{\Delta t}=1.2\x \frac{50}{1/200}=1.2\x 10^{3}{\rm V}\]
    \end{solution}
    
    \item 一个线圈的电流强度在0.001秒内有0.02安的变化时,产生50伏的自感电动势,求线圈的自感系数,如果这个电路中电流强度的变化率变为40${\rm A}/{\rm s}$,自感电动势是多大?

    \begin{solution}
\[\begin{split}
    L&=\frac{\mathcal{E}\cdot \Delta t}{\Delta I}=\frac{50\x 0.001}{0.02}=2.5{\rm H}\\
    \mathcal{E}&=L\frac{\Delta I}{\Delta t}=2.5\x 40=100{\rm V}
\end{split}\]
    \end{solution}
    
    \item 试证明$1{\rm H}=1\Omega\cdot {\rm s}$.

    \begin{solution}
    因为$ L=\dfrac{\mathcal{E}\cdot \Delta t}{\Delta I}$,所以
    \[1{\rm H}=1\frac{\rm V\cdot s}{\rm A}=1\Omega\cdot {\rm s}\]
    \end{solution}
    
    \item 根据感生电动势$\mathcal{E}=\dfrac{\Delta \phi}{\Delta t}$和 $\mathcal{E}=L\dfrac{\Delta I}{\Delta t}$,证明$L=\dfrac{\Delta \phi}{\Delta I}$,
    并说明这个式子的物理意义.

    \begin{solution}
因为$\mathcal{E}=\dfrac{\Delta \phi}{\Delta t}$,$\mathcal{E}=L\dfrac{\Delta I}{\Delta t}$,所以
\[\dfrac{\Delta \phi}{\Delta t}=L\cdot \dfrac{\Delta I}{\Delta t}\]
即:$L=\dfrac{\Delta \phi}{\Delta I}$

这个式子说明,自感系数在数值上等于电流改变1安所
产生的磁通量的变化量。
    \end{solution}
    
\end{enumerate}




\subsection{练习六}
\begin{enumerate}
    \item 如课本图2.33所示,$A$是可以在电磁铁两磁极间摆动的铝片(或铜片).电磁铁线围没有通电时,铝片可以摆动较长时间才停下来.电磁铁线圈通电时,铝片很快就会停下来.解释这个现象.
    
这种现象叫做电磁阻尼,在实际中有很多应用.课文中
讲的,使电学测量仪表的指针很快地停下来,就是电磁阻尼的应用.电磁阻尼还常用于电气机车的电磁制动器中.

\begin{solution}
    电磁铁通电时,电磁铁两磁极之间产生磁场,金属片
    $A$在两磁极间摆动时作切割磁力线的运动,因而在金属片$A$
    中形成涡流。涡流又受到磁场力作用,由楞次定律可知,这个磁场力是阻碍金属片$A$作切割磁力线的运动的,所以金属片
    很快就会停下来。
\end{solution}


\item 如课本图2.34所示,把蹄形磁铁的两极靠近一个金属圆盘,但不接触.当磁铁绕轴转动时,圆盘也绕轴转动起来,解释这个现象.

这种现象叫做电磁驱动,在实际中也有很多应用,下一章要讲的感应电动机就是利用这个道理驱动的.家庭中用的电度表,汽车上用的电磁式速度表,也利用这种电磁驱动.

\begin{solution}
    当蹄形磁铁转动时,相当于金属圆盘向反方向作切割
    磁力线的转动,在金属圆盘里形成涡流。根据楞次定律可知,
    金属圆盘中的涡流受到的磁场力作用是阻碍圆盘和蹄形磁铁
    之间的相对运动的,所以圆盘随磁铁一起转动,但圆盘的转速
    要小于磁铁的转速。
\end{solution}

\end{enumerate}



\subsection{习题}
\begin{enumerate}
    \item 在课本图2.35中,条形磁铁以速度$v$向螺线管靠近,下面哪种说法是正确的:


    \begin{enumerate}
\item 螺线管中不产生感生电流;
\item 螺线管中产生感生电流,方向如课本图2.35所示;
\item 螺线管中感生电流的方向与图中所示的方向相反.
    \end{enumerate}
    

\begin{solution}
(b)正确。
\end{solution}

    \item 在课本图2.36中,线圈$M$和线圈$P$绕在同一铁芯上.
    \begin{enumerate}
        \item 当合上电键$K$的一瞬时,线圈$P$里有没有感生电流?
        \item 当线圈$M$里有稳恒电流通过时,线圈$P$里有没有感生电流?
        \item 当断开电键$K$的一瞬时,线圈$P$里有没有感生电流?
    \end{enumerate}
    在上面三种情况里,如果线圈$P$里有感生电流,指出线圈$P$的哪一端是$N$极.

    \begin{solution}
\begin{enumerate}
    \item 有感生电流,线圈$P$右端为$N$极。
    \item 没有感生电流。
    \item 有感生电流,线圈$P$左端为$N$极。
\end{enumerate}
    \end{solution}
   
    \item 宇航员飞到某一个不熟悉的行星上,他们想用一只灵敏电流表和一个线圈来探测一下行星周围是否有磁场,应当怎样办?
   

\begin{solution}
    将线圈和灵敏电流表串联,然后使线圈在星球表面运
    动(平动和转动),观察电流表指针是否发生偏转。如果有磁
    场,通过线圈的磁通量发生变化,电流表指针要发生偏转。如
    果不管线圈如何运动,电流表始终不显示有电流存在,则可断
    定行星周围没有磁场。
\end{solution}

 \item 如图2.20所示,在匀强磁场中有一个线圈.
\begin{figure}[htp]\centering
\includegraphics{fig/2-37.pdf}
\caption{}
\end{figure}
    \begin{enumerate}
        \item 当线圈分别以$P_1$和$P_2$为轴按逆时针方向转动时(如图中箭头所示),感生电流的方向各是什么?
        \item 当转速恒定时,线圈以$P_1$和$P_2$为轴转动时,两种情况下感生电流的大小有何关系?
        \item 当转速恒定时,感生电动势的大小跟线圈面积有何关系?
        \item 设磁感应强度B为1.5特,$AB$为10厘米,$BC$为4厘米,转速为每秒$120/2\pi$转,分别求出以$P_1$和$P_2$为转轴时感生电动势的最大值.
    \end{enumerate}

    \begin{solution}
\begin{enumerate}
    \item 感生电流方向均为顺时针方向(即按$B\to C\to D
    \to A\to B$方向流动)。
    \item 两种情况下感生电流的大小相等,变化规律相同。
    \item 以$P_1$为转轴,则$BC$边切割磁力线.图示线圈位置
    为起始时刻,$t$秒后线圈平面与磁力线夹角为$\theta$, 设$AB=L_1$, 
    $BC=L_2$, 角速度为$\omega$,根据感生电动势公式可得
\[\begin{split}
    \mathcal{E}&=BL_2\cdot v\cdot \sin(90^{\circ}+\theta)\\
    &=BL_2\cdot L_1\cdot \omega\cdot \sin(90^{\circ}+\theta)\\
    &=BS\omega \cdot \cos\theta\\
    &=BS\omega \cos\omega t
\end{split}\]

    当线圈平面与磁力线平行时,$\theta=0^{\circ}$, 则$\mathcal{E}$为最大值,$\mathcal{E}_m=BS\omega$. 如果以$P_2$为转轴可得上述同样结果.可以证明
    上述结果与转轴的位置无关,可见当转速恒定时感生电动势
    的最大值跟线圈面积成正比。
    \item $\mathcal{E}_m=BS\omega=1.5\x0.1\x0.04\x\dfrac{120}{2\pi}\x2\pi=0.7{\rm V}$
\end{enumerate}
    \end{solution}
    
    \item 如图2.21所示,在磁感应强度为0.5特的匀强磁场中,让长为0.2米的导体$AB$在无摩擦的框架上以5$\ms$的速度向右滑动,如果$R_1=R_2=2$欧,其他导线的电阻不计,外力做功的功率有多大?感生电流的功率有多大?在电阻$R_1$和$R_2$上消耗的功率有多大?验证一下:能的转化是否符合守恒定律?
\begin{figure}[htp]
    \centering
\begin{circuitikz}[>=latex, european, scale=.7]
\foreach \x in {-3,-2,...,3}
   \foreach \y in {-1,0,1}
{
   \node at (\x,\y){$\times$};
}

\draw (-4,-1.5)--(4,-1.5) to [R=$R_2$] (4,1.5)--(-4,1.5) to [R=$R_1$] (-4,-1.5);
\draw [ultra thick] (.5,-1.8)node[right]{$B$}--(.5,1.8)node[right]{$A$};

\end{circuitikz}
    \caption{}
\end{figure}

    \begin{solution}
导体$AB$作切割磁力线运动,其上产生感生电动势。
所以导体$AB$可视为电源.$R_1$、$R_2$为外电路,且$R_1$与$R_2$并联.
\[\begin{split}
    \mathcal{E}&=B\ell v=0.5\x0.2\x5=0.5{\rm V}\\
I&=\frac{\mathcal{E}}{R}=\frac{0.5}{1}=0.5{\rm A}\\
I_{R_1}&=I_{R_2}=0.25{\rm A}\\
P_{\text{外}}&=F\cdot v=BI\ell\cdot v=0.5\x0.5\x0.2\x5=0.25{\rm W}\\
P_{\text{电流}}&=I\cdot \mathcal{E} =0.5 \x0.5=0.25{\rm W}\\
P_{\text{热}}&=I_1^2R_1+I_2^2R_2=0.25\x2+0.252\x2=0.25{\rm W}   
\end{split}\]
计算结果表明,外力克服安培力作功将机械能全部转化
为电能,在电路上电流作功将电能又全部转化为电阻上的内
能,上述计算结果证明能的转化是符合守恒定律的。
    \end{solution}
    
\item 如课本图2.39所示,让铜线圈$A$自由落下,并通过一段有匀强磁场的空间,试定性说明线圈的运动情况.(不考虑空气阻力)


\begin{solution}
    当线圈$A$未进入磁场时作自由落体运动,其加速度
    $a=g$. 当线圈$A$的下边进入磁场时线圈中有感生电流,线圈
    $A$受到方向相反的重力和安培力作用,故加速度$a<g$. 当线
    圈$A$全部进入磁场后,因只受重力作用,线圈$A$向下运动的
    加速度$a=g$. 当线圈的下边开始离开磁场时,线圈中产生感
    生电流,由于安培力的存在,线圈的加速度$a<g$. 当线圈全
    部离开磁场后,线圈的加速度$a=g$.
\end{solution}

\item 弹簧上端固定,下端悬挂一根磁铁.将磁铁抬到某一高度后放开,磁铁能上下振动较长时间才停下来.如果在磁铁下端放一个固定的闭合线圈使磁铁上下振动时穿过它
(课本图2.40),磁铁就会很快地停下来,试用能量观点来解释这个现象.

\begin{solution}
    在没有闭合线圈时,磁铁上下振动要不断克服空气
    阻力做功,把原有的机械能转化为空气和磁铁的内能,因而振
    幅逐渐减小,最后会停下来。如果在磁铁下方放一个固定的
    闭合线圈,磁铁上下振动,在闭合线圈中会产生感生电流。根
    据楞次定律可知,闭合线圈中感生电流的磁场总是阻碍线圈
    和磁铁之间的相对运动。因此,磁铁除了克服空气阻力做功
    外,还要克服这种阻力做功,把一部分机械能转化为闭合线圈
    中的电能,这比起没有线圈,只需克服空气阻力做功时,机械
    能的消耗要快得多,所以磁铁会较快地停下来。
\end{solution}

\item 课本图2.41是生产中常用的一种延时继电器的示意图.铁芯上有两个线圈$A$和$B$.线圈$A$跟电源连接,线圈$B$的两端接在一起,构成一个闭合电路,在拉开电键$K$的时候,弹簧$S$并不能立即将衔铁$D$拉起,从而使触头$C$(连接工作电路)
立即离开,过一段短时间后触头$C$才能离开;延时继电器就是这样得名的,试说明这种继电器的原理.

\begin{solution}
    当拉开电键$K$时,线圈$A$中电流发生由有到无的变
    化,在线圈$B$中引起感生电流,而且感生电流的磁场和原磁
    场方向相同。这就使铁心仍处于磁化状态,因而继续将衔铁
    吸住,直到感生电流消失。这时铁心失去磁性,弹簧$S$将衔铁
    $D$拉起。从而使触头$C$离开。
\end{solution}


\item 如图2.22所示,电源的电动势$\mathcal{E}=1.5$伏,内电阻$r=0.5$欧,$AB=0.5$米,$AB$的电阻$R=0.1$欧,框架的电阻不计,磁感应强度为0.5特,金属框对$AB$的滑动摩擦力为$0.25$牛.
\begin{enumerate}
    \item 分析一下,当电键$K$闭合后,会发生哪些电磁现象?
    \item 当$AB$的速度达到稳定时(即速度为最大时),电路中的电流强度是多大?
    \item $AB$的最大速度是多少?
    \item 这时电源消耗的电能转化为什么形式的能?通过计算验证一下:能的转化是否符合守恒定律?
\end{enumerate}

\begin{figure}[htp]
\centering
\begin{circuitikz}[>=latex, scale=.8]
\draw (0,0) to [cute open switch] (3,0)--(6,0);
\draw (6,4)--(0,4) to [battery2] (0,0);
\foreach \x in {.5,1.25,...,6}
\foreach \y in {.5,1.25,...,3.5}
{
   \node at (\x, \y) {$\times$};
}

\node at (1.5,-.2){$K$};

\draw[very thick] (4,0)node[below]{$B$}--(4,4)node[above]{$A$};
\draw[very thick] (4,0) arc (0:-180:0.1);
\draw[very thick] (4,4) arc (0:180:0.1);

\draw[->] (4,2.3)--(4.8,2.3)node[above]{$F$};

\end{circuitikz}
\caption{}
\end{figure}

\begin{solution}
\begin{enumerate}
    \item 当电键$K$闭合后会发生如下一些电磁现象:其一是
    闭合电路中的电流在周围空间激发磁场;其二是通电导体
    $AB$在磁场中受到安培力作用,其方向如图中$F$所示;其三是
    当导体$AB$开始运动后,因切割磁力线而在其上产生感生电
    动势,它的方向与$AB$中电流的方向相反。
    \item 导体$AB$在运动过程中受安培力$F$和滑动摩擦力$f$
    的作用,这两个力方向相反,当$F=f$时,导体$AB$达到稳定。
    因为$F=BI\ell$,所以$BI\ell=f$.
    
    因此
    \[I=\frac{f}{B\ell}=\frac{0.25}{0.5\x 0.5}=1{\rm A}\]

    \item 因感生电动势为反电动势,可得:
    \[I=\frac{\mathcal{E}-\mathcal{E}_{\text{感}}}{R+r}\]
由于:$\mathcal{E}_{\text{感}}=B\ell v$,所以
\[I=\frac{\mathcal{E}-B\ell v}{R+r}\]
因此,可得
\[v=\frac{\mathcal{E}-I(R+r)}{B\ell }=\frac{1.5-1\x(0.1+0.5)}{0.5\x 0.5}=3.6\ms\]
\item 这时电源消耗的电能一部分转化为电路中的内能,使
电阻发热。另一部分通过导体克服摩擦力作功,转化为内能。
注意,这时导体$AB$达到速度稳定,导体的动能无变化。

电源消耗的功率$P=I\mathcal{E}=1\x1.5=1.5{\rm W}$

导体AB克服摩擦力作功而消耗的功率$P'=f\cdot v=0.25\x3.6=0.9{\rm W}$

电路中电阻上消耗的功率为$P=I^2(R+r)=1^2\x(0.1+0.5)=0.6{\rm W}$.根据计算可知,能的转化符合守恒定律.
\end{enumerate}
\end{solution}


\end{enumerate}


\section{参考资料}
\subsection{由洛仑兹力所引起的电磁感应现象}

按照磁通量变化的原因不同,可将电磁感应现象分为两
种情况具体讨论。一种是在稳定磁场中运动着的导体内产生
的电动势。另一种是导体不动,因磁场变化而产生的电动势。
为了区别这两种情况,可把前者叫动生电动势,后者叫感生电
动势。引起这两种电动势的非静电力的来源是不同的,动生电
动势是由洛仑兹力所引起的。而感生电动势是由感应电场力
所引起的。实验证明,这两种电动势的大小和方向均用磁通量
的观点来统一描述,这就是法拉第电磁感应定律和楞次定律。

动生电动势的决定式,可以从洛仑兹力来推导。一般情
况下,当导线$L$在磁场$B$中以速度$V$运动时,导线中单位正电
荷在洛仑兹力($\vec{v}\x \vec{B}$)作用下沿导线移动所做的功等于动生电动势$\mathcal{E}$, 即
\[\mathcal{E}=\int_L \left(\vec{v}\x \vec{B}\right)\cdot \dd \vec{L}\]

我们以一特殊情况来讨论,设$B$为一匀强磁场,$L$为一
段直导线,而且$L$上各点$v$都相同,上式积分运算变为
\[\mathcal{E}=\left(\vec{v}\x \vec{B}\right)\cdot \vec{L}\]
\begin{figure}[htp]
    \centering
\includegraphics[scale=.6]{fig/2-23.png}
    \caption{}
\end{figure}

如图2.23所示,我们把$V$和$B$分别沿着平行和垂直于导
线$L$的方向进行正交分解可得
\[\vec{v}=\vec{v}_L+\vec{v}_{\bot},\qquad \vec{B}=\vec{B}_L+\vec{B}_{\bot}\]
这样,$\vec{v}\x \vec{B}$可以分解为以下四个分量:
\[\vec{v}_L\x \vec{B}_L,\quad \vec{v}_L\x \vec{B}_{\bot},\quad \vec{v}_{\bot}\x \vec{B}_L,\quad \vec{v}_{\bot}\x \vec{B}_{\bot}\]
它们的每一个决定一个对单位正电荷的洛
仑兹力的分量,$\vec{v}_L$与$\vec{B}_L$之间夹角为零,它们所引起的洛仑兹
力分量为零。$\vec{v}_L$与$\vec{B}_{\bot}$之间的夹角和$\vec{v}_{\bot}$与$\vec{B}_L$之间的夹角均为$90^{\circ}$, 它们引起的两个洛仑兹力的分量均垂直于导线$L$,对
电荷不做功.$\vec{v}_{\bot}$与$\vec{B}_{\bot}$之间的夹角为$\theta$, 所引起的洛仑兹力
的分量平行于导线$L$, 对电荷做功,只有洛仑兹力的这个分
量才是产生电动势的非静电力,其大小为
\[\left|\vec{v}_{\bot}\x \vec{B}_{\bot}\right|=v_{\bot}\cdot B_{\bot}\sin\theta\]
因此,
\[\mathcal{E}=\left(\vec{v}\x \vec{B}\right)\cdot \vec{L}=v_{\bot}\cdot B_{\bot}\cdot L\sin\theta\]

在一般情况下,$\vec{v}$、$\vec{B}$与$\vec{L}$的夹角分别为$\alpha$和$\beta$, 如图2.23
所示,则有
\[\mathcal{E}=vBL\sin\alpha\cdot \sin\beta\cdot \sin\theta\]

当导线方向垂直于磁场方向,速度方向又跟导线垂直,即$L$垂直
于$B$和$v$时,可得
\[\mathcal{E}=vBL \sin\theta\]
这时$\theta$等于$B$与$v$之间的夹角(图2.24)。这就是课本
上的公式2.4。

\begin{figure}[htp]\centering
    \begin{minipage}[t]{0.48\textwidth}
    \centering
\begin{tikzpicture}[>=latex]
\foreach \x in {.5,-.5,-1.5,1.5}
{
    \draw[->](-2,\x)--(3,\x);
}
\node at (3.2,0) {$B$};
\draw[<->](1.2,-1.2)node[right]{$v$}--(0,0)--(1.2,0)node[right]{$B$};
\draw(.5,0) arc (0:-45:.5)node[right]{$\theta$};
\draw (0,0)[fill=white] circle(1.5pt);

\end{tikzpicture}
    \caption{}
    \end{minipage}
    \begin{minipage}[t]{0.48\textwidth}
    \centering
\begin{tikzpicture}[>=latex, scale=.8]
    \foreach \x in {-1.8,-.6,.6,1.8}
\foreach \y in {1,2,...,6}
{
    \node at (\x,\y){$\times$};
}
\draw (-.15,.5) rectangle (.15,5.5);
\node at (0,.5)[below]{$C$};
\node at (0,5.5)[above]{$D$};

\draw[<->](-1.5,3.5)node[left]{$f'$}--(1.5,3.5)node[right]{$V$};
\draw[<->](1.5,2.6345)node[right]{$v$}--(0,3.5)--(-1.5,0.902)node[below]{$F$};

\draw[dashed](-1.5,3.5)--(-1.5,0.902)--(0,0.902);
\draw[dashed](1.5,3.5)--(1.5,2.6345)--(0,2.6345);
\draw[->, thick](0,3.5)--(0,2.6345)node[right]{$u$};
\draw[->](0,3.5)--(0,0.902)node[right]{$f$};


\end{tikzpicture}

    \caption{}
    \end{minipage}
    \end{figure}

\subsection{产生动生电动势的能量来源}

在导体切割磁力线时,动生电动势只可能存在于运动的
这一段导体上。而不动的那一段导体上没有电动势,它只是
提供电流的通路。如果仅有一段导体在磁场中运动,而没有
回路,在这一段导线上虽然没有感生电流,但仍然可能有动生
电动势。在前一篇资料中讲过,动生电动势是由洛仑兹力引
起的。


我们知道,洛仑兹力的方向总是跟电荷的速度方向垂直
的,也就是跟电荷的运动方向垂直,所以洛仑兹力永远不对
电荷做功。而这里又说动生电动势是由洛仑兹力引起的,两者
是否矛盾?其实这并不矛盾。我们来全面考虑一下运动导体
中电子的运动。在运动导体中的电子,不但具有导体本身的
速度$V$, 而且还有相对导体的定向速度$u$, 它们的合速度为
$v$, 如图2.25所示,正是由于电子的后一种运动构成了感生
电流。电子所受的总的洛仑兹力为
\[F=evB\]
它与合速度$v$垂直,总的洛仑兹力不对电子做功。$F$的一个
分量是
\[f=eVB\]
它对电子做功,形成动生电动势;另一个分量是
\[f'=euB\]
它的方向与$V$方向相反,是阻碍导体运动的,从而做负功。可
以证明,两个分量所做的功的代数和等于零。因此,洛仑兹
力的作用并不提供能量,只是传递能量,即外力克服洛仑兹力
的一个分量$f'$所做的功通过另一个分量$f$转化为感生电流
的能量。

\subsection{公式$\mathcal{E}=B\ell v\sin\theta$ 和$\mathcal{E}=\Delta\phi/\Delta t$的关系}
课本中公式
\begin{equation}
    \mathcal{E}=B\ell v\sin\theta
\end{equation}
是作为一个特例从公式
\begin{equation}
    \mathcal{E}=\frac{\Delta\phi}{\Delta t}
\end{equation}
中推导而来的.这两个公式既有统一的一面,又
有不同的特点,因此从对比中来分析两个公式不同的方面,有
利于加深对公式的理解和正确地使用这两个公式。

从研究对象来看:公式(2.1)的研究对象是一段直导
线。$\mathcal{E}$是这段直导线在磁场中运动时产生的电动势的大小,
$\mathcal{E}$是属于这段直导线的.公式(2.2)的研究对象是一个闭合回路,
$\mathcal{E}$是通过回路的磁通量变化时在回路中产生的电动势,一般
说来$\mathcal{E}$属于整个闭合回路,如果没有附加条件,则不能求出回
路各段的电动势。

从物理内容来看,公式(2.1)是研究导线在磁场中运动
时产生的电动势,是动生电动势,它的产生是由洛仑兹力引起
的.公式(2.2)是研究通过闭合回路磁通量变化时产生的电动
势,引起磁通量变化有多种多样的原因,归纳起来不外有下述
三种类型:
\begin{enumerate}
\item 回路在$B$一定的匀强磁场中运动(包括回路的
平动,转动及回路变形,面积变化等),这样产生的是动生电动
势。
\item 回路不运动但磁场随时间变化,这样产生的电动势是
感生电动势。其本质是感应电场力充当非静电力。
\item 回路在
磁场中运动,同时磁场也随着时间变化。这时产生的电动势
是动生电动势与感生电动势之和.
\end{enumerate}


\begin{figure}[htp]
    \centering
\begin{tikzpicture}[>=latex]
\draw(6,3)--(0,3)node[left]{$b$}--(0,0)node[left]{$a$}--(6,0);
\draw[<->](1,0)--node[fill=white]{$L$}(1,3);
\draw (3,0)node[below]{$d$} rectangle (3.1,3)node[above]{$c$};
\draw[->](3.1,1.5)--(3.7,1.5)node[above]{$v$};
\draw[dashed] (5,0)node[below]{$v\Delta t$}  rectangle (5.1,3);
\node at (6,1.5){$B$};
\foreach \x in {3,4.5,6}
\foreach \y in {.5,1.5,2.5}
{
    \node at (\x-.5,\y){$\times$};
}



\end{tikzpicture}
    \caption{}
\end{figure}

如图2.26所示,杆$cd$在
导轨上匀速运动,$L$、$v$均已知。磁感应强度$B$随时间均匀增
大,即$B=kt$. $t$时刻回路面积为$S$. 则在$t$到$t+\Delta t$这段时间
内磁通量的变化 
\[\Delta\phi =k(t+\Delta t)(S+Lv\Delta t)-ktS=ktLv\Delta t
+kS\Delta t+kLv(\Delta t)^2\]
则电动势为:
\[\mathcal{E}=\frac{\Delta\phi}{\Delta t}=ktLv
+kS+kLv\Delta t\]
在$\Delta t$很小时,可略去$kLv\Delta t$这一项。$ktLv$就是动生电
动势$\mathcal{E}_{\text{动}}$,$kS$就是感生电动势$\mathcal{E}_{\text{感}}$. 如果杆不动,则$\mathcal{E}_{\text{动}}=0$, 此
时$\mathcal{E}=\mathcal{E}_{\text{感}}=kS$。

从适用范围来看:公式(2.1)只适于求动生电动势,不
能求感生电动势.公式(2.2)既可求动生电动势,又可求感生电
动势,其适用范围较广.从公式(2.2)可推导公式(2.1), 反之也可
以从公式(2.1)推导公式(2.2), 这说明了两个公式的统一性.但
这种推导是有条件的,也就是必须保持磁场不随时间变化。一
般说来,公式(2.2)是由实验得到的,不能由公式(2.1)导出.课本
中强调公式(2.2)是有道理的.

\subsection{接通和断开电路的暂态电流}
为了理解课本中图2.25和图2.26的演示现象,必须
研究由自感现象引起的、不能忽略的接通和断开电路时的暂
态过程。

\begin{figure}[htp]\centering
    \begin{minipage}[t]{0.48\textwidth}
    \centering
    \includegraphics[scale=.6]{fig/2-27.png}
    \caption{}
    \end{minipage}
    \begin{minipage}[t]{0.48\textwidth}
    \centering
    \includegraphics[scale=.6]{fig/2-28.png}
    \caption{}
    \end{minipage}
    \end{figure}

设一回路如图2.27所示,回路中包含电动势为$\mathcal{E}$的电
源、电阻$R$、自感为$L$的线圈.将$K$合到1时,即将电源突然引
入$RL$电路,电路可简化为图2.28所示,在这个电路中,变
化的电流通过线图$L$时产生的自感电动势为
\[\mathcal{E}_L=-L\frac{\dd i}{\dd t}\]
因此,在任何时刻,电路中的总电动势为
\[\mathcal{E}+\mathcal{E}_L=\mathcal{E}-L\frac{\dd i}{\dd t}\]
根据欧姆定律可得
\[\mathcal{E}-L\frac{\dd i}{\dd t}=iR\]
即
\[L\frac{\dd i}{\dd t}+iR=\mathcal{E}\]
这个方程的解为
\[i= \frac{\mathcal{E}}{R}\left[1-\exp\left(-\frac{Rt}{L}\right)\right] \]

由此式可知,只有电路接通足够时间,在电路中才能建
立起稳定电流,其值为$i_0=\mathcal{E}/{R}$. 
当电路接通时,电路中的电流
$i$只能是逐渐地增加到$i_0$,而且
$L/R$的比值越大,增到$i_0$所需
时间就越长.因此课本中图2.25的实验中,所用自感线圈的
$L$越大,灯泡$A_1$的电阻$R$越小,电流增长越慢,演示效果就
越好。

如图2.27所示.当电流达到稳定值$i_0$后,如果将$K$迅
速拨到2的位置,即将电源从电路中撤除,此时电路中$\mathcal{E}=0$, 
则电流的衰减由下述方程决定
\[L\frac{\dd i}{\dd t}+iR=0\]
这个微分方程的解为
\[i=\frac{\mathcal{E}}{R}\exp\left(-\frac{Rt}{L}\right)\]
或为
\[i=i_0 \exp\left(-\frac{Rt}{L}\right)\]

上述表明,断开电源时,暂态电流是按指数规律衰减的,
虽然电源已被切断,但电路中的电流不立刻停止。

根据暂态电流的衰减规律,我们来说明课本中图2.26的
实验。设自感线圈$L$的电阻为$R_0$, 灯$A$的电阻为$R$, 因一般
$R_0\ll R$, 因此可设$R=nR_0$. 在电流稳定时,如果通过$R$的电流
为$I_1$, 通过$L$的电流为$I_2$. 因$L$与$R$并联,则$I_2R_0=I_1R$, 可得
$I_2=nI_1$. 当电源突然断开时,原来通过$R$的$I_1$立即消失.但
是,由于线圈$L$和灯泡$A$成为一个闭合电路,从而有暂态电流
$I'$存在。显然,
\[I'=I_2\cdot \exp\left(-\frac{Rt}{L}\right)\]
即
\[I'=nI_1 \exp\left(-\frac{Rt}{L}\right)\]

由此式可知,通过灯泡$A$的电流$I'$, 它的最大值$I'_{m}=nI_1$. 
因此在电路断开后的一个暂短时间内,$I'$可以超过原来通过
它的电流$I_1$, 从而灯泡亮度增加。此后,由于暂态电流随时间
迅速衰减并趋于零,灯泡才熄灭下来。如果灯泡的电阻$R$小
于线圈的电阻$R_0$, 则$I_2<I_1$, 断开后暂态电流从$I_2$开始衰减,
就不会有大于原来电流$I_1$的时候,也就不会出现灯泡更亮的
现象。可见,要做好这一演示实验,必须使灯泡电阻$R$大于线
圈电阻$R_0$, 而且
$R/R_0$
的值越大,实验效果越好。




\subsection{法拉第}

法拉第(1791—1867),英国著名物调学家和化学家。
法拉第生于伦敦一个铁匠家庭,由于家境贫苦,12岁就
当报童,后又当书店徒工,他利用业余时间学习文化知识。
1812年,法拉第听了大化学家戴维的讲演以后,产生了参加科
学工作的愿望。第二年在戴维的帮助下,进入皇家学院实验
室,作戴维的助手.1813年随戴维出访和讲学,受到了很好
的锻炼和提高.1825年任英国皇家学院实验室主任,1824年
被选为伦敦皇家学会会员,他还是法国科学院院士,1846年
他荣获伦福德奖章和皇家勋章。

法拉第在物理学方面的主要贡献是对电磁学进行了比较
系统的实验研究。他发现了电磁感应现象,总结出了电磁感
应定律,他发明了电动机和发电机,发现了电解定律,建立了
电场、磁场、力线等重要概念。

在他写成的《电学实验研究》的著作中,收集了3362个条
目,详细地记述了他做过的实验,总结出带有规律性的成果,
这是一部珍贵的科学文献。

法拉第在化学方面也做出了很大的贡献。

法拉第是靠自学成为科学家的。他在科学的征途上走了
半个多世纪,实现了自己的献身于科学的诺言。

后人为纪念法拉第,用他的名字命名电容的单位,简
称“法”。



\chapter{交流电}\minitoc[n]
\section{教学要求}
本章讲述交流电的性质、规律以及有关交流电的实际知
识。这些知识,不仅是前面所讲的电学基础知识的具体运用,
具有综合性,而且能够广泛联系实际,有较强的实用性,基于
上述两个特点,本章知识对扩展学生的知识面,培养学生运用
知识的能力是很有好处的。

本章教材可分为四个单元。第一单元包括第一节到第三
节,讲述交流电的产生和变化规律,以及表征交流电的物理
量。第二单元包括第四节到第八节,讲述交流电路的基本知
识。第三单元包括第九节到第十二节,讲述变压器、整流、滤
波。第四单元包括第十三节到第十五节,讲述三相交流电和感
应电动机。

交流电的变化规律和表征交流电的物理量,是有关交流
电的最基本的知识,是学习本章内容的基础,因而是本章的
重点。

交流电的产生是建立在电磁感应知识基础上的。教学时
要着重讲清线圈在磁场中转动时产生的电动势和电流方向的
变化。要明确中性面的位置,知道线圈经过中性面时,电动势
和电流的方向要改变。考虑到用切割磁力线来说明线圈中产
生感生电动势和感生电流更便于理解,而且计算感生电动势
也比较方便,因此这里是用一段导体切割磁力线来说明交流
电的产生的。在教学时不要求用磁通量的变化对此作补充
说明。

用图象可以形象地表示交流电的变化规律,在教学中应
使学生熟悉图象:看到图象就能想象到交流电的变化情况;根
据正弦交流电的三要素,能写出瞬时值表达式,会画出图象;
反过来,根据图象,能确定三要素,知道相位超前或落后,写出
瞬时值表达式。

交流电的最大值和有效值的关系,教学中只要求了解,不
要求推导。交流电的相位,是描述某一时刻交流电所达到的
状态的物理量,是比较抽象、难理解的概念。教学中可与机
械振动中学过的相、相差的概念对比,并可具体分析两个装
在同一转轴上、以恒定相差在匀强磁场中转动的线圈,比较它
们的感生电动势变化规律的图象和表达式,帮助学生对相和
相差的概念获得较具体的认识。

相、相差、电感和电容对交流电相位的影响、交流电的功
率等内容,明显地表现出交流电的特点,这些内容虽然教学要
求不高,却难于为学生所接受,属于选讲内容,讲述这些内
容时,要通过演示实验加强感性认识,使学生在事实上承认交
流电的特点。

本章讲述理想变压器的作用原理,不涉及实际变压器,由
于变压器的输入电流随输出电流改变的道理在中学很难讲清
楚,教学中不宜再补充讲解这个内容。

对三相交流电,要求学生明确知道三相电的三个电动势
的相位关系,熟悉三相交流电的图象,电源的连接,课文只讲
了星形连接,而把三角形连接安排在练习中,这是因为电源的
三角形连接只用于负载相同的情况,一般人接触机会较少,教
师认为有必要也可以在课堂上讲一讲。关于电源的连接,主要
讲述线电压和相电压的关系。关于负载的连接,主要讲述线
电流和相电流的关系。

感应电动机一节,教材中讲述了三相交流电产生旋转磁
场的原理。这个问题讲起来要占用较多的时间,教师可根据
实际情况灵活处理,时间紧可以不讲,相应的练习也可以
不做。

为使学生对交流电的波形、整流、滤波有深刻的印象,同
时练习示波器的使用,教材安排了两个学生实验:“用示波器
观察交流电的波形”、“用示波器观察交流电的整流和滤波”。
如果没有条件做分组实验,希望教师做好演示让学生观察,使
学生看清波形,知道如何使用示波器。

这一章知识涉及的面较宽,但大多数知识都不作深入、细
致的讨论。教学时应根据学生情况掌握教学要求的分寸。在
介绍实用知识时,要着重讲清它的基本原理,不要过细地讲技
术问题。

本章的教学要求如下:
\begin{enumerate}
\item 了解交流电产生的原理,掌握正弦交流电的变化规
律。理解交流电的瞬时值、最大值、有效值、周期和频率等概
念,理解相和相差的概念。
\item 了解纯电阻电路、纯电感电路、纯电容电路中电流与
电压的关系。了解感抗和容抗。
\item 理解变压器的原理,掌握理想变压器的电压、电流公
式,了解电能输送的原理。了解交流电的整流和滤波原理,知
道常见的整流电路。
\item 了解三相交流电的产生和电路连接,了解感应电动机
的工作原理。
\end{enumerate}

\section{教学建议}
\subsection{交流电的基本知识}
这一单元讲述交流电的产生。交流电的变化规律和表征
交流电的物理量,是这一章的核心知识,除有效值概念外,其
他内容都是前面所学知识的综合运用。这一单元实际上是围
绕交流电变化规律而展开的。

\subsubsection{交流电的产生}

初中已学过交流电的产生,这节教材
带有复习性质。在教学中应注意这样几个问题:

着重从分析线圈的每边切割磁力线的情况出发,先
弄清哪些边产生感生电动势,其方向怎样;然后弄清在旋转一
周过程中,感生电动势在哪些位置为零,在哪些位置最大,在
何处改变方向;最后讨论交流电的产生过程,使学生获得清
晰、完整的物理图景。在这里可以配合演示实验,但作为高中
学生,在已学完前面几章电学知识之后,应当注意训练他们的
分析推理与想象能力。

本节教学可以提出“闭合线框中产生的感生电动势
大小与哪些因素有关”的问题,让学生定性分析,以激发学生
思考和进一步讨论的兴趣。

这一节提到的实际发电机,是介绍性的,可以通过出
示挂图、照片或结合模型或小型发电机实物进行介绍,不要求
太具体、细致。

这一节教材及下一节教材在编写上没有采取从磁通
变化角度分析交流电产生的方法,这是因为对于切割情况产
生感生电动势,用公式$\mathcal{E}=B\ell v\sin\theta$讨论电动势比较方便。教
学中不要求用$\mathcal{E}=\Delta\phi/\Delta t$来讨论.

\subsubsection{交流电的变化规律}

这是本章也是本单元的重点教
学内容。这一节教材是按以下三个层次展开的,第一个层次
是从$\phi_0=0$的简单情况出发进行定量讨论;同时引出交流电
的瞬时值和最大值的概念.第二个层次讨论$\phi_0\ne 0$的情况,
得出交流电的一般表达式,并引出正弦交流电的概念。第三
个层次是从正弦交流电一般表达式出发,绘出正弦交流电图
象的表达方法,这一节所用知识都是学生学过的物理与数学
知识,教学中要注意复习有关的数学知识,使学生能够较好地
把所学的数学知识迁移到物理中来。

在讲解交流电图象这部分内容时,主要难点在于学生不
能正确地根据图象写出表达式或根据表达式画出图象。虽然
学生在数学课中学习过正弦函数图象,但用来理解正弦交流
电的变化,还需要着重讲述图象的物理意义。

\subsubsection{表征交流电的物理量}

教学要着重通过对交流电表
达式的分析,并可与简谐振动作适当的类比来进行。这一节
的重点内容是相与相差的概念。

有效值概念是建立在交流电与直流电(稳恒电流)在
通过相同电阻时产生的热效应相当的基础上的,虽然有效值
与最大值的关系并不要求在教学上加以推导,但要求学生能
熟练换算。应使学生知道交流电表(电压表与电流表)测定的
都是交流电的有效值。

相和相差的概念是这一节教学的重点与难点.为了
使学生能够较好理解相和相差的概念,可以通过分析两个相
同矩形线框(二线平面成一角度)绕同一转轴在匀强磁场中
以相同角速度一起旋转的实例,让学生讨论两个线框中感生
电动势的变化规律有何异同,并写出表达式,画出图象。在讨
论中,学生容易得出它们的差异:在表达式中初相不同,同一
时刻即时值不同,到达最大值时间有先后。而最大值、周期频
率都相同。这样教师就容易引导学生认识,由$\omega t+\phi_0$决定某
一时刻$t$交流电的即时值,从上例来说,也就是反映了导线切
割磁力线的角度不同。进而指出$\omega t+\phi_0$这个相当于角度的
量称为相。怎样定量反应它们到达最大值先后的差异呢?对于
上述情况来说,当一个线框中的交流电到达最大值时,另一个
还需转过一个角度$\phi=\phi_2-\phi_1$才能到达最大值,从而使学生
领会相差实质上反映两个交流电在变化过程中的步调存在时
差。在此基础上再进一步给出一般定义,以及超前或滞后的
概念。教学时应指出只有同频率的交流电才能比较变化过程
中的相差,对不同频率的交流电是无法比较相差的。


\subsection{交流电路的基本知识}
这一单元主要是从电路角度说明交流电与稳恒电流的区
别,使学生了解这些区别,是本单元教学的重要目的。为学
生知识基础所限,这一单元主要是通过演示实验、定性分析介
绍交流电路的基本知识,通过纯电感、纯电容电路建立容抗和
感抗概念,了解容抗和感抗对交流电相位的影响,了解交流电
的功率概念。这一单元重点是建立容抗和感抗概念。

\subsubsection{纯电阻电路}


 这一节从知识上讲没有什么新内容,但
课本上安排了三个演示实验,目的是为后面研究纯电感电路
和纯电容电路提供了研究方法,因而教学时应予以注意。第
一个演示实验提出了通过实验研究电路中电压和电流有效值
之间关系的方法,第二个演示实验提出了通过实验研究电路
中电压和电流相位关系的方法。第三个实验则提出了研究交
流电功率与电压、电流有效值关系的实验方法。同时,这三个
实验也给学生提供了与后类似实验进行比较的感性知识。

\subsubsection{纯电感电路与纯电容电路}

这两节内容是本单元教
学的重点。

做好演示实验是上好这两节课的重要保证.但教师
应通过这两节课,重点使学生理解感抗$X_L$与频率$f$、自
感系数$L$成正比及容抗$X_C$与频率$f$、电容$C$成反比的物理
原因,防学生不加理解地死记两个关系式。教学中需要提
醒学生,必须将频率、自感系数及电容等单位化为国际单位
制主单位才能代入公式计算。

在解释电感对交流电产生的阻碍作用以及感抗和自
感系数的关系时,要着重从电感的物理特性即由于电磁感应
而产生自感电动势阻碍电流变化这一点说明,不宜进一步讨
论自感电动势的方向和大小问题。

电感和电容对交流电相位的影响这一节是选学内
容,教材只要求通过演示实验使学生认识在纯电感电路中电
流落后电压$\pi/2$,在纯电容电路中电流超前电压$\pi/2$这两个
结论,并使学生懂得在一段含电阻、电容与电感的交流电路
中,不能简单套用直流电路的总电压、总电流等公式。

为了使学生对交流电路不能简单使用直流电路的有
关总电压、总电流、总电阻公式有一个深刻印象,可以增加一
个演示实验:实际测量一个电感(或电容)与电阻的串联电路
上各部分的电压,通过实测,可明显看到$U_{\text{总}}<U_L+U_R$. 这说
明交流电路上,串联各部分电压之和并不一定与总电压相同。
这个实验可用一个40W日光灯镇流器与一个40W白炽灯串
联后接入220V市电后进行.

“交流电的功率”这一节教材也是选学内容.教材通
过演示实验建立有关交流电功率的几个概念,并不要求说明
道理,主要是使学生对交流电功率的有关概念有一个初步了
解。对电路中电压与电流有效值乘积$UI$大于电路实际功率
$P$的现象,教学中不要求加以解释。

在进行本单元知识小结时,可以将三种纯电路从对
电流阻碍作用、欧姆定律表达式、电流电压相差、有功功率等
方面列表比较。学生情况较好的,可以由学生自己完成。


\subsection{变压器及交流电的整流与滤波}
本单元的重点是关于变压器的基本知识,在这一单元教
学中仍需十分注意研究问题的方法。这一单元相当多的内容
是属于学过知识的综合运用,因此有的问题要尽可能由学生
讨论分析解决,也可以在学生自己阅读的基础上,教师抓住主
要问题讲解并配合进行一些演示实验。

\subsubsection{变压器}
由于现行教材初中部分不讲变压器的知识,
因而变压器的教学显得更加重要了。在进行本节教学时,应
注意这样几个问题:

变压器的工作原理是建立在互感现象基础上的.可
以按互感现象、变压器结构(主要是铁心作用)、定性分析工作
原理这样的思路进行教学:先画出如图3.1甲所示的两个线
圈,根据电磁感应现象介绍互感现象,然后在图中补画入铁心
(图3.1乙).说明铁心的增强磁通和使磁路闭合两个作用.
线圈中加入铁心后,在相同电流情况下磁感应强度可比无铁
心时大几千倍,然后讨论原线圈中通入交流电时的情况,最
后应出示一些实际使用的变压器,包括心式(教学用演示万用
变压器口字形铁心)和壳式(学生实验用的小变压器,山字形
铁心),以加强学生的感性认识。
\begin{figure}[htp]
    \centering
\includegraphics[scale=.7]{fig/3-1.png}
    \caption{}
\end{figure}

在讨论变压器变压比公式时,教师应注意,教材中并
不强调严格的推导,在进行教学时,也不宜过分深究推导的严
密性。对于“在原线圈中,感生电动势$\mathcal{E}_1$起着阻碍作用,跟加
在原线圈两端电压$U_1$作用相反,是反电动势.原线圈的电阻
很小,如果略去不计,则有$U_1=\mathcal{E}_1$.”这一段话,可以类比地举
一个直流电动机接入直流电时,电源端电压与电动机电动
势似相等的例子来说明,不必展开讨论。如果在前面没有
选讲反电动势这一节,可以通过一个直流电源向蓄电池充电
的例子简要说明一下反电动势的概念。

为了加深学生对变压比公式的理解,在课堂上可以增加
一个演示实验:利用一个教学演示用大型可拆式变压器(告诉
学生初级匝数),并用大型示教交流电压表测量初级电压。用
一根长两米左右的普通塑料电线,当场在铁心另一侧绕上二
十余匝作副线圈,闭合好铁心,并另用一大型示教交流电压表
与副线圈塑料电线两端相接,接入电源,由学生直接观察原副
线圈电压比与匝数比。然后再打开铁心,继续增加匝数进行
实验。做好这个实验有很多好处,很重要的一点是学生现场看
到了变压器的线圈绕制,大大加深了对变压器的实际印象,对
匝数比有了感性认识。

在讨论课本上电流比公式时,特别需要指出的是应
从一般能量守恒出发,首先使学生明确,副线圈向负载提供的
电能是由原线圈吸收电源能量后通过电磁感应现象转化而来
的,如果不计转化过程中各项能量损失,应当有$U_1I_1=U_2I_2$.

在介绍几种常用变压
器时,应出示实物,特别是自耦
变压器的实物。而且对该变压
器的滑动端与线圈是怎样接触
的,应向学生介绍。将课本
上图3.27调压变压器的示意
图改为图3.2较好,因为实际调压器都兼有升降压作用。
\begin{figure}[htp]
    \centering
\includegraphics[scale=.7]{fig/3-2.png}
    \caption{}
\end{figure}

\subsubsection{电能的输送}

从知识讲,这一节是学过知识的综合运
用。在进行本节教学时,主要地是应引导学生自己阅读与讨
论,注意训练学生论述两种方案选择的理由。应使学生抓住
技术理由与经济理由这两个方面考虑实际问题。在本节教学
中,有些学生搞不清输送功率的含义,在求通过导线上电流
时,往往将输送电压除以导线自身电阻当作通过输电导线上
的电流。教在讲解时,应着重从建立物理模型和研究对象
的角度进行说明,输电过程的最基本模型是由输送端(发电机
等)、输电线及用户组成,如图3.3所示.从图中可以清楚地
看出:
\[\begin{split}
   \text{输送端电压}&=\text{导线损失电压}+\text{用户实际电压}\\
   \text{输送功率}&=\text{导线损失功率}+\text{用户实际获得功率} 
\end{split}\]
换言之,输送电压
与输送功率是以由导线、用户组成串联电路为研究对象的,而
导线损失功率、导线损失电压是以导线为研究对象的,只有输
送电流由于是串联电路,各部分都相同。造成前述错误的原
因在于运用公式时,代入的各量并非是同一研究对象上相应
的物理量。
\begin{figure}[htp]
    \centering
\includegraphics[scale=.7]{fig/3-3.png}
    \caption{}
\end{figure}


\subsubsection{交流电的整流与滤波}

这两节主要是介绍将交流电
转变为直流电的一种方法。在讲述整流电路和滤波电路工作
原理时,应充分运用学生已学过的知识,着重从分析方法上进
行教学,同时尽量做好演示实验。

在讲述二极管整流电路时,注意使学生掌握以下两
点分析方法:
\begin{enumerate}
\item 将二极管与负载电阻$R$串联,二极管处于正
向导通时,可将二极管看作阻值远小于$R$的极小电阻,根据
串联电路特点,电压主要分配在电阻$R$上,此时电流$I$可由
电源电压和电阻$R$决定。当二极管反向截止时,可将二极管
看作阻值远大于$R$的极大电阻,电压主要分配在二极管上,
此时电流由电源电压与二极管反向电阻决定,显然几乎为零。
\item 判断二极管是正向导通还是反向截止要从二极管两极电势
高低的比较进行判断,只有正极电势高于负极电势,极管才
能导通。掌握了以上两点,几种整流电路的原理就容易理解
了。由于桥式整流电路不大好画,学生也不易弄清四个二极
管的接法,教师画时可以告诉学生简单的记忆方法:正负相连
成两串(课本图3.35中$D_1$、$D_4$与$D_2$、$D_3$),正正相连只能进
($D_3$、$D_4$正极),负负相连只能出($D_1$、$D_2$负极),又进又出
($D_1D_4$, $D_2D_3$正负相连处)接交流。
\end{enumerate}


在讲述滤波电路时,课本为了便于和后面检波电路
衔接,首先讲述了脉动直流电可以分解为直流成分和若干交
流成分。几种滤波电路的原理介绍主要是从电路分析角度进
行解释的,教师在讲解时,可以在分析前或分析后简要指出
电容与负载并联、电感与负载串联所起滤波作用的物理实质
仍是电容的充、放电现象与线圈自感阻碍电流变化现象。

值得注意的是,在讲述整流电路时,由于电路比较复杂,
有些学生始终弄不清电路究竟在讨论什么,因此教师在一开
始讲半波整流到最后讲完整流滤波电路时,必须明确指出,整
流器是一种换能电路,输入交流电,输出直流电。在每讲一种
电路时,要先交待哪是交流电输入端,哪是直流电输出端,输
出端正负极怎样,需要直流电的负载接在什么地方,最好每画
一个图就用虚线框把这些标出来,如图3.4所示,使学生获得
清晰的印象。
\begin{figure}[htp]
    \centering
\includegraphics[scale=.7]{fig/3-4.png}
    \caption{}
\end{figure}

\subsection{三相交流电和感应电动机}
这一单元的内容是常识性的。因此,在教学时,尽可能处
理得简捷一些。抓住主要教学要求,避免过多展开,另一方
面要尽可能作好演示实验,以加深学生对某些结论的理解。

\subsubsection{三相交流电}

这一节主要是使学生了解三相交流电
的产生,通过三相交流电的表达式和图象明确三相交流电的
相差,了解三相四线制送电的道理。对于三相交流电的产生、
表达式与图象,学生根据前面知识可以自己分析得出,有条件
的学校可以演示一下三相交流电各相的波形。对于三相交流
电,也可以通过实验使学生有个感性认识,将三相交流发电机
的三相绕组分别与三只电流计连接起来,缓慢摇动发电机,通
过观察现象可以得出:
\begin{enumerate}
  \item 三相交流电最大值相同(电流计指
针最大偏角相同);
\item 三相交流电周期相同(指针摆动频率档
同);
\item 三相交流电的位相不同(各电流计指针偏转到最大角
度的时刻不同)。
\end{enumerate}
如果用示波器演示三相交流电各相的波形,
可以看出三相之间的相差。对于三相四线制供电,最好能通
过观察实际输电线路,以使学生尽可能结合实际了解如火线、
地线与负载的接法这样一些密切结合实际的知识。

\subsubsection{三相电路的连接}

这一节的教学,主要应使学生明确
以下几点:
\begin{enumerate}
\item 电源星形接法时,相电压与线电压的概念及两
者关系.    \item 负载星形接法时,加在负载上的电压为相电压。
\item 负载星形接法时只有三相负载相同,才能省略中性线。
\item 负载三角形接法时加在负载两端电压为线电压,此时相
电流与线电流不等。
\end{enumerate}

在讲解电源星形接法时,最好演示一下实际供电电路相
电压与线电压的测量,既可弄清关系,也使学生获得实际知
识。在教学中还应向学生指出,无论负载采取什么接法,每个
负载上的电流均可根据前面讲过的知识求出,当然只能求纯
电阻、纯电容或纯电感电路中的电流。

\subsubsection{感应电动机}

这一节教材主要研究两个问题:
\begin{enumerate}
\item 在
旋转磁场作用下,闭合导体环跟随转动这一感应电动机的原
理。这一点,教师可以在演示实验基础上提出几个问题,由学
生讨论解决。在讨论中应使学生明确,感生电流产生原因、方
向判断、受力判断、转动方向及为什么存在转速差异等问题。
至于转动平衡与平衡破坏时产生的现象不宜在课堂上讨论,
可作为课外活动的讨论内容在课外进行.
\item 利用三相交流
电产生旋转磁场的原理。这个道理分析起来并不复杂,只是较烦,关键是弄清线圈头尾和电流正负流向在图中的表达。
至于实际感应电动机结构,出示一下实物或图片作一扼要介
绍就可以了。
\end{enumerate}

\section{实验指导}
\subsection{演示实验}
\subsubsection{线圈在磁场中旋转产生交流电}

这个实验可以用J2416型电机原理说明器配合大型
演示用灵敏电流计进行演示.若使用J0401型大型演示电
表,则应先将选择扭扳至“G”挡,并将指针调至中央位置。由
于电机说明器磁极由励磁线圈通电产生,因此需6V蓄电池
或低压直流电源(J1201型)供电,在实验时还需先将电刷位
置调整到集电环整环部分,并保持良好接触如图3.5所示.演
示时,要使转子线圈均匀地旋转,尽量使它的转动周期接近电
表指针系统的自由振动周期。这时可以清楚看到,电表指针在中央零点左右摆动,说明圈中的感生电流是交流电。演示
中,应注意使学生观察线圈在磁场中的位置与交流电瞬时值
的关系,注意电流最大值时、电流换向时线圈平面所处位置。
进一步提高转速可发现周期变短;再继续增大转速,指针将几
乎不动了。从而使学生了解不能用磁电式电表直接测量交流
电,也为后面讲简单收音机时,未经检波的高频调幅电流不能
使耳机发声的原因作好感性知识的准备。
\begin{figure}[htp]
    \centering
\includegraphics[scale=.7]{fig/3-5.png}
    \caption{}
\end{figure}

这个实验也可利用自制仪器代替。仪器制作方法与
第一章实验指导中演示电流计工作原理中介绍方法相同,只
是不需将线头留长绕制游丝及安装指针了,将线头漆皮除去
后直接在转轴上紧绕4—5匝固定即可,转轴两端及其支架就
是电源两端,实验时,用导线将这两端与大型灵敏电流计相
连,并用万用支架附件固定卡将底座与桌子紧固,以防止整个
装置滑动。为了使线圈旋转均匀稳定,可以用一段粗棉线在
轴上绕两匝,两手拉紧两端,缓慢均匀地向一个方向移动,线
圈就会均匀地旋转起来(图3.6)。

\begin{figure}[htp]
    \centering
\includegraphics[scale=.7]{fig/3-6.png}
    \caption{}
\end{figure}

\subsubsection{用示波器观察交流电的波形}
配合正弦交流电图象的教学,可利用示波器观察照明电
路的交流电波形.实验时,将低压交流电源(J1201型)输出
2—4V的交流低压,接至示波器Y输入端.示波器通电预热
后调好亮度与聚焦,将Y衰减拨至100(或10)挡,Y增益调
节至竖直亮线范围在屏上坐标格之内,扫描频率拨至10—
100Hz挡,X增益调节至亮线宽度也在屏幕方格范围之内,
调节一下频率微调即可观察到稳定的正弦交流电波形,一般
以出现两个完整波形效果较好.如果用J2458型教学示波器
时,亮度较为理想,用其他各种型号的示波器也可。实验时,
不要利用示波器观察手摇交直流发电机发出的交流电波形,
因为此种发电机由于结构原因发出的交流电并不是正弦变化
的交流电,且频率不易稳定、波形不好,效果不理想。

如果由于电网的原因,使交流电波形出现畸变,则可改用
教学信号源输出的正弦交流信号。

\subsubsection{纯电阻电路中电阻的作用}
\begin{figure}[htp]
    \centering
\includegraphics[scale=.7]{fig/3-7.png}
    \caption{}
\end{figure}

这个实验主要是为了和后面感抗、容抗相比较而进行演
示的,因此需在数据选择上作些考虑,实验电路如图3.7所
示.如果交流电压表与交流电流表都采用J0401型示教电
表,其量程应选交流25伏与交流1安挡,电阻可选J2355型
滑线变阻器(阻值50欧,额定电流1.5安)或阻值30欧左右、额
定功率10瓦的线绕电阻,电源部分,可将J1201型低压电源
接在一自耦调压变压器输出端,通过调整低压电源输入电压
来控制低压电源交流(20伏挡)的实际输出电压,以便于读数
选择.实验时,先将变阻器阻值调整在30欧左右,调压器调
压手柄位置调整在50伏位置左右.插上调压器电源,再开启
低压电源的开关,观察伏特计示数。再调整调压器手柄位
置,使伏特计示数分别取5伏、10伏、15伏、20伏,读出相应
电流值,即可验证交流电路中电阻仍服从欧姆定律。由于电
阻取值30欧左右,其电流值变化在0.2至0.7安之间,演示效
果较好.图3.7中右边虚线框所示部分可以作一示教板,图
中输入端、伏特计符号两端、安培计符号两端及电阻符号两端
均用固定接线柱。实验时,逐一接入,以加强演示效果。


\subsubsection{电感对交流电的阻碍作用}

\begin{figure}[htp]
    \centering
\includegraphics[scale=.7]{fig/3-8.png}
    \caption{}
\end{figure}

这个演示实验的电路如图3.8所示,可以做一个示教
板.双刀双掷开关$K_1$,单刀双掷开关$K_2$, 6—8伏的小电珠
$D$, 灯座均固定在示教板上,在示教板上画出$R$、$L$位置并
在两端引出接线柱.$L$可用J2423型可拆变压器400匝绕组
(绿色线圈)。使用时,从变压器上拆下该线圈,只在其中插上
条形铁心部分,再用导线将400匝绕组(0、400两个接线柱)
接入示教板,这个绕组直流电阻约3欧.$R$可用J2354型滑
线变阻器(10欧,2安),为便于对比,实验前要将电阻调
至3欧.为加强演示直观效果,直流电源用6伏蓄电池(或四
节新的1号干电池串联),交流电源仍采用上一实验的办法,
将低压电源输入端接至调压器,低压电源交流输出端拨至8
伏挡,再利用调压器调压控制8伏挡的实际输出电压.

实验时,先将$K_2$拨至$R$端,再将$K_1$分别拨至交流与直
流电源,观察到$D$的发光相同,表明交流电压有效值与直流电
电压相同.然后将$K_1$拨至直流电源端,将$K_2$多次重复接$R$
与$L$端,发现$D$发光相同,说明线圈L在直流电路中所起
作用与$R$相同.最后将$K_1$拨至交流电源端,再多次将$K_2$分
别接于$R$与$L$端,可看到由于电感线圈对交流电的阻碍作用
而使$D$发光显著减弱,说明了电感对交流电的阻碍作用。

\subsubsection{在纯电感电路中,电流强度跟电压成正比}
\begin{figure}[htp]
    \centering
\includegraphics[scale=.7]{fig/3-9.png}
    \caption{}
\end{figure}

这个实验的演示较为困难。首先是带有铁心的电感线圈
的电感与通入的电流大小有关,因而感抗亦随电流大小而变,
无铁心线圈电感量较大时,其直流电阻又很难忽略。中学的演
示电源要用50赫兹低压电源,又要受演示电表局限.综合考
虑上述因素,实验可按下述方法进行:实验电路如图3.9所
示,其中$L$仍用上一实验中所用J2423型可拆变压器0—400
匝绕组,但应装入闭合铁心,伏特计用J0401示教电表交流
25伏挡或J0402型示教电表交流30伏挡,安培计使用J0402
型电表交流100毫安挡.电源仍采取低压电源输入电压由调
压器控制的办法来控制低压输出(仍拨在20伏挡)实际电压。
实验时可取电压值为5伏、10伏、15伏进行读数,相应电流值
约为16毫安、32毫安与48毫安左右.由此说明纯电感电路
中,电流强度与电压成正比。应向学生说明的是,此实验中的
感抗约300欧,远大于自身电阻3欧,因起主要阻碍作用的是
感抗,可将该电路看成纯电感电路。

\subsubsection{感抗与自感系数及交流电的频率有关}
实验电路可用图3.9所示电路,只是将安培计换为一个
6—8V小电珠.实验时,将输入电压调至6V, 松开铁心紧固
螺丝,将铁心慢慢拉开,可以明显看到小电珠逐渐明亮起来。
这表明由于铁心拉开,自感系数变小,感抗变小,电路中电流
变大。

演示感抗交流电频率有关,需要一个变频电源,下面介
绍演示该实验的两种方法。

\begin{figure}[htp]
    \centering
\includegraphics[scale=.7]{fig/3-10.png}
    \caption{}
\end{figure}

方法一:自制一简易换流变频电源。用一块双面铜箔板
(电路印制板),加工成直径2.5厘米左右的圆板,中央开一小
孔,孔径由玩具直流电动机的转轴直径决定(略小于轴的直
径)。铜箔板一面用喷漆(磁漆也可)画出两个半圆面,另一面
画两个同心圆环如图3.10甲、乙阴影部分所示.待漆干透,
将板浸入三氯化铁溶液中,使未涂漆部分铜箔腐蚀掉后,取出
用清水洗净。板干燥后,除去漆皮,一面相当于集流环,另一面
相当于换流半环。在$A$、$B$两点各钻一小孔,用导线将两半环
分别与另一面里外圆环焊接好。再将玩具电动机轴与圆环中
心孔套紧,将电动机固定在木底座上,并用弹性铜片(磷铜片)
制四个电刷,分别固定在底座上并与半环、整环部分接触好,
与里外整环相接触的电刷通过接线柱接6伏直流电源(取自
电池或低压电源稳压输出),与两半环接触电刷处通过接线柱
输出变频交流电,其波形为矩形波,电机部分通过变阻器
(J2354型10欧)与3伏直流电源相连,实验时,调节变阻器
阻值,即可改变电机转速,从而改变输出交流电的频率。

在演示感抗与频率关系时,将J2423型变压器100匝绕
组(闭合铁心)与6—8伏0.1安指示灯串联后接入这个变频
电源。当电机转速增加时,小电珠发光减弱,表明电流减小,
感抗增大,即频率越高,线圈感抗越大。

\begin{figure}[htp]
    \centering
\includegraphics[scale=.7]{fig/3-11.png}
    \caption{}
\end{figure}

方法二:利用有功率输出的低频信号发生器如音讯-1
甲型、XFD-7A型等作变频电源.将J2423型变压器0—100
匝绕组(闭合铁心)与6—8伏0.1安指示灯串联后接在信号
发生器功率输出端,输出电阻拨在“50欧”挡.为了使学生了
解到频率变化,可在输出端接一带有电子管收音机输出变压
器(6P1管所用)的喇叭,用音调高低说明频率高低,同时应并
联一个大型演示电压表(J0401型25伏交流挡),如图3.11
所示。演示时,先将输出微调旋至输出最小(逆时针旋到底),
频率调至50赫.开启电源后,预热两分钟,再接通K,逐渐
调节输出微调,直至灯泡发光(不必太亮)。再调节频率,使其
逐步增高,并略微调节一下输出微调以保证电压输出与原来
相同。此时,喇叭发声音调变高说明频率变高、灯泡亮度减弱
说明电流变小。电压未变,电路电流减小是由于感抗增大的
缘故,从而定性演示了感抗与频率的关系。

若无带功率输出的低频信号发生器,可以用教学用信号
源的低频信号输出接入25瓦扩音机的唱片输入端,以代替变
频电源.实验时,从扩音机16欧输出端输出交流信号.特别
要注意的是,一定要在16欧输出端接一个20欧2安的滑线
变阻器(用两个固定端)作假负载,以免造成扩音机空载而被
损坏.由于教学信号源低频信号频率在400赫以上,故电路
中电感不宜过大,否则感抗太大,电流太小,不好演示,因此
电感线圈可改用J2423型可拆变压器红色线圈的0—200匝
绕组,并且不用封闭铁心,只将条形铁心插在线圈中使用,小
灯泡仍用6—8V的,并可在电路中串入100毫安交流电流表
(J0402型).在进行实验前,应先将扩音机唱片音量控制旋
钮旋至最小音量处(逆时针旋到底),如果是电子管式的扩音
机,还应先开启扩音机预热3分钟,然后再开启信号发生器,
逐渐旋动唱片音量钮,观察输出电压升压是否正常,并观察电
路中灯泡亮度和电流计指示,实验时也应将一带输出变压器
的喇叭接在输出端作为频率变化的指示,在利用扩音机作实
验电源时,其输出电压不宜超过10伏,否则扩音机会因连续
处于满负荷状态而损坏。

\subsubsection{交流电能够通过电容器,电容对交流电的阻碍作用}
仍可做一示教电路板以加强直观性。实验电路如图3.12
所示,图中$D$仍用6—8伏灯泡,电容可用200微法、25伏
的两个电解电容负极与负极相连,两个正极引线作为无极性
电容的两端接入电路.电源与实验方法同4.对现象应说明
以下两点,一是交流电路接入了电容器,D发光表明交流电可
以“通过”电容器,而“通过”的物理实质是电容器充放电现象。
二是从$D$发光比直接接入交流电时弱,说明电容对交流电有
阻碍作用。

\begin{figure}[htp]
    \centering
\includegraphics[scale=.7]{fig/3-12.png}
    \caption{}
\end{figure}

\subsubsection{容抗跟电容和频率有关系}
这个实验仍可利用上一个实验中的示教板。将$K_1$扳向
交流6伏端,$K_2$扳向接电容端,交流6伏端接线柱接入实验6
方法一中所介绍的换流变频电源。当改变直流电机转速时,
输出频率改变,且频率越高,$D$越亮,表明容抗与频率有关,即
频率越高容抗越小,若实验时再并联一个电容(仍用两个
200微法25伏电解电容制成),则可能在频率相同时,电容越
大,灯泡越亮。定性演示了电容越大,容抗越小的关系。

如果采用音讯-1甲型等有功率输出的信号发生器,或将
教学信号源低频输入到25瓦扩音机代替信号发生器作为变
频交流电源,实验时使频率在几百至一千赫之间变化,则电容
应取小一些,可将两个20微法、25伏的电解电容两个同极相
接串联当作一无极电容使用,实验时输出电压仍以10伏以下
为宜。

\subsubsection{三种交流电路中电流跟电压的相位关系}

\begin{figure}[htp]\centering
    \begin{minipage}[t]{0.48\textwidth}
    \centering
    \includegraphics[scale=.7]{fig/3-13.png}
    \caption{}
    \end{minipage}
    \begin{minipage}[t]{0.48\textwidth}
    \centering    \includegraphics[scale=.7]{fig/3-14.png}
    \caption{}
    \end{minipage}
    \end{figure}

    \begin{figure}[htp]
        \centering
    \includegraphics[scale=.7]{fig/3-15.png}
        \caption{}
    \end{figure}

方法一:利用超低频电源与示教电表进行演示。这三个
实验的电路分别如图3.13、图3.14、图3.15所示,所用伏特
计均为J0401型演示电表直流10伏挡,纯电阻与纯电容电路
中所用微安表与纯电感电路中所用毫安表为J0401型演示
电表直流200微安挡(检流计)和100毫安挡,以上各表在使
用前均需先调整指针下方的调零旋钮,使指针位于中央零刻
度。在接入电路时必须注意电表“正”“负”接线柱接入位置正
确(图中均已标明).图3.13中电阻R可用50千欧1瓦的
碳膜电阻.图3.14中电容为5微法,可用几个耐压63伏或
160伏的无极性电容并联而成,也可将两个耐压25伏、10微
法的电解电容两个负极相接串联成无极性电容来代替。图
3.15中电感线圈可用J2423或J2425型可拆变压器的红色
线圈1400匝或1600匝绕组(闭合铁心使用).接入超低频电
源后,都需要调整超低频电源的“调零”电位器,使电表指针左
右摆动幅度相同。但纯电感电路中相差演示,电流较大,须用
后面图3.17及图3.18所示的手动超低频电源才有较好效
果。为了演示方便,可以将伏特表与微安表(毫安表)并列放
置.实验时就可看到,在图3.13所示纯电阻电路中,伏特表与
微安表指针是同时左右摆动的,表明纯电阻电路中电压与电
流同相。在演示纯电容电路和纯电感电路时,以电表指针摆动
到右端为准.在演示图3.14所示的纯电容电路时,可看到微
安表指针总是先于伏特表指针到达右端的,表明纯电容电路
的电流超前于电压,在演示图3.15所示的纯电感电路时,可看
到伏特计指针总是先于毫安表指针摆到右端,这表明纯电感
电路中的电压超前于电流。但是,在纯电感电路的演示中,由
于超低频频率太低(约0.3—0.5赫),即使电感较大,感抗也不
是很大,而且线圈电阻不能忽略。实验中,只能看到伏特表先
于毫安表指针摆到右端,不能演示出相差$90^{\circ}$, 即差1/4周期.

\begin{figure}[htp]
    \centering
\includegraphics[scale=.7]{fig/3-16.png}
    \caption{}
\end{figure}

实验所用超低频电源可使用J2464型教学信号源的超低
频输出。实验时,只需将频率旋钮拨至超低频挡开启电源开
关,从低频输出接线柱端就可获得超低频信号,如需自制,可
按图3.16所示电路制作,三极管均可使用3DK型,其中$D_1$与
$D_2$尽可能挑选相同的$\beta$值和$I_{ceo}$值。电路由$D_1$与$D_2$组成振
荡器产生三角波,$D_3$为射极跟随,$D_4$为一级电压放大,由$D_5$射
极跟随器输出信号。信号波形不是严格的正弦波。此电路一
般无需调整,因而制作与使用都较为方便。图中各元件规格
已注明,电阻可用1/4瓦或1/8瓦的,电容耐压应在16伏以
上,输出信号幅度$U_{p-p}\ge 6$伏(容载).

\begin{figure}[htp]
    \centering
\includegraphics[scale=.7]{fig/3-17.png}
    \caption{}
\end{figure}

\begin{figure}[htp]
    \centering
\includegraphics[scale=.7]{fig/3-18.png}
    \caption{}
\end{figure}

更简单的超低频电源,可以用一个学生实验用滑动变
阻器按图3.17所示连接电路,用手均匀地来回拉动变阻
器滑动端便会产生超低频信号.变阻器规格为20欧或
50欧(额定电流大于1安).或者在硬塑料管(直径3厘米
左右,长3厘米左右)上,象图3.18那样用漆包电阻丝(可从
旧变阻器上拆出后,再涂以绝缘漆使用)密绕一周,涂清漆固
定绕组,再将它固定在木座上。在穿过绕组中心孔的轴上安
一旋转滑片,滑片另一端与电阻丝接触。注意应将电阻丝与
滑片接触处的漆皮用砂纸打掉。电阻丝两端接串连电源,从电
源正负串接处和滑片轴引出输出端。实验时,只要均匀摇动
旋转滑片轴就可输出低频信号。若无硬塑料管,也可用硬纸
板卷粘成圆筒,干透后涂几层清漆代替塑料管。

方法二:用信号发生器、电子开关与示波器观察。电子开
关可将两路输入信号放大后按一定频率交替输出,送入示波
器输入端后,荧光屏上交替出现两个信号波形,当交替频率足
够高时,电子示波器光点的余辉作用和人眼的视觉暂留,使观
察者“同时”看到两个信号的波形,形成双踪示波器。实验中,
我们同时将纯电阻(或纯电容、纯电感)上的电压波形与电流
波形显示在屏上,以观察电压与电流的相差,这三个实验
的电路如图3.19、3.20、3.21所示,也应做成示教板,以加强
演示的直观性。三个图中,470欧电阻可用电位器,以便作适
当调整,这是一个电流取样电阻,由于电子开关或示波器输入
端均是高阻抗的电压输入端,因而一般利用示波器观察电流
波形时,总是在电路中串联一个小电阻(对电路总阻抗而言小
很多),利用电阻上电压与电流同相来观察电流波形.图3.19
中$R$可取10千欧碳膜电阻,图3.20中$L$可取8瓦日光灯的
镇流器,其电感约为1.2亨,也可用J2423型可拆变压器中红
色线圈0—800匝绕组(闭合铁心),图3.21中$C$取0.05微法
的纸介质电容。

\begin{figure}[htp]
    \centering
\includegraphics[scale=.7]{fig/3-19.png}
    \caption{}
\end{figure}

\begin{figure}[htp]\centering
    \begin{minipage}[t]{0.48\textwidth}
    \centering
\includegraphics[scale=.7]{fig/3-20.png}
    \caption{}
    \end{minipage}
    \begin{minipage}[t]{0.48\textwidth}
    \centering
\includegraphics[scale=.7]{fig/3-21.png}
    \caption{}
    \end{minipage}
    \end{figure}

实验时,500赫交流信号从J2464型(或其他低频信号发
生器)教学信号源低频输出端获得.电子开关可使用J2460
型教学用电子开关,示波器为J2458型教学示波器.实验接
线示意图如图3.22所示.进行实验的步骤如下:
\begin{enumerate}
    \item 开启示
波器预热,面板上各控制器应调整为:Y输入耦合开关拨至
“AC”,Y衰减钮拨至10或1,Y增益顺时针旋到底,X扫描
范围拨至“10—100Hz”.预热后调整好亮线位置与长度.
\item 开
启电子开关前将面板各钮调整如下:频率范围“25—50kHz”;
频率微调钮顺时针旋到底;A增幅与B增幅都逆时针旋到底
(增幅最小位置)。开启电子开关后,将在示波器上观察到两条
水平亮线,调节电子开关上的“相对位移”钮,使这两条亮线重
合.
\item 开启低频信号发生器,面板上频率选择拨至“500Hz”,
低频增幅钮顺时针旋到底(输出最大)。
\item 慢慢顺时针旋转电
子开关面板上的“A增幅”钮,这时示波器上将出现加在电路
两端的500Hz正弦电压波形.慢慢顺时针旋转电子开关面板
上的“B增幅”钮,这时示波器上将同时出现电路中的电流波
形(取样电阻上电压波形)。一般我们使电流波形幅度小于电
压波形,以利比较。若增幅均感不足,可将示波器Y衰减拨
至“1”.
\end{enumerate}

\begin{figure}[htp]
    \centering
\includegraphics[scale=.7]{fig/3-22.png}
    \caption{}
\end{figure}

必须注意的是,出厂日期在1982年以前的J2458型教学
示波器,X偏转板与Y偏转板都是反接的,即X轴扫描自右
向左,Y轴显示为下正上负。Y轴反接对本实验及后面三相
交流电等需比较波形相位的实验无影响,但X轴反接将使相
位比较必须从右向左看,即X轴正向向左,与习惯上我们画
图时X轴正向向右相反,因而显示观察很不方便.在1982年
以后生产的J2458型教学示波器,X轴与Y轴偏转板均已改
为正接,对本实验及后面相差演示实验就没有影响了。如果
示波器X轴是反接的,在演示前应将示波器右侧盖板和下底
盖板打开,对照示波器说明书电路图将电路板上X轴末级放
大6NI电子管两个屏极通向示波管X偏转板的两根引线对
调焊好,或直接将示波管管座上第10、11脚两根引线对调.

\begin{figure}[htp]
    \centering
\includegraphics[scale=.7]{fig/3-23.png}
    \caption{}
\end{figure}

如果想自制一个电子开关,可参阅图3.23所示电路,这
是一个效果较好且易于装制的电路,该电路由$G_1$的两个三极
管组成多谐振荡器,产生10千赫以上方波,用以轮流使$G_2$的
两个三极管导通放大,输出两路信号至示波器,图中$R_4$为“相
对位移”,$R_9$与$R_{10}$分别为“A增幅”与“B增幅”.如欲改变开
关频率,可以在$C_1$与$C_2$处加接双刀多掷开关,分别同时接入
200皮法、2000皮法、0.02微法、0.2微法电容,容量越大,开关
频率越低.电路中各电阻均应使用1/2瓦以上的碳膜电阻,
电容均应耐压400伏以上.电源可用教学用高压整流电源如
J1205型,亦可自制.

\subsubsection{交流电的有功功率与视在功率}
由于教材并不深入从理论上讨论交流电的有功功率和视
在功率,因此,需要强化这两个实验的直观感受性,用学生熟
悉的照明电路来演示。演示时,教师应充分注意安全。实验
电路分别如图3.24和图3.25所示,亦可做成示教板.
\begin{figure}[htp]
    \centering
\includegraphics[scale=.7]{fig/3-24.png}
    \caption{}
\end{figure}

\begin{figure}[htp]
    \centering
\includegraphics[scale=.7]{fig/3-25.png}
    \caption{}
\end{figure}

图中瓦特表为J0404型演示瓦特表,用250伏1安挡;伏特表为
J0401型演示电表,用交流250伏挡(亦可用J0402型);安培
表为J0402型演示电表,用交流1安挡.两图中D可选用220
伏、100瓦的白炽电灯(使用60瓦电灯也可,但瓦特计改用
250伏、0.5安挡).图中$L$可用J2423型可拆变压器红色线圈
0—800匝绕(闭合铁心).演示时,通过调压变压器控制输
入电压为200伏,以便于读数与计算.在进行实验之前,教师
应简单向学生说明瓦特计四个接线柱的用法。

\subsubsection{用并联电容器的办法来提高功率因数}
实验电路用图3.27实验电路,只是再用一个日光灯电容
器(4.75微法、40瓦日光灯用)或用几个耐压450伏以上的油浸
纸介质电容并联为5微法的电容器组接入电感元件电路两端
(注意,操作时必须切断电源),在接入后可以看到瓦特表与伏
特表读数未变,而安培表读数减小,从而表明功率因数的提
高。实验后,可以向学生说明,在机关、工厂等大量使用日光灯
的地方,按国家规定,必须接入电容以提高功率因数,日光灯
电容器就是与日光灯电路配套进行生产的。
\begin{figure}[htp]\centering
    \begin{minipage}[t]{0.48\textwidth}
    \centering
\includegraphics[scale=.7]{fig/3-26.png}
    \caption{}
    \end{minipage}
    \begin{minipage}[t]{0.48\textwidth}
    \centering
\includegraphics[scale=.7]{fig/3-27.png}
    \caption{}
    \end{minipage}
    \end{figure}

\subsubsection{整流的演示}
方法一:实验电路如图3.26所示,图中伏特表用J0401
型演示电表直流10伏挡.毫安表用J0401型演示电表直流5
毫安挡,两个电表的指针均应调为中心零位式。负载电阻$R$
由1千欧电位器与一个500欧电阻串联而成,电源采用前面
实验9中所介绍的任一种超低频信号发生器,如使用晶体管
电路的超低频电源(如J2464型教学信号源),则在演示前先
调整一下“调零”旋钮,使指针两侧摆动幅度相同,同时调整一
下$R$值,使毫安表指针摆动幅度合适。演示时,先将二极管$D$
取下,暂时用导线连接,可以观察到未整流前,两表都左右
摆动,说明输入交流电压,负载上获得的是交流电流。再将二
极管$D$接入,看到毫安表单侧摆动,且有半个周期不动,表明
负载中通过的是半波脉动直流电。而改变二极管$D$接法,又
可看到毫安表指针向另一侧单侧摆动,表示负载中通过半波
脉动电流方向与刚才相反。

全波整流的演示电路如图3.27所示,其中$R_1$与$R_2$均
为1千欧电阻.实验时,可先分别接入$D_1$与$D_2$观察两个半
波整流,然后同时接入$D_1$、$D_2$观察全波整流情况.亦可用同
样方法演示桥式整流电路。

\begin{figure}[htp]
    \centering
\includegraphics[scale=.7]{fig/3-28.png}
    \caption{}
\end{figure}

方法二:用示波器演示,此时演示的分别是整流前电压
波形与整流后负载两端电压波形.电路如图3.28所示,为
加强演示效果,$R$可用6—8V小灯泡,二极管$D$应采用2CZ
型的.整个装置分为电源部分(初级220V, 次级6V的小型
变压器)、整流部分与负载部分,可制成三块分立的、相同高度
示教板,并同时做好全波与桥式整流电路以及滤波电路的示
教板,形成一套整流滤波演示示教板组。实验前,先调整好示
波器(J2458型),Y轴输入端用一金属屏蔽线制成探针,外皮
金属导线两端分别安装夹头,为连接示波器地线端使用。中心
导线两端一端安夹头,接示波器Y输入,另一端安一万用表
用的探针或夹头。实验时,将地线夹头夹在$C$点,探针分别与
$A$点、$B$点相接,就可对比地看到未整流的波形与整流后的波
形.全波与桥式整流电路示教板如图3.29甲、乙所示,演示
方法同上,但可分别断开其中任一二极管观察对比全波与半
波的区别。

\begin{figure}[htp]
    \centering
\includegraphics[scale=.7]{fig/3-29.png}
    \caption{}
\end{figure}

\subsubsection{滤波的演示}
滤波的较好演示方法是利用示波器观察波形,实验器材
与整流演示中方法二所用器材相同,另需准备一个100微法、
16伏和一个1000微法、16伏的电解电容,J2423型可拆变压
器(只使用0—100—400匝绕组、封闭铁心),以及一块$\pi$型
滤波电路示教板,实验线圈如图3.30中所示.实验时先演
示未接入滤波元件时情况,然后在负载$R$两端先后并入100
微法与1000微法电容观察负载两端电压波形,在并入100
微法电容时,可以明显看到如课本上图3.37所示波形,加大
滤波电容时(1000微法),可以看到波形基本平滑.在这两次
演示中,表示负载$R$的小灯泡始终发光,表明电路中仍有电流
通过,但已是不含交流成分的直流电了。演示电感滤波时,可
在负载与整流电路“正”输出端两个接线柱间先后接入可拆变
压器100匝与400匝绕组,观察比较其波形.$\pi$型滤波器
示教板上的电阻可用10欧滑动变阻器(J2354型)代替,两
个滤波电容均可用470微法(或200微法)、16伏电解电容器:
实验时,将示教板接在整流(任一种整流电路)与负载示教板
之间即可演示,为加强演示效果,还可在负载两端接一带输
出变压器的喇叭,通过喇叭发出交流“嗡”声的变化,了解滤波
效果。

\begin{figure}[htp]
    \centering
\includegraphics[scale=.7]{fig/3-30.png}
    \caption{}
\end{figure}

\subsubsection{变压器的构造与工作原理}
这个演示实验可使用J2423型可拆变压器或J2425型变
压器原理说明器,实验所用电表,可用J0401型演示电表,低
压电源为J1201型.下面以用J2423型变压器为例说明实验
步骤。
\begin{figure}[htp]
    \centering
\includegraphics[scale=.7]{fig/3-31.png}
    \caption{}
\end{figure}

(1)演示变压器是根据电磁感应现象制成的,它只能用
来改变交流电压.实验电路如图3.31所示.变压器初级线
圈用绿色线圈0—400匝绕组,次级线圈用红色线圈800—1400
匝绕组(600匝),电压表均使用电表的直流10伏挡,且均
调节为中心零位式.电源&1与82均用两节干电池(一号)串
联而成,或用两节2V蓄电池,R可用20欧滑动变阻器.演示
时可以看到,只有滑动端移动时,次级电压表指针才有摆动。
如果按一定周期移动变阻器滑动端,可以清楚地看到次级线
圈中产生与初级线圈中频率相同的交变电动势。

(2)演示变压器铁心的作用.实验装置同(1),演示时取
下封闭铁心所用的条形铁心,以相同周期移动变阻器滑动端,
可以从接次级线圈的电表指针摆动幅度看到比演示(1)时小
得多,说明通过次级线圈的磁通量比封闭铁心时少得多。

这个演示也可直接将初级线圈接20伏交流低压(可用J
1201型低压电源),次级线圈改用0—200匝绕组并接一个6—8
伏指示灯,不用电表,从铁心闭合前后小灯亮度变化情况来说
明铁心的作用。

(3)演示变压器初次级线圈端电压之比等于线圈匝数之
比。实验所用变压器初级用绿色线圈0—400匝绕组,接入10伏
交流电压,用演示电表交流25伏挡测红色线圈0—200匝、
0—800匝及600—1400匝(即600匝)绕组的电压,可以在误
差允许范围(10\%)内验证初次级线圈端电压之比等于其匝
数之比。演示时,为便于控制初级电压值大小,低压电源输入
端应接一自耦调压变压器。

为加强演示的直观性,还可以将变压器红色线圈0—1400
匝绕组直接接220伏照明电路(次级绿色线圈不用拆下).再
准备一段长3米多的塑料外皮的多股导线,在铁心另一侧当
堂绕20匝作为次级线圈,两端接演示电流表交流25伏挡.闭
合铁心后,将初级线圈通电,读出电压表读数(约3伏).再将
初级线圈断电,拆开铁心,次级线圈加绕20匝.重新闭合铁心,
给初级线圈通电,再读出电压表上的读数(约6伏),从而比较
鲜明地演示了变压器绕组上电压与其匝数成正比的规律。

\begin{figure}[htp]
    \centering
\includegraphics[scale=.7]{fig/3-32.png}
    \caption{}
\end{figure}

(4)演示变压器初、次级线圈中的电流与其匝数成反比。
变压器初级线圈用绿色0—400匝绕组,次级线圈用红色0—
200匝绕组,实验电路如图3.32所示.初级、次线电流表均接
交流1安挡,两个灯泡均用汽车用12伏20/8瓦灯的8瓦灯
丝部分.演示时,先接通$K_1$, 为便于对比,通电后调节调压器,
使次级线圈中电流取一整值,如0.4安.然后再闭合$K_2$, 重
新读出初、次级线圈中的电流,可以在实验允许误差范围内,
验证初、次级线圈中的电流与线圈匝数成反比的规律,在教
学中还应将$K_1$与$K_2$都断开,此时初级线圈中电流并不为零,
它相当于一个自感线圈,从而说明该规律是有一定适用条件
的近似规律。这个实验还可说明次级线圈消耗的功率是由初
级线圈通过电网吸收功率经电磁感应现象提供的。

\subsubsection{高压输电}

\begin{figure}[htp]
    \centering
\includegraphics[scale=.7]{fig/3-33.png}
    \caption{}
\end{figure}

应先制作一如图3.33所示的示教板.图中$A$、$B$、$C$、$D$、
$E$、$F$、$G$、$H$为8个接线柱,$BE$与$DG$间分别用阻值为10欧
左右的电阻丝(康铜丝或把220伏500瓦电炉丝拉直后截
取),电阻丝应拉紧,另用一个6—8伏指示灯两端分别引出导
线并安上夹头,再准备两个J2423型可拆变压器或者小型
220伏、6伏变压器(录音机等用的).演示时,先从低压电源(如
J1201型教学低压电源)引出6伏交流电,接至$B$、$D$两点,将
指示灯两端分别夹在两电阻丝不同位置,可以看到,越远离输
电端,导线上电压损失越大,灯越暗。然后按图3.34接入两个
J2423型变压器,将灯接在$F$、$H$间,$A$、$C$间接6伏交流电。可
以明显地看到指示灯发光正常,并可用交流电压表(如J0401
型演示电表交流250伏挡)去测试$B$、$D$间与$E$、$G$间电压,可
以看出,相比之下电压损失很小。

\begin{figure}[htp]
    \centering
\includegraphics[scale=.7]{fig/3-34.png}
    \caption{}
\end{figure}

\subsubsection{三相交流电的产生}

\begin{figure}[htp]
    \centering
\begin{tikzpicture}[>=latex]
\draw(0.2,0.5) rectangle (4-.2,2.5);
\foreach \x in {1,2,3}
\foreach \y in {1,2}
{
    \draw(\x,\y)circle(4pt);
}
\node at (1,1)[below right]{$D_1$};
\node at (2,1)[below right]{$D_2$};
\node at (3,1)[below right]{$D_3$};
\node at (1,2)[below right]{$D_6$};
\node at (2,2)[below right]{$D_4$};
\node at (3,2)[below right]{$D_5$};

\end{tikzpicture}
    \caption{}
\end{figure}

演示这个实验最好用J2420型手摇三相交流发电机.实
验时将发电机输出端六个接线柱按星形接法接好,该机输出
端接线柱排列情况如图3.35所示,$D_1$、$D_2$、$D_3$分别为$A$、$B$、
$C$三相线圈的头,$D_4$、$D_5$、$D_6$分别为$A$、$B$、$C$三相线圈的尾,排
列成如图情况是为了便于进行星形连接与三角形连接。星形
接法只需将$D_6$、$D_4$、$D_5$端用铜片连在一起。用三个J0401型
演示电流表的检流计挡依次接在$D_1$、$D_2$、$D_3$与公共端(零线)
之间,并将三个表按$A$、$B$、$C$三相顺序放好,注意接线时电表
的三个负接线柱接零线。演示前在该机底板下电池盒内先装
入励磁用的四节干电池(6伏),演示时应先接通励磁开关,再
缓慢摇动手柄。注意不要使检流计超过满度电流,并尽可能使
转子转动周期与电表指针系统周期接近,这样可以明显演示
出三个电表指针依次摇动到右端。

\begin{figure}[htp]
    \centering
\includegraphics[scale=.7]{fig/3-36.png}
    \caption{}
\end{figure}

如果演示三相交流电的波形,最好让学生直接观察电网
中三相交流电的波形,实验电路如图3.36所示.示波器为
J2458型教学示波器,电子开关为J2460型教学用电子开关,
三个降压变压器可用三个学生实验用低压电源(J1202型)代
替,实验前需要确定相序。按图接好电路后,把示波器的Y
衰减拨至“10”,Y增益钮顺时针旋到底(最大增益),扫描频率
拨至“10—100”;使电子开关A增幅与B增幅钮反时针旋到
底(最小增益),开关频率拨至“5k—50k”挡;将三个变压器初
级线圈接入三相四线制电网。再调电子开关A增幅和示波器
扫描微调,使示波器荧光屏显示出两个周期的波形。调电子
开关B增幅与相对位移,使两个波形在屏上幅度相同,X轴重
合.然后观察B输出波形比A输出波形是否滞后$120^{\circ}$. 如
果不是,可将该变压器初级线圈接线颠倒一下。同理,再将电
子开关B输入端改接第三个变压器2伏输出端,调整后使其
波形滞后A$240^{\circ}$. 在确定好相序后,就可在课堂上进行演示
了。需要说明的是,经上述方法调整的相序可能不是电网原
来的相序。若想严格显示电网的相序,需要弄清三个变压器
初次级头尾,或者使用教学用三相变压器,如JYB-1型变
压器,由于电子开关只有两路,因而只能先后显示第一相与
第二相、第一相与第三相的波形,如想同时演示三相波形,
则需再用一台电子开关接在上述电子开关与示波器之间,接
法如图3.37所示.
\begin{figure}[htp]
    \centering
\includegraphics[scale=.7]{fig/3-37.png}
    \caption{}
\end{figure}

\subsubsection{星形接法与三角形接法的相电压与线电压}
按图3.38所示,制作电源接法示教板和负载接法示
教板。
\begin{figure}[htp]
    \centering
\includegraphics[scale=.7]{fig/3-38.png}
    \caption{}
\end{figure}

演示电源的两种接法时,可将J2420型手摇三相发电机
三组线圈按头尾顺序与图3.38中甲示教板相应接线柱相
接,或将演示实验16第二种方法所确定的三个变压器次级相
序与头尾接入甲示教板中(相应电压拨至6V挡).用J0401
型演示电表交流25伏挡分别测出各组线圈两端电压。再先后
按星形接法与三角形接法连接示教板上各接线柱,用伏特表
分别测试这两种情况下的相电压与线电压得出相应结论。
J2420型手摇三相发电机在手轮转速为120转/分时,空载每
相绕组电压在12伏左右.实验时,可请学生摇发电机,教师用
电压表进行测量演示。

\begin{figure}[htp]
    \centering
\includegraphics[scale=.7]{fig/3-39.png}
    \caption{}
\end{figure}

在演示三相四线制送电,负载星形接法相电压与线电压
关系及三相负载平衡时中性线中电流为零的实验时,可以将
4个6—8伏指示灯分别用较短的导线悬挂连接在图3.38所
示乙示教板相应接线柱上,并从甲示教板上接出三相四线制
电源,如图3.39所示,摇动发电机或接通电源时,可以看
到三相指示灯都亮了,只有串在中性线上的灯泡不亮,用
J0401型演示电表交流25伏挡分别测出各相相电压与线电
压,可以看出$U_{\text{线}}=\sqrt{3}U_{\text{相}}$的关系.由于三相负载平衡,在
除去串联在中性线上的小灯时,可以看出不影响三相负载用
电,说明中性线上此时电流为零,可以省去。如果中性线上
仍串联小灯,若除去其中任一相的负载灯泡,都可以看到串联
在中性线上的小灯发光,另两相上灯光减弱,表明负载不平衡
时中性线上电流不为零。如果接着再除去中性线上串联的小
灯,则相当于另两相小灯串联接在线电压之间,此时,用伏特
计可以测出每灯两端电压并与线电压值进行比较说明。

在进行负载作三角形连接演示实验时,可将三个小灯按
三角形接法接在示教板上,但如果仍使用6—8伏指示泡,注
意手摇发电机转速不要太高,三个变压器的电源应拨至4伏
挡,以免烧毁小灯。

\subsubsection{旋转磁场和感应电动机原理}

\begin{figure}[htp]\centering
    \begin{minipage}[t]{0.48\textwidth}
    \centering
\includegraphics[scale=.7]{fig/3-40.png}
    \caption{}
    \end{minipage}
    \begin{minipage}[t]{0.48\textwidth}
    \centering
\includegraphics[scale=.7]{fig/3-41.png}
    \caption{}
    \end{minipage}
    \end{figure}

演示这个实验可用J2421型三相电机原理演示器,如果
没有,自制也并不困难,装置主要由三部分组成,带支架的可
旋转的蹄形磁铁,互成$120^{\circ}$角的三相线圈及一指针式支架(附
小磁针和铝框)。蹄形磁铁可用教学用蹄形磁铁,但需制作一
夹具与轴连接,可参考图3.40所示装置.三组线圈均可用
直径0.3毫米左右的漆包线绕成(300—400匝),分别包上
红、黄、绿三色胶带或布带,互成$120^{\circ}$角安装在木底座上,
将其接成星形,三个头与底座接线柱相接,如图3.41所示.
可用长针固定在小木座上制成支架。小磁针可用教学用的
小磁针或自制;封闭金属框可以用薄铜片焊成:一边中央开
一小孔(略大于针直径),相对一边只需用圆头小钉冲出一小
槽(不要穿透铜片),使用时安放在支架上,如图3.42所示。

\begin{figure}[htp]\centering
    \begin{minipage}[t]{0.48\textwidth}
    \centering
\includegraphics[scale=.7]{fig/3-42.png}
    \caption{}
    \end{minipage}
    \begin{minipage}[t]{0.48\textwidth}
    \centering
\includegraphics[scale=.7]{fig/3-43.png}
    \caption{}
    \end{minipage}
    \end{figure}

演示分两个步骤进行。
(1)将小磁针支在支架上放在可旋转的磁铁中间,旋转
磁铁,将看到小磁针随之转动,表明了磁铁旋转产生旋转磁
场。然后将小磁针放在三相线圈中,利用手摇三相发电机或上
面介绍的通过三个变压器接入三相电网获得的低压三相交流
电给三组线圈通电,就可发现小磁针也旋转,电源也可采用如
图3.43所示裂相电源模拟三相交流电进行演示.图中$R$可
将10欧2安(J2354型)滑动变阻器调至最大值使用,$C$可将
两个470微法耐压25伏以上电 解电容器 负极与负极相接串
联后将两个正极接入电路,电源可使用J1201型或J1202型
低压电源的交流6伏输出,或者用一降压小变压器与以上
电路组装在一起使用。

(2)将金属闭合导体(即铜片线框)支于支架上,先后放
在可旋转磁铁和三相线圈中,旋转磁铁或通入三相电观察闭
合导体的旋转情况,分别加以讨论。实验时还可反向旋转磁铁
或任意更换三相电接入的两个线头位置,观察闭合导体与刚
才旋转方向相反的情况。

\subsection{学生实验}
\subsubsection{用示波器观察交流电的波形}
这个实验要进行电学重要仪器操作技能的训练。在实际
教学时,由于课本上对仪器的使用、有关实验内容都有较详细
的说明,教师不宜在课堂上再详细讲解,可要求学生自己阅
读,以培养学生独立学习新仪器使用方法的能力,还可以提出
一些问题以引导学生阅读和操作,下面几个问题可供教学时
参考:
\begin{enumerate}
\item 学生示波器(J2459型)面板上调节旋钮和开关可分
为三组,这三组分别控制什么?分别包括哪些旋钮和开关?各
起什么作用?Y衰减置于“$\infty$”时是什么意思?
\item 如果被观察的信号电压幅度估计为20伏,Y衰减应
置于哪个位置?为什么?
\item 如果被观察信号频率是200赫,想观察到两个完整
波形,扫描范围应置于哪个位置?为什么?调整扫描微调钮的
目的是什么?
\item 学生信号源(J2465型)可以输出哪几种频率的正弦
低频信号?输出信号幅度由什么调节?一般我们所说低频信号
是指哪一频率范围的信号?
\item 设计一个实验步骤,定量测出信号发生器500赫低
频输出最大时的正弦交流电电压的最大值和有效值。
\end{enumerate}

\subsubsection{用示器观察交流电的整流和滤波}
这个实验所用器材除J2459型学生示波器及学生用低压
电源(J1202型)外,所用元器件均可由J2467型(或J2467-1
型)学生电子实验箱取出使用,课本上图10-5所示实验元件
即为该型号电子实验箱中元件外形。但需注意此型号电子实
验箱中电解电容耐压只有10伏,因而实验时,最好将课本上
所说6伏交流电源改为4伏进行实验,以确保电容不被击穿。
如果没有该型号的电子实验箱,各元件也可自制:二极管用
2CP或2CZ型,电解电容可用耐压16伏、100微法(两个)与
10微法(一个)的,电阻可选用200欧及1千欧1/4瓦(或1/8
瓦)碳膜电阻,各元件都可焊接在宽2厘米长5厘米的铜箔
板上,铜箔板两端只保留1.5厘米长铜箔,中间部分可用刀刻
去。为便于接线,两端可直立焊接一段较粗的裸铜丝,导线系
用两端均带夹子的短塑料多股导线,元件板中央可用白漆画
出相应元件符号与正负极。

\begin{figure}[htp]
    \centering
\includegraphics[scale=.7]{fig/3-44.png}
    \caption{}
\end{figure}

为使实验取得较好效果,可将实验步骤作如下改变:
\begin{enumerate}
    \item 先按图3.44所示电路在白纸上画出电路图,并将元件放好,
用导线将元件连接起来。开启示波器,通过观察$A$、$B$两点输
入交流波形将示波器调好,使屏上出现3、4个完整稳定波形.再观察$A'$、$B$两点向半波整流波形,先后在$A$、$B$间接入10
微法与100微法电容器,观察电容滤波作用和不同电容的效
果。要求学生在坐标纸上记录上述观察到的电压波形,并进
行比较说明.
\item 再按课本上图10.4所示电路实验观察$A''$、
$B'$间电压波形.如果观察$C_1$、$C_2$与$R$数值变化对滤波效果
影响,则另需准备两个5微法16伏电解电容与20欧左右的
电阻,才能得到较显著的效果。
\end{enumerate}

在进行这个实验时,可能有的学生会提出不接负载时能
否观察相应波形问题,教师可以从串联电路特点出发,说明示
波器相当一个内阻极高(1兆欧以上)的电压表,不接负载时,
相当于二极管与电压表串联,此时电路正反向电阻相差不多,
因而正反向电流接近,看到波形将几乎仍为交流波形。(实际
上是二极管正向时,所加电压小于导通电压,对2CP管来说,
这个电压约为0.6伏,是由二极管此时内阻很大而引起的,
或者说是二极管的非线性特性造成的。)因而不接负载电阻
$R_f$是不能观察整流效果的。可实际观察一下,以加深理解。

\subsubsection{研究变压器的作用}
做这个实验,每一组需要一块交流电压表,一块交流电流
表,在学校实验室中,可用万用表交流10伏或25伏挡代替电
压表,可用J0414型500毫安交流电流表。如果没有该型号
交流电流表,可以用以下两种方法解决,一是将电路(初级或
次级电路)中串入一个1欧0.5瓦精密电阻(RJJ精密金属膜
电阻,误差$\pm 1\%$)作为示波器的取样电阻,通过示波器测量电
阻两端电压再换算为电流值。实验前需对示波器进行调整,使
示波器Y衰减为1、Y增益顺时针旋到底(增益最大)时屏上
Y轴上每一格相当50毫伏,实验时可将这个1欧取样电阻接
在示波器Y输入与地接线柱间并固定好,相当于一个电流表
串入电路,读出交流波形峰峰值$U_{PP}$, 按下式计算电流有效
值(Y衰减为1时):
\[I=\frac{U_{PP}}{2\sqrt{2}R}\approx 0.35U_{PP}{\rm mA}\]
或
\[I\approx 0.35\x 50\x Y_{PP}(\text{格数})=17.5Y_{PP}{\rm mA}\]
这个方
法虽然麻烦一些,但也使学生得到一个实际利用示波器作定
量测试的机会,对提高学生使用示波器能力是有好处的。

另一种方法是将学生所用直流安培表加以改装,改装方
法主要是在电表电路中接入两个2AP型晶体二极管,接法
如图3.45所示,由于各种直流安培计表头内阻与满度电
流不同,一般选满度电流小些的,如3毫安的表头.分流电阻
详细计算方法可参考有关书籍,并需实际与标准表校准。由于
二极管的非线性特性,刻度盘也是非线性的,一般可实测绘制
刻度盘。
\begin{figure}[htp]\centering
    \begin{minipage}[t]{0.48\textwidth}
    \centering
\includegraphics[scale=.7]{fig/3-45.png}
    \caption{}
    \end{minipage}
    \begin{minipage}[t]{0.48\textwidth}
    \centering
\includegraphics[scale=.7]{fig/3-46.png}
    \caption{}
    \end{minipage}
    \end{figure}

该实验所需其他仪器有J1202型学生电源,J2426型小
型变压器,J2354-1型10欧2安滑线变阻器(四个接头可当分
压器使用),2.5伏及6—8伏指示灯各两个(连同灯座),实验
电路如图3.46所示.实验按以下步骤进行:

(1)首先观察变压器结构,该变压器有三组线圈,分别标
有“I120T”,“II240T”、“III60T”,作降压变压器时,初级线圈
用“I120T”即120匝绕组,额定电压为4伏.次级线圈用“III60
T”即60匝绕组.作升压变压器时,初级线圈仍用“I120”绕组,
次级线圈用“II240T”即240匝绕组。

(2)按图3.46所示电路连接好实际电路,若进行升压实
验,小灯泡应用6—8伏的;若进行降压实验,小灯泡应用2.5
伏的。在接通电源之前,应将变阻器滑动端置于输出电压较
小位置,接通电路,逐渐滑动变阻器滑动端,使小电灯发光(不
要太亮)。

(3)用万用表交流10伏挡分别测出初次级电压(即$AB$
间与$CD$间电压),并作出记录.验证$U_1:U_2=n_1:n_2$.

(4)断开$A$点,串入电流表(或在输入端并有取样电阻的
示波器Y输入端),读出初级线圈电流值,而后接好$A$点,断
开$B$点,再串入电流表测出次级线圈电流值,作出记录,验证
$I_1:I_2=n_2:n_1$. 计算$U_1I_1$与$U_2I_2$值,求出变压器效率$n$.

(5)在次级电路小灯两端并联一个相同的小灯,重复步骤(3)与(4)。

(6)切断电源,将变压器从电路中取出,拆开铁心,装入
另一线圈,准备作降压(或升压)实验。在装铁心时,注意两点:
一是插铁心片时要交叉叠放,即一层按图3.47甲位置放,另
一层按图3.47乙位置放,二是在铁心快装满时,最后几片
铁心片的插入要十分小心,不要损坏线圈。可以先用小改锥
在一侧铁心片与线圈间插入一些,使铁心片靠里挤紧,再将最
后几片对正逐一插入,插一片,用改锥挤一次。同时,还可用
小木锤轻轻敲击铁片,使铁片对正插入。最后用木锤将铁片
敲击整齐,再装上外壳。

\begin{figure}[htp]
    \centering
    \begin{tikzpicture}[scale=.7]
\begin{scope}
    \draw(0,0) rectangle (4,4);
    \draw(0,3)-- (4,1);
    \draw[fill=white](.8,.5) rectangle (1.5,3.5);
    \draw[fill=white](2.5,.5) rectangle (3.2,3.5);
    \node at (2,-.5){奇数层};
    \node at (2,-1.5){甲};
\end{scope}
\begin{scope}[xshift=6cm]
    \draw(0,0) rectangle (4,4);
    \draw(0,1)-- (4,3);
    \draw[fill=white](.8,.5) rectangle (1.5,3.5);
    \draw[fill=white](2.5,.5) rectangle (3.2,3.5);
    \node at (2,-.5){偶数层};
    \node at (2,-1.5){乙};
\end{scope}
    \end{tikzpicture}
    \caption{}
\end{figure}

(7)重复(2)至(5)步骤,进行实验.

以上所设实验器材与实验步骤连同课本的问题可以根据
学校情况印发实验提纲发给学生阅读,并要求学生画好实验
记录表格再进行实验。如果时间允许,可在测电流时增加测空
载时电流这一项内容。

\subsection{课外实验活动}
\subsubsection{自制测电笔}

这个课外制作,需向学生简单说明测
电笔的工作原理,氖管是一种稀薄气体辉光放电管。在试电
笔中氖管两极间加60伏左右电压,即可引起氖气放电,发出
桔红色辉光,此时通过氖管的电流约几十微安,但两极电压
增大时,管中电流相应增大而且增长较快,试电笔中,电阻起
降压和限流作用,它的阻值可以通过以下方法估算出来:在测
试火线时,人体一端与地相接,另一端(手)通过氖管与电阻串
联接入火线,此时串联电路总电压为220伏,通过电流约几十
微安(人体安全电流1毫安以下),氖管两端压降为60伏左
右,因此降落在电阻器和人体电阻上的电压为160伏左右。
由欧姆定律估算出电阻器和人体电阻的总电阻约为几兆欧。
实际上人体电阻在几百千欧以下,因此,在试电笔使用过程
中,在人体上降落的电压是很小的,是安全的。只有给学生
讲清原理后,学生课外制作时才会懂得如何安全进行实验。

在具体进行这个小制作时,可以发动学生想办法把测电
笔制作得更实用一些,例如利用废旧圆珠笔或钢笔套进行制
作,不一定局限于课本上介绍的材料。测试笔与火线接触部分
长金属探针最好套上一段塑料管,以免与其他导体或人体接
触发生事故。

有些学生实验后可能提出这样的问题:为什么人体并没
有和大地直接接触(如穿着胶鞋、站在木凳上)试电笔也能
亮?这个问题可启发学生从人体与大地间是一个电容器,而测
试的又是交流电这一点上进行思考。

\section{习题解答}
\subsection{练习-}
\begin{enumerate}
    \item 有人说线圈平面转到中性面的瞬间,穿过线圈的磁通量最大,因而线圈中的感生电动势最大;线圈平面跟中性面垂直的瞬间,穿过线圈的磁通量为零,因而线圈中的感生电动势为零,这种说法对不对?为什么?


    \begin{solution}
        这种说法是错误的。根据法拉第电磁感应定律,感
        生电动势的大小与穿过闭合回路的磁通量变化率成正比,而
        与磁通量大小无关。当线圈平面转到中性面时,虽然磁通量
        最大,但磁通量变化率为零,线圈中感生电动势也为零;而当
        线圈平面转到与中性面垂直位置时,虽然磁通量为零,但磁
        通量变化率最大,因而线圈中感生电动势最大。
    \end{solution}
    
    \item 在课本图3.1所示的实验中,在不改变$B$、$\ell$和$v$的数值的情况下,你有什么办法来提高电动势的最大值?并说明所依据的原理.


    \begin{solution}
        可以通过增加线圈匝数来提高电动势的最大值。因
        为线圈在切割磁力线的过程中,每匝线圈的感生电动势都相
        同,$n$匝线圈是$n$个单匝线圈的串联,产生的感生电动势是单
        匝线圈的$n$倍。
    \end{solution}
    
    \item 在课本图3.1所示的实验中,设$AB$边的长度为20厘米,线圈的宽$AD$为10厘米,磁感应强度$B$为0.01特,线圈的转数为50转/秒,求电动势的最大值,如果从线圈平面转过中性面30$^\circ$角的瞬时开始计时,经过0.01秒时电动势的瞬时值是多大?


    \begin{solution}
        用$\ell$和$\ell'$分别表示$ab$和$ad$的长度。当线圈平面
        与中性面垂直时,电动势最大,这时的电动势
    \[\mathcal{E}_m=2B\ell v\]
    因为
\[v=\omega r=2\pi n\frac{\ell'}{2}\qquad \text{($n$为转速)}\]
所以
\[\mathcal{E}_m=2B\ell2\pi n\cdot \frac{\ell'}{2} \]
代入数值得
\[\mathcal{E}_m=2\x3.14\x50\x0.01\x20\x10^{-2}\x10\x10^{-2}=0.06{\rm V}\]
若从线圈平面转过中性面$30^{\circ}$角瞬时开始计时,那么该
交流电表达式
\[\begin{split}
   e&= \mathcal{E}_m \sin(\omega t+\phi_0)\\
   &=0.06\sin\left(2\pi \x50\x t+\frac{\pi}{6}\right)\\
   &=0.06\sin\left(100\pi  t+\frac{\pi}{6}\right)
\end{split}\]
因此经0.01秒时电动势瞬时值为
\[\begin{split}
    e &=0.06\sin\left(100\pi  0.01+\frac{\pi}{6}\right)\\
    &=0.06\sin\left(\pi +\frac{\pi}{6}\right)\\
    &=-0.06\sin\frac{\pi}{6}=-0.06\x \frac{1}{2}= -0.03{\rm V}
 \end{split}\]
    \end{solution}
    
    \item 画出$u=30\sin(\omega t+\pi/2)$伏的电压图象和$i=2\sin(\omega t-\pi/4)$安的电流图象,以$\omega t$为横轴.

    \begin{solution}
        所求图象如图3.48甲、乙所示.
\begin{figure}[htp]
    \centering
\begin{tikzpicture}[>=latex, scale=.7]
\begin{scope}
\draw[->](-2,0)--(7,0)node[right]{$\omega t$};
\draw[->](0,-4)--(0,4)node[right]{$u({\rm V})$};
\draw[very thick, domain=-pi/2:2*pi, samples=300]plot(\x,{3*cos(\x r)});
\foreach \x/\xtext in {0.5*pi/\frac{\pi}{2}, pi/\pi, 1.5*pi/\frac{3}{2}\pi, 2*pi/2\pi}
{
    \draw(\x,0)node[below]{$\xtext$}--(\x,.2);
}
\draw[dashed](0,3)node[left]{30}--(2*pi,3)--(2*pi,0);
\draw[dashed](0,-3)node[left]{$-30$}--(pi,-3)--(pi,0);
\node at (-.3,-.3){$0$};
\node at (pi,-5){甲};
\end{scope}
\begin{scope}[xshift=11cm]
    \draw[->](-2,0)--(8,0)node[right]{$\omega t$};
    \draw(0,3)node[left]{2}--(.2,3);
\draw[->](0,-4)--(0,4)node[right]{$i({\rm A})$};
\draw[very thick, domain=-pi/2:2.25*pi, samples=300]plot(\x,{3*sin(\x r -.25*pi r)});
\foreach \x/\xtext in {0.5*pi/\frac{\pi}{2}, pi/\pi}
{
    \draw(\x,0)node[below]{$\xtext$}--(\x,.2);
}
\foreach \x/\xtext in {1.5*pi/\frac{3}{2}\pi, 2*pi/2\pi}
{
    \draw(\x,0)--(\x,.2)node[above]{$\xtext$};
}
\draw[dashed](0,2.121)node[left]{$1.41$}--(pi,2.121)--(pi,0);
\draw[dashed] (.5*pi,2.121)--(.5*pi,0);
\draw[dashed](0,-3)node[left]{$-2$}--(1.75*pi,-3)--(1.75*pi,0);
\draw[dashed](0,-2.121)node[left]{$-1.41$}--(2*pi,-2.121)--(2*pi,0);
\draw[dashed](1.5*pi,-2.121)--(1.5*pi,0);
\node at (pi,-5){乙}; 
\node at (-.3,-.3){$0$};
\end{scope}
\end{tikzpicture}
    \caption{}
\end{figure}
    \end{solution}
    
    \item 课本图3.1是一个发电机模型,只能供课堂演示之用,如果你有兴趣,可约请几位同学,共同研究一下怎样在此模型的	
	基础上加以改进,设计一个小发电机,如果感到知识不足,可自学或查阅有关电工学的书籍.


    \begin{solution}
        说明:此题有关制作可参考前面实验指导部分有关
        内容。
    \end{solution}
\end{enumerate}

\subsection{练习二}
\begin{enumerate}
    \item 有一个电容器,能耐压250伏,是否能接在220伏的交流电路上?为什么?


    \begin{solution}
        耐压250伏的电容器不允许接在220伏的交流电路上.这是由于220伏交流电路上电压的最大值可达到有效值
        的$\sqrt{2}$倍,即
        \[U_m=\sqrt{2}U=\sqrt{2}\x220=310{\rm V}\]
        把耐压
        250伏的电容器接入这个电路就有击穿并引起短路的可能。
    \end{solution}
    
    \item 线圈转动的角速度$\omega$也叫角频率(或圆频率),试就课本图3.1导出角频率$\omega$跟周期$T$或频率$f$的关系式.


    \begin{solution}
        线圈从课本图3.1中甲图位量匀速转到乙图所示位
        置,转过角度为$\pi$, 经历时间为$T/2$, 故由角速度定义式可得
\[\omega=\frac{\theta}{t}=\frac{\pi}{T/2}=\frac{2\pi}{T}\]
又因$f=1/T$,所以
\[\omega=\frac{2\pi}{T}=2\pi f\]
    \end{solution}
    
    \item 已知:$u_1=220\sqrt{2}\sin(100\pi t+\pi /6)$伏,$u_2=380\sqrt{2}\sin(100\pi t+\pi /3)$伏,求这两个交流电压的最大值、有效值、周期、频率、角频率和初相.这两个电压哪个超前?相差是多大?


    \begin{solution}
        由题目可知,$U_{m_1}=220\sqrt{2}$伏,
        $U_{m_2}=380\sqrt{2}$伏。
\[        U_1=\frac{U_{m_1}}{\sqrt{2}}=220{\rm V},\qquad U_2=
 \frac{V_{m_2}}{\sqrt{2}}=380{\rm V}\]
    \[\omega_1=100\pi {\rm rad/s},\qquad \omega_2=100\pi {\rm rad/s}\]
\[f_1=\frac{\omega_1}{2\pi}=\frac{100\pi}{2\pi}=50{\rm Hz},\qquad f_2=\frac{\omega_2}{2\pi}=\frac{100\pi}{2\pi}=50{\rm Hz}\]
\[T_1=\frac{1}{f_1}=\frac{1}{50}=0.02{\rm s},\qquad T_2=\frac{1}{f_2}=\frac{1}{50}=0.02{\rm s}\]
\[\phi_1=\frac{\pi}{6},\qquad \phi_2=\frac{\pi}{3}\]
因$\phi_2>\phi_1$,所以$u_2$超前$u_1$, 相差
$$\Delta\phi=\frac{\pi}{6}-\frac{\pi}{3}=-\frac{\pi}{6}$$
即$u_2$超前$u_1$ $\pi/6$
    \end{solution}
    
    \item 有一正弦交流电,频率是50赫,有效值是5安,初相
    是$-\pi/2$.写出瞬时值的表达式,并画出图象.

    \begin{solution}
\[\begin{split}
    I_m&=\sqrt{2}I=\sqrt{2}\x 5=7.07{\rm A}\\
    \omega&=2\pi f=2\pi\x 50=100\pi {\rm rad/s}\\
    \phi_0&=-\frac{\pi}{2}
\end{split}\]
瞬时值表达式为
\[i=7.07\sin\left(100\pi t-\frac{\pi}{2}\right){\rm A}\]
其图象如3.49图。
    \end{solution}


\begin{figure}[htp]\centering
    \begin{minipage}[t]{0.48\textwidth}
    \centering
\begin{tikzpicture}[>=latex, scale=.6]
    \draw[->](-1,0)--(9,0)node[above]{$\omega t$};
\draw[->](0,-4)--(0,4)node[right]{$i({\rm A})$};
\draw[very thick, domain=0:2.5*pi, samples=300]plot(\x,{3*sin(\x r -.5*pi r)});
\foreach \x/\xtext in {0.5*pi/\frac{\pi}{2}, pi/\pi}
{
    \draw(\x,0)node[below]{$\xtext$}--(\x,.2);
}
\foreach \x/\xtext in {1.5*pi/\frac{3}{2}\pi, 2*pi/2\pi}
{
    \draw(\x,0)--(\x,.2)node[above]{$\xtext$};
}
\draw[dashed](0,-3)node[left]{$-0.707$}--(2*pi,-3)--(2*pi,0);
\draw[dashed](0,3)node[left]{$0.707$}--(pi,3)--(pi,0);
\node at (-.25,-.25){$0$};
    \end{tikzpicture}
    \caption{}
    \end{minipage}
    \begin{minipage}[t]{0.48\textwidth}
    \centering
    \begin{tikzpicture}[>=latex, yscale=1.4, xscale=.8]
       \draw [->](-.5,0)--(8.5,0)node[above]{$t$(s)};
    \draw [->](0,-1.5)--(0,2)node[right]{$i$(A)};
    \draw [ultra thick] plot[domain=0:3.14*2.5, samples=1000] function{cos(x)};
    \draw [dashed](0,1)--(2*3.1416,1);
    \draw [dashed](0,-1)--(3.1416,-1);
    \node at (0,1) [left]{$+10$};
    \node at (0,-1) [left]{$-10$};
    \foreach \x in {0.05,0.1,0.15,0.2,0.25}
    {
        \node at (\x*31.416, 0)[below]{$\x$};
        \draw(\x*31.416, 0)--(\x*31.416, .1);
    }
    \node at (-.25,-.25){$0$};
    \end{tikzpicture}
    \caption{}
    \end{minipage}
    \end{figure}

    \item 图3.50是某一正弦交流电的图象.根据图象求出最大值、有效值、周期、角频率和初相,并写出瞬时值的表达式.

    \begin{solution}
$I_m=10$安;$I=I_m/\sqrt{2}=0.707\x10=7.07$安;
$T=0.2$秒;

$\omega=\dfrac{2\pi}{T}=\dfrac{2\pi}{0.2}=10\pi{\rm rad/s}$;$\phi_0=-\dfrac{\pi}{2}$。

该交流电流瞬时值表达式为:
\[i=10\sin\left(10\pi t -\frac{\pi}{2}\right){\rm A}\]
    \end{solution}
    
    \item 课本图3.10是两个正弦交流电的图象,哪个超前,哪个落后?超前或落后的角度是多大?其中$\phi_1$和$\phi_2$的绝对值都是60$^\circ$.
    
\begin{solution}
$i_1$超前$i_2$ $120^{\circ}$, $i_2$落后$i_1$ $120^{\circ}$.
\end{solution}

\end{enumerate}


\subsection{练习三}
\begin{enumerate}
    \item 把“220V,40W”的灯泡接到照明电路中,通过灯泡的电流的最大值是多大?


    \begin{solution}
        因为$P=UI$, 所以在额定电压220V情况下,通过灯泡的电流有效值
 \[I=\frac{P}{U} =\frac{40}{220}=0.18{\rm A}\]
        因此电流的最大值
        $$I_m=\sqrt{2}\cdot I=\sqrt{2}\x0.18=0.25{\rm A}$$
    \end{solution}
    
    \item 在电阻$R$的两端加交变电压$u=220\sqrt{2}\sin(\omega t+\phi)$伏,$R=110$欧,写出电流瞬时值的表达式.


    \begin{solution}
    \[i=\frac{u}{R}=\frac{220\sqrt{2}}{110}\sin(\omega t+\phi)=2\sqrt{2}\sin(\omega t+\phi){\rm A}\]
    \end{solution}
    
    \item 在电阻为500欧的电阻丝中通以交流电,每秒钟产生5焦的热.求电流和电压的有效值和最大值,交流电的功率是多大?


    \begin{solution}
        由焦耳定律$\theta=I^2Rt$可得:交流电电流有效值
    \[I=\sqrt{\frac{\theta}{Rt}}=\sqrt{\frac{5}{500\x 1}}=0.1{\rm A}\]
电流最大值
\[i_m=\sqrt{2}I=0.14{\rm A}\]
        所以电压有效值
$$U=I\cdot R=0.1\x500=50{\rm V}$$
电压最大值
\[u_m=\sqrt{2}U=1.4\x50=70{\rm V}\]
交流电功率
\[P=\frac{W}{t}=\frac{5}{1}=5{\rm W}\]
    \end{solution}
    
    \item 把一个电热器接到10伏的直流电路中,每秒钟产生的热量为$Q$.现在把它改接到交流电路中,每秒钟产生的热量为$Q/2$,求交流电压的最大值.

    \begin{solution}
        对纯电阻电路而言,电路产生的热$$Q=\frac{U^2}{R}t$$
        接入直流电路时
        \[Q=\frac{U^2}{R}\x 1\]
        接入交流电路时
     \[\frac{Q}{2}=\frac{(U_m/\sqrt{2})^2}{R}t=\frac{U^2_m}{2R}\]
        两式相除可得:$U^2_m=U^2$, 
        所以
   \[     U_m=U=10{\rm V}\]
        即交流电压的最大值为10伏.
    \end{solution}
    
\end{enumerate}

\subsection{练习四}
\begin{enumerate}
    \item 一个线圈的自感系数为0.6亨,电阻只有几欧姆,把这个线圈接到50赫的交流电路中,它的感抗是多大?比较感抗和电阻的大小,说明为什么可以略去电阻,而认为它只有电感.


    \begin{solution}
        这个线圈对50赫交流电的感抗
    \[    X_L=2\pi fL=2\pi \x50\x0.6=188\Omega\]
        由于$X_L\gg R$, 因此在电路中起主要作用的是感抗而不是
        电阻,因而通常可以忽略电阻的作用而认为它只有电感。
    \end{solution}
    
    \item 有一个高频扼流圈,自感系数是25毫亨,对于1兆赫的交流电,它的感抗是多大?对于1千赫的交流电,它的感抗又是多大?


    \begin{solution}
        对1兆赫交流电的感抗
        \[X_{L_1}=2\pi f_1L=2\pi\x10^6\x25\x10^{-3}=157\x10^3\Omega\]
        对1千赫交流电的感抗
        \[X_{L_2}=2\pi f_2L=2\pi\x10^3\x25\x10^{-3}=157\Omega\]
    \end{solution}
    
    \item 一线圈的自感系数为0.5亨,电阻可以忽略.把它接到频率为50赫,电压为220伏的交流电路中,求通过线圈的电流.


    \begin{solution}
 \[I=\frac{U}{X_L}=\frac{U}{2\pi fL}=\frac{220}{2\x 3.14\x 50\x 0.5}=1.4{\rm A}\]   
    \end{solution}
    
    \item 有一线圈,电阻可略去不计.把它接到220伏、50赫的交流电路中,测得通过线圈的电流为2安.求线圈的自感系数.


    \begin{solution}
    因为\[I=\frac{U}{X_L}=\frac{U}{2\pi fL }\]
    所以
    \[L=\frac{U}{2\pi fI}=\frac{220}{2\pi\x 50\x 2}=0.35{\rm H}\]
    \end{solution}
    
\end{enumerate}


\subsection{练习五}
\begin{enumerate}
    \item 电容是100皮法的电容器,对频率是$10^8$赫的高频电流和频率是$10^3$赫的音频电流,容抗各是多大?


    \begin{solution}
对$10^6$赫高频电流,电容容抗
\[X_{C_1}=\frac{1}{2\pi f_1C}=\frac{1}{2\x3.14\x10^6\x100\x10^{-12}}=1.59\x10^3\Omega\]
对$10^3$赫音频电流,容抗
\[X_{C_2}=\frac{1}{2\pi f_2C}=\frac{1}{2\x3.14\x10^3\x100\x10^{-12}}=1.59\x10^6\Omega\]
    \end{solution}
    
    \item 把电容为5微法的电容器接到220伏、50赫的交流电路中,通过电容器的电流是多少?把电容器换为0.05微法的,通过的电流是多少?

    \begin{solution}
电容为5微法时,通过电容器的电流
\[\begin{split}
    I=\frac{U}{X_C}=\frac{U}{\dfrac{1}{2\pi fC}}&=U\cdot 2\pi fC\\
    &=220\x 2\pi\x 50\x 5\x 10^{-6}=0.345{\rm A}
\end{split}\]
当电容器电容为0.05微法时,通过电流
\[\begin{split}
    I'=\frac{U}{X'_C}=\frac{U}{\dfrac{1}{2\pi fC'}}&=U\cdot 2\pi fC'\\
    &=220\x 2\pi\x 50\x 0。05\x 10^{-6}\\
    &=3.45\x 10^{-3}{\rm A}=3.45{\rm mA}
\end{split}\]
    \end{solution}
    
    \item 在有220伏、50赫交流电源的地方,使用一个交流电流表可以测定电容器(耐压在311伏以上)的电容,说明测定的方法和原理.


    \begin{solution}
        根据$I=\dfrac{U}{X_C}$及$X_C=\frac{1}{2\pi fC}$可得:
\[        I=U\cdot 2\pi fC\]
        式中,$U$、$f$已知,$I$为通过电容器的电流,可用交流电流
        表测出,因而$C$可由下式求出:
        \[C=\frac{1}{2\pi f\cdot U}\]
    \end{solution}
    
    \item 线圈的自感系数为$L$,电容器的电容为$C$.要使感抗和容抗相等,交流电的频率应该满足什么条件?


    \begin{solution}
对同一电源要使线圈感抗和电容容抗相等,应满足
的关系是$X_L=X_C$, 即
\[2\pi fL=\frac{1}{2\pi fC}\]
因此电源频率应满足条件是
\[f^2=\frac{1}{4\pi^2 LC}\]
即
\[f=\frac{1}{2\pi\sqrt{LC}}\]
    \end{solution}
    
\end{enumerate}




\subsection{练习六}
\begin{enumerate}
    \item 变压器能不能改变直流电的电压?说明理由.


    \begin{solution}
变压器是利用交流电通过线圈在铁心中产生交变磁
通,从而使副线圈中引起感生电动势而完成变压作用的,对于
稳恒电流,由于不会引起磁通变化,也就无法完成变压作用。
因此,变压器不能改变直流电压。    
    \end{solution}
    
    \item 收音机中的变压器,原线圈有1210匝,接在220伏
的交流电源上,要得到5伏、6.3伏和350伏三种输出电压,
三个副线圈的匝数各是多少?


\begin{solution}
根据变压比公式可得
\[n_{\text{次}}=\frac{U_{\text{次}}}{U_{\text{初}}}\cdot n_{\text{初}}\]

5伏绕组匝数
\[n_1=\frac{U_1}{U_{\text{初}}}\cdot n_{\text{初}}=\frac{5}{220}\x 1210\approx 28\text{(匝)}\]
6.3伏绕组匝数
\[n_2=\frac{U_2}{U_{\text{初}}}\cdot n_{\text{初}}=\frac{6.3}{220}\x 1210\approx 35\text{(匝)}\]
350伏绕组匝数
\[n_3=\frac{U_3}{U_{\text{初}}}\cdot n_{\text{初}}=\frac{350}{220}\x 1210\approx 1925\text{(匝)}\]
\end{solution}

\item 为了安全,机床上照明电灯用的电压是36伏,这个
电压是把220伏的电压降压后得到的,变压器的原线圈是
1140匝,副线圈是多少匝?用这台变压器给40瓦的电灯供
电,原副线圈中的电流强度各是多大?


\begin{solution}
由变压比公式$\dfrac{U_1}{U_2}=\dfrac{n_1}{n_2}$,
可得副线圈匝数
\[n_2=\frac{U_2}{U_1}\cdot n_1=\frac{36}{220}\x 1140=187\text{(匝)}\]

当变压器副线圈向40瓦电灯供电时,通过副线圈的电流
\[I_2=\frac{P}{U}=\frac{40}{36}\approx 1.1{\rm A}\]
因为变压器的输出功率和输入功率相等,即$P_{\text{原}}=P_{\text{副}}=40$瓦,所以通过原线圈的电流
\[I_1=\frac{P_{\text{原}}}{U_{\text{原}}}=\frac{40}{220}=0.18{\rm A}\]
或:由$\dfrac{I_1}{I_2}=\dfrac{n_2}{n_1}$,可得:
\[I_1=\dfrac{n_2}{n_1}\cdot I_2=\frac{187}{1140}\x 1.1=0.18{\rm A}\]
\end{solution}

\item 利用变压器的原理可以测量线圈的匝数;用被测线
圈作原线圈,用一个匝数已知的线圈作副线圈,通入交流电,
测出两线圈的电压,就可以求出被测线圈的匝数,已知副线
圈有400匝,把原线圈接到220伏的线路中,测得副线圈的电
压是55伏,求原线圈的匝数.


\begin{solution}
    设原线圈匝数为$n_1$, 副线圈匝数为$n_2$, 原线圈电压
    为$U_1$, 副线圈电压为$U_2$. 根据变压比公式$\dfrac{U_1}{U_2}=\dfrac{n_1}{n_2}$,
    可得
\[n_1=\frac{U_1}{U_2}\cdot n_2=\frac{220}{55}\x 400=1600\text{(匝)}\]

\end{solution}

\item 在图3.28所示的电压互感器的电路中,为什么副线
圈的匝数比原线圈的少?在图3.29所示的电流互感器的电路
中,为什么副线圈的匝数比原线圈的多?


\begin{solution}
使用电压互感器是为了将被测的高电压转化成低电
压,根据变压器变压比公式可知,副线圈匝数要比原线圈少
得多。而使用电流互感器是为了将被测的强电流转化为弱电
流,根据变流比公式$I_1:I_2=n_2:n_1$可知,副线圈匝数比原线圈
多得多(事实上电流互感器的初级线圈只有一、二匝)。
\end{solution}

\end{enumerate}



\subsection{练习七}
\begin{enumerate}
    \item 在课文所给的例子中,用110伏和11千伏的电压输
电,分别要用96000${\rm mm}^2$和9.6${\rm mm}^2$的铝导线.如果不用铝
导线,而用铜导线,导线的横截面积分别要多大?

\begin{solution}
根据题意,导线的长度和电阻仍应与原来铝导线的
相同。根据电阻定律有:
\[R_{\text{铜}}=\rho_{\text{铜}}\cdot \frac{\ell_{\text{铜}}}{S_{\text{铜}}},\qquad R_{\text{铝}}=\rho_{\text{铝}}\cdot \frac{\ell_{\text{铝}}}{S_{\text{铝}}}\]
两式相除可得
\[\frac{S_{\text{铜}}}{S_{\text{铝}}}=\frac{\rho_{\text{铜}}}{\rho_{\text{铝}}}\]
所以,
\[S_{\text{铜}}=\frac{\rho_{\text{铜}}}{\rho_{\text{铝}}}\cdot S_{\text{铝}}\]
查课本第二册第180页电阻率表可知:
\[\rho_{\text{铜}}=1.7\x 10^{-8}\Omega\cdot {\rm m},\qquad   \rho_{\text{铝}}=2.9\x 10^{-8}\Omega\cdot {\rm m}\]

所以,用110伏输电时,使用铜导线截面积应为:
\[S_1=\frac{1.7\x 10^{-8}}{2.9\x 10^{-8}}\x 96000=56640{\rm mm^2}\]
用11千伏输电时,使用的铜导线截面积为:
\[S_2=\frac{1.7\x 10^{-8}}{2.9\x 10^{-8}}\x 9.6=5.7{\rm mm^2}\]
\end{solution}

\item 从发电站输出的功率为200千瓦,输电线的总电阻
为0.05欧,用110伏和11千伏的电压输电,在这两种情况
下,在输电线上损失的电压各是多少伏?输送到用户的电压各
是多少伏?在输电线上损失的功率各是多少千瓦?

\begin{solution}
输电线电阻与用户负载构成串联电路,电路消耗功
率为电站输出功率200千瓦.

因为$P=U\cdot I$, 所以$I=P/U$。
用110伏电压送电时,电路电流
\[I_1=\frac{200\x 10^3}{110}=1.82\x 10^3{\rm A}\]
用11千伏电压送电时,电路电流
\[I_2=\frac{200\x 10^3}{11\x 10^3}=18.2{\rm A}\]
\begin{enumerate}
    \item 用110伏电压送电时,输电线上损失电压
\[U_{\text{损}1}=I_1\cdot R=1.82\x10^3\x0.05=91{\rm V}\]
用户电压
\[U_{\text{用}1}=U_1-U_{\text{损}1}=110-91=19{\rm V}\]
输电线损失功率 
\[P_{\text{损}1}=I^2_1\cdot R=(1.82\x10^3)^2\x0.05{\rm W}=166{\rm kW}\]
\item 用11千伏电压送电时,输电线上损失电压
\[U_{\text{损}2}=I_2\cdot R=1.82\x 0.05=0.91{\rm V}\]
用户电压
\[U_{\text{用}2}=U_2-U_{\text{损}2}=11\x 10^3-0.91 {\rm V}\approx 11{\rm kV}\]
输电线损失功率 
\[P_{\text{损}2}=I^2_2\cdot R=1.82^2\x0.05{\rm W}\approx 0.017 {\rm kW}\]
\end{enumerate}
\end{solution}

\item 用220伏和11千伏两种电压来输电,输送的功率相
同,在输电线上损失的功率相同,导线的长度和电阻率也相
同,求导线的横截面积之比.

\begin{solution}
本题应根据损失功率列式进行比较。设输送端电压
为$U$, 输送功率为$P$, 导线电阻为$R$, 导线长度为$\ell$, 电阻率为
$\rho$、横截面积为$S$. 由输送电压与输送功率求出通过导线
电流:$I=P/U$。

由于导线上损失功率$P=I^2R$, 式中$R=\rho\cdot\dfrac{\ell}{S}$。
故损失功率可表示为
\[P=I^2R=\left(\frac{P}{U}\right)^2\cdot \rho\frac{\ell}{S}\]
若分别以220伏和11千伏输电,有
\[P_{\text{损1}}=\left(\frac{P}{U_1}\right)^2\cdot \rho\frac{\ell}{S_1},\qquad P_{\text{损2}}=\left(\frac{P}{U_2}\right)^2\cdot \rho\frac{\ell}{S_2}\]
因为$P_{\text{损1}}=P_{\text{损2}}$, 所以
\[\left(\frac{P}{U_1}\right)^2\cdot \rho\frac{\ell}{S_1}=\left(\frac{P}{U_2}\right)^2\cdot \rho\frac{\ell}{S_2}\]
整理后可得:
\[\frac{S_1}{S_2}=\frac{U^2_2}{U^2_1}=\frac{(11\x 10^3)^2}{220^2}=\frac{2.5\x 10^3}{1}\]
即:用220伏和11千伏输电两种情况导线横截面积之比
为$2.5\x10^3:1$.
\end{solution}

\end{enumerate}




\subsection{练习八}
\begin{enumerate}
	\item 图3.33中变压器副线圈$aO$间和$bO$间的交变电压都是6伏特.当$D_1$导通而$D_2$截止时,加在$D_2$上的反向电压的最大值是多大?

    \begin{solution}
设图3.51中的$O$点为零电势点.当变压器输出电
压处于正半周最大值时,$a$点电势为$+6\sqrt{2}$伏,$b$点电势为
$-6\sqrt{2}$伏,由于$D_1$处于正向电压作用而导通,$c$点和$a$点
电势接近相同,亦为$+6\sqrt{2}$伏.对二极管$D_2$而言,其正极
电势为$b$点电势为$-6\sqrt{2}$伏,而负极电势为$c$点电势$+6\sqrt{2}$伏,处于反向电压而截止,这个反向电压为$12\sqrt{2}\approx 17$伏($U_{cb}\approx 17$伏).因此加在$D_2$上反向电压最大值可达
17伏.
    \end{solution}

\begin{figure}[htp]\centering
    \begin{minipage}[t]{0.48\textwidth}
    \centering
\includegraphics[scale=.8]{fig/3-51.png}
    \caption{}
    \end{minipage}
    \begin{minipage}[t]{0.48\textwidth}
    \centering
\includegraphics[scale=.8]{fig/3-52.png}
    \caption{}
    \end{minipage}
    \end{figure}

	\item 课本图3.40是哪一种整流电路?试用带箭头的线画出当$a$正$b$负时电流的通路.

    \begin{solution}
        这是桥式整流电路,$a$正$b$负时电流通路如图3.52
        所示。
    \end{solution}
    
	\item 课本图3.41中的信号是正弦交流电,最大值为1伏,直流电源的电动势为4.5伏.画出负载$R$两端的电压波形,如果二极管$D$反接,负载$R$两端的电压又怎样?直流电源的内电阻略去不计.

\begin{solution}
    由于正弦交流电电压最大值为1伏,而电源电动势
为4.5伏,因此无论正弦交流信号处于正半周还是负半周,加
在二极管两端的电压始终是正向电压,处于导通状态。$R$两
端电压变化范围为3.5伏至5.5伏,$R$两端电压波形如图
3.53所示,如果二极管反接,则无论是交流信号的正半周还
是负半周,对二极管来说始终处于反向截止状态,通过$R$的电
流可认为是零,$R$两端电压也是零。
\begin{figure}[htp]
    \centering
\begin{tikzpicture}[>=latex, xscale=1, yscale=.6]
\draw[<->](0,7)node[right]{$u$(V)}--(0,0)--(8,0)node[right]{$t$};
\foreach \x in {0,1,2,...,6}
{
    \draw(0,\x)node[left]{\x}--(.1,\x);
}
\foreach \x in {4,5,6}
{
    \draw[dashed](0,\x)--(8,\x);
}
\draw[domain=0:2*pi+.3, samples=100, very thick] plot(\x, {sin(\x r)+5});

\foreach \x/\xtext in {1/\dfrac{T}{4},2/\dfrac{T}{2},3/\dfrac{3}{4}T,4/T}
{
    \draw(\x*pi/2,0)node[below]{$\xtext$}--(\x*pi/2,0.2);
}

\end{tikzpicture}
    \caption{}
\end{figure}
    \end{solution}
    
	\item 在课本图3.35所示的桥式整流电路中,如果二极管$D_1$的极性接反,会发生什么现象?如果二极管$D_1$已被击穿,又会发生什么现象?说明理由,并画出负载电阻上的电压波形.

    \begin{solution}
        二极管$D_1$极性接反电路如图3.54所示,当交流电
压处于正半周时,$a$正$b$负,由于$D_1$和$D_4$均处于反向电压
作用之下,各二极管与负载电阻$R$上都没有电流通过。而当
交流电压处于负半周时,即$a$负$b$正,此时$D_2$与$D_1$都处于正
向电压作用之下导通,造成$ab$两点短路,因而将造成$D_2$与
$D_1$因通过较大短路电流而烧毁,严重时将烧毁变压器,$R$上
也没有电流。


\begin{figure}[htp]\centering
    \begin{minipage}[t]{0.48\textwidth}
    \centering
\begin{circuitikz}[>=latex, scale=1]

        \draw (-3,.25)--(-3,1)--(-2,1) to [american inductor] (-2,-1)--(-3,-1)--(-3,-.25);
        \draw (1,-1)--(1,-1.5)--(-1.25,-1.5) to [american inductor] (-1.25,1.5)--(1,1.5)--(1,1);
        \draw [ultra thick] (-1.6, -.5)--(-1.6, .5);
      \ctikzset{diodes/scale=0.6}  
      \draw (0,0) to [full diode] (1,1) ;
      \draw (2,0) to [full diode] (1,1) ;
     \ctikzset{diodes/scale=0.6}  \draw (0,0) to [full diode] (1,-1) to [full diode] (2,0) ;
        \draw (2,0)--(2.5,0) to [european, R=$R$] (2.5,-2) --(-.5,-2)--(-0.5,0)--(0,0);
        
        \draw [fill=white] (-3,.25) circle (1pt);
        \draw [fill=white] (-3,-.25) circle (1pt);
        \draw [fill=black] (0,0) circle (1pt);
        \draw [fill=black] (2,0) circle (1pt);\draw [fill=black] (1,-1) circle (1pt);
        \draw [fill=black] (1,1) circle (1pt);
        \node at (-3,0){$\sim $};
        \node at (1.5,1){$D_1$};        \node at (1.5,-1){$D_2$};
        \node at (.5,-1){$D_3$};        \node at (.5,1){$D_4$};
        \node at (-1.1,.5){$a$};        \node at (-1.1,-.5){$b$};
        \end{circuitikz}

    \caption{}
    \end{minipage}
    \begin{minipage}[t]{0.48\textwidth}
    \centering
\begin{tikzpicture}[>=latex, xscale=1, yscale=1.2]
\draw[<->](0,1.7)node[right]{$U_R$}--(0,0)--(5.75,0)node[right]{$t$};
\draw(0,0)--(0,-1);
\foreach \y in {0,2,4}
{
    \draw[domain=0:1, samples=50, very thick] plot(\x+\y, {sin(\x*pi r)});
}
\foreach \x/\xtext in {1/\dfrac{T}{2},2/T,3/\dfrac{3}{2}T,4/2T,5/\dfrac{5}{2}T}
{
    \draw(\x,0)node[below]{$\xtext$}--(\x,0.1);
}
\draw(0,1)node[left]{$U_m$}--(.1,1);
\end{tikzpicture}
    \caption{}
    \end{minipage}
    \end{figure}

如果$D_1$被击穿,相当于$D_1$被短路.在交流电正半周,即
$a$正$b$负时,$R$上将有电流通过。而当交流电负半周,即$a$负$b$
正时,$D_2$处于正向导通,直接将$ab$两点短路.$R$因两端电势相
同,而没有电流通过.因此仍将出现烧毁$D_2$现象,并有烧毁
变压器可能.负载$R$上的电压波形如图3.55所示。
    \end{solution}
\end{enumerate}


\subsection{练习九}
\begin{enumerate}
    \item 根据课本图3.49写出三相负载相同时的相电流的瞬时值的表达式,并利用数学中学过的三角知识证明:三个相电流之和在任何时刻都等于零.

    \begin{proof}
        相电流的瞬时值表达式分别为
        \[\begin{split}
        i_a&=I_m\sin \omega t\\
        i_b&=I_m\sin(\omega t-120^{\circ})\\
        i_c&=I_m\sin(\omega t-240^{\circ})        
        \end{split}\]
        三个相电流之和
       \[ i_a+i_b+i_c=I_m\sin \omega t+I_m\sin(\omega t-120^{\circ})+I_m\sin(\omega t-240^{\circ})  \]
        根据三角函数和差化积公式
\[\sin A+\sin B=2\sin\frac{A+B}{2}\cdot\cos\frac{A-B}{2}\]
        可得:
\[\begin{split}
    i_a+i_b+i_c&=I_m\sin \omega t+I_m\left[\sin(\omega t-120^{\circ})+\sin(\omega t-240^{\circ}) \right]\\
    &=I_m\sin \omega t+2I_m\sin\frac{(\omega t-120^{\circ})+(\omega t-240^{\circ})}{2}\cdot \cos\frac{(\omega t-120^{\circ})-(\omega t-240^{\circ})}{2}\\
    &=I_m\sin \omega t+2I_m\cdot \sin(\omega t-180^{\circ})\cdot \cos 60^{\circ}\\
    &=I_m\sin \omega t+2I_m\cdot (-\sin \omega t)\cdot \frac{1}{2}\\
    &=I_m\sin \omega t - I_m\sin\omega t=0 
\end{split}\]
    \end{proof}
    
    \item 在课本图3.46所示的三相四线制电路中,相电压是220伏.现有“220V,100W”的灯泡90盏,第一相安装40盏,第二相安装30盏,第三相安装20盏.求灯泡全接通时,各个相电流和线电流.

    \begin{solution}
        在三相四线制电路中,相电流和线电流相同。
        因此,第一相电流
\[I_{\text{相1}}=I_{\text{线1}}=\frac{P}{U}\cdot n_1=\frac{100}{220}\x 40=18.2{\rm A}\]
第二相电流
\[I_{\text{相2}}=I_{\text{线2}}=\frac{P}{U}\cdot n_2=\frac{100}{220}\x 30=13.6{\rm A}\]
第三相电流
\[I_{\text{相3}}=I_{\text{线3}}=\frac{P}{U}\cdot n_3=\frac{100}{220}\x 20=9.1{\rm A}\]
    \end{solution}
    
    \item 如果电源采用课本图3.53所示的三角形连接,线电压和相电压的关系是怎样的?设各相线圈的情况相同.

    \begin{solution}
这种情况下两根端线间电压和直接与这两端线相连
的那一相电压相同(图3.56甲).即线电压和相电压相同.
\begin{figure}[htp]\centering
    \begin{circuitikz}[scale=1.2]
\begin{scope}
    \draw (0,0) to [american inductors, L=3] (-1.5, -2);
        \draw (0,0) to [american inductors, L=1] (1.5, -2);	
            \draw (1.5, -2) to [american inductors, L=2] (-1.5, -2);
        \draw (0,0)-- (2.5,0)node [right]{$A$};
        \draw (1.5, -2)-- (2.5,-2)node [right]{$B$};	
        \draw (-1.5, -2)--(-1.5, -3)-- (2.5,-3)node [right]{$C$};	
        \foreach \x in {-2,-3,0}
        {
            \draw[fill=white] (2.5,\x) circle({1.5pt});
        }
        \draw [fill=black](0,0) circle(1.5pt);
        \draw [fill=black](1.5, -2) circle(1.5pt);
        \draw [fill=black](-1.5, -2) circle(1.5pt);
        \node at (0.5,-3.5){甲};
\node at (0,0)[above left]{尾};
\node at (0,0)[below right]{头};
\node at (-1.5, -2)[above left]{头};
\node at (-1.5, -2)[below right]{尾};
\node at (1.5, -2)[above right]{尾};
\node at (1.5, -2)[below left]{头};

\end{scope}
       \begin{scope}[xshift=6cm]
    \draw (0,0) to [american inductors, L=3] (-1.5, -2);

        \draw (1.5,-.25)node[right]{$A'$}--(.25,-.25) to [american inductors, L=1] (1.25, -1.75)--(2,-1.75)node[right]{$B'$};	


            \draw (1.5, -2) to [american inductors, L=2] (-1.5, -2);
        \draw (0,0)-- (2.5,0)node [right]{$A$};
        \draw (1.5, -2)-- (2.5,-2)node [right]{$B$};	
        \draw (-1.5, -2)--(-1.5, -3)-- (2.5,-3)node [right]{$C$};	
        \foreach \x in {-2,-3,0}
        {
            \draw[fill=white] (2.5,\x) circle({1.5pt});
        }

        \node at (0.5,-3.5){乙};

        \node at (0,0)[above left]{尾};
        \node at (.25,-.25)[below]{头};
        \node at (-1.5, -2)[above left]{头};
        \node at (-1.5, -2)[below right]{尾};
        \node at (1.25, -1.75)[above right]{尾};
        \node at (1.5, -2)[below left]{头};

        \draw[fill=white] (1.5,-.25) circle({1.5pt});
        \draw[fill=white] (2,-1.75) circle({1.5pt});
\end{scope} 
    \end{circuitikz}
    \caption{}
    \end{figure}


说明:能否这样连接取决于能否证明任一时刻$u_{AB}=u_{AC}+u_{CB}$。
实际上三角形连接是将各相线圈头尾相接的,
为证明方便,我们先假定将第一相头尾与电路脱离开。如
图3.56乙所示,那么

第一相头尾电压瞬时值为:$u_{AB}=U_m\sin\omega t$.

第二相与第三相头尾相接后$BA$两点电压(注意顺序)瞬
时值为:
\[\begin{split}
    u_{BA}&=u_{BC}+u_{CA}\\
&=U_m\sin (\omega t-120^{\circ})+U_m\sin (\omega t-240^{\circ})\\
&=U_m[\sin (\omega t-120^{\circ})+\sin (\omega t-240^{\circ})]\\
&=U_m\cdot 2\sin(\omega t-180^{\circ})\cdot \cos 60^{\circ}\\
&=U_m\cdot 2\cdot (-\sin\omega t)\cdot \frac{1}{2}\\
&=-U_m\sin\omega t
\end{split}\]
\[u_{AB}=-u_{BA}=U_m\sin\omega t=U_{A'B'}\]
上式表明$AB$两点电压变化与$A'B'$两点电压变化是完全
相同的,即任一时刻$AB$两点电压与$A'B'$两点电压相同,因
此可以将$A$与$A'$点相接,$B$与$B'$相接而形成电源的三角形
连接。
    \end{solution}
    
\end{enumerate}



\subsection{练习十}
\begin{enumerate}
    \item 有一台感应电动机,铭牌上标有“220/380”“$\triangle$/Y”的字祥.这表示这台电动机每相定于线圈的额定电压是 220伏,如果线电压是220伏,定子线圈要连成三角形;如果线电压是380伏,定子线圈要连成星形,为什么?


    \begin{solution}
当定子线圈接成三角形时,加在每相定子线圈上的
电压与供电电路的线电压相同,所以供电电路的线电压为
220伏时,加在每相线圈上的电压也是220伏,就可使电机正
常运转.如果供电电路的线电压是380伏,就需要把电动机定
子线圈连成星形。这是因为,星形连接时,每相线圈的电压与
供电电路线电压关系为$U_{\text{线}}=\sqrt{3}U_{\text{相}}$. 所以$U_{\text{相}}=380/\sqrt{3}=220$伏,才可使电机正常运转.
    \end{solution}
    
    \item 实验表明:只要把定子上任意两组线圈的电流互换一下,旋转磁场就向相反方向旋转.画出类似于课本图3.56的
    图,并加以分析说明.


    \begin{solution}
    课本图3.56中,$AX$绕组通以$i_{AX}$, $BY$通以$i_{BY}$,
$CZ$通以$i_{CZ}$. 那么产生磁场将沿逆时针方向旋转。现在,将
某两个组定子绕组中通过的电流互换,例如将$BY$绕组中通
以$i_{CZ}$电流,而$CZ$绕组中通以$i_{BY}$电流,则其旋转磁场情况
如下:
\begin{itemize}
\item 在$t=0$时刻,$AX$中电流为零;$BY$中电流$i_{CZ}$, 由$B$端流
入、$Y$端流出;$CZ$中电流$i_{BY}$, 由$Z$端流入、$C$端流出。合成
磁场方向如图3.57甲中虚线所示,其方向指向$X$.

\item 在$t=T/3$时刻,$CZ$中电流为$i_{BY}=0$; $AX$中电流$i_{AX}$, 
由$A$端流入、$X$端流出;$BY$中电流$i_{CZ}$由$Y$端流入、$B$端流
出,合成磁场方向如图3.57乙中虚线所示,其方向指向$Z$.
\end{itemize}

显然经过$T/3$, 合成磁场方向指向沿顺时 针方向由$X$移
至$Z$, 转过$120^{\circ}$. 图3.57丙和丁中虚线分别表示在$2T/3$及$T$
时刻合成磁场情况,可以看出经过一个周期$T$, 合成磁场也相
应沿顺时针方向转过一圈。因此任意对调两个绕组中的电流,
就可使旋转磁场方向改变转向。
\begin{figure}[htp]
    \centering
    \includegraphics[scale=.7]{fig/3-57.png}
    \caption{}
\end{figure}
    \end{solution}
    
    \item 在课本图3.54所示的实验中,铝框总要比磁铁转得慢,即比旋转磁场转得慢,而不能跟磁场转得一样快,即不能同步旋转,因此感应电动机也叫异步电动机,试说明铝框为什么不能跟磁铁同步旋转.

    \begin{solution}
使铝框转动的旋转力矩,是由于在铝框中产生的感
生电流受到磁场力而引起的。因此,铝框与磁铁间有相对运
动而发生切割磁力线现象是铝框转动的前提。如果某一瞬间
铝框与旋转磁场同步转动,那么,铝框就不切割磁力线,感生
电流也就消失了,所受磁场力矩随之消失。在阻力矩作用下,
铝框转速会减小,铝框转速减小了,又会切割磁力线产生感
生电流,且随着转速减小,电流逐步增大,铝框所受磁场力矩
也将增大,直至转速减小到一定程度,使电磁力矩和阻力矩平
衡,而使铝框维持匀速转动为止。这就是铝框转速永远小于
磁场转速,不能同步旋转的道理。 
    \end{solution}
    
\end{enumerate}




\subsection{习题}
\begin{enumerate}
    \item 把电阻和电容器串联在交流电路中,测得电阻上的电压为30伏,电容器上的电压为40伏.已知电阻的阻值为200欧,交流电的频率为50赫,求电容器的电容,这道题提供了一种用伏特表测定电容的方法.


    \begin{solution}
在交流电路中,任一瞬间通过串联电路各处的电流
都是相同的。电路中电流的有效值为
\[I=\dfrac{U_R}{R}=\frac{30}{200}=0.15{\rm A}\]
对于电容器部分,可以看成一个纯电容电路,因而有:
\[I=\frac{U_C}{X_C}=\frac{U_C}{\dfrac{1}{2\pi fC}}=U_C\cdot 2\pi fC\]
\[C=\frac{I}{U_C\cdot 2\pi f}=\frac{0.15}{40\x 2\pi\x 50}=12\x ^{-6}{\rm F}=12{\rm mF}\]
    \end{solution}
    
    \item 把电阻和电容器并联在交流电路中,已知电阻的阻值是500欧,电容器的电容是30微法.当交流电的频率为50赫时,通过电阻和电容器的电流之比是多大?当交流电的频率是500赫时,电流之比又是多大?


    \begin{solution}
在交流电路中,并联各支路两端电压在任一瞬间都
是相同的,因此设电阻与电容器两端电压为$U$. 因为
\[I_R=\frac{U}{R},\qquad I_C=U\cdot 2\pi fC\]
所以
\[\frac{I_R}{I_C}=\frac{U/R}{U\cdot 2\pi fC}=\frac{1}{R\cdot 2\pi fC}\]
当电源频率为50赫时
\[\frac{I_R}{I_C}=\frac{1}{R\cdot 2\pi fC}=\frac{1}{500\x 2\pi\x 50\x 30\x 10^{-6}}\approx 0.21\]
当电源频率为500赫时
\[\frac{I_R}{I_C}=\frac{1}{R\cdot 2\pi fC}=\frac{1}{500\x 2\pi\x 500\x 30\x 10^{-6}}\approx 0.021\]
    \end{solution}
    
    \item 变压器的原线圈为1100匝,副线圈为180匝,原线圈接到220伏的交流电路中,副线圈上并联了3个阻值都是90欧的用电器.如果原线圈允许通过的最大电流为0.9安,副线圈上最多还可以并联多少个阻值为60欧的用电器?


    \begin{solution}
根据变压比公式$\dfrac{U_1}{U_2}=\dfrac{n_1}{n_2}$
可得:
副线圈电压
\[U_2=\frac{n_2}{n_1}\cdot U_1=\frac{180}{1100}\x220=36{\rm V}\]
根据变流比公式$\dfrac{I_1}{I_2}=\dfrac{n_2}{n_1}$
可得:
副线圈允许通过的最大电流为
\[I_2=\frac{n_1}{n_2}\cdot I_1=\frac{1100}{180}\x 0.9=5.5{\rm A}\]
3个90欧的用电器并联在副线圈上,副线圈上通过的
电流为:
\[I=\frac{U_2}{R_{\text{并}}}=\frac{36}{90/3}=1.2{\rm A}\]
副线圈还可增加负载电流 
\[I'=I_2-I=5.5-1.2=4.3{\rm A}\]
设还可并联$m$个60欧用电器,其总电流不超过4.3安.即
\[m\cdot\frac{U_2}{60\Omega}\le I'\]
所以
\[m\cdot \frac{36}{60}{\rm A}\le 4.3{\rm A}\]
即:$m\le 7.17$.

因此,最多还可并联7个60欧的用电器.
    \end{solution}
    
    \item 有一个教学用的可拆变压器,它的原副线圈外部还可绕线,现在要测定原副线圈的匝数.现有一根足够长的绝缘导线,还需要什么器材?简要说明实验原理.

    \begin{solution}
还需一块能测较高电压的交流电压表和一块可测几
伏特的小量程电压表,将绝缘导线在副线圈(或原线圈)外部
绕几十匝,将变压器重新组装好后,按原线圈规定电压接入电
源,分别实测原线圈两端、副线圈两端和加绕线圈两端电压
值,根据变压比公式就可以计算出原副线圈匝数。

由$\dfrac{U_1}{U_2}=\dfrac{n_1}{n_2}$可得$n_1=\dfrac{U_1}{U_2}\cdot n_2$. 其中$U_1$、$U_2$可实测,$n_2$
为加绕线圈匝数,便可求出原副线圈匝数。

注意:实际上进行这个实验时,需要根据所给变压器的
-具体情况和要求测量的准确程度进行测量,所用伏特计一般
采用多量程的,如万用表交流电压挡。实验电源也可以用低
压电源,将低电压加在加绕线圈上进行测量,如果仅有一块
单量程较高电压的电压表,也可以用如下方法测量:

将加绕线圈与原线圈(或副线圈)串联起来。测出原线圈
两端电压和串联绕组总电压,由下式求出原线圈匝数。
\[\frac{U_{\text{原}}}{U_{\text{总}}-U_{\text{原}}}=\frac{n_{\text{原}}}{n_{\text{加}}}\]
    \end{solution}
    
    \item 课本图3.59是电工常用的钳形电流表,可以用来测定交变电流,把钳口打开,把被测的通电导线放在钳口中间(右图),交流电表就可以测出导线中的电流强度.试说明钳形电流表的工作原理.

\begin{solution}
    钳形电流表实际上是一种电流互感器,用以测定强
    交流电电流。当导线中有交变电流通过时,在环绕导线的铁
    心中有变化磁通,从而使绕在铁心上的线圈产生感生电流,接
    在绕组两端的电流表即可读出读数。通电导线中电流越大,
    在铁心中磁通变化率就越大,感生电流也就越大,电流表的读
    数相应就越大,这就是钳形电流表的工作原理。
\end{solution}

    \item 在课本图3.35所示的桥式整流电路中,变压器原副线圈的匝数比等于8,原线圈接在220伏的交流电路中,能不能选用最高反向工作电压为50伏的晶体二极管进行整流?为什么?

\begin{solution}
    副线圈两端电压
    \[U_{\text{副}}=\frac{n_{\text{副}}}{n_{\text{原}}} \cdot U_{\text{原}}= \frac{1}{8}\x220=27.5{\rm V}\]
在课本图3.35所示桥式整流电路中,每个二极管承受的反向
电压可从面分析得知:

在交流电压正半周时,$D_1$与$D_3$导通,这两个二极管两端
电势可认为相同(忽略二极管自身压降),而$D_2$与$D_4$反向截
止,相当于跨接在$a,b$两点间,因而其反向承受最大电压为
变压器次级电压最大值.同理可知$D_1$与$D_4$情况,即
\[U_{km}=U_m=\sqrt{2}\cdot U_{\text{副}}=\sqrt{2}\x27.5=38.9{\rm V}<50{\rm V}\]
因此可以选
用最高反向工作电压为50伏的晶体二极管进行整流。
    \end{solution}
    
    \item 课本图3.60是电子技术中用到的限幅电路,电池组的电动势都为$\mathcal{E}$,左端输入的是正弦交流电,电压$u_1$的最大值为$2\mathcal{E}$.试画出右端输出的电压$u_2$的图象,并分析说明理由.电池组的内电阻略去不计.

\begin{solution}
右端输出电压$u_2$的图象如图3.58乙所示.图甲表
示输入端$u_1$的图象。

\begin{figure}[htp]
    \centering
\begin{tikzpicture}[>=latex, xscale=1.7, yscale=1.3]
\begin{scope}
    \draw[->](0,-1.5)--(0,1.5)node[right]{$u_1$};
    \draw[->](0,0)node[left]{0}--(3.5,0)node[right]{$t$};
\draw[very thick, domain=0:3, samples=500] plot(\x,{sin(\x*pi r)});
\draw[dashed](0,1)node[left]{$2\mathcal{E}$}--(2.8,1);
\draw[dashed](0,-1)node[left]{$-2\mathcal{E}$}--(1.8,-1);
\node at (1.5,-1.5){甲};
\draw(0,-.5)node[left]{$-\mathcal{E}$}--(.1,-.5);
\draw(0,.5)node[left]{$\mathcal{E}$}--(.1,.5);
\node at (1,0)[below left]{$\dfrac{T}{2}$};\node at (2,0)[below right]{$T$};
\end{scope}
\begin{scope}[xshift=4.5cm]
    \draw[->](0,-1.5)--(0,1.5)node[right]{$u_2$};
    \draw[->](0,0)node[left]{0}--(3.5,0)node[right]{$t$};
\draw[very thick, domain=0:1/6, samples=50] plot(\x,{sin(\x*pi r)});
\draw[very thick](1/6,.5)--(5/6,.5);
\draw[very thick, domain=5/6:7/6, samples=50] plot(\x,{sin(\x*pi r)});
\draw[very thick](7/6,-.5)--(11/6,-.5);
\draw[very thick, domain=11/6:13/6, samples=50] plot(\x,{sin(\x*pi r)});
\draw[very thick](13/6,.5)--(17/6,.5);
\draw[very thick, domain=17/6:3, samples=50] plot(\x,{sin(\x*pi r)});
\node at (1,0)[below left]{$\dfrac{T}{2}$};\node at (2,0)[below right]{$T$};
\draw[dashed](0,.5)node[left]{$\mathcal{E}$}--(2.8,.5);
\draw[dashed](0,-.5)node[left]{$-\mathcal{E}$}--(1.8,-.5);
\draw(0,-1)node[left]{$-2\mathcal{E}$}--(.1,-1);
\draw(0,1)node[left]{$2\mathcal{E}$}--(.1,1);
\node at (1.5,-1.5){乙};
\end{scope}
\end{tikzpicture}
    \caption{}
\end{figure}

当$u_1$处于正半周时,如果$u_1<\mathcal{E}$, 对$D_2$所在支路来说,由
于$D_2$始终处于反向电压作用,因此$D_2$截止,该支路被视为断
路状态,对于$D_1$所在支路来说,由于$D_1$正极电势仍低于负极
电势,因而也处于截止状态,所在支路也可视为断路状态,因
此$R$上几乎没有压降,$u_2$与$u_1$电压变化相同.如果$u_1$增大
至$u_1>\mathcal{E}$, $D_2$仍然受反向电压的作用,$D_2$所在支路仍可视为断
路状态,而对$D_1$所在支路,由于$D_1$正极电势大于负极电势,
$D_1$导通.$R$两端电压为$u_1-\mathcal{E}$, $R$中有电流通过.而输出端电
压由于电池组作用而基本维持与电动势$\mathcal{E}$相同。换言之,当
正半周$u_1<\mathcal{E}$时,输出与输入相同,当$u_1>\mathcal{E}$时,输出端电压
维持$\mathcal{E}$不变,从而起到限幅作用,同理可分析负半周情况。
\end{solution}

    \item 在课本图3.46所示的三相四线制照明电路中,设$A$相接通了8盏“220V,100W”的灯泡,$B$相接通了2盏“220V,100W”的灯泡,$C$相中没有接通灯泡,这时接通的灯泡都正常发光,因某种原因中性线断开了(在该图中$O$处断开),将会发生什么现象?说明理由.
 
    \begin{solution}
如果中性线断开,$A$相与$B$相灯泡串联起来,串联
电路两端电压将是线电压380伏,由于两组灯泡是串联,其电
阻之比为
\[R_A:R_B=\frac{R}{8}:\frac{R}{2}=1:4\]
式中$R$表示每一灯泡电阻,因而这两组灯泡电压分配之比
\[U_A:U_B=R_A:R_B=1:4\]
所以$A$相灯泡两端电压为\[U_A=\frac{1}{5}\x 380=76{\rm V}\]
$B$相灯泡两端电压为\[U_B=\frac{4}{5}\x 380=304{\rm V}\]
因此,$B$相灯泡将因电压过高而烧毁。

由此可见,在三相四线制供电电路中,中性线在任何情
况下不允许断开,以免造成用电器因电压变高而烧毁的现象
发生。这也是中性线不允许安装开关和保险丝的原因。
    \end{solution}
    
    \item 如课本图3.61所示,电源采用星形连接,负载采用三角形连接.电源的相电压是220伏,各相负载相同,阻值都是110欧,求通过各相负载的相电流和线电流.

\begin{solution}
    电源部分的线电压
   \[ U_{\text{线}}=\sqrt{3}U_{\text{相}}=\sqrt{3}\x220=380{\rm V}\]
   负载电路是三角形连接,其相电压与线电压相同,
\[  U'_{\text{相}}=U'_{\text{线}}=220\sqrt{3}=380{\rm V}\]
   负载相电流
   \[I_{\text{相}}=\frac{U'_{\text{相}}}{R}=\frac{220\sqrt{3}}{110}=3.46{\rm A}\]
   负载线电流 
\[I_{\text{线}}=\sqrt{3} I_{\text{相}}=\sqrt{3}\x\frac{220}{110}\sqrt{3}=6.00{\rm A}\]
\end{solution}

    \item 每一台电动机都有一定的额定功率.在实际中要根据负载的功率来选择电动机的功率,使电动机的额定功率等于或稍大于负载的功率,有一台水泵,抽水量$Q=0.03{\rm m^3}/{\rm s}$,抽水高度$h=20{\rm m}$,效率$\eta_1=0.55$,用一台感应电动机道过皮带传动来带动,皮带传动的效率$\eta_2=0.8$.现有三台感应电动机,额定功率分别是14千瓦、20千瓦、28千瓦,应当选择哪一台?计算时取$g=10\msq$.

    \begin{solution}
设电动机的功率为$P$. 那么根据效率的意义可知有
用功率
\[P_{\text{有}}=\eta_1\cdot \eta_2\cdot P\]
由于有用功率在单位时间内使一定量的水抽至高处增加
水势能,因此
\[P_{\text{有}}=\frac{mgh}{t}=\frac{\rho V\cdot gh}{t}=\rho gh\cdot Q\]
所以
\[\eta_1\cdot \eta_2\cdot Ρ=\rho \cdot g\cdot h\cdot Q\]
\[P=\frac{\rho \cdot g\cdot h\cdot Q}{\eta_1\cdot \eta_2}=\frac{1.0\x 10^3\x 10\x 20\x 0.03}{0.55\x 0.8}=13.6\x 10^3{\rm W}=13.6{\rm kW}\]
所以应选用14千瓦的电动机来带动这台水泵。
    \end{solution}    
\end{enumerate}



\section{参考资料}
\subsection{我国电力工业发展的一些资料}

旧中国的电力工业极端落后.中国的火力发电始于
1882年,在外国人建立的上海电光公司中安装了一台供应16
盏弧光灯的发电机,水力发电始于1912年,在距云南昆明40
公里的螳螂川上建成石龙坝水电站.从1882年到1949年的
68年中,电力工业发展十分缓慢,到1949年全国发电量才达
43亿千瓦时,均每人每年仅8.6千瓦时,全国发电设备容
量仅185万千瓦,多数设备还残缺不全,不能正常运转,而且
分布极不均衡,80\%以上集中在沿海省市。

建国以后,在党和政府的领导下,我国电力工业得到迅速
发展.1986年全年发电量达4455亿千瓦时,为1949年的
103.6倍.现在每四天发电量就超过1949年一年的发电量。
110千伏以上高压输电线路1949年为1937公里,1983年增
加到11.6万公里,增长59倍,这在世界先进国家电力工业
发展史上也是少有的。截止到1983年,我国现有发电设备容
量达7644千瓦,相当于1949年的41.3倍,由建国初期居世界
第21位跃升到第八位;发电量由第25位跃升到第六位,仅次
于美国、苏联、日本、联邦德国和加拿大,大体达到了美国
1949年、苏联1963年和日本1970年的水平。

1983年全国水电装机容量已达2416万千瓦,是1949年
16万千瓦的151倍。1986年,全国水电发电量已达932亿千
瓦时,是1949年7.1亿度的131倍。

截止到1983年,全国已有500千瓦以上的电厂4315
座,其中装机容量在10万千瓦以上的水力发电厂42座,火力
发电厂128座,全国形成六大电网(见后).我国高压远距离
输电从解放初的35千伏线路发展到大量建设110千伏、220
千伏、330千伏超高压线路.近年又发展到500千伏超高压
线路,建成总长度1594公里。直流输电在我国过去是个空
白,七十年代在上海市内进行了31千伏直流输电试验,八十
年代开始在舟山建造100千伏直流输电试验工程,并着手兴
建葛洲坝至上海的正负500千伏超高压工程。

多年来,我国为发展核电创造了必要的条件。目前正在
兴建广东大亚湾核电站(装机容量为180万千瓦)和浙江秦山
核电站(装机容量30万千瓦).

在我国第七个五年计划(1986年—1990年)中,电力工
业是能源工业发展的中心,要积极发展火电,大力开发水电,
有步骤有计划地建设核电站.争取到1990年全国发电量达到
5500亿千瓦时.

我国年发电量发展情况:
\begin{center}
\begin{tabular}{c|ccccc}
    \hline
   & 1949年&    1952年&    1978年&    1983年&    1986年\\
   \hline
   发电量  &43.1&   72.6 &  2565.5  & 3514   &    4455\\
   (亿千瓦时)&  其中水电7.1 &    &    &   其中水电864 &    其中水电932\\
   \hline
\end{tabular}
\end{center}

1985年若干省市发电量情况:
\begin{center}
\begin{tabular}{c|cccccc}
    \hline
   & 北京&   河北&   河南&   山西&   辽宁&   湖北\\
   \hline
   发电量 (亿千瓦时) &103.3&   262.9&   208.45&   184.6&   352.1&   220.54\\
   \hline
\end{tabular}
\end{center}

我国七大电网:
目前我国六大电网已初具规模,这六大电网是:
\begin{enumerate}
\item 华北电网 由京津唐、石邯、山西和内蒙呼包电网组成。
“七五”计划期间,华北、西北电网将并联,再与东北电网并联,
形成我国北部跨越九省两市的大型电网。
西北电网 由陕西、甘肃、天水、西宁地区电网合并成的
陕甘青、宁夏石银青电网组成,这两个电网的装机容量均超
过500万千瓦.1983年初,西北电网和西南电网以22万伏线
路并联。
\item 东北电网 由原东电网主网发展而成.1981年装机
容量达880万千瓦,并人黑龙江东部及吉林延边的“佳鸡牡
延”电后,容量可达1000万千瓦.
\item 华东电网 在上海电网基础上发展起来的,由上海、江
苏、安徽、浙江电网组成,已装机990多万千瓦,还将与山东电
网、福建电网并联,“七五”计划期间,华东电网、华中电网将以
50万伏线路并联,形成我国中部东西大跨区电网.
\item 华中电网 由湖北、河南、湖南、江西电网组成,这四个电
网装机容量均超过1000万千瓦.
\item 西南电网 由四川、贵州、云南电网组成。
\item 另外,华南电网正在建设之中,它由广东、广西电网组成。
目前已装机300多万千瓦.红水河梯级电站投产后,将形成
近2000万千瓦电网.
\end{enumerate}

根据“七五”计划,华中与华东,西南与华南,西北、华北
和华中将在“七五”期间实行联网,到2000年,全国将实现大
联网。

西藏、新疆与台湾在较长时间内仍将单独发供电。

我国已建和在建的装机容量超过100万千瓦的大型
水电站有三个:黄河上游的龙羊峡水电站,装机容量达128
万千瓦,年发电量60亿千瓦时,坝高达175米,刘家峡水电
站,装机容量达116万千瓦,年发电量57亿千瓦时,坝高达
147米,正在兴建的我国最大水电站长江中游葛洲坝水电
站,装机容量达271.5万千瓦,由2台17万千瓦与19台12.5
万千瓦的发电机组组成,年发电量可达140亿千瓦时.


\subsection{交流电的平均值与有效值}
交流电的平均值与有效值概念在教学中易造成混淆。一
个变化的物理量对时间取平均值究竟有没有意义,要看实际
而定。对正弦交流电来说,其电流(或电压、电动势)按正弦规
律变化,在一个周期内取平均值,显然为零,就失去其意义了。
但在考虑电流的热效应时,由于瞬时功率$P_i=i^2R$, 因而产生
热量不是与电流平均值有关,而是与电流平方的平均值有关
的,因此,交流电热功率的平均值是有意义的,它相当于某一
直流电通电阻时产生热功率。由于
\[\sin^2\omega t=\frac{1}{2}-\frac{1}{2}\cos2\omega t\]
所以交流电瞬时功率为:
\[\begin{split}
    P_i=i^2R &=I^2_m\cdot  R\cdot \sin^2 \omega t\\
    &=I^2_m\cdot  R\cdot \left(\frac{1}{2}-\frac{1}{2}\cos2\omega t\right)\\
    &=\frac{I_m^2 R}{2}-\frac{I_m^2 R}{2}\cdot \cos2\omega t
\end{split}\]
上式中,后一项在一个周期内平均值为零,因此,在一个周期
内,交流电平均功率为:
\[\bar P=\frac{I_m^2 R}{2}\qquad (\text{最大瞬时功率的一半})\]
如果考虑一个直流电与其等效,即$P=I^2\cdot R$, 就有$P=\bar P$
即:
\[I^2\cdot R=\frac{I^2_m\cdot R}{2}\]
所以
\[I=\frac{I_m}{\sqrt{2}}\approx 0.707I_m\]
这个与交流电通过电阻时产生热功率等效的直流电流值
$I$, 称为交流电的有效值,同样可知电压有效值与电动势有效
值分别为:
\[U=\frac{U_m}{\sqrt{2}},\qquad \mathcal{E}=\frac{\mathcal{E}_m}{\sqrt{2}}\]

从上面讨论知道,交流电有效值概念是从交流电热功率
平均值与直流电热功率等效角度引入的。在另外一些场合
下,例如整流电路中,交流电经过半波整流或全波整流,其电
流变为单向波形,这时,交流电流(或电压)在$T/2$时间内的平
均值就显得很有意义了,这个平均值应为$T/2$时间内通过的
总电量与时间$T/2$的比值,即
\[\begin{split}
\bar i=\frac{2Q}{T}&=\frac{2}{T}\cdot \int^{T/2}_0 I_m Sm\frac{2\pi t}{T}\dd t\\
&=\frac{2}{T}\cdot \frac{I_m T}{2\pi} \left[-\cos\frac{2\pi t}{T}\right]^{T/2}_0\\
&=\frac{I_m}{\pi}\cdot (1+1)=\frac{2}{\pi}\cdot I_m
\end{split}
    \]
相应电压平均值为:
\[\bar u=\frac{2}{\pi}U_m\]

在讨论整流电路及有关计算时经常运用交流电电流平均
值与电压平均值,由于日常情况下交流电均用有效值,这两个
值的关系是:
\[\bar i=\frac{2\sqrt{2}}{\pi}I\approx 0.90I,\qquad \bar u=\frac{2\sqrt{2}}{\pi}U\approx 0.90U\]

从以上讨论可知,需根据实际物理意义来理解交流电的
平均值与有效值的具体含义。

\subsection{变压器匝数的问题}
变压器的铁心截面与初级匝数都是有一定要求的,不能
简单地根据变压比公式随意选择.如果要将220伏市电变为
110伏电压输出,可否用初级线圈为2匝,次级线圈为1匝的
变压器呢?下面简单地作一讨论:

对变压器初级线圈来说,当外加电压$U$为正弦交流电时,
线圈中的电流为正弦交流电流,这个电流将激起变压器铁心
中的磁通变化。由于磁通变化又产生自感电动势$E$, 它是一
个反电动势,始终与外加电压相平衡,即$E=U$.

设$\phi=\Phi_m\sin\omega t$(不考虑漏磁通),
若初级线圈有$N$匝,则自感电动势即时值
\[\begin{split}
    e=-N\cdot \frac{\dd\phi}{\dd t}&=-N\frac{\dd}{\dd t}\Phi_m\sin\omega t\\
&=-N\Phi_m\omega \cos\omega t\\
&=N\cdot \Phi_m\omega \sin(\omega t-90^{\circ})\\
&=2\pi f N \Phi_m \sin(\omega t-90^{\circ})
\end{split}\]

式中$2\pi f N \Phi_m$为自感电动势最大值,记作$E_m$, 则自感电
动势有效值为
\[E=\frac{E_m}{\sqrt{2}}=\frac{2\pi}{\sqrt{2}}f N \Phi_m\approx 4.44 f N \Phi_m\]

故外加电压有效值与频率$f$、匝数$N$、铁心中最大磁通
$\Phi_m$关系为
\[U=E\approx 4.44fN\Phi_m=4.44fNB_mS\]
上式是电工计算时的一个重要关系式,由上式可以看到,
当初级线圈所加电压$U$一定、频率$f$一定时,匝数$N$与$B_mS$
成反比,对各种材料的铁心,为防止磁饱和,$B_m$有一定允许值,
例如优质热轧硅钢片可取到1.0—1.2特,因此在铁心材料与
截面$S$选定的情况下,初级线圈匝数可由上式计算出来,而并
不是根据变压比公式随意选取的。

设计小型变压器要进行计算,计算主要依据以下三个
公式:

铁心截面:由铁心材料、铁心形状及变压器设计功率
决定,一般用下式计算:
\[S=\frac{K}{\sqrt{P_0}}\]
式中,$K$根据铁心材料形状而定,数值一般取1—2. 对于热
轧硅钢片,$K$值取1.25至1.5, 对于冷轧硅钢片,质量好的可取
小些,对于一般黑铁片$K$值可取为2左右.式中$P_0$单位为
瓦,$S$单位为${\rm cm^2}$.

初级线圈每伏匝数:由$U=4.44fNB_mS$,得:
\[N=\frac{U}{4.44fB_mS}\]
取$f=50{\rm Hz}$,
\[N\approx \frac{0.0045}{B_mS}\cdot U \text{(匝)}\]
每伏匝数
\[N_0=\frac{N}{U}=\frac{0.0045}{B_mS}(\text{匝/伏})\]
式中,$B_m$单位为T,$S$单位为${\rm m^2}$.

根据每伏匝数计算初次级线圈应绕匝数,但次级线圈匝
数要加5\%, 用来抵消阻抗损失。

导线截面直径
\[d=1.13\sqrt{\frac{I}{j}}\]
式中$I$为通过该绕组电流强度,单位为安,$j$为导线单位
截面上允许通过的电流强度,称电流密度,一般100瓦以下变
压器,$j$取$2.5{\rm A/mm^2}$, 100瓦以上小型变压器$j$取$2{\rm A/mm^2}$。由上式求得导线直径$d$的单位为毫米。

不能根据变压比公式随意选取匝数,还可以从另外一个
角度进行粗浅解释。对于理想变压器,根据变流比公式可知,
次级电流越小,初级电流也越小,次级空载时,初级电流并不
为零,这时相当一个自感线圈,正是存在这个初级电流,才能
产生自感电动势(反电动势)与外电压平衡,我们当然希望这
个电流越小越好,这样损失就少,由纯电感电路$I=\dfrac{U}{2\pi fL}$可知,这需要线圈有足够大的自感系数,并且满足$X_L\gg R$的关
系,而自感系数$L$的大小是与铁心、匝数有关的。匝数越多,铁
心越大,自感系数越大。因而变压器初级匝数铁心不是随意选
取的。另一方面也不是匝数越多越好,铁心越大越好,这是因
为还有一个经济原则,具体选择,需根据理论作出计算。

\subsection{特斯拉(1856—1943)}

特斯拉是美国的发明家,1856年7月10日生于南斯拉
夫克罗地亚的斯米良,1880年毕业于布拉格大学,1884年在
美国定居。

特斯拉在科学技术上的最大贡献是开创了交流电体系,
他发现了旋转磁场,被称为迎来电力时代的天才.1883年在
斯特拉斯堡工作时,利用业余时间制成了第一台感应电动机。
十九世纪八十年代在爱迪生的推动下,直流电已普遍使用。但
直流电动机价格昂贵、使用麻烦,每1平方英里的范围内,需
要一个单独的发电机供电,很不经济.1885年,特斯拉出 售
了他的多相交流发电机、变压器和电动机的专利权,触发了交
流电体系和直流电体系的激烈竞争。由于交流电既安全可靠,
又能用细导线实现远距离输电,使交流电的应用得到了迅速
发展。

特斯拉一生的发明很多:发明了电话系统的增音器,提
出了无线电通讯的基本要素是天线、发射回路和接收回路。他
发明的特斯拉线圈,广泛用于无线电、电视机中.1900年,他
还证明了大地能作导体。他还发明了供医疗用的高频电热疗
法,等等,总共获得112项美国专利.他把出售专利获得的资
金用于研究技术,表现了科学家的良好品格。他设想过科学
的某些未来课题,如对其他行星的通讯等,在他留下来的笔
记中,有大量创造性的天才设想。

特斯拉在晚年经济上经常处于困境,1943年1月7日,
他孤独地在旅馆里去世。有三位诺贝尔奖获得者在给他的祭
文中说:“这位世上显赫的智者之一,为现代技术发展的许多
方面铺平了道路。”

南斯拉夫人民为纪念他,在贝尔格莱德修建了“特斯拉纪
念馆”。在他诞生百年纪念时(1956年),国际电气技术协会
将M.K.S制系统中的磁通量密度命名为“特斯拉”。














\chapter{电磁振荡和电磁波}\minitoc[n]
\section{教学要求}

这一章在电场、磁场、电磁感应等知识的基础上,讲述电
磁振荡和电磁波,以及无线电发射和接收的初步知识,是电学
知识的继续,也是振动和波的知识的发展。这一章的知识又
为以后认识光的电磁本质作准备。

本章内容既讲述基本知识又讲解实际应用,全章知识多
是介绍性的,这是本章的特点,讲述基本知识时,应着重介
绍主要之点。这一章的核心内容是麦克斯韦电磁场的理论。
电磁波的实际应用中,需要解决许多实际技术问题,教材对解
决发射与接收的主要问题——开放电路、调制、调谐和检波
作了简要介绍,讲述时应着重介绍原理和解决问题的思路,
不宜过多地涉及技术细节。

本章教材可分为三个单元。第一单元包括第一节和第二
节,讲述电磁振荡的产生和$LC$振荡电路的周期和频率公式。
第二单元包括第三节至第五节,讲述麦克斯韦电磁理论要点
和电磁波的知识。第三单元包括第六节至第十二节,讲述电
磁波应用技术初步知识和发展简况。

$LC$回路可以产生振荡电流,要通过演示实验来使学生确
信。电磁振荡的产生过程,要突出电场能和磁场能的相互转
化。由于在中学不容易搞清楚$LC$回路中为什么电流最大时电
压最小这类问题,而进一步深究会加重学生负担,在教学中不
要过多探讨,对$LC$振荡电路的周期公式,应使学生掌握它
的物理意义。由于在后面学习谐时要用到改变电磁振荡周
期的内容,要使学生切实理解电磁振荡的周期怎样随电容和
感的改变而改变。

麦克斯韦电磁场的理论把电场和磁场统一起来,要让学
生了解麦克斯韦理论的两个要点,这两个要点都是用场的观
点分析得出的,虽然学生已经学过电场和磁场,但不熟悉用
变化的场分析现象。要使学生明确由变化磁场产生电场跟是
否有闭合电路无关,由变化的电场产生的磁场跟空间里存在
电流而产生的磁场是一样的,但无需提出位移电流的概念。教
材还指出了均匀变化的场产生稳定的场,非均匀变化的场产
变化的场,这是为讲电磁波作准备的。在分析时,要向学生
说明什么是均匀变化的场和非均匀变化的场,以帮助学生理
解怎样用变化的场分析有关现象。

在讲解电磁波的性质时,要突出它可以脱离电荷而独立
存在、具有能量、不需要别的物质做媒质等特点。正是电磁波
的这些特点,表明电磁场是客观存在的物质。赫兹实验证实
了电磁波的存在,确立了光的电磁说。有条件的学校应尽量
演示这个实验,使学生确信麦克斯韦理论的正确,也有助于认
识电磁场的物质性。

麦克斯韦电磁理论在教学中之所以重要,还在于它在思
想方法上给人们以重大的启示。许多重大发现的提出往往不
是在其理论系统完成之时,而是在人们根据种种物理现象定
性分析、深入思考并进行猜测之时,这是创造性思维的特点。
这一章无论是麦克斯韦理论的提出,还是$LC$电路产生电磁
振荡的物理过程,以及电磁波发送与接收过程的教学都要充
分重视从人们是怎样提出和思考问题的角度进行教学,避免
仅作一般知识介绍,这样会有利于进行创造能力的培养,对于
掌握教材内容也是有益的。

本章安排的学生实验是“安装简单的收音机”,其目的是
通过安装、调试,使学生确信接收原理,培养他们的动手能力,
而不在于熟悉具体的线路。因此,只要求了解实验线路图中
的可变电容器、二极管、三极管、耳机的作用,对线路中的其他
元件的作用不要求讲解。

本章的教学要求是:
\begin{enumerate}
\item 了解电磁振荡产生的过程,掌握电磁振荡的周期和频
率的公式。
\item 了解电磁场理论的要点,了解电磁波的产生和特点.
知道电磁场是一种物质形态。知道赫兹实验。
\item 了解无线电发射和接收的基本原理,了解无线电传播
的特性。
\end{enumerate}

\section{教学建议}
\subsection{第一单元~~ 电磁振荡}
这一单元的教材中,首先安排了$LC$振荡电路的演示
实验,以使学生获得$LC$振荡电路中产生振荡电流的初步感性
知识,在此基础上运用电容器充放电现象与电感线圈的自感
现象对$LC$电路产生振荡电流的物理原因作了定性分析,
并进一步从能量转化角度突出指出了$LC$电路产生振荡电流
的物理实质,由此引出了电磁振荡概念以及阻尼、无阻尼振荡
概念,说明了在$LC$振荡电路中维持电磁振荡的条件,可以看
出,教材的基本思路是:
\begin{center}
\begin{tikzpicture}[>=latex]
\node (A) at (0,0){物理现象};
\node (B) at (3,0){物理原因};
\node (C) at (6,0){物理实质};
\node (D) at (9,0){有关概念};
\draw[->](A)--(B);\draw[->](B)--(C);\draw[->](C)--(D);
\end{tikzpicture}
\end{center}
教材的结构是:
\begin{center}
\begin{tikzpicture}[>=latex, align=center]
\node (A) at (0,0){$LC$电路的\\[-1ex]振荡电流};
\node (B) at (3,0){电容与电感\\[-1ex]基本特性};
\node (C) at (6.5,0){电场能与磁场能\\[-1ex]的互相转化};
\node (D) at (9.5,0){电磁振荡};
\node (D1) at (12.5,0.5){无阻尼振荡};
\node (D2) at (12.5,-0.5){阻尼振荡};
\draw[->](A)--(B);\draw[->](B)--(C);\draw[->](C)--(D);
\draw[->] (D)-- (D1);\draw[->] (D)-- (D2);
\end{tikzpicture}
\end{center}




































































































































































































\chapter{光的反射和折射}
\minitoc[n]
\section{教学要求}
    本章讲述几何光学知识,包括两部分内容:一部分是几何
光学的基本规律,讲述光的反射定律和折射定律;另一部分是
运用基本规律来研究平面镜、球面镜、棱镜和透镜等基本光学
元件,这些元件在成像和控制光路方面各有其自身的规律和
特点.学习这些知识,可以便我们认识自然界和生活中的许
多光现象,理解基本光学元件的作用和光学仪器的原理.因
此,这一章知识具有实际意义.

全章教材分为五个单元.一、二节为第一单元,讲述光的
直线传播和光速.三、四节为第二单元,讲述光的反射和平面
镜、球面镜.五—八节为第三单元,讲述光的折射、全反射和
棱镜.九—十二节为第四单元,讲述透镜知识.十三、十四节
为第五单元,讲述眼睛、显微镜和望远镜的构造和成像原理.

本章教材相当大一部分内容是在复习初中知识的基础
上,进一步扩大、加深和提高的.光的直线传播、光的反射等
知识,是复习性的教材。光的折射则在初中学过的知识基础
上予以提高,引入了折射定律和折射率的概念,是进一步学习
棱镜、色散、透镜的知识基础。透镜是许多光学仪器的主要元
件,掌握透镜成像的规律,可以理解一些助视仪器的成像原
理.所以,光的折射和透镜成像是本章教学的重点.

这一章的教学要求是:
\begin{enumerate}
\item 了解光在同一媒质中沿直线传播的性质,会用光的
直线传播解释有关的现象。知道光在真空中的速度。
\item 掌握光的反射定律,会用反射定律解释有关的现象.
掌握平面镜成像的原理和规律。了解凹镜对光的会聚作用和
凸镜对光的发散作用,了解它们的成像规律,
\item 掌握光的折射定律和折射率的概念,了解媒质的折射
率与光速的关系;会用折射定律解释简单的现象。
\item 理解光的全反射现象,掌握临界角的概念和发生全反
射的条件,了解全反射现象的应用。
\item 了解棱镜对光的偏折作用和光通过棱镜后的色散
现象。
\item 了解凸透镜对光的会聚作用和凹透镜对光的发散作
用。了解透镜的主轴、光心、焦点和焦距,掌握透镜成像规律、
成像作图法和成像公式。
\item 了解眼睛、显微镜和望远镜的基本构造和成像原理,
了解近视眼和远视眼以及眼镜的作用。
\end{enumerate}

虽然学生每天都接触光现象,但是却不能直接看到光的
传播路径,讲解光的反射、折射和全反射,要做好演示,让学
生看到光的传播路径,以利于揭示规律,加深印象,同时也可
以提高学生的学习兴趣。

在几何光学的教学中,研究成像是一个重要的问题,要让
学生了解成像规律和所成像的虚实。但是要注意,不要机械
地记忆什么镜成什么像,而要着重让学生从根本上弄清实像
和虚像的含义,掌握分析成像的规律和成像性质的方法。

在反射和折射现象中光路的可逆性对分析实际问题很有
用。教学中要通过具体例子让学生领会到这一点。

媒质的折射率是个重要的概念,要让学生掌握绝对折射
率的概念,为了使学生了解媒质的折射率总是相对于另一种
媒质而言的,教材介绍了相对折射率,教学中不要求把相对
折射率与绝对折射率的关系作为讲述重点,也不要求利用相
对折射率进行计算。

学生对于全反射现象是比较生疏的。教学中应通过演示
让学生认识这个现象,理解产生全反射现象的条件。在演示
时,还应引导学生注意观察光强分布的变化情况,这对于正确
理解全反射现象是有帮助的。光导纤维是全反射现象的应
用,教学中可根据实际情况,向学生介绍这一新技术的发展情
况,以开阔学生的眼界。

透镜是许多光学仪器的重要元件。在透镜成像的教学
中,要求学生会用三条特殊光线求出像的位置、大小和倒正,
判断像的虚实。对于透镜成像公式,应使学生能独立地把它
推导出来,知道什么情况下$f$取正值,什么情况下$f$取负值,
并能根据$v$的正负,判断出像的虚实、倒正和位置。

最后一个单元介绍眼睛、显微镜和望远镜。讲眼睛时,主
要从物理方面讲解眼睛的成像原理,眼睛的调节以及近视眼、
远视眼和眼镜的作用,不要过多地牵涉与生理机制有关的内
容,对于显微镜和望远镜,主要学习它们的基本构造和成像
原理。要求学生能看懂光路图,了解视角放大的情况。如果
有条件,最好在教学中能结合望远镜和显微镜的实物来讲解,
并介绍一些使用和维护方面的知识。

\section{教学建议}
\subsection{光的直线传播和光的速度}
光的直线传播是几何光学的基础,应通过演示实验让
学生看到光传播的路径。演示内容应包括:
\begin{enumerate}
\item 光在同一种均
匀媒质中沿直线传播;    \item 光到达两种媒质的界面,发生反射和
折射;    \item 条件较好的学校还可演示光在非均匀媒质中传播时
光线发生弯曲的现象(为阅读教材“海市蜃楼”做准备)。
\end{enumerate}


光线是几何光学的最基本的概念之一,它为应用几何
学的方法研究光的传播奠定了基础,光的反射定律和折射定
律等都是应用光线的概念来表述的,要使学生理解:
\begin{enumerate}
\item 光线
的概念建立在光的直线传播的基础上,所以,作图时画光线必
须要有表示光传播方向的头。    
\item 光线和光是有区别的,光
线只代表了光的传播方向,并不是实际存在的东西,而光是实
际存在的。    
\item 有时用光线代表很窄的平行光束的传播路径,
从这个意义上讲,光线是光束的抽象;有时又用两条光线代表
一束光束,这两条光线表示光束的边缘,也表示出光束的会
聚、发散或平行的性质。
\end{enumerate}

要注意应用几何学的方法研究与光的传播有关的实
际问题。在对影、本影和半影等问题进行分析时可参考下面
五点逐步深入:
\begin{enumerate}

\item 点光源发出的光照到不透明物体上,在物
体背后的光屏上可以看到光线照不到的黑暗区域——影子。
作出光路图,说明影子的范围;
\item 如果物体大小是已知的,应
用相似形的关系可以说明影子的大小决定于点光源、物体和光屏的相对位置;\item 利用发光面较大的面光源代替点光源,
说明本影和半影的概念;
\item 面光源的线度大于物体时,移动
光屏的位置,讨论屏上影的情况,说明本影、半影等概念的关
系.
\item 观察者处于图5.1中$A$、$B$、$C$、$D$等不同位置时看面
光源$PQ$, 见到的现象会不同。通过以上分析,可以为学生认
识日蚀或月蚀等现象、解答课本练习一第2题打下基
础,教学时应着重引导学生体会几何光学的研究方法:从光
传播的基本定律出发,做出光路图,运用几何学的方法研究
有关线段、角度等量的关系。
\end{enumerate}

\begin{figure}[htp]
    \centering
    \includegraphics[scale=.6]{fig/5-1.png}
    \caption{}
\end{figure}

光的速度是物理学中的重要基本常数,为了测量光
速,从十七世纪伽利略开始一直到今天,经过了许多科学家
的努力,测量结果越来越精确。讲述光的速度要结合物理学
史介绍科学家百折不挠、精益求精的探索精神,要重点讲述光
的传播速度虽然很大,但是有限的,分析测量光速实验的设计
思想。伽利略根据$v=s/t$
的原理进行了历史上第一次测光速
的实验,尽管没能测出光的速度来,但是他的实验设计思想却
给人们指出了努力方向,由于光速是很大的,早期测量光速
的方法或者是增大光传播的距离,或者是利用更精巧的办法
准确地测量光传播的时间。

罗默观测木星的卫星蚀,第一次证实了光以有限的速度
传播,用的就是前一种方法。学生理解罗默方法的原理是有
困难的,教材并不要求掌握具体细节,但应使学生理解:虽然
卫星绕木星运动的周期是固定的,但在地球上观测木星的两
次卫星蚀的时间间隔却是不同的,这是由于地球位于公转轨
道上的不同位置时,与木星的卫星间的距离不同。而光以有
限速度传播,从卫星发出的光到达地球时经过的距离不等,因
此,观测到的卫星蚀的时间间隔也就不等。

迈克耳逊用旋转棱镜法测光速,是采用精巧的办法准确
地测量出光传播的时间。他是早期测量光速最准确的科学家。
讲述旋转棱镜法测光速的实验设计思想时,要着重理解光由
光源$S$发出最后到达望远镜$C$中(课本图5.5)所经过
的路径和旋转棱镜能精巧地测量时间的原理,调节八面棱镜
的转速,使光由凹镜反射返回来时八面镜恰好转过$1/8$转,在
望远镜$C$中能看到$S$的像(还可进一步引导学生讨论转速增
加到转过$\frac{2}{8},\frac{3}{8},\ldots$
转的情况)。测出转速便可求得光在两
个山峰之间往返一次所用的时间。

近代科学家还提出用其他的实验原理来测量光速。例如
测量激光的频率$\nu$,和波长$\lambda$, 根据公式$c=\lambda\cdot \nu$计算光速,大
大提高了测量的精度。应该要求学生记住真空中光速的近似
值$c=3.00\x10^8\ms$.

\subsection{光的反射}
\subsubsection{反射定律}

反射定律、镜面反射和漫反射在初中讲得
都比较清楚,教材中这部分内容是按照复习旧知识安排的。可
以通过演示光的反射光路、镜面反射和漫反射,再现这些现
象,再进一步采用讨论的方法加深对反射定律的认识。选择
讨论题目要针对学生的具体情况,下列问题可供参考:
\begin{enumerate}
\item 在什么条件下发生光的反射现象?
\item 反射定律的内容为什么一定要包括课本中讲的两条?
    \item 怎样确定反射界面的法线?如
果界面是球面,它的法线是什么?
    \item 发生镜面反射时,反射光
线和入射光线之间的夹角取值范围可能多大?
    \item 漫反射和镜面反射有什么区别和联系?
\end{enumerate}



\subsubsection{光路的可逆性}

光路的可逆性是几何光学的重要原
理之一。要结合反射定律(包括以后的折射定律)理解反射时
的光路(包括折射时的光路)是可逆的。几何光学中的光路(面
镜、棱镜和透镜的光路和成像)都建立在反射定律和折射定
律的基础之上,并且几何光学中的所有光路都是可逆的,应
该注意,物理光学中的光路是不可逆的。

做为一种方法,光路的可逆性可以用来解决一些具体问
题。例如,人眼的视野问题。可以设想在人眼所在的位置处
放一个点光源,求出由该点光源发出的光所能照亮的区域,根
据光路可逆性,在这个区域内的物体发出的光都能到达人眼,
被人眼看到。因此,这个区域是人眼的视野范围。

\subsubsection{平面镜}

平面镜的内容也是在初中知识的基础上讲
述的。讲述的重点放在虚像的概念、虚像的形成以及平面镜
成像的特点上。

虚像的概念及其形成分三个层次讲解:
\begin{enumerate}
\item 在光的直
线传播一节中讲述了根据光的直线传播确定发光点的位置,
这为讲解虚像做了准备、由发光点发出的光向各个方向传播,
可以将其中的任意两条光线反向延长,它们的交点就是发光
点的位置。应该注意,这两条光线必须是同一发光点发出的,
由不同发光点发出的两条光线的交点,在成像问题上没有意
义;
\item 由某一物点$S$发出的、经平面镜反射后的光线进入了眼
睛,人们根据光的直线传播的经验认为这些光线都是由它们
的反向延长线的交点$S'$射来的,好象在$S'$点有一个发光点
一样,$S'$点就是物点$S$的虚像。这里说明了虚像不是光线的
实际交点,是“虚”的,需要借助其他光学仪器(如眼睛)才能观
察到;
\item 组成物体的每一个物点都在平面镜里产生一个虚像
点,这些虚像点的集合就是物体的虚像。教学中要注意培养
学生抽象思维的能力,不仅要从实验现象中看到物体通过平
面镜可以成像,还要学会从理论上分析推理得出像的概念,
再根据反射定律进一步推出物体通过平面镜所成的像是等
大、正立的虚像,物和像对于镜面是对称的。
\end{enumerate}

平面镜的应用包括应用平面镜成像,应用平面镜控
制光路等,在课文中提到了利用平面镜将微小效应放大、潜望
镜的光路图等实例,在习题中安排了平面镜成像,利用激光测
量月球和地球间的距离以及人眼通过平面镜观察的视野范围
等问题。在分析实际问题时要画好光路图,使学生了解几何光
学的研究方法,培养他们灵活应用物理知识解决实际问题的
能力。

\subsubsection{球面镜}

通过演示实验复习凹镜的会聚作用和凸镜
的发散作用,注意从实验观察到的光路出发讲述球面镜的顶
点、主轴、焦点(实焦点和虚焦点)和焦距等概念。进一步理解
课本193页讲述的凹镜和凸镜的实际应用的例子.

近轴光线是球面镜和透镜能成理想像的前提条件,教材
中给出了在近轴条件下,球面镜的焦距$f=r/2$
是为了使学
生对球面镜焦距的大小有一个具体的了解,并不要求深入展
开讨论和证明这个公式。根据反射定律,可以分别画出平行
于主轴的近轴光线和远轴光线经过凹镜反射后与主轴的交
点,如图5.2中的$F$和$E$点,说明球面镜的非近轴光线不能会
聚于一点。实验给出的几条光线能够在焦点会聚仅是近似
的;近似的条件就是入射光线都是近轴光线。为了使远轴的
平行光线也能会聚于焦点,实际应用的大口径凹镜都用抛物
面镜而不用球面镜。
\begin{figure}[htp]
    \centering
    \includegraphics[scale=.6]{fig/5-2.png}
    \caption{}
\end{figure}

球面镜成像是从实验得出的,教材不要求画成像的光路
图,也不要求讲述球面镜成像公式,只是从实验得出物体通过
凹镜可以成放大实像、缩小实像和放大虚像,通过凸镜只能成
缩小虚像的情况 演示实验中要让学生注意观察用光屏能够
接收到实像,而无论把光屏放在什么位置都不能接收到虚像。
这表明实像是反射光线实际会聚而成的;虚像不是反射光线
的实际会聚点,而是光线的反向延长线的交点,这是实像和虚
像的主要区别。


\subsection{光的折射}

折射定律和折射率是本章的重点内容,可以按下述
几个层次安排教学。
\begin{enumerate}
\item 通过演示实验观察光在两种媒质的界面上的折射现
象,观察折射角随入射角的增大而增大的规律,激发学生深入
研究折射角和入射角的关系的学习动机。
\item 分析课本197页表中的实验数据,得出折射定律.
\item 折射定律中的常数决定于两种媒质的性质,引入相
对折射率$n_{21}$·
\item 引入绝对折射率,进而求出相对折射率和绝对折射
率的关系。
\end{enumerate}

\subsubsection{折射定律}

课本结合物理学史讲述了历史上经历了
一千多年,由托勒密到斯涅耳经过许多科学家的努力,从大量
的实验数据中归纳总结出折射定律,教学过程也要从演示折
射现象开始,演示的内容有:
\begin{enumerate}
\item 光达到两种媒质的界面上发生
折射的现象;
\item 折射角随入射角增大而增大的规律;
\item 光由光
疏媒质进入光密媒质向靠近法线方向偏折,光由光密媒质进
入光疏媒质向远离法线方向偏折。
\end{enumerate}
通过演示,使学生获得光
的折射现象的感性认识。在定性实验的基础上,再分析课本
197页表中所给出的实验数据,说明托勒密最初得出的折射
角与入射角成正比的结论是片面的,给出斯涅耳经过深入的
分析得出的折射定律内容。(折射定律现代表述的形式是由
笛卡儿提出的。)

和反射定律一样,掌握折射定律也要注意培养学生的想
象力,要使学生建立起光由媒质I进入媒质II的折射图景。
使学生在理解界面及其法线、入射光线和折射光线,入射角和
折射角等概念的基础上,进一步掌握折射定律。折射定律的
第一点指出了入射光线、折射光线、法线在平面内的大致方
位,第二点定量地确定折射角,确定折射光线的确切方向。

\subsubsection{折射率}

媒质的折射率是由折射定律引入的.教材
从光在两种媒质界面上发生折射的现象人手,引人相对折射
率,并且直接给出了相对折射率和光在两种媒质中传播速度
的关系,然后从光从真空进入媒质发生折射的特殊情况,引
入绝对折射率的概念,并通过折射率和光速的关系,求出相对
折射率和绝对折射率的关系。这部分内容,特别是绝对折射
率和相对折射率的关系,涉及的物理量比较多,初学者较难
掌握,容易造成混乱。教学中应强调$n_{21}$下标的表述及其意
义.讲解公式$n_{21}=\dfrac{n_2}{n_1}=\frac{v_1}{v_2}$时,要和具体的光疏、光密媒质以
及光线的偏折情况相联系。

教学中要注意培养学生的想象力,建立光由媒质I射入
媒质II发生折射的图景,理解其中的平面、直线、角度之间的
关系,进一步寻求入射角和折射角、媒质的折射率及光速之
间的关系,例如,已知水中的光速是真空中光速的$3/4$,光由
空气进入水中,以多大的入射角射入时,反射光线和折射光线
垂直,学生在头脑中建立了光在空气和水的界面上发生反射
和折射的图景之后,就能找到入射角、反射角和折射角间的关
系,求出入射角来。

课本201页讨论的光通过两面平行的玻璃砖发生侧移的
例题,既应用了折射定律分析解决具体问题,又为“测定玻璃
折射率”的学生实验做准备。可以先演示光通过玻璃砖发生平
行侧移的现象,再提出如下问题:①光线穿过玻璃砖后,为什
么只发生平移而不改变传播方向?②光线平行侧移的距离、方
向决定于什么条件?让学生自己通过讨论去研究解释这个现
象,寻求例题的解。

\subsubsection{全反射}
 由于学生先接受了光到达两种媒质的界面
上既要发生反射,又要发生折射的现象,要进一步认识在一定
条件下,光到达界面上只发生反射不发生折射的全反射现象
比较困难,学生往往对于发生全反射的条件理解不深刻。教
学中,要使学生在实验的基础上,认识全反射现象,了解发生
全反射的条件。

做演示实验时,要使学生看到光从光密媒质射入光疏媒
质时,折射角大于入射角。随着入射角的增大,折射角也逐渐
增大.当入射角达到某一角度时,折射角达到$90^{\circ}$. 再增大
入射角,就只有反射光线,没有折射光线。通过现象的观察,
使学生了解全反射现象是光由光密媒质射入光疏媒质时,入
射角达到某一特定值以后产生的现象。这有助于使学生弄清
楚“临界角”的含义。

还要使学生理解全反射现象并不违背折射定律。当光从
光密媒质进入光疏媒质时,有$\sin r=\dfrac{\sin i}{n_{21}}$, 而光疏媒质对
光密媒质的相对折射率$n_{21}=\dfrac{v_1}{v_2}<1$, 所以折射角$r$大于入射
角$i$. 当折射角$r=90^{\circ}$时,即使再增大入射角,也不会存
在大于$90^{\circ}$的折射角$r$, 这时就发生了全反射现象。

教学时,可结合实验观察使学生理解临界角的概念,掌
握计算临界角的方法。

为使学生获得全面的认识,在讨论全反射现象时,也可以
分析光由光疏媒质进入光密媒质的情况。通过比较,加深学
生对产生全反射条件的认识。

在演示时,还要注意引导学生观察入射光的能量在反射
光和折射光中的分配随入射角的大小而变化。使学生认识到
发生全反射时,入射光线的能量几乎都被反射回原来的媒质
中,没有光能量进入另一种媒质,因而反射光最强。由此可以
认识生活中的全反射现象,如课本205页讲述的实例,以及使
用光导纤维和全反射棱镜的意义。

\subsubsection{棱镜}
三棱镜对光的偏折作用和色散现象是折射定律的具体应
用,设计如下的教学过程可供参考。

演示单色光通过三棱镜发生偏折的光路,分别演示
棱镜的底面在下方和底面在上方的两种情况,让学生观察现
象,归纳出光过三棱镜总是向底面偏折的结论。

根据折射定律讨论光在三棱镜两个侧面的折射,定
性地说明光通过光密三棱镜向底面偏折的光路。也要提醒学
生注意,如果是光疏三棱镜,光的偏折情况将会怎样?让学生
在做课本练习六第1题时回答.

定性说明偏折角度$\theta$与棱镜的折射率有关。在保持
入射角$i$和棱镜顶角$A$不变的条件下,棱镜材料的折射率越
大,偏折角度也越大。这是由于棱镜的折射率大,在$AB$面
上光的偏折角度也大,也就是$AC$面上的入射角变大。在$AC$
面,由于入射角和棱镜材料的折射率都变大,偏折角度也要变
大。因此总的偏折角度随折射率增大而增大。

演示白光过三棱镜的色散现象,可观察到由色散
产生的彩色光带(连续光谱)。要注意引导学生观察色散时光线
的偏折方向以及光谱中不同颜色光的偏折角度。演示时还应
在三棱镜的入射侧面前加红、黄、蓝等不同颜色的滤色片,显示
单色光通过棱镜的偏折情况。可以看到红光偏折最小,蓝光
偏折最大。由此揭示白光包含有各种颜色的色光,不同色光通
过棱镜的偏折角度不同,进而可以解释产生色散现象的原因。

组成白光的各种单色光,对三棱镜的入射角都相同,
但出射时各色光偏折角度不同,这是由于棱镜对于不同色光
的折射率不同造成的,红光的偏折最小,表明棱镜材料对于红
光的折射率最小;蓝光的偏折大,表明棱镜对于蓝光的折射率
较大,要让学生了解媒质对于不同颜色的光的折射率不同,说
明不同色光在该媒质中的传播速度不一样,使学生进一步加
深对折射率的认识。

可结合练习五中的题目讲一讲全反射棱镜.常用的
全反射棱镜是等腰直角三棱镜。在光学仪器中常用它来控制
光路,改变光线的传播方向或使像倒转等。例如,潜望镜中用
全反射棱镜代替平面反光镜,改变光线的传播方向;望远镜中
常用一对全反射棱镜使像的上下和左右都反转,观察者能够
直接看到正立的像。

常用一对全反射棱镜使像的上下和左右都反转,观察者能够
直接看到正立的像。

\subsection{透镜}
这一单元讲述透镜的成像规律,成像作图法和成像公式。
这些内容是本章中重要的基础知识,也是教学的重点。

实验是讲解透镜知识的基础.凸透镜对于光线的会
聚作用,凹透镜对于光线的发散作用,物体通过透镜可以成像
(包括实像和虚像),都是通过实验得出结论的。通过实验建
立起物和像的一一对应关系,为进一步用作图法和公式法研
究物像的关系打好基础,要做好下述演示实验:

平行于主轴的光线通过凸透镜后都会聚于主轴上的
一点;平行于主轴的光线通过凹透镜后变得发散,借助于直尺
看出出射光线的延长线交于主轴上的一点。通过实验的光路
讲述主轴、光心、焦点(包括实焦点和虚焦点)和焦距等概念。

从主轴、光心、焦点的概念分别引出凸透镜和凹透镜
的三条主要光线,再用实验复现这三条主要光线,强化学生的
认识。

通过演示实验复习物体通过凸透镜成像的五种情况
($u=f$时不成像)和凹透镜成像的一种情况。演示过程要注
意:
\begin{enumerate}
\item 利用光屏接收实像,确定实像的位置要注意实验操作的
示范,让学生看到光屏在成像位置附近移动时,像由模糊到清
晰再到模糊的过程。
\item 改变物距的大小应由物距较大逐渐变
为较小,使学生对于凸透镜和凹透镜成像的几种情况有比较
完整的认识,体会研究方法,切忌只孤立地演示成像的几种情
况,而不体现物距(或像距)连续变化的过程。
\end{enumerate}

\subsubsection{成像作图}

透镜成像作图建立在物体通过透镜可以
成像的实验基础之上,物体上的每一个发光点都对应着一个
像点,即由物点发出的光线经过透镜折射后,所有出射光线,包
括三条主要光线,都汇交于它对应的像点.课本安排练
习九第1题的目的就是为了加深这个认识,为了确定像点的
位置,可以在这些光线中任意选取两条出射光线求得交点,选
用三条主要光线中的两条比较方便。同样,确定物点的位置,
也可求出与像点相应的任意两条入射光线的交点。这样,可以
使学生认识物点和像点的意义,认识成像作图的基本出发点,
正确理解作图方法,避免得出只有三条主要光线才能成像的
错误印象。

教学中要注意示范,使学生养成作图规范的良好习惯,作
图规范包括:每条光线在透镜折射前后都应标出表示光线行
进方向的箭头;正确绘出透镜的符号,标注光心、焦点以及物
和像的位置和正倒;分清图中的实线和虚线。

教学中还要注意:
\begin{enumerate}
\item 采用对比的方法区别凸透镜和凹透镜三条主要光线
的不同点。学生对凹透镜的三条主要光线往往掌握不好,其
主要原因是对于虚焦点的概念和位置理解不深刻,混淆了凹
透镜和凸透镜对光线的作用。
\item 强调物点是透镜的入射光线的出射点,像点则是上
述人射光线经透镜折射后的出射光线或其延长线的交点,不
是随意两条光线的交点,学生往往不注意这点而造成错误。
\item 利用作图法确定物像关系,要和物体通过透镜成像
的情况相联系,许多实际问题往往是在分析成像情况做出判
断后才能正确做图。
\end{enumerate}


































\chapter{光的波动性}\minitoc[n]
\section{教学要求}

本章讲述光的波动性、光的电磁理论和电磁波谱,以及光
谱和光谱分析的知识。认识光的波动性及其电磁本质,在物
理学史上有重要的意义。光谱学的知识,为研究原子结构提
供了信息。光的干涉、衍射、偏振等现象,在现代科学技术中
有主要的应用。学生应该对这些知识有所了解。

光的干涉现象,根据学生的知识水平和接受能力,有条件
分析得深入一些。因此,作为本章的重点内容。衍射现象的
分析,要复杂一些,因此只介绍了现象,本章的其他知识都是
介绍性的。

本章教材分为三个单元。第一节至第五节为第一单
元,讲述光的波动性,第六、七节为第二单元,讲述光的电磁
本性,第八节为第三单元,介绍光谱的初步知识。

讲述光的微粒说与波动说的矛盾是为了使学生了解人们
对光的本性的认识经历了曲折的过程,也是讲解光的波动性
的引子,激发学生的学习兴趣。

光的干涉和衍射现象,在日常生活中极易忽略,学生不熟
悉。教学中应注意通过演示使学生观察现象,联系机械波的
干涉和衍射现象,运用波动知识进行分析。在讲干涉现象时,
不要求解释为什么不同光源发出的光不是相干光,要把教学
的重点放在用波动说解释产生明暗条纹的条件上。在推导时
要注意突出推导的思路。通过研究光的干涉,还应该让学生
知道光的颜色跟波长(频率)有关,不同色光的频率是不同的。

光的衍射现象也证明了光具有波动性。在教学中可以指
导学生用一些简单的方法进行观察。

光的偏振现象,证明了光是横波。但是学生对偏振现象
比较生疏。教学中应跟学生比较熟悉的机械波的有关现象类
比,并在充分观察实验现象的基础上进行讲述。

从表面上看,光现象和电磁现象之间似乎没有什么联系。
麦克斯韦提出的光的电磁说和赫兹对电磁波进行的实验研
究,揭示了光的电磁本质。通过光的电磁说的教学,应该使学
生体会自然现象间的相互联系,对无线电波、红外线、可见光、
紫外线、伦琴射线等不同波长的电磁波有一个统一的认识,并
对它们的性质和产生机理有所了解。

这一章的教学要求如下:
\begin{enumerate}
\item 理解光的干涉现象,理解产生明暗条纹的条件,了解
光的干涉现象的应用。
\item 了解光的衍射现象和产生衍射现象的条件。
\item 了解光的偏振现象和光是一种横波.
\item 了解光是一种电磁波;了解无线电波、红外线、可见
光、紫外线、伦琴射线等都是波长不同的电磁波。
\item 了解光谱和光谱分析的初步知识。
\end{enumerate}

\section{教学建议}
\subsection{光的波动性}
\subsubsection{人类对光的本性的两种认识}

人类对光的本性的认
识经历了一个辩证发展的过程,到十七世纪,在人类已经积累
了许多几何光学知识的基础上,形成了对光的本性的两种认
识——微粒说和波动说。

教学中要讲述两种理论在解释实验规律上各有其成功的
一面,也要讲述两种理论的不足之处。还要说明两种理论的
尖锐矛盾。例如,微粒说难以说明光在媒质界面上同时发生
反射和折射,波动说却可以解释这个事实,波动说难以说明光
的直线传播,在微粒说看来,这都是很自然的现象。还可适当
补充一些历史事实进一步说明微粒说和波动说的尖锐矛盾,
例如,笛卡儿从微粒说推导出光的折射定律,得出了光在媒质
中的速度大于真空中的光速;惠更斯从波动说也推导出光的
折射定律,但得出了光在媒质中的速度小于真空中的光速。对
于两束光相交时各自独立传播的事实,微粒说难以解释,波动
说却很容易说明,等等。通过这些事实的讲述,使学生认识到
理论和实践的矛盾,两种理论对光的本性认识的矛盾,是推动
人类认识光的本性的内在动力。

\subsubsection{双缝干涉}

光的干涉现象是光具有波动性的重要依
据。教材首先介绍了托马斯·杨在历史上第一次解决了相干
光源的问题,成功地做出了光的干涉实验,然后具体介绍了双
缝干涉实验,并用波动理论对实验现象进行了定量的分析,得
出了干涉条纹间距与波长的关系,这对于学生认识光的波动
性重要的意义,由于学生对于光的干涉现象比较生疏,难
以把光想象成是一种波动,而且运用波动理论分析光的干涉
现象,内容比较抽象,学生接受起来会有一定的困难。建议教
学中注意如下几点:

首先要复习机械波的干涉的知识,复习重点放在什
么是波的干涉现象,相干波源的条件。运用波动理论分析空间
振动加强区域和削弱区域的产生条件,还要进一步用光程差
来分析两个振动的位相关系,为讲授本节内容做准备。

要做好演示实验,使全班同学都亲眼看到光的干涉
现象,这是认识光的干涉的基础。实验内容包括单色光干涉
的明暗条纹和白光干涉的彩色条纹。要注意介绍仪器装置。教
学要结合课本图6.2进行.

运用波动理论分析双缝干涉实验要注意:
\begin{enumerate}
\item 说明如
何获得相干光源,这是得到稳定的干涉现象的关键。结合实验
装置,对照课本图6.2说明通过双缝得到的$S_1$和$S_2$
两个光源,是在任何时刻频率和位相都相同的相干光源。
\item 
类比机械波的干涉,运用波动理论分析光的干涉现象.把$S_1$
和$S_2$相干光源发出的光想象成两列光波,两列光波在屏上相
遇。两个光振动叠加产生亮、暗条纹。亮条纹处的光能量较
强,对应着合振动互相加强;暗条纹处的光能量较弱,对应着
合振动互相削弱。
\item 推导屏上亮暗条纹的位置公式要讲清思
路:屏上亮纹或暗纹的位置用距$O$点的距离$x$表示,屏上任
一点的振动加强或削弱的情况决定于该点到光源$S_1$和$S_2$的
距离之差(光程差)$\delta=r_2-r_1$. 在$\delta$等于波长整数倍的位置,
产生亮纹;在$\delta$等于半波长奇数倍的位置时产生暗纹.寻求
$\delta$与$d$, $\ell$和$x$的关系,推出$\delta\approx \dfrac{d}{\ell}\cdot x$
\item 学生对于公式
\[x=\pm k\frac{\ell }{d}\lambda, \qquad k=0,1,2,\ldots\]
和
\[x=\pm (2k-1)\frac{\ell }{d}\cdot \frac{\lambda}{2}, \qquad k=1,2,\ldots\]
的表述方法不习惯,讲述时要从具体的中央亮纹、第1条、第2
条……亮纹或暗纹入手,归纳出一般的表述。例如,亮纹的条
件是$\delta=r_2-r_1=0,\lambda,2\lambda,\ldots$和$-\lambda,-2\lambda,\ldots$(“$-$”的意义
是$r_2<r_1$, 位置在$O$点的下方)。归纳出$\delta=\dfrac{d}{\ell}x=\pm k\lambda$, 解出
\[x=\pm k\frac{\ell }{d}\lambda, \qquad k=0,\pm 1,\pm 2,\ldots\]
然后具体说明$k=0,1,
2,\ldots$所对应的
$x=0,\pm\dfrac{\ell}{d}\lambda,\pm 2\dfrac{\ell}{d}\lambda,\ldots$
的意义,使学生理
解这种数学表述的内容。
\item 双缝干涉公式的近似条件是
$\ell\gg d$和$\ell\gg x$. 双缝间的距离$d$一般仅为十分之几毫米。而
屏上偏离开中央亮纹较远处的亮纹的强度是十分弱的,几乎
无法观测到。通常能观察到亮纹范围远小于$\ell$.
\end{enumerate}


根据相邻亮纹或暗纹间的距离公式$\Delta x=\dfrac{\ell}{d}\lambda$
可以测量光波的波长。红光的干涉条纹间距最宽,紫光的最窄,由此
可认识不同颜色的光的波长不等,由红到紫,波长越来越短,
频率越来越高。进一步说明白光的彩色干涉条纹的产生原因。
学生通过测量光波的波长,对于可见光的波长和频率可以有
一个大致的认识。

说明双缝干涉的亮暗条纹反映了光源$S_1$和$S_2$发出
的光的能量在空间的分布情况。暗条纹处的光能量几乎是零,
表明两列光波叠加彼此相互抵消。这并不是光能量损耗了或
转变成了其他形式的能量,而是按照波的传播规律,没有能
量传到该处;亮条纹处的光能量比较强,光能量增加也不是光
的干涉可以产生能量,而是按照波的传播规律,到达该处的能
量比较集中。

\subsubsection{薄膜干涉 }
学生平时都见过薄膜干涉现象(雨后路面
上的油膜形成的彩色条纹,色彩绚丽的肥皂泡等)只是没有引
起注意。提出一些现象,并让学生动手做肥皂液薄膜的干涉
实验,观察单色光的明暗条纹和白光的彩色条纹,会给学生
留下深刻的印象。实验时,可引导学生细致地观察干涉条纹,
例如,可观察到干涉条纹产生在薄膜的表面上,肥皂液薄膜的
干涉条纹基本上是水平的等等。(这些都是等厚干涉的特点决
定的。)

分析薄膜干涉实验的重点应放在如何得到相干的
两列波,薄膜上是怎样出现明暗相间的干涉条纹或彩色条纹
的.课本图6.4是肥皂液薄膜干涉的示意图.图中只给
出了从楔形薄膜的前后表面反射的两列光波的位相关系和叠
加的结果。该图表示的是光波几乎垂直地入射到楔形薄膜上
后,又从薄膜的前后表面反射回来的情况,薄膜的两个表面是
近于平行的。图中薄膜的楔形被大大夸张了,应该指出,分别
从前、后表面反射回来的两列波都来自于同一入射波,因而是
相干的。由于从后表面反射的光波比前表面反射的光波通过
的路程较长,因而位相要落后一些,落后的相差与膜的厚度有
关。在膜上某些厚度的地方,两列反射波是同相的,形成相互
加强的亮纹;在另一些厚度的地方两列反射波是反相的,形成
相互削弱的暗纹。教材没有深入分析计算光程差和薄膜厚度
的关系,也不涉及光在光疏媒质中达到光密媒质的界面反射
时的半波损失以及在液膜内的光波波长小于空气中的波长等
问题,教学中应注意掌握讲授的深度,以免加重学生的负担。

\begin{figure}[htp]
    \centering
    \includegraphics[scale=.6]{fig/6-1.png}
    \caption{}
\end{figure}


薄膜干涉在科学技术上有重要的应用,除了教材中
介绍的“用干涉法检验光学元件表面加工的质量”之外,还
可适当增加介绍在检验钢球的直径、透镜表面的曲率半径,测
量长度的微小变化等方面的应用,讲授增透膜时,要注意说明
反射光和透射光的能量之和等于入射光的能量(不考虑媒质
对光的吸收)。增透膜的作用只是使反射的两列光波产生相消
干涉,反射光的能量趋于零,因增加了透射光在入射光中所
占的比例。并不是增加了光的能量,以免学生误解。增透膜
的厚度是光在薄膜媒质中传播的波长的1/4(不是真空中波长
的1/4)。由于光垂直于薄膜表面入射时,从前后表面反射的两
列光波的路程差等于薄膜厚度的2倍,如图6.1所示,当膜的
玻璃厚度是$\lambda/4$
时,路程差恰好是$\lambda/2$,
从前后表面反射的光波的相位恰好相反,便产生相消干涉。制作增透膜的材料是氟化镁
${\rm MgF_2}$, 折射率$n=1.38$, 介于空气和玻璃之间.因此在空气
和增透膜的界面上、增透膜和玻璃的界面上,反射情况相同,
不会再额外增加两列光波的光程差。增透膜只对人眼或感光
胶片最敏感的绿光起增透作用。如果白光照射到增透膜上,
由于绿光产生相消干涉,在反射光中绿光的强度几乎是零,
而其他波长的反射光并没有完全抵消,因此,增透膜呈绿光的
互补色——淡紫色。

\subsubsection{光的衍射}

光的衍射现象进一步证明了光具有波动
性,对发展光的波动理论起了重要的作用。教材讲述光的衍射
的思路是先说明一般情况下不容易观察到光的衍射现象的原
因,同时也就说明了观察衍射现象的条件,再来做衍射实验。
由于衍射现象产生的物理过程分析起来比较复杂,课本中对
于衍射现象未作理论上的分析。

做好光过小孔或单缝发生衍射现象的实验,是学
生认识光的衍射的基础。实验中要让学生观察:
\begin{enumerate}
\item 光偏离直
线传播,绕过障碍物进入阴影中,并且在屏上出现明暗相间的
衍射条纹。
\item 只有孔或狭缝较小时,光的衍射现象才比较显
著。
\item 为了认识光的直线传播和光的衍射的关系,观察屏上的
光斑变化情况时应使孔径或狭缝的宽度逐渐变小。当孔径或
缝宽较大时,屏上光斑的边缘清晰,显示出光沿直线传播;孔
径或缝宽逐渐变小时,屏上光斑的边缘逐渐变得不清晰了,衍
射现象逐渐变得显著起来,直至出现明暗相间的衍射条纹。
\end{enumerate}
上述演示可以使学生认识到,由于光是波动,遇到障碍物时发生
衍射现象是不可避免的,只是由于在一般情况下障碍物的尺
寸比光的长大得多,因此衍射现象很不明显。当衍射现象
可以忽略时,才可以认为光是沿着直线传播的。

建议采用课堂讲授和学生实验并进的方式进行教
学。用游标卡尺的测脚形成可调宽度的狭缝观察线光源的衍
射现象,用不同孔径的小孔观察点光源的衍射现象。学生自
已动手观察到光的衍射条纹,可使他们了解到缝宽和孔径多
大时能够观察到比较显著的衍射现象。

菲涅耳圆盘衍射的中心亮斑能够引起学生的极大兴
趣,但在课堂上完成这个实验比较困难。教学时可以用细金
属丝产生的衍射来代替,由于观察到在金属丝的阴影中间有
一条亮线,学生会更加信服光的衍射现象。

光栅的衍射是衍射现象在科学技术上的重要应用.
光通过衍射光栅的亮条纹随着光栅缝数的增加而变窄和变亮
的特点是从实验现象中得出的。教学中不要求从理论上加以
分析。可介绍一些衍射光栅的应用,例如利用光栅衍射条纹的
特点可以比较精确地测量光波的波长和产生均匀分布的光
谱等。

\subsubsection{光的偏振}
横波的偏振是新概念.为了从机械波入手来认识偏
振.应将课本257页的实验演示给学生看.这个实验,形象
地给出了横波是偏振的机械模型,偏振现象是横波区别于纵
波的最明显的标志,通过机械波和光波的类比,可以从光的
偏振现象使学生认识光是横波。

光的偏振现象要通过实验给出.实验的设计思想和
横波偏振的机械模型一样。电气石晶体薄片或人造偏振片对
于某一振动方向的光具有选择吸收的本领,它们的作用相当
于课本257页的演示中限制或检验振动方向的“狭缝”.实验
应先演示光通过两个偏振片时,转动其中任一个偏振片的方
位,透射光的强度出现周期性的变化,给学生以鲜明突出的印
象,再演示光通过一个偏振片的情况,使学生产生悬念,激发
他们的探索热情。分析光的偏振实验,要引导学生把光波和
机械横波相类比,建立光的波动模型,能想象出光是一种横
波,它的振动方向跟光的传播方向垂直,教学中应注意讲清起
偏器和检偏器的不同作用,自然光和偏振光的区别和联系。

通过演示反射光的偏振现象,说明光的偏振现象是
普遍的(即教材中讲的“除了从光源直接射来的,基本上都是
偏振光),也便于使学生理解光的偏振现象在科学技术上的一
些应用,教材还安排了“偏振光与立体电影”的阅读教材,如能
配合教学看一场立体电影,学生一定会有浓厚的兴趣。

\subsection{光的电磁本性}

\subsubsection{光的电磁说}

光的电磁说比光的波动说前进了一大步.课本结合物
理学史讲述了人类认识光的本性的发展过程。教学中要注意
以下四个环节:
\begin{enumerate}
    \item 从十七世纪开始到十九世纪初,光的波动说不断地
发展和完善,逐渐为人们所接受,但是人们对光的本性认识不
足,以为光也是一种机械波,这种认识在光的传播媒质等问题
上遇到了严重的困难。
\item 法拉第发现在强磁场作用下,偏振光的振动面发生
偏转的现象。它启示人们把表面上很不相同的光现象和电磁
现象联系起来。
\item 光是一种电磁波.电磁波和机械波在本质上不同,
它可以在真空中传播而不需要任何媒质。这就不仅解决了波
动说在光的传播媒质问题上遇到的困难,而且对光的本质有
了进一步的认识。
\item 历史上,光的电磁学说是麦克斯韦作为假说提出的,
赫兹实验证实了电磁波的客观存在,也证明了光是一种电磁
波,使光的电磁理论得以确立。
\end{enumerate}

通过光的电磁说的教学,不仅要使学生了解光的电磁说
的基本内容。还要通过光的电磁理论的建立和发展过程,认
识理论和实践的辩证关系,从中受到辩证唯物主义世界观的
教育。

\subsubsection{电磁波谱} 
电磁波谱的教学,应该着重使学生领会光
的电磁说把光现象和电磁现象统一起来了。

在把无线电波、红外线、可见光、紫外线、伦琴射线和γ射
线按照频率(或波长)的顺序排列成波谱,使学生对电磁波有
一个全面的了解后,还要进一步使学生认识这些电磁波既具
有共同的本质,又有各自的特性和不同的产生机理,介绍这
些内容可以为进一步学习光的波粒二象性和原子的内部结构
做准备。

在电磁波谱的教学中,还要注意介绍红外线、紫外线、伦
琴射线的一些应用,以开阔学生的视野,激发他们学习科学
技术的志趣。

通过这一单元的教学,应当使学生体会到自然现象之间
是相互联系的,有些表面上很不相同的现象却存在着共同本
质。把光现象和电磁现象统一起来,是物理学的伟大成果,现
在物理学的研究还在更加广泛的范围上进行着类似的统一
工作。

\subsection{光谱的初步知识}
\subsubsection{介绍光谱学的知识}

要让学生认识观察光谱的仪
器——分光镜。讲述分光镜的构造原理要结合实物和挂图。应
先讲述单色光通过三棱镜的光路,单色光照亮狭缝$S$, 经过凸
透镜$L_1$形成平行光,通过三棱镜$P$发生偏折,再会聚在凸透
镜$L_2$的焦平面$MN$上成实像,狭缝$S$的实像是一条亮线,颜
色和入射光相同。再讲复色光通过三棱镜的光路,使学生理
解不同颜色的光谱线实际上是照亮的狭缝$S$在焦平面$MN$上
形成的不同颜色的实像,这些实像按光的波长顺序排列形成
光谱。教学中还可以介绍标尺管的作用及其光路,使学生对分
光镜有比较全面的认识。

\subsubsection{光谱分析}

结合观察连续光谱、明线光谱(特别是
氢原子的光谱)和吸收光谱,使学生了解有关的几个概念,了
解各种光谱产生的机制,观察同一元素原子的发射光谱和吸
收光谱时,要注意观察吸收光谱的暗线位置和发射光谱的明
线位置一致,而不同元素原子有不同的光谱。从而理解光谱
代表了每种原子的特征,为介绍原子结构的内容作些准备。正
是每种原子都有自己的特征谱线,利用光谱可以鉴别物质和
确定物质的化学组成。

需要指出的是,由于中学配备的分光镜分辨本领较差,对
于太阳的吸收光谱的暗线,有的仪器不能观察到。

建议采用以学生自学为主的方式进行教学,指导学生
通过观察实验和阅读教材,理解光谱产生的机理和光谱的分
类,理解发射光谱、连续光谱、明线光谱和吸收光谱之间的关
系,认识光谱分析的原理及其方法。

\section{实验指导}
\subsection{演示实验}
本章的双缝干涉、衍射、偏振演示实验,要用J2508型光
的干涉衍射偏振演示仪。

J2508型光的干涉衍射偏振演示仪是由可转动的光具
座、滑块、观察筒、盒式光源、光学元件组成。

光具座上部是附有长80厘米标度的单导轨,导轨支撑在
光具座的底座上并可在水平面上任意转动。滑块分三种规格,
可套在光具座的单导轨上沿导轨滑动。滑块顶部的孔上可安
插各种光具。

观察筒由遮光筒、放大镜、玻璃屏和遮光板组成。

盒式光源由低压电源供电.盒内装有12V50W的卤钨
灯,盒前有聚光透镜。在出光口上装上单缝光栏,便可作线光
源使用。

主要光具有:狭缝、牛顿环、双面镜、反射器、毛玻璃屏、双
凸透镜等。其中,狭缝包括双缝(缝宽0.016毫米,缝距分别
为0.04毫米和0.08毫米两种)、单缝(缝宽0.08毫米),光栅
(1\%)。多缝缝宽0.02毫米,缝距0.08毫米。牛顿环装在圆
形胶木架中,胶木架上有三个调整螺钉。

光源、光具,观察筒均可安装在滑块顶部的小孔上,绕安
装轴转动。

图6.2的甲、乙、丙分别为光具座、观察筒和盒式光源的
结构示意图。

\begin{figure}[htp]
    \centering
    \includegraphics[scale=.6]{fig/6-2.png}
    \caption{}
\end{figure}

\subsubsection{双缝干涉}
用J2508型光的干涉衍射偏振演示仪做双缝干涉实验.
装置按图6.3配置.套在光源前的光源单缝缝宽为0.11毫
米,双缝的缝宽0.08毫米,装在光具架上,缝上的指示刻线
对齐光具架上的零刻线。

调整光源、单缝、双缝和观察筒的共轴是实验能否成功的
关键。

\begin{figure}[htp]
    \centering
    \includegraphics[scale=.6]{fig/6-3.png}
    \caption{}
\end{figure}

光具的调整步骤如下:先使单缝和双缝大致平行,相距约
5—10厘米(双缝离单缝近一些可以增加干涉亮条纹的亮度,
但光源单缝和双缝的共轴必须调得很好),观察筒的轴线和光
具座导轨平行,调整时可在观察筒前放一张白纸作为观察
屏,转动光源使得光通过单缝和双缝后落在白纸屏上的光斑
位于观察筒的中心处。再转动光源单缝,使它和双缝平行,在
白纸屏上见到清晰的干涉条纹,如果撤去白纸屏,便在毛玻
璃屏上呈现出清晰的干涉条纹。学生可以直接看毛玻璃屏上
的干涉条纹,也可以通过透镜观察干涉条纹的放大虚像。如
果在光源单缝和双缝之间加滤色片(也可用红色、绿色或紫
色的玻璃纸),可以看到单色光的明暗相间的干涉条纹。改变
观察筒与双缝的距离,可以看到干涉条纹的宽度随观察筒与
双缝间的距离增大而增大的情况。

由于光源用卤钨灯并有聚光透镜,从而使通过单缝和双
缝的光通量增加,提高了屏上亮条纹的亮度。为延长灯泡寿
命,开始用6伏电源点亮灯泡,然后再根据实际需要逐渐升高
电压,但不得超过12伏,接收干涉条纹的毛玻璃屏位于观察
筒内,遮挡了其他杂散光,提高了屏上干涉条纹的可见度,因
此可在一般亮度的教室中观察到清晰的干涉条纹。

实验时应缓慢转动光具座,使全班同学都能看到实验
现象。

\begin{figure}[htp]
    \centering
    \includegraphics[scale=.6]{fig/6-4.png}
    \caption{}
\end{figure}

由于激光的平行度好,单色性好,亮度高,是做光的干涉、
衍射实验的理想的单色光源,实验装置如图6.4所示,将激
光光束直接照射到双缝上,通过双缝后的光再投影到远处的
屏上,在屏上便可以见到明暗相间的单色光的干涉条纹,由于
激光光束很细,屏上的干涉条纹近于明暗相间的亮点,若采用
扩束装置将激光扩束后照到双缝上,可以得到双缝干涉的平
行条纹。但扩束后明条纹的亮度很低,实验需在暗室中进行。

\subsubsection{薄膜干涉}

可让学生分组做“肥皂液膜上光的干涉”实验.金属
丝圆环用黄铜丝或铁丝自制,肥皂液应清洁无杂物,浓度适
当。往酒精灯芯上撒食盐,火焰是黄色的。把肥皂液膜靠近酒
精灯,通过薄膜的反射去看黄色火焰,在薄膜上可见到明暗相
间的干涉条纹。若用白光照射,在薄膜上见到彩色的干涉
条纹。

对于一般的光源,能够见到干涉条纹时,薄膜的厚度须足
够薄。因此,肥皂液的浓度不能太浓。干涉条纹一般先出现在薄
膜的上部,往往当薄膜将要破裂时,才能见到较多的干涉条纹。

仪器所附的牛顿环,是由一块圆的平板玻璃和一个
凸透镜叠合而成的。球面与平板玻璃间形成空气膜,其厚度
由接触点向外逐渐增大。调节框上的三个调整螺钉,使干涉
图样位于中心部分。但不要拧得过紧,以免玻璃破碎。由牛
顿环的空气膜产生的干涉条纹是等厚条纹,利用光的干涉、
衍射、偏振演示仪做牛顿环投影实验装置如图6.5。将牛
顿环置于导轨的一端,把光源、凸透镜、毛玻璃屏按图中所示
位置放置,使凸透镜($f=7$厘米)距牛顿环大约8厘米。光源
的光斜射到牛顿环上,使反射光过凸透镜在毛玻璃屏上成
像,稍稍调整牛顿环的位置,在毛玻璃屏上可见到清晰的圆
环形彩色干涉条纹。条纹越向外越密,条纹的中心是暗
斑。

\begin{figure}[htp]
    \centering
    \includegraphics[scale=.6]{fig/6-5.png}
    \caption{}
\end{figure}

若将凸透镜放在牛顿环的另一侧,并移动它的位置,也可
在光屏(如白墙)上见到牛顿环的干涉条纹,这是光透过牛顿
环后经凸透镜成的像,牛顿环的透射干涉条纹的中心是亮
斑,和反射干涉条纹是互补的。拧动牛顿环边缘上的调整螺
钉,干涉条纹的形状、位置和圆环半径的大小都会发生变化。

本实验用低压电源供电,电压为6—10伏.

\subsubsection{单缝衍射}
用光的干涉、衍射、偏振演示仪(J2508型)做单缝衍
射实验.装置与图6.3相似,需将图中的双缝换成宽度为
0.08毫米的衍射单缝,如图6.6所示,调整光源单缝,使衍射
单缝和观察筒共轴。调整方法和双缝干涉实验相同。
\begin{figure}[htp]
    \centering
    \includegraphics[scale=.6]{fig/6-6.png}
    \caption{}
\end{figure}

用游标卡尺的外测脚做宽度可调的单缝代替衍射单缝,
可以演示衍射条纹随单缝宽度变化的情况。缝宽较大时,如
缝宽为2毫米,屏上为边缘清晰的亮线.当缝宽减小时,屏上
的亮线的宽度也减小。如果进一步减小单缝的宽度,可以见
到亮线的两侧出现衍射条纹。缝宽再减小,衍射条纹的宽度
反而变大,只是明条纹的亮度降低。实验应该注意:
\begin{enumerate}
\item 缝宽较
大时,屏上的亮线中似有明暗的条纹,这不是衍射条纹,而是
仪器的光源卤钨灯灯丝的像的一部分(灯丝像的其余部分
被可调狭缝屏遮挡住了)。
\item 用卡尺的测脚做狭缝,缝宽较大
时,屏上有时也出现明暗相间的条纹,这是从卡尺反射的光和
从光源单缝直接射出的光相干涉的条纹。解决的办法是稍稍
转动卡尺,使测脚平面反射的光不能达到观察筒内的毛玻璃
屏上。
\end{enumerate}


利用这套实验装置还可以做不透明的单丝衍射实验,取
直径在0.1毫米左右的细丝(如头发丝、多股绞合电线中的一
股等),用胶水竖直地粘在光具架上,替换衍射单缝。光源单缝
换用宽0.025毫米的,转动单缝的位置,使它和不透明的单丝
平行,在屏上可以看到与单丝平行的衍射条纹,特别是在单丝
的“影子”中央有一条亮线。

用激光光源做单缝衍射实验,激光光束直接照到狭
缝上,通过狭缝后再投影到远处光屏上,可以看到明暗相间的
衍射条纹,改变狭缝的宽度,可以看到狭缝越窄,衍射条纹越
远,但亮度减弱。

用激光光源还可以做圆孔衍射实验,用细针在牙膏皮上
扎出不同孔径的圆孔,激光光束照射到小孔上,通过小孔后直
接投射到远处的光屏上,可以看中心处是最亮的圆形光斑,
周围还有明暗相间的圆环形条纹,孔径越小,中心亮班及周围
的圆环形条纹的面积越大。

\subsubsection{衍射光栅}
\begin{figure}[htp]
    \centering
    \includegraphics[scale=.6]{fig/6-7.png}
    \caption{}
\end{figure}

用光的干涉、衍射、偏振演示器(J2508型)做光栅衍射的
实验,装置如图6.7。实验时将光源、光源单缝、凸透镜和毛
玻璃屏共轴放置在光具座上。调整凸透镜($f=7$厘米)的位
置(距光源单缝稍大于7厘米),使得光源单缝在毛玻璃屏中
央成清晰的放大实像,再将衍射光栅装在光具架上,缝座的指
示刻线对齐光具架的零刻线,把光具架插到凸透镜和光屏之
间,并靠近凸透镜。调整光源单缝与光栅的刻线平行,在毛玻
璃屏上就可以见到光栅的衍射条纹。

为了说明衍射明条纹随着缝数增加而变窄且亮度增强的
特点,实验时,可在光具架上顺序装置单缝、双缝、多缝和光
栅,比较它们的衍射条纹,上述特点很容易由实验得到。单缝、
双缝和多缝的衍射亮条纹的亮度较低,可用观察筒代替毛玻
璃屏,便于同学观察。

实验用6—10伏交流电.

\subsubsection{光的偏振}

演示自然光通过偏振片产生偏振光可用光的干涉、
衍射、偏振演示仪。将光源、两个偏振片(分别装在两只光具
架上)和毛玻璃屏依次装在光具座上,使它们的中心在平行于
光具座导轨的同一条直线上。当两个偏振片的偏振化方向
(用偏振片座上的指针表示)平行时,毛玻璃屏上有明亮的光
斑;固定一个偏振片(为方便演示,偏振化方向可取竖直方
向),转动另一个偏振片,当两个偏振片的偏振化方向垂直时,
屏上的光斑几乎消失。继续转动偏振片,屏上光斑的亮度出
现周期性变化。

\begin{figure}[htp]
    \centering
    \includegraphics[scale=.6]{fig/6-8.png}
    \caption{}
\end{figure}

反射光的偏振实验装置如图6.8所示.玻片反射起
偏器置于光具架上,使其框上的刻线对准入射角为$57^{\circ}$的定
位点,先将光源置于起偏器左侧,转动光源,使出射光束经玻
片反射后,光斑落在毛玻璃屏的中央。再把偏振片装入光具
架内。转动偏振片,当指示偏振化方向的指针处于竖直方向
时,屏上有明亮的光斑。指针处于水平方向时,屏上的光斑几
乎消失。

上述两个实验用6—10伏低压电源.

偏振片可以用观看立体电影偏光眼镜片代替,左
右两只眼镜片的偏振化方向互相垂直。光源可选取平行光源
(幻灯、手电筒、激光光源等),通过两个偏振片将光斑投影在
墙上,转动其中任一偏振片,很容易观察到光的偏振现象。

反射光偏振起偏器,可以用一般的平板玻璃,将其一面
涂黑,用以吸收透过玻璃的光,当光线的入射角为布儒斯特
角时,从玻璃表面反射的光是平面偏振光,其振动方向和光的
入射平面垂直。再让反射光通过偏振片,转动偏振片的方位,
可看到屏上光斑的亮度周期性的变化,利用反射偏振光可以
检查偏振片的偏振化方向。






















































\chapter{光的粒子性}\minitoc[n]
\section{教学要求}

本章讲述光电效应、爱因斯坦的光子说的主要内容,还简要讲述光的波粒二象性以及物质波的概念。通过本章的教学,使学生进一步了解光的本性,初步接触一些量子观念,对微观粒子的特性有一个初步印象,本章知识也为学习原子物理作了准备。

本章知识大都只需定性了解。但由于微观粒子的性质不同于宏观物体,学生往往习惯于用观察宏观现象时形成的观念去认识微观粒子,因而在理解微观粒子的性质时会遇到困难,教学中要特别注意使学生了解微观现象与宏观现象有着本质上的差别,以逐步适应这一章内容的教学。

本章可分为两个单元。第一、二、三节为第一单元,讲述光电效应-光的粒子性;第四、五节为第二单元,讲述微观粒子的波粒二象性。

光电效应现象及其规律,是认识光的粒子性的基础,最好是做好紫外线照射锌板时发生光电效应的演示,以加深学生对现象的认识,但是,教材中介绍的研究光电效应规律的实验,比较复杂,不易做准确。因此教学中只要讲清实验现象,使学生能在了解实验现象的基础上,理解光电效应的规律就可以了,不要求一定在课堂上做演示。

光的波动说不能解释光电效应的规律,为了理解这一点要用到下述知识:按照波动理论,光的能量由光的强度决定,而光的强度又由光的振幅决定,与频率无关。这一知识学生过去没有学过,可联系声波的知识(振幅越大声音越强)给学生讲一讲,以帮助学生理解光电效应与光的波动说之间的矛盾。爱因斯坦对光电效应的解释是在普朗克量子说的基础上作出的,教学中也应对普朗克量子说的主要内容作些简单的介绍。

爱因斯坦的光子说及其对光电效应的解释,是本章的重点内容,讲述爱因斯坦对光电效应的解释时,要抓住入射光与金属中自由电子间能量接受关系的线索,着重说明光电效应的极限频率,要让学生理解逸出功的概念,知道不同金属的逸出功不同。

光的波粒二象性是光的两大对立学说的历史性的综合。学生在分别认识了光的波动性和粒子性后,要把这两种认识统一起来仍然是有困难的。教学时要提醒学生,不能用机械模型来想象波粒二象性,爱因斯坦的光子能量公式和动量公式与表征波的特征的频率和波长相联系,表明他的光子说与牛顿的微粒说有着本质的区别,对于公式$p=E/c$, 学生知道就行,不要求进行讲解。

物质波一节教材,是选讲内容,对学生不作要求,但这节内容扩大了对波粒二象性的认识,说明了一切微观粒子都具有波粒二象性。如果时间允许,讲一讲还是必要的和有益的。

本章的教学要求是:
\begin{enumerate}
\item 了解光电效应的基本规律和光电效应的应用.
\item 知道光子说的基本内容及其对光电效应的解释,掌握
爱因斯坦光电效应方程。
\item 了解光的波粒二象性的初步概念.
\end{enumerate}


\section{教学建议}
\subsection{光电效应——光的粒子性}

\subsubsection{光电效应}

光电效应是学生认识光的粒子性的实验基础,所以最好是做好课本图7.1的演示.实验时,要注意验电器金箔闭合或张开的现象都不能简单地说明光电效应,需要对实验现象进行分析后才能得出结论。用紫外线照射带负电的锌板,验电器的金箔立即闭合的实验,需用紫外线照射带正电的锌板,验电器的金箔不闭合的实验做对比,表明紫外光照射锌板只能打出负电荷;用紫外线照射不带电的锌板时,验电器的金箔张开,应补充检验锌板上带正电荷而不是负电荷,这样,通过对这些实验现象的深入分析,才能归纳出现象的共同本质,即光照射下物体中能够发射电子,进而引出光电效应的概念。

光电效应规律的实验,有条件的学校也可以做,没有条件的学校可按照课本274页的装置简图描述实验的过程及现象,使学生理解怎样通过实验得出光电效应的实验规律。

有的学生不理解课本图7.2电路中的光电流的回路,光电子到达极板$A$后,经电流表再通过电路(包括滑动变阻器和
电池组两个支路)到达极板$K$形成回路,这是由于极板$K$发射光电子后带正电荷,光电子到达极板$A$后,极板$A$带负电荷。只有光电子从极板$A$经光电管以外的电路到达极板$K$形成回路,才能持续不断地维持光电流。

光电效应的教学中,应强调它的主要规律,即极限频率的存在以及光电子的最大初动能只与入射光的频率有关而与人射光的强度无关,这是波动说无法解释的实验事实,也是引入光子说的主要依据。把这两点讲清楚了,也就为讲述本单元的核心内容——爱因斯坦的光子说打下了基础。

\subsubsection{光子和光子说}
学生接受光子说的主要困难在于建立光量子的观念。教学中要强调光子说与波动说不同,光子说认为光的能量不是连续分布的,而是一份一份地集中在一个个粒子——光子上。光的能量既跟光子的数目有关,也跟每个光子的能量有关。每个光子的能量$E=h\nu$, 与光波的频率成正比。

在介绍爱因斯坦的光子说时,可联系普朗克的量子假设,说明理论的继承和发展的关系,学生没有学过黑体辐射的规律,这里也不必补充这方面的知识,只需简要介绍普朗克提出电磁辐射的发射和吸收时,能量是不连续的、一份一份进行的。爱因斯坦则进一步推广了普朗克的假设,认为自由电子与电磁辐射相互作用时,吸收的能量也是不连续的、一份一份的。光就是由这种不连续的粒子组成的。

帮助学生按照光子说建立光的粒子模型:光的能量分布在一个个光子上,光束可以看作光子流,每个光子都以光速运动着。光和其他物质相互作用,就是光和其他物质交换
一份一份的光子的能量。

在运用光电效应解释光电效应的规律时,要抓住爱因斯坦的光电效应方程$\frac{1}{2}mv^2_m=h\nu-W$, 着重说明极限频率的存在和入射光子的频率决定光子的最大初动能等实验规律。要让学生了解光电效应方程中各个量所代表的物理意义,知道这个方程是对单个光子而言的,并能从方程上看出,如果入射光的频率很低,$h\nu$小于金属的逸出功$W$, 自由电子就不会从金属中逸出。只有当光子的能量达到或超过金属的逸出功时,才能产生光电效应。

逸出功是个很重要的概念,应使学生掌握它的意义和计算式。学生还应该知道,不同金属的逸出功不同,它们的极限频率也不同。

教学时要注意能量的单位焦耳与电子伏的换算关系。

在运用光电方程解释光电效应规律时,还可就粒子说和波动说对应的能量、光的强度的认识进行对比,以加深对光子说的认识。

\subsubsection{光电效应的应用} 

这一节教材讲述的重点是光电管
的作用。因此,教学时要使学生清楚地了解光电管的工作原理。

在讲述具体应用时,应使学生了解光控继电器和有声电影是怎样利用光电管来工作的。对光控继电器,应了解电路的基本组成,应清楚是光电管将光信号变为电信号达到自动控制电路的作用的。在有声电影的放映中,是利用光电管把声音信号还原为声音的。对于装置中的其他问题,教学中不必
详细介绍。

\subsection{波粒二象性}
\subsubsection{光的波粒二象性}

光的波粒二象性是十七世纪以来关于光的本性的微粒说和波动说两大对立学说的历史的综合,是对光的本性进行长期研究得出的结论。通过光的波粒二象性的教学,一方面使学生对光的本性有比较完整的了解;另一方面使学生对于描述微观粒子的量子概念有初步的了解。

光的波粒二象性的观点,用经典物理观念是无法理解的。因此只有破除旧观念的影响,才能接受近代物理对微观粒子运动规律的认识,才能理解波粒二象性,但是限于学生的知识基础,在高中物理课中不可能以丰富的事实和有很强说服力的论证去破除学生头脑中已有的经典物理观念,所以对于学生对波粒二象性的了解不能要求过高。教学中只要学生对课文中所述内容大体知道即可,不宜再补充内容。

\subsubsection{物质波}

把光的波粒二象性推广到一切微观粒子上,就是德布罗意提出的物质波的假说。电子衍射现象证明了电子的波动性,说明了物质波的理论是正确的。在教学中,要注意介绍电子衍射实验,使学生了解物质波理论的实验验证,并说明物质波也是一种几率波。同时,还要使学生了解,虽然一切微观粒子具有波粒二象性,但光子跟其他微观粒子还是有区别的,光子是永远以光速运动的、没有静止质量的粒子。

这一节的教学,对扩展学生的知识面是有益的。通过对物质波假说的提出和检验过程的了解,还可以对学生进行科
学的方法论和辩证唯物主义认识论的教育。

\section{实验指导}
\subsection{演示实验}
\subsubsection{光电效应现象}

被紫外线照射的锌板能发射出电子,紫外线光源可用弧光灯(摘掉聚光透镜)、紫外线灯或用荧光高压汞灯(除去涂有荧光粉的外壳,从灯芯直接射出的光中含有紫外线)。 实验室中高压汞灯的接线电路如图7.1所示.图中的镇流器可用万用变压器的初级0—110伏的线圈代替.
\begin{figure}[htp]\centering
    \begin{minipage}[t]{0.48\textwidth}
    \centering
    \includegraphics[scale=.6]{fig/7-1.png}
    \caption{}
    \end{minipage}
    \begin{minipage}[t]{0.48\textwidth}
    \centering
    \includegraphics[scale=.6]{fig/7-2.png}
    \caption{}
    \end{minipage}
    \end{figure}

图7.2中的锌板必须用细砂纸打磨,除去表面的氧化层,
把表面打磨干净的锌板用细尼龙线直接捆绑在指针验电器的小球旁边。

用毛皮摩擦过的橡胶棒使锌板带上负电,验电器的指针偏开一定的角度。用紫外线照射锌板时,见到验电器的指针逐渐回到0处,表明在紫外线照射下锌板所带的负电荷消失了。

若使锌板带上正电,用紫外线照射时,验电器的指针的偏角几乎保持不变,表明锌板上的正电荷并不因紫外线的照射而消失。

只用紫外线照射不带电的锌板时,验电器的指针不偏转。若在锌板附近放置带正电的玻璃棒,可以看到验电器的指针发生偏转。进一步用带正电的玻璃棒或带负电的橡胶棒靠近锌板,可以用实验检验出锌板上带的是正电荷。若用带负电的橡胶棒靠近锌板时,验电器的指针也不偏转。

这是由于,锌板在紫外线作用下能够放出电子,锌板带正电。释放出的电子聚集在锌板周围形成空间电荷区,使锌板和周围空间电荷区之间形成反向电压,阻碍光电子的发射,只有用带正电的物体吸附了锌板周围的负电荷,才能使锌板发射出较多的光电子,验电器的指针才能发生偏转。

实验时要注意防止紫外线伤害眼睛。

\subsubsection{研究光电效应的规律}

研究光电效应的规律的实验线路如图7.3, 其中光电管的型号是GD-28,它的阴极材料为铯锑(CsSb), 极限波长约为0. 6500微米,灵敏度大干10微安/流明,工作电压24伏,光源可用低压白炽灯,如双缝干涉仪(J2515型)的光源,电表可用一般的演示用电表,如J0401型等.
\begin{figure}[htp]
    \centering
\includegraphics[scale=.6]{fig/7-3.png}
    \caption{}
\end{figure}

(1)光电效应的产生

按图连好电路,光电管两极加正向电压约15伏左右,这时电路中没有电流。接通照明电路,当光照到光电管的光入射窗口时,电路中有了光电流。用黑纸板遮挡入射窗口,电流随即消失;移开挡板,又有光电流。

(2)光电管的饱和光电流随入射光强度增大而增大

接通电路,微安表中有光电流通过。移动滑动变阻器的滑动触头,使加在光电管$A$, $C$两极间的正向电压增大,光电流也增大,直到光电流达到饱和值,照到光电管阴极的光强度要适当(入射光不能过强,否则会影响光电管的使用寿命),要使光电流的饱和值小于80微安,正向工作电压小于24伏.提高入射光的强度(如把光源移近光电管或提高白炽灯的供电电压),光电流的值继续变大,饱和光电流的值也随之增大。

(3)将光电管的两极改接反向电压

电压较小时,电路中仍有光电流,逐渐提高反向电压的值,光电流逐渐减小,直到电压达到某一值时,光电流变为0. 再提高入射光的强度,光电流仍为零,在光入射窗口换上不同颜色的滤色片,可以见到,对于紫光阻止光电流的反向电压最大。

(4)光电管阴极的极限波长,极限效率

电路接正向电压,在入射窗口放蓝色(或绿色)滤色片时,电路中有光电流,若加红色滤色片时,即使是增大入射光的强度,电路中也没有光电流。

\subsubsection{光电效应的应用}

光电管、电流放大器(光电效应演示器中的附件)、演示用继电器(J2413型)和小灯泡按图7.4连好电路.用灯泡代表
计数器,接继电器的常闭触点,继电路的原线圈和电流放大器相连,光照在光电管上,电路中有电流,继电器动作,常闭触点打开,电灯不亮,代表计数器不工作,用手挡住射入光电管的光流,继电器原线圈中的电流消失,常闭触点接通,灯泡亮代
表计数器工作一次.若没有电流放大器,可按图7.5电路自制。

\begin{figure}[htp]\centering
    \begin{minipage}[t]{0.48\textwidth}
    \centering
    \includegraphics[scale=.6]{fig/7-4.png}
    \caption{}
    \end{minipage}
    \begin{minipage}[t]{0.48\textwidth}
    \centering
    \includegraphics[scale=.6]{fig/7-5.png}
    \caption{}
    \end{minipage}
    \end{figure}


\section{习题解答}


\subsection{练习一}

\begin{enumerate}
    \item 使锌板产生光电效应的光子的最长波长是0.37微
米,这种光子的能量是多少电子伏?锌的逸出功是多少?

\begin{solution}
光子的能量
\[E=h\nu=\frac{hc}{\lambda}=\frac{6.63\x 10^{-34}\x 3.0\x 10^8}{0.37\x 10^{-6}\x 1.6\x 10^{-19}}=3.4{\rm eV}\]
锌的逸出功$W=3.4{\rm eV}$
\end{solution}
\item 可见光的光子,能量范围是多大(用电子伏表示)?为
什么用可见光不能使锌板产生光电效应?

\begin{solution}
可见光的频率范围是$3.9\x10^{14}$—$7.7\x10^{14}$赫.光
子的能量
\[\begin{split}
E_1&=h\nu_1=\frac{6.63\x 10^{-34}\x 3.9\x10^{14}}{1.60\x10^{-19}}=1.6{\rm eV}\\
E_2&=h\nu_2=\frac{6.63\x 10^{-34}\x 7.7\x10^{14}}{1.60\x10^{-19}}=3.2{\rm eV}\\
\end{split}\]
可见光光子的能量范围为$1.6$—$3.2$电子伏,都小于锌的逸出功,因此不能使锌板产生光电效应。
\end{solution}
\item 铯的逸出功是$3.0\x10^{-19}$焦,用波长是0.59微米的
黄光照射铯,电子从铯表面飞出的最大初动能是多大?

\begin{solution}
最大初动能
\[\begin{split}
    \frac{1}{2}mv^2_m &=\frac{hc}{\lambda}-W\\
    &=\frac{6.63\x 10^{-34}\x 3.0\x 10^8}{0.59\x 10^{-6}}-3.0\x 10^{-19}\\
    &=3.7\x 10^{-20}{\rm J}
\end{split}\]
\end{solution}
\item 钨的逸出功是4.52电子伏,使钨产生光电效应的最
长波长是多少?这种波长是可见光吗?

\begin{solution}
因为$W=\dfrac{hc}{\lambda}$,所以极限波长是$\lambda_0=\dfrac{hc}{W}$,
\[\lambda_0=\frac{6.63\x 10^{-34}\x 3.0\x 10^8}{4.52\x 1.6\x 10^{-19}}{\rm m}=0.275\mu{\rm m}\]
这种波长的光不是可见光。
\end{solution}
\end{enumerate}



\subsection{习题}
\begin{enumerate}
    \item 功率为1瓦的手电筒灯泡大约有5\%的电能转化为
可见光,试估算它1秒钟能释放出多少个可见光的光子.

\begin{solution}
 可见光的频率按$6\x10^{14}$赫来估算,光子的能量$E=h\nu$, 一秒钟释放的光子数
\[n=\frac{Pt\x 5\%}{h\nu}=\frac{1\x 1\x 5\%}{6.63\x 10^{-34}\x 6\x 10^{14}}=10^{17}\text{(个)}\]
\end{solution}
\item 使铜产生光电效应的最低频率是$1.1\x10^{15}$
赫,用频率为$1.5\x10^{15}$赫的紫外线照射铜时,它发射出的光电子的
最大速度是多大?

\begin{solution}
\[\frac{1}{2}mv^2_m=h\nu-h\nu_0\]
最大速度
\[\begin{split}
    v_m&=\sqrt{\frac{2h(\nu-nu_0)}{m}}\\
    &=\sqrt{\frac{2\x 6.63\x 10^{-34}\x (1.5-1.1)\x 10^{15}}{9.1\x 10^{-31}}}\\
    &=7.6\x 10^5\ms
\end{split}\]
\end{solution}
\item 一个质量为$4\x10^{-4}$克的尘埃颗粒,以$1{\rm cm}/{\rm s}$
的速度在空气中下落,计算它的德布罗意波长.

\begin{solution}
    德布罗意波长
\[\lambda=\frac{h}{mv}=\frac{6.63\x 10^{-34}}{4\x 10^{-7}\x 1\x 10^{-2}}=1.7\x 10^{-25}{\rm m}\]
\end{solution}
\item 计算速度为$10^3\ms$的中子的德布罗意波长.这
个波长跟$\gamma$射线的波长相比如何?中子的质量是$1.67\x10^{-27}$
kg.

\begin{solution}
中子的德布罗意波长
\[\lambda=\frac{h}{mv}=\frac{6.63\x 10^{-34}}{1.67\x 10^{-27}\x 10^{3}}=4\x 10^{-10}{\rm m}\]
在$\gamma$射线的波长范围内。
\end{solution}
\end{enumerate}



\section{参考资料}
\subsection{早期量子论和光电效应}

量子的概念是德国物理学家普朗克研究黑体辐射时最先提出的。

在任何温度下都能全部吸收落在它上面的一切电磁辐射的物体叫做绝对黑体,简称黑体,一个不透射任何辐射的器壁围住带有一个小孔的空腔,可以作为绝对黑体的模型,实
验表明,黑体辐射的能谱(即光谱的能量分布曲线)与组成黑体的材料无关,只与黑体的温度有关,图7.6就是黑体辐射的能谱图.图中的$E_{\lambda}$表示波长在$\lambda$和$\lambda+\dd \lambda$之间时辐射的能量密度。可以看出,能量的分布有极其显著的最大值,这个最大值对应的波长$\lambda_m$随着黑体温度的升高而向短波方向移动。
\begin{figure}[htp]
    \centering
\includegraphics[scale=.6]{fig/7-6.png}
    \caption{}
\end{figure}

许多物理学家试图从经典理论推导黑体辐射的能量分布公式,结果都失败了。例如,英国物理学家瑞利和琼斯根据经典物理的电磁场理论和统计物理推得黑体辐射的能量分布公式,只在波长相当长的部分才与实验曲线相符合,而随着波长的减小,辐射通量趋于无限大,而被称为黑体辐射的“紫外灾难”。

1900年,德国物理学家普朗克假设黑体的腔壁由无数带电谐振子组成,并且假设这些谐振子的能量不能连续变化,只能取一些分立值:$0,e,2e,\ldots$这些分立值的能量是$e=h\nu$的整数倍。从而推出黑体辐射的普朗克公式,这个公式与实验曲线吻合得很好。

普朗克能量量子假设是对经典物理学的一个巨大的突破,1918年,普朗克由于对量子理论的贡献而荣获诺贝尔物理学奖。

爱因斯坦认识到普朗克量子概念带来的将是物理学理论基础的根本变革,无论是经典力学或经典电动力学都不能应用到微观世界所发生的过程,他进一步发展了能量子的假设,提出了光量子假说,普朗克理论认为空腔中的辐射场(电磁场)本质上仍然是连续的,只是当它们与腔壁谐振子发生能量交
换时才显示出量子性,爱因斯坦进一步认为光不仅是一份一份地被辐射或吸收,而且它的能量也是聚集成一份一份地在空间传播,光是由光子组成的粒子流,每个光子的能量$e=h\nu$.

光子学说最初是以假说的形式提出的,这是由于当时的实验事实还不充分,光电效应实验也是非常原始的,是在真空条件不很好的条件下做出的。爱因斯坦根据光子假说得出的光电效应方程成功地解释了光电效应的实验规律,但在当时并未被物理学家广泛承认,他们不赞成光子说,因为它违背了光的波动理论。

美国物理学家密立根花了十年的时间做光电效应实验,开始他不相信光量子理论,想用实验否定它,结果恰恰相反,他于1915年宣布:证实了爱因斯坦的光电效应方程,并根据该方程用实验测得普朗克常数$h$与普朗克公式提出的$h$值完全一致,其他的一些实验,如光压,康普顿效应等都证明了光量子假设的正确性,1922年爱因斯坦由于理论物理方面的贡献和光电效应方程而荣获诺贝尔物理奖,密立根也于1923年由于他在基本电荷和光电效应方面的研究工作荣获诺贝尔物理奖。


\subsection{外光电效应和内光电效应}

通常所说的光电效应是指外光电效应,即物体在光的照射下光电子飞到物体外部的现象。另一种光电效应叫内光电效应,它是物体在光的照射下,内部原子中的一部分束缚电子变为自由电子,这些电子仍留在物体内部,使物体的导电性
加强。

利用内光电效应可以制成光敏电阻、光敏二极管和光电池。光敏电阻的阻值随着光照射的强弱而明显地变化。光敏二极管的顶部有玻璃透镜,通常给光敏二极管接上反向电压,如果没有光照,反向电阻很大,电路里几乎没有电流;有光照时,反向电流就随着增大,经三极管放大后,可推动继电器工作。

\begin{figure}[htp]
    \centering
\includegraphics[scale=.6]{fig/7-7.png}
    \caption{}
\end{figure}

光电池的种类很多,早期有氧化铜光电池和硒光电池,现在又有硅、砷化镓、硫化镉、磷化铟等光电池。硅光电池是由单晶硅材料制成的、具有大面积PN结的半导体器件.图7.7
是硅光电池的结构示意图,PN结两边各有一个引出电极,就是光电池的正负极。光电池的表面镀有一层增透膜,其作用是为了减小光的反射,提高光电池的转换效率。光电池之所以能把光能转变为能是应用了半导体PN结的光生伏打效应。在光照射下,物体内部原子的束缚电子变成自由电子,
形成自由电子和空穴。在结电场作用下,自由电子向N区运动,空穴向P区运动,于是在N区和P区分别聚集了电子和空穴,产生了电动势。接有负载时,负载中就有电流通过。现在的硅光电池在强光照射下,能产生0.5伏的电动势,每平方厘米工作面积输出24毫安的光电流,相当输出功率10—12毫瓦。使用时,可把大量的硅光电池串联和并联起来,以得到所需要的电压和电流。光电池作为电源应用在人造地球卫星和灯塔、无人气象站等处。










\chapter{原子结构}\minitoc[n]
\section{教学要求}
人类对原子结构的认识,是逐步深入的.本章教材是沿着历史的线索来讲述人类对原子结构的认识的.由于玻尔的原子理论在一定程度上反映了原子的真实情况,又比较适合于中学生的理解能力和知识水平,因此,把这个内容作为本章教学的重点.但教材也指出了玻尔理论的局限性,粗略地介绍了现代原子理论中关于原子结构的一般图景.

这一章的教学可分为三个单元.第一节和第二节为第一单元,讲述原子的核式结构学说的建立.第三节至第五节为第二单元,讲述玻尔理论.第六节为第三单元,讲述原子的受激辐射和激光.

电子的发现,对人类认识原子的结构有重大的意义,它使人们改变了认为原子是组成物质的最小微粒的看法,使人们认识了原子是有结构的.教学中应注意讲清汤姆生研究电子的方法.汤姆生的原子模型是早期的原子模型,讲述这个模型的目的,一方面是为使学生了解人类对原子的认识经历了曲折的发展过程,更重要的是为了使学生认识$\alpha$粒子散射实验的重要意义.

$\alpha$粒子散射实验证明了原子核的存在,通过教学,应当
使学生了解$\alpha$粒子散射实验及其结果,了解卢瑟福是怎样分析$\alpha$粒子的散射现象,否定了汤姆生原子模型,提出原子具有核式结构的学说的.

在分析卢瑟福原子模型的困难时,要用到电学、力学和光谱发射的知识.其中有的知识,学生没有学过.例如,根据经典电磁理论,绕核做加速运动的电子要向外辐射电磁波,电磁波的频率等于电子绕核运行的频率,学生就没有学过.对这些内容,不宜过高要求,让学生知道卢瑟福原子模型与原子的稳定性和光谱的发射性质有矛盾就可以了.

玻尔的原子理论,突破了经典理论的束缚,引入了量子化假设,把原子理论推进了一步.应该要求学生了解玻尔理论的主要内容,对原子的轨道和能量的量子化获得具体的认识.明确知道玻尔对原子结构提出的新概念,体会玻尔的创新精神.对于玻尔理论遇到的困难和建立在量子力学基础上的现代原子理论,只作很初步的一般性介绍.

本章中关于激光产生的原理和应用是选讲教材,向学生介绍这个内容可以扩大学生的眼界,引起学习的兴趣,只要教学时间允许,就应该讲一讲这节教材.

本章的教学要求是:
\begin{enumerate}
\item 了解$\alpha$粒子的散射实验和卢瑟福的原子核式结构.
\item 了解玻尔原子理论的主要内容及其对氢光谱规律的解释.
\end{enumerate}

\section{教学建议}
\subsection{原子核式结构的建立}
本单元讲述了电子的发现以及早期的原子结构模型——汤姆生模型,重点介绍了$\alpha$粒子散射实验.该实验否定了汤姆生模型,奠定了原子核式结构模型的基础.

本单元的教学,应注意以下几点:

\subsubsection{电子的发现}

这部分内容的教学,要突出电子的发现
对人类认识原子结构的重要作用.

电子是怎样发现的.汤姆生用测定粒子荷质比的方法发现了电子,这个问题在本书第一章中曾经提到过.这里要注意联系已学过的知识,讲清汤姆生方法的原理.汤姆生发现阴极射线在电场和磁场中的偏转现象,根据偏转方向,确认阴极射线是带负电的粒子流,当他测定阴极射线粒子的荷质比时发现,不同物质做成的阴极发出的射线都有相同的荷质比,这表明它们都能发射相同的带电粒子.因此这种带电粒子是构成物质的共同成份,这就是电子.

电子的发现对人类认识原子结构的重要性.电子的发现,使人们认识到原子不是组成物质的最小微粒,原子本身也具有结构.由于原子含有带负电的电子,从物质的电中性出发,推想到原子中还有带正电的部分.这就提出了进一步探索原子的结构、建立原子模型的问题,

\subsubsection{汤姆生的原子模型}

汤姆生原子模型是在发现电子的基础上建立起来的,通过教学要使学生对汤姆生原子模型有一个形象的了解,教学中应该注意,讲述汤姆生模型的目的是为了使学生从原子学说的历史发展上来认识$\alpha$粒子散射实验的重大意义,为进一步理解原子的核式结构作好准备.

\subsubsection{原子的核式结构的发现}

$\alpha$粒子散射实验以及卢瑟福对实验现象的分析为原子的核式结构模型奠定了基础.教学中可按以下几个层次进行教学:为什么用$\alpha$粒子的散射现象可以研究原子的结构,$\alpha$粒子的散射实验是怎么做的,实验结果是什么,分析实验结果得到怎样的原子模型.

原子的结构非常紧密,用一般的方法无法探测它内部的结构,要认识原子的结构,需要用高速粒子对它进行轰击,由于$\alpha$粒子具有足够的能量,可以接近原子的中心,它还可以使荧光物质发光,如果$\alpha$粒子与其他粒子发生相互作用,改变了运动的方向,荧光屏便能够显示出它的方向变化,因此卢瑟福采用$\alpha$粒子散射的方法来研究原子的结构.

$\alpha$粒子散射实验的装置,可根据课本上的示意图来讲述,主要由放射源、金箔、荧光屏、显微镜和转动圆盘儿部分组成,每一部分的作用应该让学生明确,实验的做法,课文中写得比较简明,重点应指出荧光屏和显微镜能够围绕金箔在一个圆周上转动,从而可以观察到穿过金箔后偏转角度不同
的$\alpha$粒子,要让学生了解,这种观察是十分艰苦细致的工作,所用的时间也是相当长的.

必须让学生明确,实验结果可以把入射的$\alpha$粒子分为三部分,这三部分$\alpha$粒子的大致多少是用“绝大多数”、“少数”和“极少数”这样的数量形容词来描述的,它们穿过金箔后的情况分别是沿原来的方向前进、发生了较大的偏转和大角度偏转.卢瑟福的原子核式结构模型就是在分析了这三部分$\alpha$粒子的情况后建立起来的.

对实验结果的分析应着重说明如下几点:
\begin{enumerate}
\item 电子不可能使$\alpha$粒子发生大角度散射.$\alpha$粒子跟电子碰撞过程中,两者动量的变化量相等,由于$\alpha$粒子的质量是电子质量的7300倍,在碰撞前后,质量大的$\alpha$粒子速度几乎不变,而质量小的电子速度要发生改变.因此,$\alpha$粒子与电子正碰时,不会出现被反弹回来的现象.发生非对心碰撞时,$\alpha$粒子也不会有大角度的偏转.可见,电子使$\alpha$粒子在速度的大小和方向上的改变都是十分微小的.
\item 按照汤姆生的原子模型,正电荷在原子内部均匀地分布,$\alpha$粒子穿过原子时,由于粒子两侧正电荷对它的斥力有相当大一部分互相抵消,使$\alpha$粒子偏转的力也不会很大.$\alpha$粒子的大角度散射现象,说明了汤姆生模型不符合原子结构的实际情况.
\item 实验中发现极少数$\alpha$粒子发生了大角度偏转,甚至反弹回来,表明这些$\alpha$粒子在原子中的某个地方受到了质量、电量均比它本身大得多的物体的作用.
\item 金箔的厚度大约是1微米,金原子的直径大约是$3\x10^{-10}$米,绝大多数$\alpha$粒子在穿过金箔时,相当于穿过几千个金原子的厚度,但它们的运动方向却没有发生明显的变化,这个现象表明了$\alpha$粒子在穿过时基本上没有受到力的作用,说明原子中的绝大部分是空的,原子的质量和电量都集中在体积很小的核上.
\end{enumerate}

\subsubsection{原子核的电荷和大小}
 原子核的电荷这部分内容,要
突出测知原子核的电荷的重要意义.应该使学生了解,根据$\alpha$粒子散射实验的数据可以算出靶元素原子核的电荷,从而推知这种原子中的电子数.计算的结果表明,元素原子中的电子数非常接近该元素在周期表中的原子序数.人们由此知道元素周期表是按原子中的电子数来排列的.这就是说,元素的化学性质,归根到底是由原子中的电子数、从而是由原子核中的电荷数来决定的.

关于原子的大小,应该让学生记住一个数量级,即原子核的大小在$10^{-14}$米以下,原子的大小是$10^{-10}$米,所以原子核的半径只相当于原子半径的万分之一,这样,原子核的体积与原子体积的比应为万亿分之一.这里突出了原子核是很小的,原子内部是很空的.

为培养学生看书学习能力,可以通过学生自学课文,
教师提出参考讨论题让学生讨论,最后由教师归纳总结出课文的基本内容.也可由学生自己提出讨论问题,归纳课文的主要内容,进行讨论,最后由教师进行小结,象这样的看书自学讨论的方式,可以作为这一章的各节课均可采用的一种教学方法.

\subsection{玻尔理论}
这一单元是本章的重点,主要讲好玻尔理论的三点假设,让学生理解能量量子化的概念,这是玻尔理论的核心,教
学中,要注意以下几点:

\subsubsection{玻尔理论产生的背景}

应该让学生清楚地认识到,卢瑟福的原子核式结构模型虽然能很好地解释$\alpha$粒子散射实验,但跟经典的电磁理论发生了矛盾.这些矛盾说明从宏观现象总结出来的经典电磁理论不适用于微观现象,不解决这个矛盾,原子理论就不能前进,这就是产生玻尔原子理论的历史背景.

卢瑟福的原子核式结构模型与经典电磁理论的矛盾主要有两点:按照经典电磁理论,电子在绕核作加速运动过程中,要向外辐射电磁波(这一点学生没有学过,可由教师告诉他们),因此能量要减少,电子轨道半径也要变小,最终会落到原子核上,因而原子是不稳定的;电子在转动过程中,随着转动半径的缩小,转动频率不断增大,辐射电磁波的频率不断变化,因而大量原子发光的光谱应该是连续光谱.然而事实上,原子是稳定的,原子光谱也不是连续光谱而是线状光谱.

\subsubsection{玻尔原子理论的主要内容}
玻尔把量子观念引入到原子理论中去,提出了不能用经典概念解释的三条假设,是一个创举.这三条内容应作为整体来理解.为了便于学生掌握玻尔理论,不妨把这三条叫作能级假设,跃迁假设和轨道假设.

能级假设说明原子只能处于一系列不连续的能量状态中,这些状态叫定态,具有一定的能量,也叫能级.能级假设是针对原子的稳定性提出的,它承认核式模型,但假定原子只能处于一系列不连续的稳定状态中,处于稳定状态的原子中的电子,虽做加速运动但不辐射能量,对于“不连续”的概念,
学生是不习惯的.一定要使学生明白,从宏观现象的“连续”的概念过渡到微观世界的“不连续”的概念,是人类对物质世界认识上的一次飞跃.

跃迁假设说明原子从一个定态跃迁到另一个定态时,它辐射或吸收一定频率的光子,光子的能量由这两个定态的能量差决定,它说明了原子发光的机制.这一条假设是针对原子光谱是线状光谱提出的,运用了普朗克的量子理论.辐射光子的能量与发生跃迁的两个轨道有关.

轨道假说明原子的不同能量状态对应于电子的不同运行轨道,由于原子的能量状态是不连续的,因此电子的轨道也是不连续的,即电子不能在任意半径的轨道上运行.轨道量子化假设也是针对原子的核式模型提出的,是对第一条假设的补充.

教学时,指出三条假设的针对性,可便于学生理解玻尔理论.教学时还可以把下一节课文中的图8.5氢原子的轨道图提前到这里让学生观看,可以把它画成较大的挂图,据图讲述其轨道“不连续”的含义,让学生对“不连续”的量子观念有一个形象而具体的了解.

\subsubsection{氢原子的大小和能级} 

这是玻尔理论对氢原子的应用获得成功的具体内容,通过这部分内容的教学,应让学生对能量量子化有比较具体的了解.在讲解原子的能级时,应该说明为什么能量是负值,懂得负值的含义是原子处于某一定态时的能量比电子距核无限远时原子具有的能量少,并使学生通过计算大致了解不同能级的能量值和相应的电子轨道半径大小,为使学生头脑中形成轨道和能量量子化的具体图
景,可在本节课上让学生研究课本图8.4和图8.5. 在此基础上来理解什么是基态,什么是激发态,了解原子发光的机制.

\subsubsection{玻尔原子理论对氢光谱的解释}

这个问题,可首先从氢光谱巴耳末线系的四条可见谱线出发,简单介绍一下巴耳末经验公式(顺便介绍里德伯恒量),说明氢光谱谱线之间是有内在规律的,然后再进一步使学生了解按照玻尔理论推导出来的谱线公式跟巴耳末经验公式在
形式上完全一致,而且由前一个公式计算出来的$\dfrac{-E_1}{hc}$的数值
与巴耳末公式中的里德伯恒量的$R$值相符.这说明了巴耳末公式完全可以由玻尔理论推导出来,玻尔理论可以解释氢光谱的规律.

还应向学生说明,氢光谱的其他线系也可以用玻尔理论来解释,由玻尔理论预言存在的新线系,后来也被人们发现,充分说明了玻尔理论的成功.

教学时还应该让学生了解,氢光谱中的每个线系,都是原子从不同的高能级向某一低能级跃迁时发出的谱线.例如,赖曼系是氢原子的电子从$n=2, 3, 4,\ldots$等能级跃迁到$n=1$的基态时发出的谱线;巴耳末线系是氢原子的电子从$n=3, 4, 5,\ldots$等能级向$n=2$能级跃迁时发出的谱线,等等.光谱线上的每一条谱线都是大量处于同一能态的原子的电子向同一低能态跃迁的结果,由于每个原子的电子所处的能态不同,大量原子的跃迁在同一时刻,会发出不同频率的光来,因此光谱线上能够出现各种谱线.

学生对同一线系中各个谱线波长的大小排列顺序与对应的能级跃迁之间的联系往往理解不好,课堂上可以通过具体计算,帮助学生理解.

\subsubsection{玻尔原子理论的困难和量子力学}

本节的教学,首先可简要指出玻尔理论遇到的主要困难,说明一下造成这种困难的原因在于理论内部的矛盾,玻尔理论是一种半经典的理论,一方面引入了量子假设,另一方面又应用经典理论计算电子轨道半径和能量.因此,玻尔理论在解释复杂的微观现象时遇到困难,乃是必然的.

关于量子力学和薛定谔方程等内容,学生不可能深入理解,可只做简单的讲述.但须指出,量子力学是彻底的量子理论,是研究微观世界的基本理论工具,它不但能解释玻尔理论所能解释的现象,而且能够解释大量玻尔理论不能解释的现象.玻尔理论中的三点假设,在量子力学中也变成理论上推导出来的直接结果.

教学中可着重说明,建立在量子力学基础上的原子理论与玻尔原子理论的区别:根据量子力学,核外电子的运动服从统计规律,而没有固定的轨道,我们只能知道它们在核外某处出现的几率大小,核外电子的这种运动情况可用“电子云”来形象描述.电子云稠密的地方就是电子出现几率大的地方.

新的原子结构理论虽然更加切合实际,但它仅给出了原子的数学模型,而没有给我们提供比较直观的物理图景,因此,学生想象这个模型有一定的困难,如果有些同学对量子力学比较感兴趣,可以引导他们阅读一些有关这方面内容的通俗书籍.

\subsection{原子的受激辐射——激光}

激光是一门重要的现代科学技术,在各个领域中有着广泛的应用,对新的技术革命有重要作用,学生了解了激光的知识,可以开阔眼界,了解现代科学技术的发展,增加学习物理的兴趣.

在教学过程中,可以按照原理和应用两个方面进行讲述.

关于激光的产生原理,首先要让学生明确原子发光的两种情形.一种是自激辐射(自然光),一种是受激辐射(激光),了解这两种辐射的不同机制和发出的光在频率、初相和偏振方向上的不同特点.再进一步讲清亚稳态和粒子数反转的问题,让学生了解,只有把处于基态的原子大量激发到亚稳态,使处于高能级的原子数超过处于低能级的原子数,才能使受激辐射持续进行下去得到激光.

讲述激光的应用要从激光的特点出发.激光的主要特点是亮度高、方向性好、单色性好和相干性好.这些特点,应该联系前面学习的受激辐射来理解,也是激光能得到广泛应用的原因.对于激光的应用,除了课本上讲述的以外,教师还可以向学生介绍一些课外读物让学生自己去阅读,以开阔学生的眼界,了解现代科学技术的发展,在有条件的情况下,也可以组织一个科学班会,让学生向全班介绍他们阅读过的有关课外读物的主要内容,介绍激光的特点和广泛应用.

由于激光器的种类很多,具体的构造都比较复杂,教学时不宜多介绍,以便把学生的注意力集中到理解激光的基
本原理和主要应用上.

\section{习题解答}

\subsection{练习一}
\begin{enumerate}
    \item $\alpha$粒子被原子散射的原因是什么?

\begin{solution}
在原子的中心有一个很小的核,原子的全部正电荷和几乎全部的质量都集中在原子核里,电子在核外定向运动.$\alpha$粒子穿过原子时,影响$\alpha$粒子运动的主要是原子核,是原子核与$\alpha$粒子间的库仑斥力影响$\alpha$粒子的运动.如果$\alpha$粒子离核较远,与原子核之间的库仑斥力很小,它运动方向的改变也很小,当$\alpha$粒子与原子核十分接近时,会受到很大的库仑斥力,要发生大角度的偏转.当$\alpha$粒子与原子核发生正碰时,就会被原子核反弹回来,由于原子核很小,能靠近原子核的$\alpha$粒子很少,所以只有极少数$\alpha$粒子能发生大角度偏转.    
\end{solution}
    \item 卢瑟福的原子模型与汤姆生的原子模型,主要区别
    是什么?

    \begin{solution}
汤姆生原子模型认为原子是一个球体,正电荷在整个球内均匀分布,电子象枣糕里的枣子那样镶嵌在球内.而户瑟福的原子模型认为在原子中心有一个很小的核,核集中了原子的全部正电荷和几乎全部质量,电子在核外绕核旋转.汤姆生原子模型是“均匀”结构,而卢瑟福原子模型是“核式”结构,这是他们的主要区别.        
    \end{solution}
    \item $\alpha$粒子的质量大约是电子质量的7300倍.如果$\alpha$
    粒子以速度$v$跟电子发生弹性正碰(假定电子原来是静止
    的),求碰撞后$\alpha$粒子的速度变化了多少,并由此说明,为什
    么原子中的电子不能使$\alpha$粒子发生明显的偏转.

    \begin{solution}
设$\alpha$粒子质量为$M$, 电子质量为$m$, 电子原来静止,$\alpha$粒子以速度$v_1$向电子运动,设发生弹性正碰后$\alpha$粒子与电子的速度分别为$v_1'$和$v_2'$.

由于是弹性正碰,动量和动能都守恒.并且两球运动在同一直线上,我们可以用代数式来计算.

根据动量守恒定律
\begin{equation}
  Mv_1=Mv_1'+mv_2'  
\end{equation}
根据动能守恒
\begin{equation}
    \frac{1}{2}Mv_1^2=\frac{1}{2}M{v_1'}^2+\frac{1}{2}m{v_2'}2 
\end{equation}
由(8.1)和(8.2)消去$v'_2$可得
\begin{equation}
    v'_1=\frac{M-m}{M+m}v_1
\end{equation}
把$M=7300m$代入(8.3), 可得
\[\begin{split}
    v'_1&=\frac{7300m-m}{7300m+m}v_1=\frac{7299}{7301}v_1\\
    \Delta v_1&= v'_1-v_1=-\frac{2}{7301}v_1=-0.0003v_1
\end{split}\]
由此可见,$\alpha$粒子的速度变化,只有初速度的万分之三,这就说明原子中的电子不能使$\alpha$粒子发生明显偏转.  
    \end{solution}
    \item 已知氢原子的半径是$0.53\x10^{-10}$米,电子不致被吸
    引到核上,按照卢瑟福的原子模型,电子绕核做匀速圆周运动
    的速度和频率各是多大?

    \begin{solution}
氢原子核与电子间的库仑引力就是电子绕核做匀速
圆周运动的向心力,所以
\[k\frac{e^2}{r^2}=m\frac{v^2}{r}\]
\[v=\sqrt{\frac{ke^2}{mr}}=\sqrt{\frac{9.0\x10^9\x (1.6\x10^{-19})^2}{9.1\x10^{-31}\x0. 53\x10^{-10}}}=2.19\x 10^{6}\ms\]
运动频率
\[f=\frac{1}{T}=\frac{v}{2\pi r}=\frac{2.19\x 10^{6}}{2\x 3.14\x 0.53\x 10^{-10}}=6.6\x 10^{15}{\rm Hz}\]
    \end{solution}
    \item 为什么在计算电子与核之间的引力作用时,可以不
    考虑万有引力?

    \begin{solution}
已知电子质量$m=9.1\x10^{-31}$kg,氢核质量$M=1.67\x10^{-27}$kg,万有引力恒量$G=6.67\x10^{-11}{\rm N\cdot m^2/kg^2}$.

电子与核之间的库仑引力
\[F=k\frac{e^2}{r^2}=9\x 10^9\x \frac{(1.6\x 10^{-19})^2}{(0.53\x 10^{-10})^2}=8.2\x 10^{-8}{\rm N}\]
电子与核之间万有引力
\[F_{\text{引}}=G\frac{Mm}{r^2}=6.67\x10^{-11}\x \frac{1.67\x 10^{-27}\x 9.1\x 10^{-31}}{(0.53\x 10^{-10})^2}=3.6\x 10^{-47}{\rm N}\]
两者比值
\[\frac{F}{F_{\text{引}}}=\frac{8.2\x 10^{-8}}{3.6\x 10^{-47}}=2.3\x 10^{39}\]
由此可知,万有引力远小于库仑引力,所以计算电子与核之间引力作用时,可以忽略万有引力.
    \end{solution}
\end{enumerate}



\subsection{练习二}

\begin{enumerate}
    \item 利用公式$r_n=n^2r_1$和$E_n=E_1/n^2$,
计算氢原子的第2、3、4轨道的半径和电子在这些轨道上的能量.

\begin{solution}
已知$r_1=0.53\x10^{-10}$米,$E_1=-13.6$电子伏,由公式$r_n=n^2r_1$和$E_n=\dfrac{E_1}{n^2}$可以计算出:
\begin{enumerate}
    \item 当$n=2$时,$r_2=2.12\x10^{-10}$米,$E_2=-3.40$电子伏;
    \item 当$n=3$时,$r_3=4.77\x10^{-10}$米,$E_3=-1.51$电子伏;
    \item 当$n=4$时,$r_4=8.48\x10^{-10}$米,$E_4=-0.85$电子伏.
\end{enumerate}

\end{solution}
\item 根据上题算出的结果,说明要把基态的氢原子激发
到$n=2$的能级上去,需要供给电子多大的能量.如果用电磁
波来供给这个能量,需要用波长多长的电磁波?这个波长属于
哪个波段?

\begin{solution}
    要把氢原子从基态激发到$n=2$的能级上去,吸收的能量
    $$\Delta E=E_2-E_1=-3.4-(-13. 6)=10. 2{\rm eV}$$

    根据光子能量公式$\Delta E=h\nu$和关系式$c=\lambda \nu$, 可得电磁波的波长
\[\lambda=\frac{hc}{\Delta E}=\frac{6.63\x 10^{-34}\x 3.00\x 10^8}{10.2\x 1.60\x 10^{-19}}=1.22\x 10^{-7}{\rm m}\]
这个波长属于紫外区.
\end{solution}
\end{enumerate}



\subsection{练习三}
\begin{enumerate}
    \item 根据玻尔理论,$H_{\alpha}$、$H_{\beta}$谱线光子的能量应该是多少
电子伏?根据实验测得的$H_{\alpha}$、$H_{\beta}$的波长算得的光子的能量是
多少电子伏?二者是否一致?

\begin{solution}
根据玻尔理论,$H_{\alpha}$和$H_{\beta}$谱线的光子,是由氢原子从量子数$n=3$和$n=4$的能级跃迁到$n=2$的能级时辐射的,它们的能量分别为
\[\begin{split}
    H_{\alpha}:&\quad E_3-E_2=-1.51-(-3. 40)=1. 89{\rm eV}\\
    H_{\beta}:&\quad E_4-E_2=-0.85-(-3.40)=2.55{\rm eV}
=2. 55电子伏.
\end{split}\]
而从实验测得的和$H_{\beta}$的波长分别是
\[H_{\alpha}:\;  0.6562\mu{\rm m},\qquad H_{\beta}:\; 0.4861\mu{\rm m}\]
根据$\Delta E=h\nu=hc/\lambda$
可计算这两条谱线对应的光子能量
\[\begin{split}
    E_{H_{\alpha}}&=\frac{hc}{\lambda}=\frac{6.63\x 10^{-34}\x 3.00\x 10^8}{0.6562\x 10^{-6}\x 1.60\x 10^{-19}}=1.89{\rm eV}\\
    E_{H_{\beta}}&=\frac{hc}{\lambda}=\frac{6.63\x 10^{-34}\x 3.00\x 10^8}{0.4861\x 10^{-6}\x 1.60\x 10^{-19}}=2.56{\rm eV}
\end{split}\]
从计算的结果看,两者是一致的.
\end{solution}
\item 计算氢原子从$n=4$,$n=5$能级分别跃迁到$n=3$能
级时辐射出的光子的波长.这两条谱线在哪个波段?它们属
于哪个线系?

\begin{solution}
\[\begin{split}
    \lambda_1=\frac{hc}{E_4-E_3}&=\frac{6.63\x 10^{-34}\x 3.00\x 10^8}{[-0.85-(-1.51)]\x 1.60\x 10^{-19}}\\
    &=1.9\x 10^{-6}{\rm m}=1.9\mu{\rm m}
\end{split}\]
\[\begin{split}
    \lambda_2=\frac{hc}{E_5-E_3}&=\frac{6.63\x 10^{-34}\x 3.00\x 10^8}{[-0.54-(-1.51)]\x 1.60\x 10^{-19}}\\
    &=1.2\x 10^{-6}{\rm m}=1.2\mu{\rm m}
\end{split}\]
这两条谱线在红外波段,属于帕邢线系.
\end{solution}
\item 怎样用玻尔原子理论解释原子吸收光谱的规律?

\begin{solution}
 玻尔理论认为原子从较低能级向较高能级跃迁时,要吸收能量;原子从较高能级向较低能级跃迁时,要辐射能量.无论是原子吸收的能量还是辐射的能量,都等于发生跃迁的两个原子能级间的能量差.对同一种原子说来,对应的能级的能量差是一定的,故其发射光谱与吸收光谱中的谱线是一一对应的,能发射什么样的谱线的原子,也必能吸收这样的谱
线.这就是白光被某种元素的原子吸收时产生的吸收光谱与这种原子发出的明线光谱相一致的原因.
\end{solution}
\end{enumerate}


\section{参考资料}

\subsection{$\alpha$粒子散射理论}

$\alpha$粒子是放射性物体放射出来的高速粒子,它的质量是
电子质量的7300倍,带有两个单位的正电荷.$\alpha$粒子穿过金属薄片时,绝大多数平均只有2—3度的偏转,但卢瑟福的学生盖革和马斯登在1909年从实验中观察到,约有1/8000的$\alpha$粒子的偏转角大于$90^{\circ}$, 其中有接近$180^{\circ}$的.经理论分析知道,这种现象不可能在汤姆生模型那样的原子中发生.当$\alpha$粒子在汤姆生模型的原子的外边时,由于原子的正负电相等且对称分布,原子对$\alpha$粒子没有库仑力的作用(考虑到原子的极化,有作用力也是很微小的). 当$\alpha$粒子接近或进入原子的实体球时,电子因质量很小,对$\alpha$粒子动量变化的影响极小,而它本身将会在$\alpha$粒子的力的作用下离去,所以可以只考虑原子的正电部分对$\alpha$粒子的作用.

设原子半径为$R$, 正电荷$Ze$均匀分布在这球体中.$\alpha$粒子带正电荷$2e$, 当它在原子球的外边,即$r\ge R$时,它所受原子正电荷的库仑力是$\dfrac{2kZe^2}{r^2}$, 到达球面时是$\dfrac{2kZe^2}{R^2}$. 当$\alpha$粒子进入球内,到达离球心$r$处时,它所受的力比在球面时所受的力还要小,这时对$\alpha$粒子起作用的电荷是以$r$为半径的球体中所含的电荷,这电荷
\[Q=\frac{Ze}{\frac{4}{3}\pi R^3}\x \frac{4}{3}\pi r^3=\frac{Zer^3}{R^3}\]
因此,$\alpha$粒子这时所受的力是
\[\frac{2keQ}{r^2}=\frac{2kZe^2r}{R^3}\]
所以进入球体后,离球心越近,所受的力越小.$\alpha$粒子在汤姆生模型中受原子正电部分的力最大是它到达原子球的表面时.$\alpha$粒子的初速度是可以知道的.按上面分析的$\alpha$粒子的受力情况来计算,结论是不能产生大角度散射的,因此汤姆生模型被否定了.

卢瑟福的原子的核式结构模型能够解释$\alpha$粒子散射实验现象.

\begin{figure}[htp]
    \centering
   \includegraphics[scale=.6]{fig/8-1.png}
    \caption{}
\end{figure}

设有一个$\alpha$粒子射到一个原子附近.在原子核的质量比$\alpha$粒子的质量大得多的情况下,可以认为原子核不会被推动,$\alpha$粒子在核的库仑力的作用下而改变了运动的方向,如图8.1所示.图中$v$是$\alpha$粒子原来的速度,$\ell$是原子核离$\alpha$粒子原
运动路径的延长线的垂直距离,叫做瞄准距离,由力学原理可以证明$\alpha$粒子的路径是双曲线,偏转角$\theta$和瞄准距离$\ell$有如下关系:
\begin{equation}
    \cot\frac{\theta}{2}=\frac{Mv^2}{2kZe^2}\ell
\end{equation}
式中$M$是$\alpha$粒子的质量.从上式可以看出,$\theta$与$\ell$有对应关系:$\ell$大,$\theta$就小;$\ell$小,$\theta$就大;对一定的$\ell$, 有一定的$\theta$. 对于不同的瞄准距离,$\alpha$粒子的轨道形状如图8.2所示.
\begin{figure}[htp]
    \centering
    \includegraphics[scale=.6]{fig/8-2.png}
    \caption{}
\end{figure}

当瞄准距离由$\ell$缩小到$\ell-\dd\ell$时,散射角将由$\theta$增大到$\theta+\dd\theta$. 我们来计算散射到两个角锥面($\theta$和$\theta+\dd\theta$)中间的$\alpha$粒子的数目.

\begin{figure}[htp]
    \centering
    \includegraphics[scale=.6]{fig/8-3.png}
    \caption{}
\end{figure}

假设有一束均匀分布的,平行而且等速的$\alpha$粒子沿$AA'$
方向向金属薄片射来(金属薄片和$AA'$垂直),单位时间内,在与$AA'$垂直的单位面积内有$n_0$个$\alpha$粒子,以原子核$O$为中心,以瞄准距离$\ell$和$\ell-\dd\ell$为半径作一个圆环,这个圆环面与$AA'$垂直,如图8.3所示.这个圆环的面积为$\dd S=2\pi\ell|\dd\ell|$.
对准这个环射来的$\alpha$粒子数为
\[\dd n_0=n_0\dd S=2\pi n_0\ell|\dd\ell|\]
为了求出$\ell\dd\ell$的值,把(8.4)式平方,得
\[\ell^2=\left(\frac{2kZe^2}{Mv^2}\right)^2\cot^2\frac{\theta}{2}\]
微分后得:
\[\ell \dd \ell=-\frac{1}{2}\left(\frac{2kZe^2}{Mv^2}\right)^2\frac{\cot\frac{\theta}{2}}{\sin^2\frac{\theta}{2}}\dd\theta \]
带回$\dd n_0$的表达式,则得:
\[\dd n_0=\pi n_0\left(\frac{2kZe^2}{Mv^2}\right)^2\frac{\cot\frac{\theta}{2}}{\sin^2\frac{\theta}{2}}\dd\theta \]
或
\[\dd n_0=n_0\left(\frac{kZe^2}{Mv^2}\right)^2\frac{2\pi \sin\theta}{\sin^4\frac{\theta}{2}}\dd\theta\]
这是被一个核$O$所散射的在两个锥面$\theta$到$\theta+\dd \theta$中间的$\alpha$粒子数.

假设金属薄片每单位面积内有$N$个原子核,由任何一个核所散射的$\alpha$粒子,将都为上式所表示,由于金属薄片到荧光屏的距离较大,$N$个散射角锥的顶点可视为集中在一点.在单位时间内,由这$N$个核所散射的包含在两个锥面($\theta$和$\theta+\dd\theta$)内的$\alpha$粒子数为
\begin{equation}
    \dd n=N\dd n_0=n_0N \left(\frac{kZe^2}{Mv^2}\right)^2\frac{2\pi \sin\theta}{\sin^4\frac{\theta}{2}}\dd\theta
\end{equation}
现在我们来求在单位时间内在荧光屏单位面积上观察到的$\alpha$粒子数.

\begin{figure}[htp]
    \centering
    \includegraphics[scale=.6]{fig/8-4.png}
    \caption{}
\end{figure}

以锥体的顶点$O$为球心作一半径为$r$的球面,如图8.4
所示,由两个锥面$\theta$及$\theta+\dd\theta$在这个球面上所划出的带区的面积为
\[2\pi r\sin\theta r\dd\theta =2\pi r^2\sin\theta\dd\theta\]
穿过这个带区面积的$\alpha$粒子总数为$\dd n$. 那么,在单位时间内穿过这个带区的单位面积上的$\alpha$粒子数(即在$\theta$方向上,每单位时间在荧光屏上每单位面积内所观察到的$\alpha$粒子数)为
\begin{equation}
    \dd n'=\frac{\dd n}{2\pi r^2\sin\theta\dd\theta}=\frac{n_0 N}{r^2}\left(\frac{kZe^2}{Mv^2}\right)^2\frac{1}{\sin^4\frac{\theta}{2}}
\end{equation}

从(8.6)式可知,在一定的实验条件下,乘积$\dd n'\cdot \sin^4\dfrac{\theta}{2}$是一个恒量.这个关系是可以用实验来验证的.下表是观察$\alpha$
粒子在金箔中散射的实验结果.

\begin{center}
\begin{tabular}{ccc||ccc}
    \hline
    散射角$\theta^{\circ}$  & 闪烁数  &  $\dd n'\cdot \sin^4\dfrac{\theta}{2}$ & 散射角$\theta^{\circ}$  & 闪烁数  &  $\dd n'\cdot \sin^4\dfrac{\theta}{2}$\\
    \hline
    150&    33.1&    28.8&    60&    477&    29.8\\
    120&51.9&29.0&45&1435&30.8\\
    105&69.5&27.5&30&7800&35.0\\
75&211&29.1&15&132000&38.4\\
\hline
\end{tabular}
\end{center}


由实验结果可以看出,当散射角由$15^{\circ}$变更到$150^{\circ}$时所得的$\dd n'\cdot \sin^4\dfrac{\theta}{2}$的数值差不多保持不变,这个结论是假设$\alpha$粒子在金属薄片中只发生一次散射而得出的,由于原子核很小,$\alpha$粒子在金属薄片中十分接近原子核的机会很少,对于大角度散射来说,这个假设是可以满足的.但对于小角度散射
来说,$\alpha$粒子是从离核比较远的地方通过的,因此发生多次散射的机会就比较大了,这就是实验数据中,$45^{\circ}$以上的散射结果与理论值符合的比较好,而$45^{\circ}$以下的散射结果与理论值偏离比较大的原因.


\subsection{卢瑟福}
卢瑟福(1871—1937),英国物理学家.卢瑟福出生于新西兰手工业工人家庭,他的父亲主要从事亚麻加工工作,当时他很可能继承这一职业.但是,由于他在学校里各门功课成绩优异,他的父母决定让他继续学习下去.

1894年卢瑟福在喀捷贝尔大学毕业.由于电磁方面的
论文获得了奖金,这个奖金使他得到了在英国最好的大学实习的机会.卢瑟福在英国剑桥大学卡文迪许实验室实习了三年,当时领导实验室的是卓越的物理学家汤姆生.

1896年,巴黎的亨利·贝克勒耳发现了天然放射性,卢瑟福对贝克勒耳的发现非常感兴趣,并立即开始研究放射线.他比较了“铀射线”和伦琴射线,查明它们并不是相同的.他于1899年肯定了放射性辐射中的两种成分,分别命名为$\alpha$射线和$\beta$射线(不久之后,维拉德又发现了第三种成分$\gamma$射线).

1902年他和英国化学家索第(1887—1956)共同提出了原
子自然衰变理论.卢瑟福明确地指出,放射现象是一种物质的原子以一定的速率自行衰变成另一种物质的原子的过程.这个理论打破了原子不可分的观念,在物理和化学理论上引起了一场剧烈的变革,卢瑟福于1908年获得诺贝尔化学奖.

1909年卢瑟福交给年青物理学家马斯登一项简单的任
务,要他数一数穿过各种物质薄片(金、铜、铝等)的$\alpha$粒子,根据当时大家所接受的汤姆生原子模型,从理论上不难作出这样的推断,$\alpha$粒子应该很容易地穿过原子,不发生散射现象,因此卢瑟福认为,薄片不会影响$\alpha$粒子的直线运动,马斯登在实验时注意到,虽然绝大多数的$\alpha$粒子穿过了薄片,但是仍然可以看到散射现象——有一些粒子好象是反弹回来了.

卢瑟福知道了这一情况以后,重复了很多次这个实验.在经过三个多星期的认真思考以后,他得出了下面的结论:原子是一个很复杂的系统,它有一个带正电的核心(原子核),在核周围的一定轨道上转动着带负电的电子.

在卢瑟福的原子模型产生后不久,他的学生亨利·莫塞莱证明了原子核内单位电荷的数目就是原子的序数,它决定元素在门捷列夫周期表上的位置.

1918年,卢瑟福接替退休的汤姆生的职位,担任著名的
卡文迪许实验室主任.逝世前,他一直在那里工作,在这实验室里,他继续做了许多重要的实验.

1919年他用$\alpha$粒子轰击氮原子,第一次实现了元素的人
工转变.

1920年,他预言了中子的存在.

各国许多天才的物理学家都曾是卢瑟福的学生,如詹姆斯·查德威克(英国),尼尔斯·玻尔(丹麦),彼得·里昂诺维奇·卡皮查(苏联)等等.

\subsection{玻尔}
玻尔(1885—1962),丹麦物理学家.
玻尔是物理学创始人之一,是卢瑟福的学生,1885年生于哥本哈根,1911年毕业于哥本哈根大学,他曾在英国剑桥大学汤姆生领导下的卡文迪许实验室工作,还在曼彻斯特卢瑟福实验室工作过.1920年他创建了哥本哈根大学理论物理研究所并担任所长,第二次世界大战期间,玻尔去了美国,战后又回到丹麦工作,主张和平利用原子能和控制原子武器.他还领导创建了欧洲核子研究中心(CERN).

他一生的主要研究工作是发展原子、分子和原子核结构的量子理论.他在普朗克量子假说和卢瑟福原子的核式结构学说的基础上,于1913年提出了氢原子结构和氢光谱的初步理论.为此获得1922年诺贝尔物理学奖.稍后,又提出了经典规律和量子规律之间的对应原理,这些工作对量子论和量子力学的建立起了重要作用.此外,玻尔在原子核反应理论和解释重核裂变现象等方面,也有重要的贡献.






\chapter{原子核}
\minitoc[n]

\section{教学要求}

本章讲述原子核的初步知识,主要包括原子核的组成、核
能及其应用两方面的内容.这些知识对于学生进一步认识微
观世界,了解研究微观世界的方法,都有重要意义.

这章内容分为三个单元,第一单元讲述原子核组成方面
的知识,包括第一节到第四节;第二单元讲述关于核能方面的
知识,包括第五节到第七节.最后一节,作为一个单元,讲述
基本粒子,是选学内容.

原子核的人工转变和原子核的组成是本章教学的重点.
在核能正逐渐广泛地被和平利用的当今社会中,了解核能知
识是十分必要的.由于原子核的结合能是讨论裂变、聚变中
释放原子能的依据,是了解核能的基础,因此,原子核的结合
能也是教学的重点内容.

天然放射现象的教学,应着重介绍三种射线的性质,确定
三种射线的本质,和放射性元素的衰变规律,说明原子核不仅
有复杂的结构,而且可以发生转变.应使学生了解衰变过程
都遵守质量数守恒和电荷数守恒的规律.学生应能根据这两
条规律得出位移定则,确定衰变后的产物.

通过教学,不仅要使学生了解原子核的组成,而且要使学
生知道人们是怎样确定原子核的组成的.通过介绍查得威克
确定中子质量的分析,要使学生理解守恒定律在物理研究中
所起的作用.

在介绍原子核的结合能时,除了要求学生会计算,还要让
学生理解核子平均结合能随质量数变化的曲线的意义.爱因
斯坦的质能方程是在说明怎样计算结合能时直接给出的.教
学中不要求论证,只要求学生会使用.

通过基本粒子的教学,应使学生了解基本粒子并不是组
成物质的最小单元,特别是强子,已发现它们仍有内部结构,
许多物理学家倾向于不再用“基本粒子”这个名称,而把它改
称为粒子.探索强子结构的夸克(我国叫层子)模型,是一种比
较成功的理论.


本章的教学要求是:
\begin{enumerate}
\item 了解天然放射现象,知道三种射线的性质以及放射性
元素衰变的规律;了解探测放射线的方法.
\item 了解原子能的人工转变,知道原子核的组成,会写核
反应方程,了解放射性同位素的应用.
\item 理解原子核的结合能的概念,会用质能方程进行计
算,了解释放核能的两种途径——裂变和聚变,以及它们的
应用.
\end{enumerate}

\section{教学建议}
\subsection{原子核的组成}
\subsubsection{天然放射现象}

通过这一节的教学应该使学生了解,天然放射现象
的发现,打开了人们认识原子内部世界的窗口,它不仅使人
类认识到原子核也是具有结构的,而且告诉人们,一种原子核
可以自发地转变为另一种原子核.这一发现,揭开了原子核物
理的新篇章,至于原子核的内部组成如何,在这里暂不涉及,在
讲第三节原子核的人工转变和原子核的组成等内容时,再作
统一处理.这一节可着重讲述三种放射线的本质、它们的特性
和放射性元素的衰变规律,还应该使学生注意,天然放射性并
不是少数元素才具有的.原子序数大于83的天然元素都具有
放射性,原子序数小于83的天然元素,也有一些具有放射性.

三种射线的本质和特性,根据教材中的叙述学生不
难掌握.教学中可通过电场中偏转的实验,使学生加深对射
线带电性质的理解.对于三种射线的特性(贯穿本领、电离作
用等),可以简要地列表对比加以说明.三种粒子的符号、质
量数和电荷数、也应该让学生正确掌握:教学中还须告诉学
生,$\alpha$、$\beta$、$\gamma$粒子都是从原子核里放射出来的,但不能认为这
三种粒子就是原子核的组成部分.

放射性元素的衰变,放射性元素的原子核发生衰变
时,要放出一个$\alpha$粒子或$\beta$粒子,使原子核转变为新核.要让
学生了解,衰变过程遵守质量数守恒和电荷数守恒的规律.
衰变后的产物(新核)可由这两个守恒定律来确定.根据放射
性元素的衰变过程可写出衰变方程.教学中,要训练学生会
正确地书写核符号和衰变时的核反应方程.

教材从衰变方程中根据质量数守恒和电荷数守恒定律得
出了位移定则,但在实际教学中,不要强调机械地记忆这个定
则 学生只要理解了衰变的物理过程,独立推导出位移定则
是并不困难的.

半衰期是研究衰变过程的一个重要概念.应使学生
明白,放射性元素的衰变规律是统计规律,只适用于含有大量
原子的样品,半衰期是表示放射性元素的大量原子核半数发
生衰变所需要的时间,表示大量原子核衰变的快慢,当样品
中的原子数目减少到统计规律不再起作用的时候,我们就不
能肯定在某一时间里这些原子核会有多少发生衰变了.因
此也就无法肯定某一放射性样品的全部原子完全衰变所需的
时间,有的同学认为,可以由半衰期推算出放射性样品完
全衰变的寿命期,当然是不正确的,在学生明白了半衰期
的概念后,再引导他们求放射性元素经过$n$个半衰期后的剩
余质量,可启发学生自己得出
\[m_n=m_0\left(\frac{1}{2}\right)^n\]
的结论($m_0$为放
射性元素的初始质量,$m_n$为$n$个半衰期后的质量).


\subsubsection{探测放射线的方法}

这一节里讲的云室、计数器和乳
胶照相,都是很早就使用了的最起码的核物理的实验手段,跟
原子核的人工转变等教材都有关系,应该让学生了解这部分
内容.教材主要讲述了三种探测方法的原理,目的是使学生
知道研究原子核变化中的微观现象,可以根据各种粒子产生
的次级效应来进行观察和判断,进而体会到研究微观现象的
规律并不神秘.因此,凡是有条件的学校都应该做好演示,让
学生观察,以加深印象,云室设备如果没有,也可以自制(参看
实验指导部分).

云室实验由于粒子径迹呈现时间较短,云室又较小,许多
学生同时观察会有困难.教学时可先将各种粒子的径迹图样
向同学讲清楚,然后让学生分组进行观察,也可以利用投影
幻灯进行观察.

本节教材较多地涉及以前学过的知识,如气体的绝热膨
胀、过饱和汽、气体的电离和气体放电等内容.教学中要注意
新旧知识之间的联系,及时提示学生回忆过去学过的知识,加
深对三种探测方法的原理的理解.

\subsubsection{原子核的人工转变和原子核的组成}
学生已经知道了原子核是由质子和中子组成的,但并不了解是根据什么得
到这种结论的.本节的教学,应该使学生认识人们是怎样通
变革原子核的实验弄清了原子核的组成的.

关于原子核的人工转变,首先要注意讲清卢瑟福用
$\alpha$粒子轰击氮核的实验装置以及怎样确定实验中产生的新粒
子就是质子.还可以结合这个实验讲一讲卢瑟福的科学态
度,早在1915年,卢瑟福的学生马斯登就观察到了用$\alpha$粒子
轰击空气时会产生不寻常的长射程粒子.一种可能的解释是
这种粒子是氢核,因为这是用$\alpha$粒子轰击氢时常常出现的现
象.卢瑟福没有轻易作出结论,而是耐心地进行实验研究,以
便弄清那些粒子到底是氮核、氦核还是氢核,实验要在荧光
屏前观察和统计微弱的闪烁,条件是相当艰苦的,经过了三年
多的时间,在1919年夏,他才总结了$\alpha$粒子与轻原子的碰撞
现象,对氮原子核的人工转变作出了无可置疑的结论.

关于中子的发现,在教学中可以讲一讲守恒定律对
物理学的意义,在中子发现之前,摆在物理学家们面前的问题
是:要么$\alpha$粒子轰击铍发出的是$\gamma$光子,它在跟质子的碰撞中
能量和动量不再守恒;要么$\alpha$粒子轰击铍发出的射线不是$\gamma$
光子,而是一种新粒子.查德威克运用了能量和动量守恒定
律,科学地分析了实验结果,终于发现了中子.还可以讨论一
下,在约里奥·居里夫妇的实验中中子已经出现了,但他们不
能识别它,一项划时代的发现,就这样从他们手中溜走了,
我们应该由此得到什么教训.

在讲过上述两部分内容的基础上,再来讨论原子核
的组成,学生就可以认识到确定原子核的组成并不是一件很
容易的事情.从1911年提出的原子的核式结构起到1932年
完成对原子核组成的认识,经历了21年.很多科学家为此付
出了辛勤劳动.

本节的教学要注意培养学生综合运用知识的能力.
还可以让学生进一步练习写核反应方程.要让学生知道核反
应方程也和以前讲过核衰变方程一样,遵守电荷数守恒和质
量数守恒的规律,不过应该注意,核反应方程反映的是客观
的物理过程,是不能反过来根据上述两条规律任意编造的.

\subsubsection{放射性同位素及其应用}

人工放射性同位素的发现
是核研究的又一项重要成果,这一发现使放射性同位素获得
了广泛的应用.

关于同位素,要使学生了解以下几点:
\begin{enumerate}
\item 质子数相同而中子数不同的原子互称同位素;
\item 同种元素的不同的同位素在元素周期表上具有相同
的位置(原子序数相同),它们的核电荷数相同,具有相同的化
学性质;
\item 同种元素的各种同位素的中子数不同,因此它们的
物理性质有差异;
\item 同一种元素的多种同位素中,有稳定的,也有不稳定
的.不稳定的同位素会自发地放出$\alpha$粒子或$\beta$粒子衰变为别
种元素.这种不稳定的同位素就叫放射性同位素.四十多种
元素具有天然放射性同位素,各种元素都有人工放射性同
位素.
\end{enumerate}

人工放射性的教学中还要注意讲清$\beta^+$衰变和什么是正
电子,说明它的性质,并让学生掌握它的符号.

关于放射性同位素的产生及其应用,应使学生了解以下
几点:
\begin{enumerate}
\item 用中子、质子、氘核、$\alpha$粒子或$\gamma$光子轰击原子核都
可制取放射性同位素,重核裂变的产物也有放射性同位素.现
在大量使用的放射性同位素有许多是利用核反应堆产生的.

\item 放射性同位素的应用主要分为两个方面:
\begin{itemize}
\item 利用它
的射线,    \item 用作示踪原子.
\end{itemize}
在这两个方面各有其在工业、农业、
医疗、科研等方面的许多应用实例.这些实例不要求学生过
多的记忆,只要理解是怎样利用射线的特性和示踪原子来解
决实际问题的就可以了.
\item 
我国在这方面取得的新成就,可以适当地补充与介
绍,以充实这部分内容,进行爱国主义教育.
\end{enumerate}


放射性同位素的应用尽管都是一般性的介绍,但它涉及
的知识原理,有些也是学生较难理解的,可适当运用挂图幻灯
等教学手段配合教学.

\subsection{核能}
\subsubsection{原子核的结合能}

原子核的结合能是讨论裂变和聚
变时释放原子能的根据,是本章的一个重要概念.

核力的概念是讲解结合能的基础.关于核力的本质
问题目前尚未完全弄清楚,仍在进一步深入研究中.已经确
定的核的主要特性有:核力比电磁力强100多倍;核力是
短程力,只有距离接近到$10^{-15}$米的数量级时才发生作用;每
个核子只跟它相邻的核子间有核力的作用,而不是跟原子中
所有的核子有核力的作用.

讲解结合能的概念时,为了便于学生接受,可以按教
材中的思路讲述,即先说明要把原子核拆散成核子,需要克服
核子间的核力做功,因而需要巨大的能量.在讲解克服核力做
功需要能量时,如果跟学生熟知的宏观的力学现象作对比,更
便于学生理解,例如把宇宙飞船发射到地球引力范围之外,相
当于把地球和飞船拆开,需要克服地球引力做功,因而需要提
供能量,然后再讲反过来,核子结合成原子核时,就要释放能
量.核子结合成原子核时放出的能量或原子核分解为核子时
吸收的能量,都叫原子核的结合能.

根据核反应时的质量亏损和爱因斯坦的质能方程可
以计算原子核的结合能$E=\Delta mc^2$. 在进行计算时,常常用u
表示原子质量单位.书上331页的$1u=931.5$兆电子伏,指
的是1个原子质量单位所对应的能量,即$1u\x c^2=931.5$兆电
子伏.教学时要讲清这个式子的意义,以免引起学生的误解.

质能方程是爱因斯坦从相对论得出的.在教学时,要引
导学生不要把质能方程理解为质量就是能量或质量可以变成
能量.质能方程指出的是质量和能量之间的联系.即物体的
质量和它具有的能量之间保持着严格的正比关系.物体质量
不但会因为它吸收或放出能量而增减,还会由于机械运动状
态的改变而发生变化,只不过由于$c^2$是一个非常大的恒量,
通常的能量变化只引起微不足道的质量变化,在中学阶段只
要求学生初步理解和会用质能方程,不要对该方程作进一步
的讨论.

讲平均结合能时,应明确以下几点:
\begin{enumerate}
\item 平均结合能反
映了核子结合成不同原子核时平均每个核子所释放的能量,
即原子核结合能对每一个核子的平均值;    \item 不同原子核的平
均结合能不相同,平均结合能的大小表征原子核的稳定程度,
平均结合能越大,原子核越稳定;    \item 轻核和重核的平均结合能
都比较小,中等质量的原子核平均结合能较大,质量数为50
—60的原子核平均结合能最大,表示这部分核最稳定.这些
内容都可以引导学生观察平均结合能随质量数变化的曲线得
出来.
\end{enumerate}
练习三第5题让学生根据平均结合能计算裂变中释
放的能量,可以巩固平均结合能的概念,同时为讲解裂变作了
准备.

\subsubsection{重核的裂变}

在讲解重核裂变时,可以先根据平
均结合能曲线说明为什么重核裂变时要放出能量,从图中可
以看出,重核的核子平均结合能小于中等质量的核子的平均
结合能,因此重核分裂成中等质量的核时,会有一部分结合能
放出来.这种由核结构发生变化而放出的能叫做核能,也叫
做原子能.

讲述裂变的过程时,还可补充下面的核裂变反应方
程
\[\atom{U}{235}{92}+\atom{n}{1}{0}\longrightarrow \atom{Xe}{139}{54}+\atom{Sr}{95}{38}+2\atom{n}{1}{0}\]

铀核裂变产物$\atom{Xe}{139}{54}$(氙)和$\atom{Sr}{95}{38}$(锶)都是有放射性的,例
如$\atom{Xe}{139}{54}$可以经过一系列的$\beta$衰变而变成$\atom{La}{139}{57}$:
\[\atom{Xe}{139}{54} \mathop{\longrightarrow}^{\beta}   \atom{Cs}{139}{55}  \mathop{\longrightarrow}^{\beta}    \atom{Ba}{139}{56}  \mathop{\longrightarrow}^{\beta}    \atom{La}{139}{57}\]
所以在重核裂变过程中可以得到多种放射性同位素,这与前
面讲述的放射性同位素有关知识可以呼应.

根据质能方程算出释放的能量跟用平均结合能计算
的结果是一致的,因为结合能本身也是用质能方程计算出来
的.教学时,可把计算出来释放的能量201兆电子伏,补充写
到课本334页的核反应方程中去,强调一下能量守恒.

讲铀核裂变的链式反应时,应着重说明铀235与铀
238跟中子的作用的区别,为讲核反应堆作准备.
最后可说明铀核裂变时还可能分裂成三部分或四部分,
这是我国科学家钱三强、何泽慧夫妇于1946年在法国首先
发现的.

核反应堆是用人工控制链式反应的装置,在教学时,
可通过挂图或幻灯片,讲清楚反应堆的构造、各部分的名称和
作用.可以把反应堆分成五部分来讲述:
\begin{enumerate}
\item 铀棒,是天然铀或浓缩铀.作为原子燃料,提供原子
能,在裂变时释放大量能量.
\item 减速剂,它的作用是使裂变时产生的快中子减速,变
成慢中子,慢中子容易被铀235俘获而引起裂变,维持链式
反应.
\item 控制棒.它的作用是调节中子数目以控制链式反应
速度,镉吸收中子的能力很强,所以用它作控制棒,当反应过
于激烈时,使镉棒插入深一些,让它多吸收一些中子,链式反
应的速度就会慢一些;当反应过于缓慢,达不到所需功率时,
使镉棒插入浅一些,让它少吸收一些中子,链式反应速度就可
增大,用电子仪器自动地调节镉棒的升降,就能使反应堆保
持一定功率.
\item 冷却剂.用水、液体金属钠或空气等,在反应堆内外
循环流动,不断地带走反应堆放出的热量,同时用来输出
热能.
\item 水泥防护层.作用是防止铀核裂变物放出的各种射
线对人体的危害,用来屏蔽射线,不让它们透射出来.
\end{enumerate}

还应说明原子能反应堆不仅可以提供强大的原子能,而
且它产生的大量中子,还可以用来进行各种原子核物理实验,
制造各种放射性同位素,利用反应堆还可以生产新的核燃料.

最后,可以介绍一下我国第一座大高通量原子反应堆
的简单情况,进行爱国主义教育.我国自行设计建造的第一
座大型高通量原子反应堆,于1978年12月安装完毕,于
1980年12月达到高功率运行,它的主体及其配套工程的设
备,全部都是我国自行设计制造的.

这座高通量反应堆,热功率设计额为12.5万千瓦,最大
热中子通量是$6.2\x10^{14}\text{个}/{\rm cm^2\cdot s}$,据不完全统计,目前
世界各国共有400座研究堆,其中中子通量在$3\x10^{14}\text{个}/{\rm cm^2\cdot s}$以上的约20座左右,这从一个侧面反映我国核科学技
术的发展水平.

这座高通量反应堆是一座试验研究反应堆.它具有一堆
多用的特点,可以同时产生多种放射性同位素和超钚元素.它
的建成有助于提高我国许多科学领域和工业部门的技术
水平.

\subsubsection{轻核的聚变}

与讲裂变相似,在教学时可先据
平均结合能曲线说明轻核的平均结合能很小,因此,当某些轻
核结合成质量较大的核时,能释放出更多的结合能,一个氘核
和一个氚核结合成一个氦核时,能释放17.6兆电子伏的能
量,也可以把这个能量,补充写到核反应方程中去.轻核结合
成质量较大的核叫做聚变.还可让学生计算一下上述聚变反
应中平均每个核子释放出来的能量为:
\[\frac{17.6}{5}=3.52{\rm MeV}\]
而铀核裂变时平均每个核子放出的能量约为1兆电子伏,说
明轻核聚变时每个核子放出的能量比重核裂变时所释放的能
量还要大几倍.关于轻核聚变的例子,可以再提供以下两个
核反应方程:
\[\begin{split}
    \atom{Li}{6}{3}+\atom{H}{2}{1}&\longrightarrow 2\atom{He}{4}{2}+22.4{\rm MeV}\\
    \atom{H}{3}{1}+\atom{H}{1}{1}&\longrightarrow \atom{He}{4}{2}+19.2{\rm MeV}\\  
\end{split}\]

关于聚变发生的条件,要着重说明聚变为什么要在
高温下才能发生.因为要使轻核接近到$10^{-15}$米,由于原子
核都是带正电的,这样就必须克服电荷之间很大的斥力作
用,这就要使核具有很大的动能,必须把它们加热到很高的
温度.因此聚变反应又叫做热核反应,这里还可以复习核力
为短程力这样一些旧知识.

教学时,应该使学生了解热核反应可分为爆炸式热
核反应和可控热核反应.现在人们已经掌握并利用爆炸式热
核反应,例如氢弹的爆炸,对可控热核反应,至今尚在研之
中,还有许多困难需要克服,这些困难主要是把几千万度以
上的高温聚变物质控制在一定足够长的时间,我国自行设计
和制造的可控热核反应试验装置“中国环流器一号”已于
1984年9月顺利起动,它标志着我国可控核聚变的研究有了
新的发展和提高,必将为人类探求新能源作出贡献.

\subsection{基本粒子}
本章第八节介绍基本粒子,这是选讲教材.编入这一节
教材,目的是使学生知道现代物理的前沿,了解人类对物质结
构的认识是不断深入的,是无穷尽的.

\subsubsection{基本粒子的概念和种类}

随着人类对基本粒子的不
断深入的认识,基本粒子的概念也是不断发展的,当人类知道
了质子和中子组成原子核,原子核和电子组成原子时,人们把
电子、质子和中子叫做基本粒子.后来又发现正电子、$\mu$介
子、$K$介子和$\pi$介子等.人们把电子、正电子称为轻子,质
子、中子称为重子,质量介于质子和电子之间的粒子叫做介
子.后来又发现了质量比质子大的粒子,名叫超子,超子也属
于重子.后来又发现了反粒子等.现在发现的基本粒子已达
几百种.按照基本粒子之间的相互作用,可以把它们分为三
类:强子、轻子和媒介子,所有强子都参与强相互作用,轻子
都不参与强相互作用,媒介子是传递粒子间相互作用的粒子,
例如光子就是其中的一种,是传递电磁相互作用的.原来按
照粒子质量所做的分类,已不能恰当地反映粒子的性质.例
如,重子和介子都属于强子,但$\mu$介子只参与弱相互作用,应
属于轻子,现已改称为$\mu$子.

应使学生明确绝大多数基本粒子都是不稳定的,在很
短时间内就发生衰变,并且能互相转化,课本上列举四个转
化的例子,前两个例子是中子转化为质子和质子转化为中子,
可以概括为实物粒子之间的转化;后两个例子是正、负电子转
化为光子和光子转化为正、负电子,这可以概括为实物粒子与
场粒子之间的转化.这些事实表明,实物粒子(如电子)与场
粒子(如光子)虽然有基本的区别,但是仍然有着深刻的联系.

应使学生明确基本粒子不是组成物质的最基本的、最
小的单元,它们也有复杂的结构.夸克(我国称为层子)模型
就是一个强子结构的理论模型,这个理论模型可以解释许多
实验现象,强子的性质已明显表现出有内部结构,在这以后,
“基本粒子”这个名称就名不副实了,于是近来又把它们改称
为“粒子”.研究粒子的种类、性质、运动规律以及它们的内部
结构的学科叫粒子物理学.由于许多科学工作者的努力,粒
子物理学已经取得了辉煌的成果,然而,就在得到这些新成果
的同时,客观世界也把新的疑问呈现在人们面前,等待人们去
研究、认识.

这一章的内容,大都只要求学生了解,以扩大学生的视
野,因此在教学过程中,可以较多地采用培养学生阅读能力为
主的“自学、讨论、总结”式的教学方法,即在堂上由学生自学
课文,然后学生自己归纳课文的基本内容,提出问题,通过讨
论,最后总结出应该掌握的几个方面的知识.教师也可以提
出一些启发性的问题,引导学生讨论,或补充讲解一些课文上
没有的内容,最后小结知识要点.

\section{实验指导}
\subsection{演示实验}
\subsubsection{用云室观察$\alpha$粒子径迹}

课本图9.2的云室,是活塞式云室,它是用杠杆机构
牵动云室底上下运动,使云室内的饱和汽因绝热膨胀而达到
过饱和.

实验前,用长约150毫米的玻璃移液管,从云室壁上的孔
中将酒精均匀地滴洒在云室内的呢子上,约10—20滴左右;
把放射源插在握子上后送入云室,拧紧螺丝不使漏气.然后
把200—300伏左右的直流电源接在云室的“$+$”“$-$”接线柱
上,不要接错.演示时可轻快迅速地向下按压杠杆,使云室骤
然膨胀,在膨胀的一刹那可看到很不清楚的$\alpha$粒子的径迹如
课本图9.3的左图所示,这时不要松手,径迹可保留一小段
时间.只要一松手,径迹立即消失,必须间隔30秒乃至1分
钟,才可进行下一次的膨胀与观察.为了让较多的学生观察,
可以在云室的上方放置倾斜$45^{\circ}$角的平面镜,让学生从镜里
观察;也可让学生分组观察(每组4—5人),配合以较强光源
从云室壁上的透明窗把$\alpha$粒子径迹照亮,观察效果更好.

演示实验中可能出现的故障:
\begin{enumerate}
    \item 密封不严密.如果在按下杠杆时,能听到轻微的嘶嘶
漏气声,需要检查一下上盖压圈上的螺钉是否松动,应再紧一
紧;或者放射源握子的螺旋没有拧紧,也应再紧一下.务使云
室密闭严密可靠.
\item 径迹模糊不清.假如云室在膨胀后,充满白雾,径迹
模糊不清,可以降低膨胀比,也就是把限制器的柱头向上拧,
然后再试验几次,直到径迹清晰为止.假如这样仍无效,应检
查一下电场电压是否确已加上,如果因电压未加而云室内离
子扫除不尽,实验效果就差.有时把电压提高一些也有帮助.
\item 演示径迹不直.由于微量漏气,使云室内的气体在数
次膨胀之后,压强发生变化,引起a粒子径迹畸变,为此,在经
过几次膨胀后,可以扭开放射源螺旋,略微放气,使云室内气
体压强恢复,再拧紧握子螺旋,即能继续演示实验.
\item 云室上盖玻璃板上结成雾珠.当滴入酒精过多或房
间温度较高时,往往会出现雾珠,使上盖不透明,影响观察.消
除雾珠的方法是把放射源握子拧下,按几次杠杆,使云室连续
进行吸气和排气动作,若干次后就可以使雾珠消失,玻璃盖恢
复透明.这时应将纱布遮住结合套的孔,防止灰尘进入云室.
同时要在通风的地方排气,以免云室内排出的射气散布在教
室内,污染教室空气.
\end{enumerate}

在实际的教学过程中,本实验还是比较容易成功的,上面
这些故障很少发生,即使出现了也不难解决.如果不是使用
杠杆式的云室,而是用老式的“橡皮球”式的云室,实验成功率
就差一些,参照以上的方法,摸索实验的规律,也可以把演示
实验做好.

云室装置也可以自制,有两种自制的方案:一种是用干冰
致冷,制作容易,但需要使用干冰;另一种还是利用绝热膨胀
致冷,演示比较困难,需要有一定技巧,教师可于课前多做几
次,以掌握演示中的一些技巧,保证课堂演示成功.现将两种
方案分述如下:

\paragraph{干冰式云室} 将一个带盖的透明塑料圆筒或玻璃圆筒
放在一块干冰上,用一条浸透
酒精的黑布,围绕在圆筒内壁
靠近上部的地方.在圆筒盖子
上的软木塞下面插一根针,把
放射源装在针眼里,这个云室
的结构如图9.1所示.

在云室中,从布条上蒸发出来的酒精,到达容器底部的冷
区域里遇冷要重新凝结,如果靠近云室底部的区域内没有能
使酒精蒸汽形成液滴的其他凝结核(例如灰尘)存在,则酒精
蒸汽附着的凝结核就只能是带电粒子或射线经过时产生的离
子.当顶部的蒸汽向低温处扩散时,在容器底部达到过饱和
状态,并以离子为凝结核形成雾珠.因此酒精液滴的径迹反
映的是粒子或射线经过的路径.这种径迹在暗底衬托下是可
见的,并可以照相.

演示时,应使云室内形成
高压电场.为便于观察,使光从
侧面沿水平方向照射云室内靠
近底部的区域.

\begin{figure}[htp]\centering
    \begin{minipage}[t]{0.48\textwidth}
    \centering
\includegraphics[scale=.6]{fig/9-1.png}
    \caption{}
    \end{minipage}
    \begin{minipage}[t]{0.48\textwidth}
    \centering
\includegraphics[scale=.6]{fig/9-2.png}
    \caption{}
    \end{minipage}
    \end{figure}

\paragraph{绝热式云室} 这种云
室与威尔逊云室原理相同.云
室的结构如图9.2所示.在一
个广口瓶上加一个中间有孔的
橡皮塞.橡皮塞中间的孔中插一根玻璃管,用橡皮管将玻璃
管连接到自行车的打气筒嘴上.瓶塞上穿过两根裸导线,瓶
内的一端焊接一条铜片电极.实验时,穿出橡皮塞的一端接
电源.放射源倒插在橡皮塞上.在瓶内要先放好浸透酒精的
脱脂棉.这种云室的结构简单,易于操作,径迹显示时间长.

演示时,在两电源间加上200—300伏的直流电压,用手
把打气筒的活塞慢慢地向下压,直到感觉到有一定压力时,突
然松手.压入瓶内的空气突然作绝热膨胀,使瓶内温度降低,
瓶中酒精蒸汽达到过饱和状态.由于放射源放出的射线粒子
使沿途的气体电离,这时过饱和的酒精蒸汽便以这些离子为
核迅速凝结成液滴,显示出放射线的径迹.演示本实验时要
注意:
\begin{enumerate}
    \item 仪器各部分应密封,不能漏气.在压气时漏气,放气
    时就不能获得绝热膨胀致冷的效果.
    \item 压气时,要缓慢地把打气筒的活塞压下,以防止压缩
    气体冲开瓶塞.瓶塞要塞紧.
    \item 加在电极上的直流高压,可以从电子管收音机取出,
    也可由自行装置的直流高压电源提供.要注意这时整流器处
    于空载情况,输出电压较高,所以应在整流器输出端并联一只
    2瓦、100千欧的电阻,以防止击穿整流器的滤波电容.
\end{enumerate}


\subsubsection{用盖革计数器进行演示}
盖革计数器是由盖革计数管、放大和显示装置组成,是利
用放射线的电离作用制成的粒子探测仪器(如图9.3).
\begin{figure}[htp]
    \centering
    \includegraphics[scale=.6]{fig/9-3.png}
    \caption{}
\end{figure}


当一个放射性粒子进入计数管时,就使计数管发生一次
短暂的放电,从而得到一个脉冲电流.把这个脉冲电流用电
子电路加以放大,送入显示装置就可以显示出来.显示的方
法可以用扬声器发声、氖泡发光和数字显示等.把计数管用
开有窗口的金属套筒保护起来,加上手柄,再用电缆跟放大显
示装置连接起来就成为盖革计数器,又叫探测器,例如目前
医学及工业用的$\beta$、$\gamma$射线探测器.

用探测器可以进行一些演示.只是有的探测器是用耳机
听的,声音不够大.演示时,可以通过合适的线路把它接到扩
音机上,使全班学生都能听到.

利用计数器可以探测放射线.做法如下:把几个火柴盒
放在讲台上,其中一个装有放射源.调节探测器,使扩音机发
出间隔约两秒钟左右的“叭”“叭”声.告诉学生,这响声是宇
宙射线进入计数管产生的.然后把探头分别接近各只火柴
盒.当探头接近装有放射源的火柴盒时,扩音机里“叭”“叭”
声响的次数马上大量增加,这表示了有放射性粒子进入了探
测器.

利用计数器还可以演示原子辐射的防护.将探头和放射
源放在一定的距离上,让学生注意扩音机中“叭”“叭”声每分
钟响的次数,然后分别在探头与放射源之间放上硬纸板、木
板、铜片、铁片和铅片,让学生注意扩音机响声间隔的变化.从
实验可知哪种材料对射线有较大的防护作用,即阻止射线的
本领较大.

\subsubsection{放射源的制作}

取一个烧坏了的汽灯纱罩,将它研成粉末,用胶水粘成直
径2—3毫米的小球,插在一根火柴杆上,干后即成为放射源.
放射源应放进试管或玻璃瓶里保存.

汽灯纱罩是用浸过硝酸钍$\rm Th(NO_3)_4$的苎麻做成的,灼
烧后的灰烬含有99\%的二氧化钍$\rm ThO_2$.二氧化钍具有放
射性,所以能作为放射源使用.

还可以用搜集到的旧夜光表指针(或文具商店出售的罗
盘指针)上的磷光粉做成放射源.


\section{习题解答}


\subsection{练习一}
\begin{enumerate}
    \item 钍230是$\alpha$放射性的,它放出一个$\alpha$粒子后变成了
什么?写出衰变方程.


\begin{solution}
    钍230经过$\alpha$衰变后变成镭226, 其衰变方程为
    \[\atom{Th}{230}{90}\longrightarrow \atom{Ra}{226}{88}+\atom{He}{4}{2}\]
\end{solution}
\item 钫223是$\beta$放射性的,它放出一个$\beta$粒子后变成了
什么?写出衰变方程.


\begin{solution}
    钫223经过$\beta$衰变后变成镭223, 其衰变方程为
    \[\atom{Fr}{223}{87}\longrightarrow \atom{Ra}{223}{88}+\atom{e}{0}{-1}\]
\end{solution}
\item 钍232经过六次$\alpha$衰变和四次$\beta$衰变后变成一种稳
定的元素.这种元素是什么?它的原子量是多少?它的原子序
数是多少?


\begin{solution}
    经过六次$\alpha$衰变,质量数减少$6\x4$, 电荷数减少
    $6\x2$; 经过四次$\beta$衰变,质量数不变,电核数增加$4\x1$. 故新
    核的质量数:$232-6\x4=208$, 
    电荷数:$90-6\x2+4\x1=82$. 
    所以新元素为$\atom{Pb}{208}{82}$(铅).
\end{solution}
\item 
$\atom{U}{238}{92}$变成$\atom{Pb}{206}{82}$,要经过几次$\alpha$衰变和几次$\beta$衰变?

\begin{solution}
    衰变过程中质量数减少为$238-206=32$, 因为$\beta$衰
    变不影响质量数,可见要经过八次$\alpha$衰变;而经过八次$\alpha$衰
    变,电荷数应减少$8\x2=16$, 现在电荷数只减少了$92-82
    =10$. 因为每经过一次$\beta$衰变,电荷数增加1, 可见经过了六
    次$\beta$衰变.
\end{solution}
\item 
$\atom{Bi}{210}{83}$的半衰期是5天.10克的铋210经过20天后
还剩下多少?


\begin{solution}
    20天为4个半衰期,故20天后剩下铋210的质量
\[m=m_0\left(\frac{1}{2}\right)^4=10\x \left(\frac{1}{2}\right)^4=0.625{\rm g}\]
\end{solution}
\item 放射性元素$\atom{Na}{24}{11}$经过6小时后只剩下1/8没有衰
变,它的半衰期是多少?


\begin{solution}
因为$\dfrac{1}{8}=\left(\dfrac{1}{2}\right)^3$,所以是经过了3个半衰期,半衰期为
$6/3=2$小时.
\end{solution}
\end{enumerate}




\subsection{练习二}

\begin{enumerate}
    \item 用$\alpha$粒子轰击氮核使它发生转变.从云室的照片中
    为什么可以确定细而长的径迹是质子产生的,粗而短的径迹
    是反冲氧核产生的.


    \begin{solution}
    质子与氧核相比,质子的质量小,氧核的质量大.根
据动量守恒定律,当质子从复核中射出时,就具有比氧核大的
速度.由于质子速度大,在气体中能够通过较长的路程.但质
子的电量少,电离本领小,所以在云室中的径迹细而长;而反
冲氧核的质量大,速度小,在气体中通过的路程短,它的带电
量大,电离本领也大,因此它的径迹粗而短.
    \end{solution}
    \item 用$\alpha$粒子轰击氩40,复核衰变时产生一个中子和一
    个反冲核,这反冲核是什么?写出核反应方程.


    \begin{solution}
        反冲核是钙43, 其核反应方程为:
    \[\atom{He}{4}{2}+\atom{Ar}{40}{18}\longrightarrow \atom{Ca}{43}{20}+\atom{n}{1}{0}\]
    \end{solution}
    \item 用$\alpha$粒子轰击硼10,产生一个中子和一个具有放射
    性的核,它是什么?这个核能放出正电子,它衰变后变成什
    么?写出核反应方程.


    \begin{solution}
        具有放射性的核为$\atom{N}{13}{7}$(氮),经过衰变后变成$\atom{C}{13}{6}$(碳),其核反应方程分别为:
\[\begin{split}
    \atom{He}{4}{2}+\atom{B}{10}{5}&\longrightarrow \atom{N}{13}{7}+\atom{n}{1}{0}\\
    \atom{N}{13}{7} &\longrightarrow \atom{C}{13}{6}+\atom{e}{0}{1}
\end{split}\]
    \end{solution}
    \item 用中子轰击氮14,产生碳14,碳14具有$\beta$放射性,
    它放出一个$\beta$粒子后衰变成什么?写出核反应方程.


    \begin{solution}
        最后衰变为氮14, 其核反应方程为:
\[\begin{split}
    \atom{N}{14}{7}+\atom{n}{1}{0}&\longrightarrow \atom{C}{14}{6}+\atom{H}{1}{1}\\
    \atom{C}{14}{6} &\longrightarrow \atom{N}{14}{7}+\atom{e}{0}{-1}
\end{split}\]    
    \end{solution}
    \item 用中子轰击铝27,产生钠24,写出核反应方程.钠
    24是具有放射性的,衰变后变成镁24,写出核反应方程.


    \begin{solution}
        它们的核反应方程分别是:
\[\begin{split}
    \atom{Al}{27}{13}+\atom{n}{1}{0}&\longrightarrow \atom{Na}{24}{11}+\atom{He}{4}{2}\\
    \atom{Na}{24}{11} &\longrightarrow \atom{Mg}{24}{12}+\atom{e}{0}{-1}
\end{split}\]  
    \end{solution}
\end{enumerate}





\subsection{练习三}

\begin{enumerate}
    \item 氘核的质量是2.013553u,根据质量亏损,计算氘核的结合能.

    \begin{solution}
氘核由一个质子和一个中子组成,质子、中子和氘核
的质量分别为:
\[m_p=1.007277u,\qquad m_n=1.008665u,\qquad m_D=2.013553u\]
质量亏损 
\[\Delta m=m_p-m_n-m_D=0.002389u\]
氘核的结合能
\[\Delta E=\Delta mc^2=0.002389\x 931.5=2.225{\rm MeV}\]
    \end{solution}
    \item 碳原子的质量是12.000000u,可以看做是由6个氢
原子(质量是1.007825u)和6个中子组成的.求碳原子核的
结合能.(在计算中可以用碳原子的质量代替碳原子核的质
量,用氢原子的质量代替质子的质量,因为电子的质量可以在
相减过程中消去.)

\begin{solution}
    6个氢原子的质量
\[M_H=6m_H=6\x 1.007825u=6.046950u\]
6个中子的质量
\[M_n=6m_n=6\x1.008665u=6.051990u\]
质量总和:
\[M_H+M_n=6.046950u+6.051990u=12.098940u\]
质量亏损
\[\Delta m=12.098940u-12.000000u=0.098940u\]
碳原子的结合能
\[\Delta E=\Delta mc^2=0.098940\x931.5=92.16{\rm MeV}\]
\end{solution}
\item  在$\atom{He}{4}{2}$,$\atom{Kr}{82}{36}$,$\atom{U}{238}{92}$等原子核中核子的平均结合能个最大?哪个最小?原子核的结合能哪个最大?哪个最小?(根据平均结合能曲线进行比较)


\begin{solution}
    根据平均结合能曲线所示,$\atom{Kr}{82}{36}$的平均结合能最
    大,$\atom{He}{4}{2}$的平均结合能最小.原子核的结合能应为平均结合能乘以核子数,故$\atom{U}{238}{92}$的原子核的结合能最大,而$\atom{He}{4}{2}$的原
    子核的结合能最小.
\end{solution}
\item 如果要把$\atom{O}{16}{8}$分成8个质子和8个中子,要给它多
少能量?要把它分成$\atom{He}{4}{2}$,要给它多少能量?已知$\atom{O}{16}{8}$的核
子平均结合能是7.98MeV,$\atom{He}{4}{2}$的核子平均结合能是
7.07MeV.

\begin{solution}
要把氧16分成8个质子和8个中子,需要的能量就
是氧16原子核的结合能,即氧原子核的核子的平均结合能乘
以核子数:
\[E=16\x7.98=128{\rm MeV}\]
要把氧16分成4个氦核,所需能量应为上述能量减去4
个氦原子核的结合能($E'$).因为
\[E'=4\x4\x7.07=113{\rm MeV}\]
所以
\[E-E'=128-113=15{\rm MeV}\]
计算把氧16分成4个氦核所需能量,还可用核子的平均
结合能之差乘以核子总数计算:
\[(7.98-7.07)\x16=14.6{\rm MeV}\]
\end{solution}
\item 在一次核反应中,铀核$\atom{U}{235}{92}$变成了氙核$\atom{Xe}{136}{54}$和锶
核$\atom{Sr}{90}{38}$(同时放出了若干中子).铀核的核子平均结合能约为
7.6MeV,氙核的核子平均结合能约为8.4MeV,锶核
的核子平均结合能约为8.7MeV.
\begin{enumerate}
    \item 把U235分解为核子,要吸收多少能量?
    \item 再使相应的核子分别结合成Xe136和Sr90,要放出多少能量?
    \item 在这个核反应中是吸收还是放出能量?这个能量大
约是多大?
\end{enumerate}

\begin{solution}
\begin{enumerate}
    \item 把U235分解为核子时吸收的能量等于U235原子
    核的结合能
\[E=7.6\x235=1.79\x10^3{\rm MeV}\]
\item 核子结合成Xe136和Sr90放出的能量,等于该原子
核的结合能
\[\begin{split}
    E_1&=8.4\x136=1.14\x10^3{\rm MeV}\\
E_2&=8.7\x90=0.78\x10^3{\rm MeV}
\end{split}\]
\item 在这个核反应中,是放出能量.这个能量为
\[E_1+E_2-E=(1.14+0.783-1.79)\x 10^3=1.3\x 10^2 {\rm MeV}\]
\end{enumerate}
\end{solution}
\end{enumerate}



\subsection{习题}
\begin{enumerate}
    \item 铀238的半衰期是$4.5\x 10^9$年,假设一块矿石中含
有1千克的铀238,经过45亿年(相当于地球的年龄)以后,
还剩有多少铀238?假设发生衰变的铀238都变成了铅206,
矿石中会有多少铅?这时铀铅的比例是多少?再经过45亿年,
矿石中的铀铅比例将变成多少? 根据这种铀铅比例能不能判
断出矿石的年龄?


\begin{solution}
经过45亿年即经过1个半衰期,剩下质量应为原
来的一半,即剩下铀238的质量为0.5千克.

因为有一半铀衰变为铅,所以铀、铅的原子核数相等,它
们的质量之比等于原子核质量数之比,即
\[238:206=0.5:x\]
所以铅的质量$x=0.433$千克.铀、铅比例为:$0.5:0.433
=1.15:1$.

再经过45亿年,即又经过一个半衰期,0.5千克铀衰变后
质量剩下一半,还有0.25千克铀,又有0.216千克铅产生.这时铀、铅比例为$0.25:(0.433+0.216)=0.385:1$.

由于铀、铅质量的比例与矿石的年龄有关,因此我们可以
根据铀、铅的比例粗略判断矿石的年龄.
\end{solution}
\item 镭核在$\alpha$衰变中放出能量为4.78MeV的$\alpha$粒
子和能量为0.19MeV的$\gamma$粒子.如果1克镭每秒钟有
$3.7\x10^{10}$个原子核发生$\alpha$衰变,算出它每秒钟释放多少能量.

\begin{solution}
1个镭原子核发生衰变时放出的能量
\[\Delta E=(4.78+0.19)=4.97{\rm MeV}\]
$3.7\x10^{10}$个原子核发生衰变时放出能量
\[E=3.7\x10^{10}\Delta E=3.7\x10^{10}\x4.97=1.8\x10^{11}{\rm MeV}\]
\end{solution}
\item 静止状态的放射性原子核镭($\atom{Ra}{226}{88}$)进行$\alpha$衰变.为
了测量$\alpha$粒子的动能$E$,让$\alpha$粒子垂直飞进$B=1$特的匀强磁
场,测得$\alpha$粒子的轨道半径$r=0.2$米.
\begin{enumerate}
    \item 写出Ra的$\alpha$衰变方程.
    \item 试计算$\alpha$粒子的动能$E$.
\end{enumerate}

\begin{solution}
\begin{enumerate}
    \item 镭的$\alpha$衰变方程
\[\atom{Ra}{226}{88}\longrightarrow \atom{Rn}{222}{86}+\atom{He}{4}{2}\]
    \item 在匀强磁场中,$\alpha$粒子在洛仑兹力作用下作圆周运
    动,洛仑兹力作为向心力,有
    \[qvB=m\frac{v^2}{R}\]
    所以 \[v=\frac{qRB}{m}\]
    $\alpha$粒子的动能
\[\begin{split}
    E=\frac{1}{2}mv^2=\frac{1}{2}m\left(\frac{qRB}{m}\right)^2    &=\frac{q^2R^2B^2}{2m}\\
&=\frac{(2\x 1.6\x 10^{-19})^2\x 0.2^2\x 1^2}{2\x 6.64\x 10^{-27}\x 1.6\x 10^{-19}}=2{\rm MeV}
\end{split}\]
\end{enumerate}


\end{solution}
\item 在某些恒星内,三个$\alpha$粒子结合成一个$\atom{C}{12}{6}$核.$\atom{C}{12}{6}$
的质量是12.0000u,$\atom{He}{4}{2}$的质量是4.0026u.这个反应中放出
多少能量?

\begin{solution}
    三个$\atom{He}{4}{2}$的质量
 \[   m=3\x4.0026u=12.0078u\]
    三个$\alpha$粒子结合成碳12核时质量亏损
\[\Delta m=12.0078u-12.0000u=0.0078u\]
    放出的能量
\[\Delta E=\Delta mc^2=0.0078\x931.5=7.3{\rm MeV}\]
\end{solution}
\item 已知$\atom{Ra}{226}{88}$,$\atom{Rn}{222}{86}$,$\atom{He}{4}{2}$的原子量分别是226.0254,
222.0175,4.0026.求在$\atom{Ra}{226}{88}$衰变
\[\atom{Ra}{226}{88}\longrightarrow \atom{Rn}{222}{86}+\atom{He}{4}{2}\]
中放出的能量是多少电子伏?如果这些能量都以Rn核和He
核的动能形式释放出来,放出的$\alpha$粒子的速度有多大?

提示:能量和动量都守恒.

\begin{solution}
    衰变过程中质量亏损
\[\Delta m=226.0254u-(222.0175u+4.0026u)=0.0053u\]
放出的能量
\[\Delta E=\Delta mc^2=0.0053\x931.5=4.9{\rm MeV}=7.9\x10^{-13}{\rm J}\]
设$\alpha$粒子质量为$m$, 速率为$v$, Rn222的质量为$M$, 速率
为$V$, 由于衰变过程中能量和动量守恒.则有
\begin{align}
    mv+MV&=0\\
    \frac{1}{2}mv^2+\frac{1}{2}MV^2&=\Delta E
\end{align}
由(9.1)可得
\[V=-\frac{m}{M}v\]
代入(9.2)式,可得
\[\begin{split}
    \frac{1}{2}mv^2\left(\frac{m}{M}+1\right)&=\Delta E\\
    \frac{1}{2}mv^2&=\frac{M}{M+m}\Delta E\\
v&=\sqrt{\frac{2M\Delta E}{(M+m)m}}\\
&=\sqrt{\frac{2\x222\x7.9\x10^{-13}}{226\x4\x1.66\x10^{-27}}}=1.5\x10^7\ms
\end{split}\]
\end{solution}
\item 在一原子反应堆中,用石墨(碳)作减速剂使快中子
减速.已知碳核的质量是中子的12倍,假设把中子与碳核的
每次碰撞都看作是弹性正碰,而且认为碰撞前碳核都是静止
的.
\begin{enumerate}
    \item 设碰撞前中子的动能是$E_0$,经过一次碰撞,中子损失的能量是多少?
    \item 至少经过多少次碰撞,中子的动能才能小于$10^{-6}E_0$?
\end{enumerate}

\begin{solution}
    设中子质量$m_1$, 初速度为$v_0$, 碳核质量为$m_2$, 初速度
    为0. 一次碰撞后,中子速度为$-v_1$, 碳核速度为$v_2$. 
    根据动量守恒
\begin{equation}
    m_1v_0=m_2v_2-m_1v_1
\end{equation}
根据动能守恒
\begin{equation}
    \frac{1}{2}m_1v_0^2=\frac{1}{2}m_2v_2^2+\frac{1}{2}m_1v_1^2
\end{equation}
由题意设$m_2=Am_1$, $A=12$. 
所以
\begin{align}
    v_0&=Av_2-v_1\\
    v^2_0&=Av^2_2+v_1^2
\end{align}
解(9.5)和(9.6)可得
\begin{equation}
v_1=\frac{A-1}{A+1}v_0
\end{equation}
\begin{enumerate}
\item 经过一次碰撞后中子损失的能量
\[\Delta E=\frac{1}{2}m_1v_0^2-\frac{1}{2}m_1v_1^2=\frac{1}{2}m_1v^2_0\left[1-\left(\frac{A-1}{A+1}\right)^2\right]=\frac{48}{169}E_0\]

\item 经过第$n$次碰撞后,中子速度为$v_n$, 则有
\[v_n=\left(\frac{A-1}{A+1}\right)^n v_0\]
由题意要求,有
\[ \frac{\frac{1}{2}m_1v_n^2}{\frac{1}{2}m_1v_0^2}<10^{-6} \]
即
\[\left(\frac{A-1}{A+1}\right)^{2n}<10^{-6}\quad \Rightarrow\quad \left(\frac{11}{13}\right)^{2n}<10^{-6}\]
由对数计算,可得$n\ge 42$. 

所以至少经过42次碰撞,中子动能才能小于$10^{-6}E_0$.
\end{enumerate}
\end{solution}
\end{enumerate}


\section{参考资料}
\subsection{正电子的发现}
威尔逊云室用于粒子物理研究,由于可以“抓拍”在短暂
瞬间内发生的现象,不仅为已知粒子留了影,也记录了原先不
为人知的新现象,帮助人们发现新的粒子.第一个被云室摄
到的新粒子就是正电子,这也是人们所发现的第一个反粒子,
它是从宇宙线中拍摄下来的.

宇宙线是来自外层空间的射线,奥地利科学家亥斯(V. 
F. Hess, 1883—1965)在研究地球上和大气中的放射性时发
现了这种射线.在分析宇宙线的成分和性质时,威尔逊云室
是一个很好的工具,密立根是位多年致力于宇宙线研究的专
家,他把云室放到强磁体的磁极中间,不间断地每15秒钟对
云室拍一次照.由于飞行的带电粒子在磁场中发生偏转,就
可根据照片中径迹的偏转方向确定其带电的状况.带有相反
电荷的粒子偏转方向也相反,在同一磁场强度下,速度的快慢
也与偏转有关.另外,在粒子径迹的单位长度上水滴的个数
与粒子的质量有一定关系.所以,磁场中的云室的照片可以
提供径迹的多种信息,1932年,密立根的同事年轻的安德孙
(C. D. Anderson, 生于1905年)在认真察看几千张云室照片
时,发现其中有一种径迹是陌生的.这种径迹的弯曲方向表
明它是一个带正电的粒子,但是与已知的带正电的粒子如质
子相比,它的雾珠密度完全不一样.经过仔细安排的进一步
实验,确认了这是一种新粒子,它的质量与电子质量相等,但
电荷相反,称为正电子.

在此以前,狄拉克从关于电子的相对论量子理论预见到,
一定有一种质量与电子完全相等,电荷大小也相等,只是符号
相反的粒子.安德孙发现的正电子恰好具有这种性质,因而,
这是狄拉克理论的一个有力支持,狄拉克的理论还预言电
子和正电子可以同时成对地由光子从真空中产生出来;相反
地,如果一个电子和正电子相撞,他们便同时湮灭,转化为光
子.这两种现象也为实验证实了.另外,在发现正电子后测
得它在物质中平均只能存在百万分之一秒的时间,但正电子
与电子一样是不会自行衰变的,它在物质中存在时间短正是
因为它十分容易与物质中的电子发生湮灭反应的缘故.这也
说明了为什么只有在云室发明后才能从宇宙线中发现正
电子.

由于狄拉克理论的成功,人们从此认识到了微观世界的
几个特征.首先,从狄拉克理论导致一个结论:每一种微观粒
子都有其反粒子.虽然正粒子和其反粒子相遇时即发生湮
灭,所幸我们生存的宇宙空间内,正粒子的数量与反粒子的数
量并不相等,例如地球上只有极少数来自外层空间的反质子
和正电子,因此不用担心,地球上的物质是不会被湮灭掉的.

\subsection{放射性的应用}
\subsubsection{工农业方面}
放射性测井.如天然$\gamma$射线测井,只需测量钻井内各
个深度的岩层天然放射的$\gamma$射线强度,加以比较,就能够了解
岩层的构造.放射性测井不必分析取出的岩心,方法简便可
靠.我国的石油勘探和煤碳勘探等已经应用了这种方法.

土壤水分的测定.可以应用吸收射线的原理来测量
土壤水分,通常采用钴60的土壤水分测量仪.这种仪器是
双管型的,一管装放射源,另一管装盖革计数管.两管按一定
的距离插入土中,计数管所测到的计数率和标准曲线相对照,
即可测出土壤的水分.

农作物的储藏.利用射线的照射可以抑制马铃薯的
发芽和杀死谷仓里的各种害虫.

\subsubsection{医疗方面}
诊断上的应用,放射性碘(碘131)已被应用来诊断甲
状腺的机能是否正常,当病人食用一定量的放射性碘后,从
小便中的含碘量可以看出正常人和甲状腺失常的人有显著差
异.还可以用来检查脑瘤的位置和测量血液循环的速度,诊
断血管有无阻塞或硬化.

治疗上的应用,放射线有生理效应,当放射性辐射作
用到活的细胞上,可以造成死亡,这种作用的机理跟快速的带
电粒子通过细胞对细胞内原子的电离和分子的分解有关.正
在迅速地成长和繁殖的细胞,对辐射的作用特别敏感,这种
作用可以用来治疗癌肿.

\subsubsection{科学研究方面}

半导体的扩散研究.把放射性元素锌(锌65)电镀到
半导体锗的表面,然后在电炉内加热,锌即从表面向里扩散.
扩散分子的浓度跟距离表层的深度有关系,有了放射性锌作
标记就容易测定这个关系,方法是把锗片一层层磨下,收集
每一层磨下的粉末,测量其放射性强度,即可求出扩散分子的
浓度.

制造发光剂.发光物质可以在放射性的作用下发光.
在发光物质(例如硫化锌)中加很少量的镭盐,就制成了一种
经常发光的漆.这种漆要是涂在仪表的表盘、指针或瞄准装
置上,就可以使它们在黑暗中也能被看见.

测定地球的年龄.从不含放射性元素的矿中采出的
通常的铅,原子量是207.20.因铀238的衰变而形成的铅,
原子量是206.含在某些矿中的铅,原子量很接近206.由此
可见,在铀矿形成的时候(从熔融体或液体中结晶出来),其中
并不含有铅,这些矿物中现含的铅都是由铀的衰变积聚起来
的.利用衰变定律,根据矿物中的含铅量与含铀量的比例,就
可以决定矿物的年龄.应用这种方法确定出来的矿物年龄,要
以几亿年来计量,最老的矿物年龄超过15亿年,通常是把硬
地形成后所经过的时间,当作地球的年龄,根据放射性测定
出地球的年龄约为45亿年.

\subsection{核电及安全}
核电站以铀为燃料,在反应堆中裂变产生热,再用冷却剂
带出,传给汽轮发电机,生产电能.这种以高温高压水为冷却
剂的核电站称为压水堆型核电站.

压水堆型核电站的安全性好.铀燃料以及产生的大量放
射性产物被包盖在耐高温、耐腐蚀的锆金属管中,形成燃料元
件.管子被称为防止放射释放的第一道屏障,燃料元件组成
堆芯装在直径约四米、壁厚约二百毫米的低合金钢压力壳中,
压力壳被称为第二道屏障,压力壳及有关设备又被完全封闭
在安全壳中,安全壳又被称为第三道屏障.

对压水堆型核电站进行事故发生可能性的分析表明,堆
芯熔化最大假想事故的可能性为$10^{-4}$—$10^{-8}$/堆年,即一个反
应堆运行一万年到一百万年,才可能产生一次这样的事故.而
同时由安全壳受损造成大量放射性释放的可能性就更小了.

在全世界已经运行的三百七十四座核电站中,压水堆型
占总数的50\%; 正在建设的核电站一百七十六座中,压水堆
型占63\%. 我国正在建设和计划建设的核电站,都是压水堆
型核电站.

在工业化、高科技社会对能源的需求日益增加的同时,环
境污染和生态问题也日益尖锐,电能是常规能源,常规电站
(尤其是燃煤电站)会排出大量的废渣与废气,煤燃料中的
20\%成为灰渣和飞尘.飞尘形成煤烟引起大气污染,烟气中
含有的大量氧化硫、氧化氮,是产生酸雨的主要因素.科学家
们预言,酸雨和二氧化碳产生的“温室效应”将成为今后严重
的环境问题.

核能是一种清洁、安全、经济的能源.压水堆型核电站在
正常运行时,由于燃料元件包壳、压力系统和安全壳三层屏障
的包容,以及放射性废气、废液排放前的一系列处理,加上严
格的排放标准和管理,最终的排放量是很小的.一座大型的压
水堆电站,允许的排放量对附近居民的最大照射仅为天然放
射性本底的5\%, 世界各国多年的运行经验表明,实际排放
量很容易被控制在允许排放量以下.对于废气、废液处理过
程中的浓缩物,经固化后运至厂外贮存库作永久贮存.在核
电站正常运行时,排出的废物很少,有利于保护环境.

现在,发展核电是世界能源发展的共同趋势.据联合国
国际原子能机构统计,全世界1985年的核发电量占总发电量
的15\%. 

\subsection{贝克勒耳}
贝克勒耳(1852—1908),法国物理学家.
1852年11月15日诞生于法国巴黎的一个科
学世家,祖父、父亲都是法国科学院院士,贝克勒耳幼年偏
爱自然科学.1872年中学毕业后,曾先后在巴黎工业大学和
土木工程学院学习,1876年到工业大学任教.1888年以光
的吸收方面的论文获得博士学位,1892年任巴黎自然博物馆
和巴黎工业大学教授,1903年贝克勒耳因为发现放射性现
象和皮埃尔·居里夫妇同获诺贝尔物理学奖.

贝克勒耳早年从事磷光和荧光现象的研究已取得不少成
就.1895年德国物理学家伦琴发现了X射线.这一发现引
起了贝克勒耳的很大兴趣.贝克勒耳试图研究X射线与可
见光的联系,他把一张照相底片用黑纸包得严严实实,再把硫
酸铀钾晶体放在上面,在太阳光下照射.几个小时后,把底片
冲出来一看,发现底片上有浅浅的黑斑,贝克勒耳认为,这是
硫酸铀钾晶体受阳光照射激发出荧光和X射线,但是后来的
实验,由于几天连阴雨而不能用阳光照射,他只好扫兴地把铀
盐晶体和用黑纸包好的照相底片一起放入写字台的抽屉里,
结果惊奇地发现:这些未经阳光照射的铀盐也能使底片感光,
冲出来的底片上清清楚楚地显现出黑斑.他作了认真的分析,
认为这种现象同荧光、阳光都无关,可能是铀盐晶体自身发出
的一种神秘的射线的结果,接着他又对铀的穿透辐射性质进
行了试验,证明这种辐射在一定距离内具有放电体的性质,即
放射性.这是科学实验中认识放射性的开端.在3月2日科
学院例会上,贝克勒耳激动地宣布了这个新发现.

会后贝克勒耳精心地做了一系列实验,分别对铀盐晶体
加热、冷冻、研成粉末、溶解在酸里等物理、化学加工,发现只
要有铀元素,就有这种神奇的贯穿辐射.他还试验了纯金属
铀,发现它所产生的贯穿辐射要比用硫酸铀钾强,在五月十
八日科学院例会上,贝克勒耳宣布,铀盐会自发放射出射线,
这是一种新的、由原子自身产生的射线,后人就把这种射线
叫做贝克勒耳射线.

贝克勒耳刚年过半百,身体健康状况就很差,医生劝他休
养,他却舍不得离开实验室,对医生说:“除非把我的实验室搬
到我疗养的地方,否则我决不离开!”1908年8月24日,贝克
勒耳离开了人世,终年56岁.后人为纪念这位放射性研究的先
驱者,把放射性活度的单位命名为“贝克勒耳”,简称“贝克”.

\subsection{玛丽·居里}
玛丽·居里(1867—1934),法国物理学家、化学家.是放射性学说的奠基人之一,原籍波兰,出生
于华沙.1884年在华沙中学毕业时得到了金质奖章,1891
年至1894年,在巴黎大自然科学系学习,毕业时得到物理
学和数学两个硕士学位,1903年在巴黎大学完成了博士论
文“放射性物质的研究”的答辩.从1906年起,为巴黎大学教
授和教研室主任,1914年起兼任镭学研究院院长.

玛丽·居里和皮埃尔·居里对铀、钍等矿物的放射现象进
行了研究,并从大量的沥青铀矿中分离出两种具有更强的放
射性物质,于1898年发现了两种元素:钋和镭.

1902年玛丽·居里分离出了零点几克纯净的镭盐,而在
1910年她和法国化学家德别爱而诺一起得到了金属镭,她
确定了镭的原子量和它在化学元素周期表中的位置.1903年
由于放射现象的研究,居里夫妇和贝克勒耳一起得到了物理
学诺贝尔奖金,而在1911年,玛丽·居里由于得到了金属状态
的镭和对放射性元素性质的研究而荣获化学诺贝尔奖金.

玛丽·居里试验了很多种放射性元素、研究它们的性质,
详细分析了放射性测量的方法,研究了放射性辐射对人体细
胞的影响,第一个在医学上利用放射性.居里夫妇的杰出发
现,开创了利用原子能的新纪元.

1914年巴黎镭学研究院成立,居里夫人任理事,后任院
长,1922年当选为巴黎医学科学院院士.同年出任国际文
化合作委员会委员,后任副主席.居里夫人除两次获得诺贝
尔奖金外,还获得各国科学奖章十六枚,获得二十五个国家的
荣誉头衔一百多个,被人们誉为“镭的母亲”.

居里夫人为人类作出了巨大的贡献,深受各国人民的爱
戴和崇敬.1920年,美国记者麦朗宁夫人获悉居里夫人很需
要一克镭继续作放射性研究,就在全美国妇女中展开了募捐
运动.她在不到一年的时间里募集到十万美元,买了一克镭
准备献给居里夫人.1921年5月21日,美国第三十四任
总统哈定代表美国妇女,向居里夫人馈赠了这一克镭.

由于长期受放射线照射,居里夫人晚年患了恶性贫血症,
双目也几乎失明.但她经常隐瞒病情,坚持进行研究,她常
说:“我的生活是不能离开实验室的”.1934年7月4日,居
里夫人在法国萨瓦去世,终年67岁.

\subsection{约里奥·居里}
约里奥·居里(1900—1958),法国物理学家,
是居里夫妇的女婿,对原子核物理有重要
贡献.

约里奥·居里1900年3月19日诞生在法国巴黎的一个
商人家庭.他从小喜欢读书,尤其喜欢钻研自然科学.他非
常崇敬法国大科学家巴斯德(1822—1895)和居里夫妇,不但
阅读他们的生平传记,还模仿他们的科学生活.约里奥1918年
考取居里夫妇发现镭的巴黎理化学院,每门功课都是第一.不
久,约里奥应召入伍服役.战后,他回巴黎理化学院一边工
作,一边在朗之万教授指导下学习,他的兴趣在物理、化学方
面.1925年约里奥作居里夫人的助手,并且结识了伊丽芙·居里.

1930年约里奥以论述放射性元素钋的电化学的论文获
得博士学位.1932年约里奥和他的妻子伊丽芙·居里(1897
—1956)合作,用放射性钋所产生的$\alpha$射线轰击铍、锂、硼等元
素,发现了前所未见的穿透性强的辐射.后经查德威克的
研究,确定为中子,1934年他俩在用$\alpha$粒子轰击铝硼时,
首次产生了人工放射性物质.由于这一重大发现,两人于
1935年获得诺贝尔奖金.约里奥·居里于1937年任法兰西学
院教授,法国国家科学研究中心原子能研究所所长.1946年
任法国科学研究中心主任,主持法国原子能委员会的领导
工作.

从本世纪四十年代中期起,约里奥·居里为了维护世界和
平,致力于和平利用原子能的事业,极力反对法国生产和发展
原子武器.为世界和平事业做出了重大贡献,1958年8月14
日,约里奥·居里在巴黎去世,终年58岁.











































































\end{document}