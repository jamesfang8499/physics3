\chapter{原子结构}\minitoc[n]
\section{教学要求}
人类对原子结构的认识,是逐步深入的.本章教材是沿着历史的线索来讲述人类对原子结构的认识的.由于玻尔的原子理论在一定程度上反映了原子的真实情况,又比较适合于中学生的理解能力和知识水平,因此,把这个内容作为本章教学的重点.但教材也指出了玻尔理论的局限性,粗略地介绍了现代原子理论中关于原子结构的一般图景.

这一章的教学可分为三个单元.第一节和第二节为第一单元,讲述原子的核式结构学说的建立.第三节至第五节为第二单元,讲述玻尔理论.第六节为第三单元,讲述原子的受激辐射和激光.

电子的发现,对人类认识原子的结构有重大的意义,它使人们改变了认为原子是组成物质的最小微粒的看法,使人们认识了原子是有结构的.教学中应注意讲清汤姆生研究电子的方法.汤姆生的原子模型是早期的原子模型,讲述这个模型的目的,一方面是为使学生了解人类对原子的认识经历了曲折的发展过程,更重要的是为了使学生认识$\alpha$粒子散射实验的重要意义.

$\alpha$粒子散射实验证明了原子核的存在,通过教学,应当
使学生了解$\alpha$粒子散射实验及其结果,了解卢瑟福是怎样分析$\alpha$粒子的散射现象,否定了汤姆生原子模型,提出原子具有核式结构的学说的.

在分析卢瑟福原子模型的困难时,要用到电学、力学和光谱发射的知识.其中有的知识,学生没有学过.例如,根据经典电磁理论,绕核做加速运动的电子要向外辐射电磁波,电磁波的频率等于电子绕核运行的频率,学生就没有学过.对这些内容,不宜过高要求,让学生知道卢瑟福原子模型与原子的稳定性和光谱的发射性质有矛盾就可以了.

玻尔的原子理论,突破了经典理论的束缚,引入了量子化假设,把原子理论推进了一步.应该要求学生了解玻尔理论的主要内容,对原子的轨道和能量的量子化获得具体的认识.明确知道玻尔对原子结构提出的新概念,体会玻尔的创新精神.对于玻尔理论遇到的困难和建立在量子力学基础上的现代原子理论,只作很初步的一般性介绍.

本章中关于激光产生的原理和应用是选讲教材,向学生介绍这个内容可以扩大学生的眼界,引起学习的兴趣,只要教学时间允许,就应该讲一讲这节教材.

本章的教学要求是:
\begin{enumerate}
\item 了解$\alpha$粒子的散射实验和卢瑟福的原子核式结构.
\item 了解玻尔原子理论的主要内容及其对氢光谱规律的解释.
\end{enumerate}

\section{教学建议}
\subsection{原子核式结构的建立}
本单元讲述了电子的发现以及早期的原子结构模型——汤姆生模型,重点介绍了$\alpha$粒子散射实验.该实验否定了汤姆生模型,奠定了原子核式结构模型的基础.

本单元的教学,应注意以下几点:

\subsubsection{电子的发现}

这部分内容的教学,要突出电子的发现
对人类认识原子结构的重要作用.

电子是怎样发现的.汤姆生用测定粒子荷质比的方法发现了电子,这个问题在本书第一章中曾经提到过.这里要注意联系已学过的知识,讲清汤姆生方法的原理.汤姆生发现阴极射线在电场和磁场中的偏转现象,根据偏转方向,确认阴极射线是带负电的粒子流,当他测定阴极射线粒子的荷质比时发现,不同物质做成的阴极发出的射线都有相同的荷质比,这表明它们都能发射相同的带电粒子.因此这种带电粒子是构成物质的共同成份,这就是电子.

电子的发现对人类认识原子结构的重要性.电子的发现,使人们认识到原子不是组成物质的最小微粒,原子本身也具有结构.由于原子含有带负电的电子,从物质的电中性出发,推想到原子中还有带正电的部分.这就提出了进一步探索原子的结构、建立原子模型的问题,

\subsubsection{汤姆生的原子模型}

汤姆生原子模型是在发现电子的基础上建立起来的,通过教学要使学生对汤姆生原子模型有一个形象的了解,教学中应该注意,讲述汤姆生模型的目的是为了使学生从原子学说的历史发展上来认识$\alpha$粒子散射实验的重大意义,为进一步理解原子的核式结构作好准备.

\subsubsection{原子的核式结构的发现}

$\alpha$粒子散射实验以及卢瑟福对实验现象的分析为原子的核式结构模型奠定了基础.教学中可按以下几个层次进行教学:为什么用$\alpha$粒子的散射现象可以研究原子的结构,$\alpha$粒子的散射实验是怎么做的,实验结果是什么,分析实验结果得到怎样的原子模型.

原子的结构非常紧密,用一般的方法无法探测它内部的结构,要认识原子的结构,需要用高速粒子对它进行轰击,由于$\alpha$粒子具有足够的能量,可以接近原子的中心,它还可以使荧光物质发光,如果$\alpha$粒子与其他粒子发生相互作用,改变了运动的方向,荧光屏便能够显示出它的方向变化,因此卢瑟福采用$\alpha$粒子散射的方法来研究原子的结构.

$\alpha$粒子散射实验的装置,可根据课本上的示意图来讲述,主要由放射源、金箔、荧光屏、显微镜和转动圆盘儿部分组成,每一部分的作用应该让学生明确,实验的做法,课文中写得比较简明,重点应指出荧光屏和显微镜能够围绕金箔在一个圆周上转动,从而可以观察到穿过金箔后偏转角度不同
的$\alpha$粒子,要让学生了解,这种观察是十分艰苦细致的工作,所用的时间也是相当长的.

必须让学生明确,实验结果可以把入射的$\alpha$粒子分为三部分,这三部分$\alpha$粒子的大致多少是用“绝大多数”、“少数”和“极少数”这样的数量形容词来描述的,它们穿过金箔后的情况分别是沿原来的方向前进、发生了较大的偏转和大角度偏转.卢瑟福的原子核式结构模型就是在分析了这三部分$\alpha$粒子的情况后建立起来的.

对实验结果的分析应着重说明如下几点:
\begin{enumerate}
\item 电子不可能使$\alpha$粒子发生大角度散射.$\alpha$粒子跟电子碰撞过程中,两者动量的变化量相等,由于$\alpha$粒子的质量是电子质量的7300倍,在碰撞前后,质量大的$\alpha$粒子速度几乎不变,而质量小的电子速度要发生改变.因此,$\alpha$粒子与电子正碰时,不会出现被反弹回来的现象.发生非对心碰撞时,$\alpha$粒子也不会有大角度的偏转.可见,电子使$\alpha$粒子在速度的大小和方向上的改变都是十分微小的.
\item 按照汤姆生的原子模型,正电荷在原子内部均匀地分布,$\alpha$粒子穿过原子时,由于粒子两侧正电荷对它的斥力有相当大一部分互相抵消,使$\alpha$粒子偏转的力也不会很大.$\alpha$粒子的大角度散射现象,说明了汤姆生模型不符合原子结构的实际情况.
\item 实验中发现极少数$\alpha$粒子发生了大角度偏转,甚至反弹回来,表明这些$\alpha$粒子在原子中的某个地方受到了质量、电量均比它本身大得多的物体的作用.
\item 金箔的厚度大约是1微米,金原子的直径大约是$3\x10^{-10}$米,绝大多数$\alpha$粒子在穿过金箔时,相当于穿过几千个金原子的厚度,但它们的运动方向却没有发生明显的变化,这个现象表明了$\alpha$粒子在穿过时基本上没有受到力的作用,说明原子中的绝大部分是空的,原子的质量和电量都集中在体积很小的核上.
\end{enumerate}

\subsubsection{原子核的电荷和大小}
 原子核的电荷这部分内容,要
突出测知原子核的电荷的重要意义.应该使学生了解,根据$\alpha$粒子散射实验的数据可以算出靶元素原子核的电荷,从而推知这种原子中的电子数.计算的结果表明,元素原子中的电子数非常接近该元素在周期表中的原子序数.人们由此知道元素周期表是按原子中的电子数来排列的.这就是说,元素的化学性质,归根到底是由原子中的电子数、从而是由原子核中的电荷数来决定的.

关于原子的大小,应该让学生记住一个数量级,即原子核的大小在$10^{-14}$米以下,原子的大小是$10^{-10}$米,所以原子核的半径只相当于原子半径的万分之一,这样,原子核的体积与原子体积的比应为万亿分之一.这里突出了原子核是很小的,原子内部是很空的.

为培养学生看书学习能力,可以通过学生自学课文,
教师提出参考讨论题让学生讨论,最后由教师归纳总结出课文的基本内容.也可由学生自己提出讨论问题,归纳课文的主要内容,进行讨论,最后由教师进行小结,象这样的看书自学讨论的方式,可以作为这一章的各节课均可采用的一种教学方法.

\subsection{玻尔理论}
这一单元是本章的重点,主要讲好玻尔理论的三点假设,让学生理解能量量子化的概念,这是玻尔理论的核心,教
学中,要注意以下几点:

\subsubsection{玻尔理论产生的背景}

应该让学生清楚地认识到,卢瑟福的原子核式结构模型虽然能很好地解释$\alpha$粒子散射实验,但跟经典的电磁理论发生了矛盾.这些矛盾说明从宏观现象总结出来的经典电磁理论不适用于微观现象,不解决这个矛盾,原子理论就不能前进,这就是产生玻尔原子理论的历史背景.

卢瑟福的原子核式结构模型与经典电磁理论的矛盾主要有两点:按照经典电磁理论,电子在绕核作加速运动过程中,要向外辐射电磁波(这一点学生没有学过,可由教师告诉他们),因此能量要减少,电子轨道半径也要变小,最终会落到原子核上,因而原子是不稳定的;电子在转动过程中,随着转动半径的缩小,转动频率不断增大,辐射电磁波的频率不断变化,因而大量原子发光的光谱应该是连续光谱.然而事实上,原子是稳定的,原子光谱也不是连续光谱而是线状光谱.

\subsubsection{玻尔原子理论的主要内容}
玻尔把量子观念引入到原子理论中去,提出了不能用经典概念解释的三条假设,是一个创举.这三条内容应作为整体来理解.为了便于学生掌握玻尔理论,不妨把这三条叫作能级假设,跃迁假设和轨道假设.

能级假设说明原子只能处于一系列不连续的能量状态中,这些状态叫定态,具有一定的能量,也叫能级.能级假设是针对原子的稳定性提出的,它承认核式模型,但假定原子只能处于一系列不连续的稳定状态中,处于稳定状态的原子中的电子,虽做加速运动但不辐射能量,对于“不连续”的概念,
学生是不习惯的.一定要使学生明白,从宏观现象的“连续”的概念过渡到微观世界的“不连续”的概念,是人类对物质世界认识上的一次飞跃.

跃迁假设说明原子从一个定态跃迁到另一个定态时,它辐射或吸收一定频率的光子,光子的能量由这两个定态的能量差决定,它说明了原子发光的机制.这一条假设是针对原子光谱是线状光谱提出的,运用了普朗克的量子理论.辐射光子的能量与发生跃迁的两个轨道有关.

轨道假说明原子的不同能量状态对应于电子的不同运行轨道,由于原子的能量状态是不连续的,因此电子的轨道也是不连续的,即电子不能在任意半径的轨道上运行.轨道量子化假设也是针对原子的核式模型提出的,是对第一条假设的补充.

教学时,指出三条假设的针对性,可便于学生理解玻尔理论.教学时还可以把下一节课文中的图8.5氢原子的轨道图提前到这里让学生观看,可以把它画成较大的挂图,据图讲述其轨道“不连续”的含义,让学生对“不连续”的量子观念有一个形象而具体的了解.

\subsubsection{氢原子的大小和能级} 

这是玻尔理论对氢原子的应用获得成功的具体内容,通过这部分内容的教学,应让学生对能量量子化有比较具体的了解.在讲解原子的能级时,应该说明为什么能量是负值,懂得负值的含义是原子处于某一定态时的能量比电子距核无限远时原子具有的能量少,并使学生通过计算大致了解不同能级的能量值和相应的电子轨道半径大小,为使学生头脑中形成轨道和能量量子化的具体图
景,可在本节课上让学生研究课本图8.4和图8.5. 在此基础上来理解什么是基态,什么是激发态,了解原子发光的机制.

\subsubsection{玻尔原子理论对氢光谱的解释}

这个问题,可首先从氢光谱巴耳末线系的四条可见谱线出发,简单介绍一下巴耳末经验公式(顺便介绍里德伯恒量),说明氢光谱谱线之间是有内在规律的,然后再进一步使学生了解按照玻尔理论推导出来的谱线公式跟巴耳末经验公式在
形式上完全一致,而且由前一个公式计算出来的$\dfrac{-E_1}{hc}$的数值
与巴耳末公式中的里德伯恒量的$R$值相符.这说明了巴耳末公式完全可以由玻尔理论推导出来,玻尔理论可以解释氢光谱的规律.

还应向学生说明,氢光谱的其他线系也可以用玻尔理论来解释,由玻尔理论预言存在的新线系,后来也被人们发现,充分说明了玻尔理论的成功.

教学时还应该让学生了解,氢光谱中的每个线系,都是原子从不同的高能级向某一低能级跃迁时发出的谱线.例如,赖曼系是氢原子的电子从$n=2, 3, 4,\ldots$等能级跃迁到$n=1$的基态时发出的谱线;巴耳末线系是氢原子的电子从$n=3, 4, 5,\ldots$等能级向$n=2$能级跃迁时发出的谱线,等等.光谱线上的每一条谱线都是大量处于同一能态的原子的电子向同一低能态跃迁的结果,由于每个原子的电子所处的能态不同,大量原子的跃迁在同一时刻,会发出不同频率的光来,因此光谱线上能够出现各种谱线.

学生对同一线系中各个谱线波长的大小排列顺序与对应的能级跃迁之间的联系往往理解不好,课堂上可以通过具体计算,帮助学生理解.

\subsubsection{玻尔原子理论的困难和量子力学}

本节的教学,首先可简要指出玻尔理论遇到的主要困难,说明一下造成这种困难的原因在于理论内部的矛盾,玻尔理论是一种半经典的理论,一方面引入了量子假设,另一方面又应用经典理论计算电子轨道半径和能量.因此,玻尔理论在解释复杂的微观现象时遇到困难,乃是必然的.

关于量子力学和薛定谔方程等内容,学生不可能深入理解,可只做简单的讲述.但须指出,量子力学是彻底的量子理论,是研究微观世界的基本理论工具,它不但能解释玻尔理论所能解释的现象,而且能够解释大量玻尔理论不能解释的现象.玻尔理论中的三点假设,在量子力学中也变成理论上推导出来的直接结果.

教学中可着重说明,建立在量子力学基础上的原子理论与玻尔原子理论的区别:根据量子力学,核外电子的运动服从统计规律,而没有固定的轨道,我们只能知道它们在核外某处出现的几率大小,核外电子的这种运动情况可用“电子云”来形象描述.电子云稠密的地方就是电子出现几率大的地方.

新的原子结构理论虽然更加切合实际,但它仅给出了原子的数学模型,而没有给我们提供比较直观的物理图景,因此,学生想象这个模型有一定的困难,如果有些同学对量子力学比较感兴趣,可以引导他们阅读一些有关这方面内容的通俗书籍.

\subsection{原子的受激辐射——激光}

激光是一门重要的现代科学技术,在各个领域中有着广泛的应用,对新的技术革命有重要作用,学生了解了激光的知识,可以开阔眼界,了解现代科学技术的发展,增加学习物理的兴趣.

在教学过程中,可以按照原理和应用两个方面进行讲述.

关于激光的产生原理,首先要让学生明确原子发光的两种情形.一种是自激辐射(自然光),一种是受激辐射(激光),了解这两种辐射的不同机制和发出的光在频率、初相和偏振方向上的不同特点.再进一步讲清亚稳态和粒子数反转的问题,让学生了解,只有把处于基态的原子大量激发到亚稳态,使处于高能级的原子数超过处于低能级的原子数,才能使受激辐射持续进行下去得到激光.

讲述激光的应用要从激光的特点出发.激光的主要特点是亮度高、方向性好、单色性好和相干性好.这些特点,应该联系前面学习的受激辐射来理解,也是激光能得到广泛应用的原因.对于激光的应用,除了课本上讲述的以外,教师还可以向学生介绍一些课外读物让学生自己去阅读,以开阔学生的眼界,了解现代科学技术的发展,在有条件的情况下,也可以组织一个科学班会,让学生向全班介绍他们阅读过的有关课外读物的主要内容,介绍激光的特点和广泛应用.

由于激光器的种类很多,具体的构造都比较复杂,教学时不宜多介绍,以便把学生的注意力集中到理解激光的基
本原理和主要应用上.

\section{习题解答}

\subsection{练习一}
\begin{enumerate}
    \item $\alpha$粒子被原子散射的原因是什么?

\begin{solution}
在原子的中心有一个很小的核,原子的全部正电荷和几乎全部的质量都集中在原子核里,电子在核外定向运动.$\alpha$粒子穿过原子时,影响$\alpha$粒子运动的主要是原子核,是原子核与$\alpha$粒子间的库仑斥力影响$\alpha$粒子的运动.如果$\alpha$粒子离核较远,与原子核之间的库仑斥力很小,它运动方向的改变也很小,当$\alpha$粒子与原子核十分接近时,会受到很大的库仑斥力,要发生大角度的偏转.当$\alpha$粒子与原子核发生正碰时,就会被原子核反弹回来,由于原子核很小,能靠近原子核的$\alpha$粒子很少,所以只有极少数$\alpha$粒子能发生大角度偏转.    
\end{solution}
    \item 卢瑟福的原子模型与汤姆生的原子模型,主要区别
    是什么?

    \begin{solution}
汤姆生原子模型认为原子是一个球体,正电荷在整个球内均匀分布,电子象枣糕里的枣子那样镶嵌在球内.而户瑟福的原子模型认为在原子中心有一个很小的核,核集中了原子的全部正电荷和几乎全部质量,电子在核外绕核旋转.汤姆生原子模型是“均匀”结构,而卢瑟福原子模型是“核式”结构,这是他们的主要区别.        
    \end{solution}
    \item $\alpha$粒子的质量大约是电子质量的7300倍.如果$\alpha$
    粒子以速度$v$跟电子发生弹性正碰(假定电子原来是静止
    的),求碰撞后$\alpha$粒子的速度变化了多少,并由此说明,为什
    么原子中的电子不能使$\alpha$粒子发生明显的偏转.

    \begin{solution}
设$\alpha$粒子质量为$M$, 电子质量为$m$, 电子原来静止,$\alpha$粒子以速度$v_1$向电子运动,设发生弹性正碰后$\alpha$粒子与电子的速度分别为$v_1'$和$v_2'$.

由于是弹性正碰,动量和动能都守恒.并且两球运动在同一直线上,我们可以用代数式来计算.

根据动量守恒定律
\begin{equation}
  Mv_1=Mv_1'+mv_2'  
\end{equation}
根据动能守恒
\begin{equation}
    \frac{1}{2}Mv_1^2=\frac{1}{2}M{v_1'}^2+\frac{1}{2}m{v_2'}2 
\end{equation}
由(8.1)和(8.2)消去$v'_2$可得
\begin{equation}
    v'_1=\frac{M-m}{M+m}v_1
\end{equation}
把$M=7300m$代入(8.3), 可得
\[\begin{split}
    v'_1&=\frac{7300m-m}{7300m+m}v_1=\frac{7299}{7301}v_1\\
    \Delta v_1&= v'_1-v_1=-\frac{2}{7301}v_1=-0.0003v_1
\end{split}\]
由此可见,$\alpha$粒子的速度变化,只有初速度的万分之三,这就说明原子中的电子不能使$\alpha$粒子发生明显偏转.  
    \end{solution}
    \item 已知氢原子的半径是$0.53\x10^{-10}$米,电子不致被吸
    引到核上,按照卢瑟福的原子模型,电子绕核做匀速圆周运动
    的速度和频率各是多大?

    \begin{solution}
氢原子核与电子间的库仑引力就是电子绕核做匀速
圆周运动的向心力,所以
\[k\frac{e^2}{r^2}=m\frac{v^2}{r}\]
\[v=\sqrt{\frac{ke^2}{mr}}=\sqrt{\frac{9.0\x10^9\x (1.6\x10^{-19})^2}{9.1\x10^{-31}\x0. 53\x10^{-10}}}=2.19\x 10^{6}\ms\]
运动频率
\[f=\frac{1}{T}=\frac{v}{2\pi r}=\frac{2.19\x 10^{6}}{2\x 3.14\x 0.53\x 10^{-10}}=6.6\x 10^{15}{\rm Hz}\]
    \end{solution}
    \item 为什么在计算电子与核之间的引力作用时,可以不
    考虑万有引力?

    \begin{solution}
已知电子质量$m=9.1\x10^{-31}$kg,氢核质量$M=1.67\x10^{-27}$kg,万有引力恒量$G=6.67\x10^{-11}{\rm N\cdot m^2/kg^2}$.

电子与核之间的库仑引力
\[F=k\frac{e^2}{r^2}=9\x 10^9\x \frac{(1.6\x 10^{-19})^2}{(0.53\x 10^{-10})^2}=8.2\x 10^{-8}{\rm N}\]
电子与核之间万有引力
\[F_{\text{引}}=G\frac{Mm}{r^2}=6.67\x10^{-11}\x \frac{1.67\x 10^{-27}\x 9.1\x 10^{-31}}{(0.53\x 10^{-10})^2}=3.6\x 10^{-47}{\rm N}\]
两者比值
\[\frac{F}{F_{\text{引}}}=\frac{8.2\x 10^{-8}}{3.6\x 10^{-47}}=2.3\x 10^{39}\]
由此可知,万有引力远小于库仑引力,所以计算电子与核之间引力作用时,可以忽略万有引力.
    \end{solution}
\end{enumerate}



\subsection{练习二}

\begin{enumerate}
    \item 利用公式$r_n=n^2r_1$和$E_n=E_1/n^2$,
计算氢原子的第2、3、4轨道的半径和电子在这些轨道上的能量.

\begin{solution}
已知$r_1=0.53\x10^{-10}$米,$E_1=-13.6$电子伏,由公式$r_n=n^2r_1$和$E_n=\dfrac{E_1}{n^2}$可以计算出:
\begin{enumerate}
    \item 当$n=2$时,$r_2=2.12\x10^{-10}$米,$E_2=-3.40$电子伏;
    \item 当$n=3$时,$r_3=4.77\x10^{-10}$米,$E_3=-1.51$电子伏;
    \item 当$n=4$时,$r_4=8.48\x10^{-10}$米,$E_4=-0.85$电子伏.
\end{enumerate}

\end{solution}
\item 根据上题算出的结果,说明要把基态的氢原子激发
到$n=2$的能级上去,需要供给电子多大的能量.如果用电磁
波来供给这个能量,需要用波长多长的电磁波?这个波长属于
哪个波段?

\begin{solution}
    要把氢原子从基态激发到$n=2$的能级上去,吸收的能量
    $$\Delta E=E_2-E_1=-3.4-(-13. 6)=10. 2{\rm eV}$$

    根据光子能量公式$\Delta E=h\nu$和关系式$c=\lambda \nu$, 可得电磁波的波长
\[\lambda=\frac{hc}{\Delta E}=\frac{6.63\x 10^{-34}\x 3.00\x 10^8}{10.2\x 1.60\x 10^{-19}}=1.22\x 10^{-7}{\rm m}\]
这个波长属于紫外区.
\end{solution}
\end{enumerate}



\subsection{练习三}
\begin{enumerate}
    \item 根据玻尔理论,$H_{\alpha}$、$H_{\beta}$谱线光子的能量应该是多少
电子伏?根据实验测得的$H_{\alpha}$、$H_{\beta}$的波长算得的光子的能量是
多少电子伏?二者是否一致?

\begin{solution}
根据玻尔理论,$H_{\alpha}$和$H_{\beta}$谱线的光子,是由氢原子从量子数$n=3$和$n=4$的能级跃迁到$n=2$的能级时辐射的,它们的能量分别为
\[\begin{split}
    H_{\alpha}:&\quad E_3-E_2=-1.51-(-3. 40)=1. 89{\rm eV}\\
    H_{\beta}:&\quad E_4-E_2=-0.85-(-3.40)=2.55{\rm eV}
=2. 55电子伏.
\end{split}\]
而从实验测得的和$H_{\beta}$的波长分别是
\[H_{\alpha}:\;  0.6562\mu{\rm m},\qquad H_{\beta}:\; 0.4861\mu{\rm m}\]
根据$\Delta E=h\nu=hc/\lambda$
可计算这两条谱线对应的光子能量
\[\begin{split}
    E_{H_{\alpha}}&=\frac{hc}{\lambda}=\frac{6.63\x 10^{-34}\x 3.00\x 10^8}{0.6562\x 10^{-6}\x 1.60\x 10^{-19}}=1.89{\rm eV}\\
    E_{H_{\beta}}&=\frac{hc}{\lambda}=\frac{6.63\x 10^{-34}\x 3.00\x 10^8}{0.4861\x 10^{-6}\x 1.60\x 10^{-19}}=2.56{\rm eV}
\end{split}\]
从计算的结果看,两者是一致的.
\end{solution}
\item 计算氢原子从$n=4$,$n=5$能级分别跃迁到$n=3$能
级时辐射出的光子的波长.这两条谱线在哪个波段?它们属
于哪个线系?

\begin{solution}
\[\begin{split}
    \lambda_1=\frac{hc}{E_4-E_3}&=\frac{6.63\x 10^{-34}\x 3.00\x 10^8}{[-0.85-(-1.51)]\x 1.60\x 10^{-19}}\\
    &=1.9\x 10^{-6}{\rm m}=1.9\mu{\rm m}
\end{split}\]
\[\begin{split}
    \lambda_2=\frac{hc}{E_5-E_3}&=\frac{6.63\x 10^{-34}\x 3.00\x 10^8}{[-0.54-(-1.51)]\x 1.60\x 10^{-19}}\\
    &=1.2\x 10^{-6}{\rm m}=1.2\mu{\rm m}
\end{split}\]
这两条谱线在红外波段,属于帕邢线系.
\end{solution}
\item 怎样用玻尔原子理论解释原子吸收光谱的规律?

\begin{solution}
 玻尔理论认为原子从较低能级向较高能级跃迁时,要吸收能量;原子从较高能级向较低能级跃迁时,要辐射能量.无论是原子吸收的能量还是辐射的能量,都等于发生跃迁的两个原子能级间的能量差.对同一种原子说来,对应的能级的能量差是一定的,故其发射光谱与吸收光谱中的谱线是一一对应的,能发射什么样的谱线的原子,也必能吸收这样的谱
线.这就是白光被某种元素的原子吸收时产生的吸收光谱与这种原子发出的明线光谱相一致的原因.
\end{solution}
\end{enumerate}


\section{参考资料}

\subsection{$\alpha$粒子散射理论}

$\alpha$粒子是放射性物体放射出来的高速粒子,它的质量是
电子质量的7300倍,带有两个单位的正电荷.$\alpha$粒子穿过金属薄片时,绝大多数平均只有2—3度的偏转,但卢瑟福的学生盖革和马斯登在1909年从实验中观察到,约有1/8000的$\alpha$粒子的偏转角大于$90^{\circ}$, 其中有接近$180^{\circ}$的.经理论分析知道,这种现象不可能在汤姆生模型那样的原子中发生.当$\alpha$粒子在汤姆生模型的原子的外边时,由于原子的正负电相等且对称分布,原子对$\alpha$粒子没有库仑力的作用(考虑到原子的极化,有作用力也是很微小的). 当$\alpha$粒子接近或进入原子的实体球时,电子因质量很小,对$\alpha$粒子动量变化的影响极小,而它本身将会在$\alpha$粒子的力的作用下离去,所以可以只考虑原子的正电部分对$\alpha$粒子的作用.

设原子半径为$R$, 正电荷$Ze$均匀分布在这球体中.$\alpha$粒子带正电荷$2e$, 当它在原子球的外边,即$r\ge R$时,它所受原子正电荷的库仑力是$\dfrac{2kZe^2}{r^2}$, 到达球面时是$\dfrac{2kZe^2}{R^2}$. 当$\alpha$粒子进入球内,到达离球心$r$处时,它所受的力比在球面时所受的力还要小,这时对$\alpha$粒子起作用的电荷是以$r$为半径的球体中所含的电荷,这电荷
\[Q=\frac{Ze}{\frac{4}{3}\pi R^3}\x \frac{4}{3}\pi r^3=\frac{Zer^3}{R^3}\]
因此,$\alpha$粒子这时所受的力是
\[\frac{2keQ}{r^2}=\frac{2kZe^2r}{R^3}\]
所以进入球体后,离球心越近,所受的力越小.$\alpha$粒子在汤姆生模型中受原子正电部分的力最大是它到达原子球的表面时.$\alpha$粒子的初速度是可以知道的.按上面分析的$\alpha$粒子的受力情况来计算,结论是不能产生大角度散射的,因此汤姆生模型被否定了.

卢瑟福的原子的核式结构模型能够解释$\alpha$粒子散射实验现象.

\begin{figure}[htp]
    \centering
   \includegraphics[scale=.6]{fig/8-1.png}
    \caption{}
\end{figure}

设有一个$\alpha$粒子射到一个原子附近.在原子核的质量比$\alpha$粒子的质量大得多的情况下,可以认为原子核不会被推动,$\alpha$粒子在核的库仑力的作用下而改变了运动的方向,如图8.1所示.图中$v$是$\alpha$粒子原来的速度,$\ell$是原子核离$\alpha$粒子原
运动路径的延长线的垂直距离,叫做瞄准距离,由力学原理可以证明$\alpha$粒子的路径是双曲线,偏转角$\theta$和瞄准距离$\ell$有如下关系:
\begin{equation}
    \cot\frac{\theta}{2}=\frac{Mv^2}{2kZe^2}\ell
\end{equation}
式中$M$是$\alpha$粒子的质量.从上式可以看出,$\theta$与$\ell$有对应关系:$\ell$大,$\theta$就小;$\ell$小,$\theta$就大;对一定的$\ell$, 有一定的$\theta$. 对于不同的瞄准距离,$\alpha$粒子的轨道形状如图8.2所示.
\begin{figure}[htp]
    \centering
    \includegraphics[scale=.6]{fig/8-2.png}
    \caption{}
\end{figure}

当瞄准距离由$\ell$缩小到$\ell-\dd\ell$时,散射角将由$\theta$增大到$\theta+\dd\theta$. 我们来计算散射到两个角锥面($\theta$和$\theta+\dd\theta$)中间的$\alpha$粒子的数目.

\begin{figure}[htp]
    \centering
    \includegraphics[scale=.6]{fig/8-3.png}
    \caption{}
\end{figure}

假设有一束均匀分布的,平行而且等速的$\alpha$粒子沿$AA'$
方向向金属薄片射来(金属薄片和$AA'$垂直),单位时间内,在与$AA'$垂直的单位面积内有$n_0$个$\alpha$粒子,以原子核$O$为中心,以瞄准距离$\ell$和$\ell-\dd\ell$为半径作一个圆环,这个圆环面与$AA'$垂直,如图8.3所示.这个圆环的面积为$\dd S=2\pi\ell|\dd\ell|$.
对准这个环射来的$\alpha$粒子数为
\[\dd n_0=n_0\dd S=2\pi n_0\ell|\dd\ell|\]
为了求出$\ell\dd\ell$的值,把(8.4)式平方,得
\[\ell^2=\left(\frac{2kZe^2}{Mv^2}\right)^2\cot^2\frac{\theta}{2}\]
微分后得:
\[\ell \dd \ell=-\frac{1}{2}\left(\frac{2kZe^2}{Mv^2}\right)^2\frac{\cot\frac{\theta}{2}}{\sin^2\frac{\theta}{2}}\dd\theta \]
带回$\dd n_0$的表达式,则得:
\[\dd n_0=\pi n_0\left(\frac{2kZe^2}{Mv^2}\right)^2\frac{\cot\frac{\theta}{2}}{\sin^2\frac{\theta}{2}}\dd\theta \]
或
\[\dd n_0=n_0\left(\frac{kZe^2}{Mv^2}\right)^2\frac{2\pi \sin\theta}{\sin^4\frac{\theta}{2}}\dd\theta\]
这是被一个核$O$所散射的在两个锥面$\theta$到$\theta+\dd \theta$中间的$\alpha$粒子数.

假设金属薄片每单位面积内有$N$个原子核,由任何一个核所散射的$\alpha$粒子,将都为上式所表示,由于金属薄片到荧光屏的距离较大,$N$个散射角锥的顶点可视为集中在一点.在单位时间内,由这$N$个核所散射的包含在两个锥面($\theta$和$\theta+\dd\theta$)内的$\alpha$粒子数为
\begin{equation}
    \dd n=N\dd n_0=n_0N \left(\frac{kZe^2}{Mv^2}\right)^2\frac{2\pi \sin\theta}{\sin^4\frac{\theta}{2}}\dd\theta
\end{equation}
现在我们来求在单位时间内在荧光屏单位面积上观察到的$\alpha$粒子数.

\begin{figure}[htp]
    \centering
    \includegraphics[scale=.6]{fig/8-4.png}
    \caption{}
\end{figure}

以锥体的顶点$O$为球心作一半径为$r$的球面,如图8.4
所示,由两个锥面$\theta$及$\theta+\dd\theta$在这个球面上所划出的带区的面积为
\[2\pi r\sin\theta r\dd\theta =2\pi r^2\sin\theta\dd\theta\]
穿过这个带区面积的$\alpha$粒子总数为$\dd n$. 那么,在单位时间内穿过这个带区的单位面积上的$\alpha$粒子数(即在$\theta$方向上,每单位时间在荧光屏上每单位面积内所观察到的$\alpha$粒子数)为
\begin{equation}
    \dd n'=\frac{\dd n}{2\pi r^2\sin\theta\dd\theta}=\frac{n_0 N}{r^2}\left(\frac{kZe^2}{Mv^2}\right)^2\frac{1}{\sin^4\frac{\theta}{2}}
\end{equation}

从(8.6)式可知,在一定的实验条件下,乘积$\dd n'\cdot \sin^4\dfrac{\theta}{2}$是一个恒量.这个关系是可以用实验来验证的.下表是观察$\alpha$
粒子在金箔中散射的实验结果.

\begin{center}
\begin{tabular}{ccc||ccc}
    \hline
    散射角$\theta^{\circ}$  & 闪烁数  &  $\dd n'\cdot \sin^4\dfrac{\theta}{2}$ & 散射角$\theta^{\circ}$  & 闪烁数  &  $\dd n'\cdot \sin^4\dfrac{\theta}{2}$\\
    \hline
    150&    33.1&    28.8&    60&    477&    29.8\\
    120&51.9&29.0&45&1435&30.8\\
    105&69.5&27.5&30&7800&35.0\\
75&211&29.1&15&132000&38.4\\
\hline
\end{tabular}
\end{center}


由实验结果可以看出,当散射角由$15^{\circ}$变更到$150^{\circ}$时所得的$\dd n'\cdot \sin^4\dfrac{\theta}{2}$的数值差不多保持不变,这个结论是假设$\alpha$粒子在金属薄片中只发生一次散射而得出的,由于原子核很小,$\alpha$粒子在金属薄片中十分接近原子核的机会很少,对于大角度散射来说,这个假设是可以满足的.但对于小角度散射
来说,$\alpha$粒子是从离核比较远的地方通过的,因此发生多次散射的机会就比较大了,这就是实验数据中,$45^{\circ}$以上的散射结果与理论值符合的比较好,而$45^{\circ}$以下的散射结果与理论值偏离比较大的原因.


\subsection{卢瑟福}
卢瑟福(1871—1937),英国物理学家.卢瑟福出生于新西兰手工业工人家庭,他的父亲主要从事亚麻加工工作,当时他很可能继承这一职业.但是,由于他在学校里各门功课成绩优异,他的父母决定让他继续学习下去.

1894年卢瑟福在喀捷贝尔大学毕业.由于电磁方面的
论文获得了奖金,这个奖金使他得到了在英国最好的大学实习的机会.卢瑟福在英国剑桥大学卡文迪许实验室实习了三年,当时领导实验室的是卓越的物理学家汤姆生.

1896年,巴黎的亨利·贝克勒耳发现了天然放射性,卢瑟福对贝克勒耳的发现非常感兴趣,并立即开始研究放射线.他比较了“铀射线”和伦琴射线,查明它们并不是相同的.他于1899年肯定了放射性辐射中的两种成分,分别命名为$\alpha$射线和$\beta$射线(不久之后,维拉德又发现了第三种成分$\gamma$射线).

1902年他和英国化学家索第(1887—1956)共同提出了原
子自然衰变理论.卢瑟福明确地指出,放射现象是一种物质的原子以一定的速率自行衰变成另一种物质的原子的过程.这个理论打破了原子不可分的观念,在物理和化学理论上引起了一场剧烈的变革,卢瑟福于1908年获得诺贝尔化学奖.

1909年卢瑟福交给年青物理学家马斯登一项简单的任
务,要他数一数穿过各种物质薄片(金、铜、铝等)的$\alpha$粒子,根据当时大家所接受的汤姆生原子模型,从理论上不难作出这样的推断,$\alpha$粒子应该很容易地穿过原子,不发生散射现象,因此卢瑟福认为,薄片不会影响$\alpha$粒子的直线运动,马斯登在实验时注意到,虽然绝大多数的$\alpha$粒子穿过了薄片,但是仍然可以看到散射现象——有一些粒子好象是反弹回来了.

卢瑟福知道了这一情况以后,重复了很多次这个实验.在经过三个多星期的认真思考以后,他得出了下面的结论:原子是一个很复杂的系统,它有一个带正电的核心(原子核),在核周围的一定轨道上转动着带负电的电子.

在卢瑟福的原子模型产生后不久,他的学生亨利·莫塞莱证明了原子核内单位电荷的数目就是原子的序数,它决定元素在门捷列夫周期表上的位置.

1918年,卢瑟福接替退休的汤姆生的职位,担任著名的
卡文迪许实验室主任.逝世前,他一直在那里工作,在这实验室里,他继续做了许多重要的实验.

1919年他用$\alpha$粒子轰击氮原子,第一次实现了元素的人
工转变.

1920年,他预言了中子的存在.

各国许多天才的物理学家都曾是卢瑟福的学生,如詹姆斯·查德威克(英国),尼尔斯·玻尔(丹麦),彼得·里昂诺维奇·卡皮查(苏联)等等.

\subsection{玻尔}
玻尔(1885—1962),丹麦物理学家.
玻尔是物理学创始人之一,是卢瑟福的学生,1885年生于哥本哈根,1911年毕业于哥本哈根大学,他曾在英国剑桥大学汤姆生领导下的卡文迪许实验室工作,还在曼彻斯特卢瑟福实验室工作过.1920年他创建了哥本哈根大学理论物理研究所并担任所长,第二次世界大战期间,玻尔去了美国,战后又回到丹麦工作,主张和平利用原子能和控制原子武器.他还领导创建了欧洲核子研究中心(CERN).

他一生的主要研究工作是发展原子、分子和原子核结构的量子理论.他在普朗克量子假说和卢瑟福原子的核式结构学说的基础上,于1913年提出了氢原子结构和氢光谱的初步理论.为此获得1922年诺贝尔物理学奖.稍后,又提出了经典规律和量子规律之间的对应原理,这些工作对量子论和量子力学的建立起了重要作用.此外,玻尔在原子核反应理论和解释重核裂变现象等方面,也有重要的贡献.





