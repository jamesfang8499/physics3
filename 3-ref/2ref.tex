\chapter{电磁感应}
\section{教学要求}
这一章教材是以初中学过的电磁现象以及高中学过的电
场、磁场等知识为基础,阐述电磁感应现象和电磁感应的基本
规律。这些内容是电磁学的基础知识,也是学习交流电、电磁
振荡和电磁波的基础。

本章教材可分为四个单元。课文的第一节为第一单元,
讲述电磁感应现象。第二、三节为第二单元,讲述楞次定律。第
四、五节为第三单元,讲述法拉第电磁感应定律和电磁感应现
象中的能量转化问题。第六节至第九节为第四单元,讲述电磁
感应现象中的几种特殊情况。

本章教材以法拉第电磁感应定律为中心,进一步揭示了
电与磁的内在联系,因而法拉第电磁感应定律是这一章的重
点。楞次定律及其应用,法拉第电磁感应定律的应用,是教
学上的难点。

用磁通量的概念来概括和表达电磁感应现象的规律,是
这一章的特点。教材是用磁通量的变化来叙述感生电流的产
生条件和楞次定律,用磁通量的变化率来叙述法拉第电磁感
应定律的。因此,理解磁通量的变化和磁通量的变化率的意
义,了解它们之间的区别与联系,对于学好全章具有重要的
意义。

研究感生电流产生的条件、判定感生电流的方向、确定感
生电动势的大小,都是通过实验分析得出规律的。因此在教
学中,做好实验是非常重要的。

用磁通量的变化来表述楞次定律,可适用于电磁感应现
象的各种情况,因而具有普遍性。在教学中,要注意强调这一
点。但是,对于由相对运动而产生感生电流的情形,用阻碍相
对运动来表述楞次定律,用起来比较方便,特别是在分析电磁
感应现象中的能量转化时要用到。因此,在教材中也注意了这
种情况下楞次定律的表述。学习楞次定律要着重使学生学会
运用这个定律来判断感生电流及其磁场的方向。学生已经学
过用右手定则判断感生电流的方向,教学中应该使学生理解
在由相对运动(导体切割磁力线)而引起的电磁感应现象中,
用右手定则和用楞次定律来判断感生电流的方向,结果是一
致的。

感生电动势是表示电磁感应现象的重要物理量,是电磁
学的基本概念,应使学生掌握。教材不要求区别感生电动势
和动生电动势,不要求学生掌握内电路中各点电势的高低。感
生电动势的大小,教材表述为$\mathcal{E}=k\dfrac{\Delta\phi}{\Delta t}$,
,而没有在等式的右边
加上“$-$”号,这是因为,对一般学生,理解负号的意义比较
困难。

能的转化和守恒定律,是普遍适用的客观规律。楞次定
律和法拉第电磁感应定律也符合这一规律。教材只就由相对
运动产生的电磁感应现象来分析能的转化和守恒问题,对于
没有相对运动而磁场变化的情形,由于能的转化涉及场能,而
没有加以分析。

直流电动机的反电动势是选讲教材,反电动势的概念在
讲变压器时要用到。这节内容,可以使学生知道电磁感应现
象不是孤立发生的,通电线圈在磁场中受力而运动,同时就切
割磁力线,产生电磁感应现象,出现反电动势。这个反电动势
又反过来影响线圈中的电流和线圈受力情况、运动情况。这
节内容对培养学生综合分析问题的能力是有好处的。如果课
时允许,希望讲一下这节教材。

自感现象是电磁感应现象的特殊情况,这个内容是学习
交流电、电磁振荡等内容的基础。教学中,要注意讲清自感系
数的物理意义。在自感现象的应用中,要讲清日光灯的两个
主要元件-起动器和镇流器的工作原理和作用。还可结合
实际情况,补充讲解一些日光灯的使用维修知识。

涡流一节是选讲教材,使学生知道产生涡流的条件,对
涡流的有利、有害两方面有所了解就可以了,不要求作进一步
的讨论。

本章的教学要求是:
\begin{enumerate}
\item 理解电磁感应现象,掌握产生感生电流的条件.
\item 掌握楞次定律和右手定则,並会应用它们判断感生电
流的方向。
\item 掌握法拉第电磁感应定律,並能用来计算有关感生电
动势的问题。
\item 理解自感现象及其在实际中的应用,理解自感电动势
的概念和自感系数的物理意义。
\end{enumerate}

\section{教学建议}
\subsection{电磁感应现象}
这一单元研究感生电流产生的条件,是本章教材的起始
课。教学时要注意从复习电生磁以及磁场对电流的作用这样
一些电与磁相互联系的已有知识中提出问题,即引导学生研
究如何利用磁场来获得电流的问题。同时可以简要介绍法拉
第其人及其在物理学上的贡献,以激发学生的学习积极性。

\subsubsection{感生电流产生的条件}
课本中安排了三个演示实验,它是本单元中研究感
生电流产生条件的依据,也是后面研究楞次定律、法拉第电磁
感应定律的基础。因此,认真做好三个演示实验,力争有较好
的观察效果,是教学的关键。同时,要引导学生逐步从实验
现象中总结出正确的结论,这不但是学生掌握概念和规律的
需要,而且也有利于培养学生的抽象概括能力。

三个实验的观察重点不同.实验一是观察闭合电路
的一部分导体在磁场中运动时产生感生电流的条件。这里要
引导学生注意观察导体向上或向下运动和向左或向右运动的
不同结果,从而了解“导体做切割磁力线运动”的含义。可以
明确告诉学生,所谓切割磁力线的运动,就是导体运动速度的
方向和磁感应强度的方向不平行,这也为推导$\mathcal{E}=B\ell v\sin\theta$
这一公式打下基础。实验二是进一步观察导体不动而磁铁运
动时是否产生感生电流。从而使学生了解,只要导体和磁场
之间发生切割磁力线的相对运动,闭合电路中就会产生感生
电流,这里要强调相对运动必须是导体切割磁力线的,否则
闭合电路中是没有感生电流的。实验三是研究导体静止在
磁场中时能否获得感生电流的问题。在这个实验里,学生能
观察到不论是线圈$A$电路的接通或断开还是变阻器滑动片的
移动,线圈$B$电路中都出现感生电流。在观察实验现象的基
础上,要引导学生分析上述现象的物理过程,在上一章的学习
中,学生已经知道由电流所激发的磁场的磁感应强度$B$总是
正比于电流强度$I$的,即$B\propto I$是一个有普遍意义的关系。
电路的闭合或断开控制了电流从无到有的变化,变阻器则是
通过改变电阻来改变电流。而电流的变化必将引起磁场的
变化,线圈$B$中出现感生电流就是由穿过它所围面积的磁场
变化所引起的。

\subsubsection{结论}

教材是在每个演示实验的后面,对实验现象加
以分析,通过对三个实验的分析,逐步得出产生感生电流的
条件,这就是:“不管是闭合电路的一部分导体做切割磁力线
的运动,还是闭合电路中的磁场发生变化,穿过闭合电路的
磁力线条数都发生变化,这时闭合电路中就有感生电流
产生。”然后,再用磁通量的概念总结出只要穿过闭合电路的
磁通量发生变化,闭合电路中就会产生感生电流这个 一般
条件。

处理这段教材时应注意以下几点:

通过提问学生来复习磁通量的概念,明确磁通量的
定义式为$\phi=BS\cos\theta$, $\theta$为面积$S$与垂直于磁感应强度方向
平面的夹角。当$S$与$B$垂直时,$\phi=BS$. 复习磁通量的概念之
后,可向学生提出“根据磁通量的定义式来分析,使一个面积
为$S$的闭合线圈中的磁通量发生变化,有哪几种方法”的问题,
以使学生进一步明确通过一闭合面积的磁通量是由哪些因
素决定的,同时为总结概括产生感生电流的条件打下基础。

利用磁通量概念概括产生感生电流的条件时,要注
意强调磁通量的变化。教学时,可画出上述三个实验的示意
图,并将它们的磁力线的分布在图上表示出来,如图2.1中
甲、乙、丙所示。
\begin{figure}[htp]
    \centering
\includegraphics[scale=.6]{fig/2-1.png}    
    \caption{}
\end{figure}

在图2.1甲中,闭合电路的一部分导体$AB$向右或向左
运动时都作切割磁力线运动,通过闭合电路的磁力线条数发
生了或增或减的变化.磁通量的变化量$\Delta\phi=B\Delta S$. 在图2.1
乙中,当$N$极向下插入线圈时,由于离$N$极近处磁感应强度较
大,使得线圈内部空间的磁感应强度变大。而线圈的面积不
变,故通过线圈的磁通量增加,图2.1丙中,接通开关$K$的瞬
间,线圈$A$中的电流由无到有,故磁感应强度由零开始增大,
而线圈$A$的面积$S$不变,所以磁通量也由零开始增大。上面
两个实验中,磁通量的变化$\Delta\phi=\Delta B\cdot S$. 通过上述分析,应使学生认识到:不论是导体作切割磁力线运动,还是磁场发生
变化,实质上都是引起穿过闭合电路的磁通量发生变化。在
此基础上给学生明确指出,产生感生电流的条件可以归结为
“穿过闭合电路的磁通量发生变化”。

因为初中教材中只讲了切割磁力线的情况,给学生
留下了一个先人为主的印象。又由于在某些情况下利用切割
磁力线来分析感生电流比较方便,因此,学生习惯于利用切割
磁力线来判断电磁感应现象,为了使学生对于用磁通量观点
来判断电磁感应现象加深认识,可通过一些典型例题让学
生进行讨论.例如课本中练习一4就可以放在课堂上处理。
通过例题分析还应使学生认识到,掌握磁体和通电导体所产
生的磁场的特点及磁力线在空间的分布情况,在判断感生电
流时也是十分重要的。


\subsubsection{培养学生用实验研究问题的能力}

在学校实验器
材较全且学生水平比较整齐的情况下,可采取在课堂上学生
随老师一起做实验的办法来组织这一节的教学,这样可对感
生电流产生的条件获得深刻的印象。

\subsection{楞次定律}
楞次定律是确定感生电流方向的一般规律。由于它的内
容抽象,涉及到电与磁间复杂的相互关系,因此它是本章教材
的一个难点。

试用本是从能量守恒定律出发,通过推理得出楞次定律
的。但是从能量守恒定律得出的是阻碍相对运动,要得出阻
碍磁通量变化的表达,还需要进一步的讲解。这就使楞次
定律的得出显得不够轻快。为了解决这个问题,本单元教材
抓住了磁通量变化这个线索,尽可能简捷地引出楞次定律。至
于电磁感应现象中的能量守恒问题,放在第三单元中去讨论。

本单元的重点是使学生明确引起感生电流的磁通量
的变化和感生电流所激发的磁场之间的关系。为了解决这个
问题,可先复习学生所学过的知识:
\begin{enumerate}
\item 磁通量的变化是产生感
生电流的条件。  
  \item 根据电流的磁效应,感生电流一定会激发
磁场,感生电流的方向与它所激发的磁场的方向间的关系可
由右手定则来判定.
\end{enumerate}
在上述知识基础上,再向学生提出“磁通
量的变化与感生电流的磁场之间有什么关系”的问题,并通过
实验引导学生来研究。

在利用课本图2.2所示的装置做实验时,要注意:
\begin{enumerate}
\item 
明确线圈导线的绕向(实验用的线圈最好是实验者自己绕制
的).   
 \item 实验前,要用旧干电池来确定电流的方向与电流表
指针偏转方向的关系。
\end{enumerate}

实验时,最好能将磁铁磁场的磁通量的变化以及感生电
流所激发的磁场的磁力线方向简明示出,使学生能顺利地找
出二者之间的关系。可将实验中观察到的现象填入下表。
\begin{center}
\begin{tabular}{c|p{.08\textwidth}p{.08\textwidth}p{.08\textwidth}p{.13\textwidth}p{.08\textwidth}c}
    \hline
   & 原磁场方向  & 原$\phi$变化情况& 感生电流方向& 感生电流磁场$B$方向& $B$与$\phi$的关系& 结论\\
   \hline
   $N$极向上\\
   $N$极向下\\
   $S$极向上\\
   $S$极向下\\
   \hline
\end{tabular}
\end{center}

为使学生更加形象地了解感生电流的磁场B和原磁场磁
通中的关系,分析实验结果时要注意强调:原磁场的磁通量
减少时,感生电流磁场与原磁场方向相同;原磁场的磁通量
增加时,感生电流磁场与原磁场方向相反,然后使学生进一
步认识,感生电流是通过其磁场与原磁场同方向或反方向来
起到阻碍磁通量变化的作用的。

\subsubsection{楞次定律的表述}

楞次定律表述为:“感生电流具有这样的方向,就是
感生电流的磁场总要阻碍引起感生电流的磁通量的变化。”虽
然这一表述是建立在实验的基础上,但为避免学生产生误解,
要针对学生容易产生的误解作一些讲解。①学生往往把“阻碍
原磁场的变化”理解为“阻碍原磁场”,从而得出“感生电流的
磁场必与原磁场的方向相反”,或者“感生电流的流向必与原
来电流的流向相反”等错误结论。②学生往往把“阻碍原磁场
的变化”理解为“阻止原磁场的变化”,从而得出“有了感生电
流,原磁场就不会变化了”或“感生磁场加原磁场等于稳恒磁
场”等错误结论。在实际教学时,应根据学生的实际问题,有
针对性的讲解,使学生真正理解楞次定律上述表述的意义。

教材通过分析磁铁和通电螺线管之间的磁极相互作
用,提出了楞次定律的另一种表述,即导体和磁体发生相对运
动时,感生电流总要阻碍相对运动。应该告诉学生,这是从不
同角度表述同一个定律,而不是两个定律。

\subsubsection{楞次定律的应用}
这一节的教学,要使学生通过实例的分析,加深对楞
次定律的理解。总结出用楞次定律判断感生电流方向的思路。
应用楞次定律来判断感生电流方向的步骤是:
\begin{enumerate}
\item 明确原来磁
场的方向。
\item 穿过闭合电路的磁通量是增还是减。
\item 根据楞次
定律判定感生电流的磁场方向。
\item 利用安培定则来判定感生
电流的方向。
\end{enumerate}
教学时,切忌由教师将这几个步骤直接讲给学
生,然后用这四个步骤来套例题。这种注入式的方法不利于
学生思维的发展,也不利于调动学生的积极思维。思路应该
让学生自己总结出来,在学生独立总结有困难时,教师可组
织学生讨论”应用之一”得出解题思路。

“应用之二”的分析难度较大.关键是引导学生把第
一步分析好。可分为:电键闭合时、电键打开时、变阻器滑动
片向左移动时、向右移动时四种情况下感生电流方向的判断。
让学生根据上述步骤进行判断后,再用实验演示进行验证。

“应用之三”的教学可先通过复习初中学过的右手定
则来判断,然后用楞次定律的第一种表述来判断。引导学生
进行比较,了解以下两点:
\begin{enumerate}
 \item 用楞次定律和用右手定则来判定
感生电流方向的结果是一致的。
\item 导体切割磁力线时,用右
手定则判感生电流方向更为简便。   
\end{enumerate}
学生条件容许时,还可
以引导学生用楞次定律的第二种表述来进行判断,这里的关
键是根据导体的运动方向来判定感生电流受到的安培力的方
向。然后根据安培力的方向和原磁场方向利用左手定则来判
断感生电流方向。最后告诉学生,究竟用什么方法来判断感
生电流方向,应根据题目的情况来决定,但以简便为原则,为
了加深这种认识,可将练习二中的题3提到课堂作巩固练
习.练习二中题6的实验应在课堂进行演示,或将仪器交给
学生在课后进行观察。

\subsection{法拉第电磁感应定律}
本单元的教学,首先讲述法拉第电磁感应定律,然后对电
磁感应现象中能量转化进行分析,使学生认识到电磁感应现
象的规律不但有实验基础,而且能用能量转化和守恒定律这
一普遍规律来认识,处理本单元教材应注意以下几点:
\begin{figure}[htp]
    \centering
\includegraphics[scale=.6]{fig/2-2.png}
    \caption{}
\end{figure}

法拉第电磁感应定律是研究感生电动势大小的规律,
感生电动势是反映电磁感应现象的重要物理量,要使学生理
解感生电动势的概念。可以从图2.2两个电路的对比中使学
生认识感生电动势,根据闭合电路的欧姆定律可知,闭合电路
的电流是由电源的电动势决定的.图2.2甲电路中的线圈相
当于电源,其电动势为感生电动势,线圈的导线电阻为电源内
阻。$a$端电势高为正极,$b$端电势低为负极,可以画出图
2.2乙的电路跟图2.2甲的情况类比,告诉学生对含有电磁
感应现象的闭合电路分析时,可首先画出如图2.2乙所示的
电路。还应通过分析使学生了解到,电磁感应现象中感生电
动势比感生电流更有本质意义。至于电路中出现的感生电
流,只是在闭合电路中有感生电动势存在的必然结果,当电
路不闭合时,也会产生电磁感应现象,这时并没有感生电流,
感生电动势却仍然存在。

要向学生说明,感生电动势方向的判定要借助于感生电
流方向的判定。即感生电动势的方向与闭合电路中感生电流
的方向相同。如果电路没有闭合,也要将它想象为闭合的。

关于电磁感应定律,课本是按如下线索来安排教材
的:
\begin{enumerate}
\item 重新演示第一节中的三个实验,使学生了解到由于磁通
量的变化快慢不同,直接观察到产生的感生电流大小不同。磁
通量变化越快,感生电流越强,感生电动势越大。    \item 提出用单
位时间内磁通量的变化来定量描述磁通量的变化快慢。从而
引出磁通量的变化率的概念。    \item 由上述思路自然引出感生电
动势大小由磁通量的变化率来决定的结论。即磁通量的变化
率越大,感生电动势就越大。
\end{enumerate}
按上述线索处理教材时,一是要
使学生对演示实验的现象观察清楚。二是要讲清变化率的
概念。教学时,可以列举速度由位移的变化率决定,加速度是
速度的变化率,等等,通过用这些已有知识的类比来加深对
变化率的理解。

讲解法拉第电磁感应定律的定量表达式时,要注意讲
清比例常数$k$, 可引导学生推导出$1{\rm Wb/s}=1{\rm V}$,从而得出
$k=1$的结论,因此法拉第电磁感应定律的表达式为$\mathcal{E}=\Delta\phi/\Delta t$
要使学生了解如果闭合电路是由$n$匝线圈串联组成,整个线
圈的总电动势是单匝线圈的$n$倍,即$\mathcal{E}=n\Delta\phi/\Delta t$。
这里要注意说
明,穿过每匝线圈的磁通量变化率是相同的。应明确指出,利
用$\mathcal{E}=\Delta\phi/\Delta t$
进行定量计算,所得结果是$\Delta t$时间内的平均电动
/
势。因为$\Delta t$是一个有限的时间间隔,在这个时间内磁通量
的变化可以是不均匀的。如果在$\Delta t$时间内磁通量的变化是
均匀的,则其变化率是恒定的,这时平均电动势和即时电动
势相等。

利用$\mathcal{E}=\Delta\phi/\Delta t$
来研究导体做切割磁力线运动,可
推导出计算感生电动势大小的公式为$\mathcal{E}=B\ell v\sin\theta$, 注意条
件是$v$与$\ell$垂直.其中,$\theta$是$B$与$v$的夹角.不论从分解速度
角度还是从分解磁感应强度角度来理解这个公式的意义,都
是等价的.因为不论以速度$v$为基准把$B$分解为$B_{\parallel}$和$B_{\bot}$, 还
是以$B$为基准把$v$分解为$v_{\parallel}$和$v_{\bot}$, 都会得出相同的结果.上
式中如果速度$v$为即时速度,则求得的$\mathcal{E}$为即时电动势,如果
速度$v$为平均速度,则电动势$\mathcal{E}$为平均电动势。切割磁力线运
动产生的感生电动势的计算式虽然是$\mathcal{E}=\Delta\phi/\Delta t$
的一个特例,
但在计算感生电动势时是一个十分重要的公式,必须使学生
熟练掌握。当闭合电路所包围的面积$S$不变时,由于磁场的
变化而引起磁通量发生变化,其感生电动势大小的计算式为
\[\mathcal{E}=S\cdot \frac{\Delta B}{\Delta t}\]
这种将磁感应强度的变化率
$\Delta B/\Delta t$
和感生电动势
相联系的公式,也应要求学生有所了解。

讲完法拉第电磁感应定律后,可引导学生从$\mathcal{E}=\Delta\phi/\Delta t$
这一公式出发,对产生感生电动势的条件、大小和方向作一总
结性分析,使学生理解公式中磁通量的变化量$\Delta\phi$是产生感
生电动势的条件,而感生电动势的方向是由磁通量的变化情
况来决定,即是增加还是减少来决定的。感生电动势大小则
是由$\Delta\phi/\Delta t$
来决定。

电磁感应现象中的能量转化问题在课本中是作为独
立的一节来安排的,教师可通过复习提问,明确能量的转化可
通过做功来实现。感生电动势的存在,意味着有其他形式的能
转变成了电能,然后引导学生来分析课本中第一节的实验一
和实验二,分析时,应注意:
\begin{enumerate}
\item 使学生定性了解机械能转化为
电能的物理过程,要讲清使导体和磁场之间维持相对运动克
服磁场力作功的过程是机械能转化为电能的过程;闭合电路
中的感生电流作功是电能转化为电路的内能等其他形式能量
的过程。
\item 要说明楞次定律是与能量转化和守恒定律相符的。
可以引导学生从反面来分析:假设感生电流方向与用楞次定
律判断的方向正好相反,所得的结果是什么?引导学生来讨
论这个问题。
\end{enumerate}


在上述定性讨论基础上,利用课本中图2.20来定量研究
电磁感应现象中的能量转化与守恒。定量讨论时,可按照物理
过程提出一系列问题进行引导。例如,$ab$导体作匀速运动时
受到的安培力大小是多少?方向如何?导体受到的外力多大?
外力克服安培力做功的表达式是什么?如果已知$\mathcal{E}$, 在$\Delta t$时
间内感生电流做功的计算式是什么?这两个功有什么关系?为
什么?使学生在回答或讨论这些问题中得到结论。

发生电磁感应现象时常伴随着其他现象发生。当闭
合电路(电阻已知)中产生感生电动势时,电路中出现感生电
流,而感生电流的强弱又由欧姆定律所决定。感生电流在磁
场中必将受到磁场力作用,可见通过感生电流可将电磁感应
与电路、力学等知识联系起来,因此本单元要安排习题课,但
题目不要选得过难,课本中作复习用的习题中的题4和题
5可作习题课的例题处理。根据学生学习中存在的问题,教
师也可以自选一些例题。




































