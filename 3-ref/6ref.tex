\chapter{光的波动性}\minitoc[n]
\section{教学要求}

本章讲述光的波动性、光的电磁理论和电磁波谱,以及光
谱和光谱分析的知识。认识光的波动性及其电磁本质,在物
理学史上有重要的意义。光谱学的知识,为研究原子结构提
供了信息。光的干涉、衍射、偏振等现象,在现代科学技术中
有主要的应用。学生应该对这些知识有所了解。

光的干涉现象,根据学生的知识水平和接受能力,有条件
分析得深入一些。因此,作为本章的重点内容。衍射现象的
分析,要复杂一些,因此只介绍了现象,本章的其他知识都是
介绍性的。

本章教材分为三个单元。第一节至第五节为第一单
元,讲述光的波动性,第六、七节为第二单元,讲述光的电磁
本性,第八节为第三单元,介绍光谱的初步知识。

讲述光的微粒说与波动说的矛盾是为了使学生了解人们
对光的本性的认识经历了曲折的过程,也是讲解光的波动性
的引子,激发学生的学习兴趣。

光的干涉和衍射现象,在日常生活中极易忽略,学生不熟
悉。教学中应注意通过演示使学生观察现象,联系机械波的
干涉和衍射现象,运用波动知识进行分析。在讲干涉现象时,
不要求解释为什么不同光源发出的光不是相干光,要把教学
的重点放在用波动说解释产生明暗条纹的条件上。在推导时
要注意突出推导的思路。通过研究光的干涉,还应该让学生
知道光的颜色跟波长(频率)有关,不同色光的频率是不同的。

光的衍射现象也证明了光具有波动性。在教学中可以指
导学生用一些简单的方法进行观察。

光的偏振现象,证明了光是横波。但是学生对偏振现象
比较生疏。教学中应跟学生比较熟悉的机械波的有关现象类
比,并在充分观察实验现象的基础上进行讲述。

从表面上看,光现象和电磁现象之间似乎没有什么联系。
麦克斯韦提出的光的电磁说和赫兹对电磁波进行的实验研
究,揭示了光的电磁本质。通过光的电磁说的教学,应该使学
生体会自然现象间的相互联系,对无线电波、红外线、可见光、
紫外线、伦琴射线等不同波长的电磁波有一个统一的认识,并
对它们的性质和产生机理有所了解。

这一章的教学要求如下:
\begin{enumerate}
\item 理解光的干涉现象,理解产生明暗条纹的条件,了解
光的干涉现象的应用。
\item 了解光的衍射现象和产生衍射现象的条件。
\item 了解光的偏振现象和光是一种横波.
\item 了解光是一种电磁波;了解无线电波、红外线、可见
光、紫外线、伦琴射线等都是波长不同的电磁波。
\item 了解光谱和光谱分析的初步知识。
\end{enumerate}

\section{教学建议}
\subsection{光的波动性}
\subsubsection{人类对光的本性的两种认识}

人类对光的本性的认
识经历了一个辩证发展的过程,到十七世纪,在人类已经积累
了许多几何光学知识的基础上,形成了对光的本性的两种认
识——微粒说和波动说。

教学中要讲述两种理论在解释实验规律上各有其成功的
一面,也要讲述两种理论的不足之处。还要说明两种理论的
尖锐矛盾。例如,微粒说难以说明光在媒质界面上同时发生
反射和折射,波动说却可以解释这个事实,波动说难以说明光
的直线传播,在微粒说看来,这都是很自然的现象。还可适当
补充一些历史事实进一步说明微粒说和波动说的尖锐矛盾,
例如,笛卡儿从微粒说推导出光的折射定律,得出了光在媒质
中的速度大于真空中的光速;惠更斯从波动说也推导出光的
折射定律,但得出了光在媒质中的速度小于真空中的光速。对
于两束光相交时各自独立传播的事实,微粒说难以解释,波动
说却很容易说明,等等。通过这些事实的讲述,使学生认识到
理论和实践的矛盾,两种理论对光的本性认识的矛盾,是推动
人类认识光的本性的内在动力。

\subsubsection{双缝干涉}

光的干涉现象是光具有波动性的重要依
据。教材首先介绍了托马斯·杨在历史上第一次解决了相干
光源的问题,成功地做出了光的干涉实验,然后具体介绍了双
缝干涉实验,并用波动理论对实验现象进行了定量的分析,得
出了干涉条纹间距与波长的关系,这对于学生认识光的波动
性重要的意义,由于学生对于光的干涉现象比较生疏,难
以把光想象成是一种波动,而且运用波动理论分析光的干涉
现象,内容比较抽象,学生接受起来会有一定的困难。建议教
学中注意如下几点:

首先要复习机械波的干涉的知识,复习重点放在什
么是波的干涉现象,相干波源的条件。运用波动理论分析空间
振动加强区域和削弱区域的产生条件,还要进一步用光程差
来分析两个振动的位相关系,为讲授本节内容做准备。

要做好演示实验,使全班同学都亲眼看到光的干涉
现象,这是认识光的干涉的基础。实验内容包括单色光干涉
的明暗条纹和白光干涉的彩色条纹。要注意介绍仪器装置。教
学要结合课本图6.2进行.

运用波动理论分析双缝干涉实验要注意:
\begin{enumerate}
\item 说明如
何获得相干光源,这是得到稳定的干涉现象的关键。结合实验
装置,对照课本图6.2说明通过双缝得到的$S_1$和$S_2$
两个光源,是在任何时刻频率和位相都相同的相干光源。
\item 
类比机械波的干涉,运用波动理论分析光的干涉现象.把$S_1$
和$S_2$相干光源发出的光想象成两列光波,两列光波在屏上相
遇。两个光振动叠加产生亮、暗条纹。亮条纹处的光能量较
强,对应着合振动互相加强;暗条纹处的光能量较弱,对应着
合振动互相削弱。
\item 推导屏上亮暗条纹的位置公式要讲清思
路:屏上亮纹或暗纹的位置用距$O$点的距离$x$表示,屏上任
一点的振动加强或削弱的情况决定于该点到光源$S_1$和$S_2$的
距离之差(光程差)$\delta=r_2-r_1$. 在$\delta$等于波长整数倍的位置,
产生亮纹;在$\delta$等于半波长奇数倍的位置时产生暗纹.寻求
$\delta$与$d$, $\ell$和$x$的关系,推出$\delta\approx \dfrac{d}{\ell}\cdot x$
\item 学生对于公式
\[x=\pm k\frac{\ell }{d}\lambda, \qquad k=0,1,2,\ldots\]
和
\[x=\pm (2k-1)\frac{\ell }{d}\cdot \frac{\lambda}{2}, \qquad k=1,2,\ldots\]
的表述方法不习惯,讲述时要从具体的中央亮纹、第1条、第2
条……亮纹或暗纹入手,归纳出一般的表述。例如,亮纹的条
件是$\delta=r_2-r_1=0,\lambda,2\lambda,\ldots$和$-\lambda,-2\lambda,\ldots$(“$-$”的意义
是$r_2<r_1$, 位置在$O$点的下方)。归纳出$\delta=\dfrac{d}{\ell}x=\pm k\lambda$, 解出
\[x=\pm k\frac{\ell }{d}\lambda, \qquad k=0,\pm 1,\pm 2,\ldots\]
然后具体说明$k=0,1,
2,\ldots$所对应的
$x=0,\pm\dfrac{\ell}{d}\lambda,\pm 2\dfrac{\ell}{d}\lambda,\ldots$
的意义,使学生理
解这种数学表述的内容。
\item 双缝干涉公式的近似条件是
$\ell\gg d$和$\ell\gg x$. 双缝间的距离$d$一般仅为十分之几毫米。而
屏上偏离开中央亮纹较远处的亮纹的强度是十分弱的,几乎
无法观测到。通常能观察到亮纹范围远小于$\ell$.
\end{enumerate}


根据相邻亮纹或暗纹间的距离公式$\Delta x=\dfrac{\ell}{d}\lambda$
可以测量光波的波长。红光的干涉条纹间距最宽,紫光的最窄,由此
可认识不同颜色的光的波长不等,由红到紫,波长越来越短,
频率越来越高。进一步说明白光的彩色干涉条纹的产生原因。
学生通过测量光波的波长,对于可见光的波长和频率可以有
一个大致的认识。

说明双缝干涉的亮暗条纹反映了光源$S_1$和$S_2$发出
的光的能量在空间的分布情况。暗条纹处的光能量几乎是零,
表明两列光波叠加彼此相互抵消。这并不是光能量损耗了或
转变成了其他形式的能量,而是按照波的传播规律,没有能
量传到该处;亮条纹处的光能量比较强,光能量增加也不是光
的干涉可以产生能量,而是按照波的传播规律,到达该处的能
量比较集中。

\subsubsection{薄膜干涉 }
学生平时都见过薄膜干涉现象(雨后路面
上的油膜形成的彩色条纹,色彩绚丽的肥皂泡等)只是没有引
起注意。提出一些现象,并让学生动手做肥皂液薄膜的干涉
实验,观察单色光的明暗条纹和白光的彩色条纹,会给学生
留下深刻的印象。实验时,可引导学生细致地观察干涉条纹,
例如,可观察到干涉条纹产生在薄膜的表面上,肥皂液薄膜的
干涉条纹基本上是水平的等等。(这些都是等厚干涉的特点决
定的。)

分析薄膜干涉实验的重点应放在如何得到相干的
两列波,薄膜上是怎样出现明暗相间的干涉条纹或彩色条纹
的.课本图6.4是肥皂液薄膜干涉的示意图.图中只给
出了从楔形薄膜的前后表面反射的两列光波的位相关系和叠
加的结果。该图表示的是光波几乎垂直地入射到楔形薄膜上
后,又从薄膜的前后表面反射回来的情况,薄膜的两个表面是
近于平行的。图中薄膜的楔形被大大夸张了,应该指出,分别
从前、后表面反射回来的两列波都来自于同一入射波,因而是
相干的。由于从后表面反射的光波比前表面反射的光波通过
的路程较长,因而位相要落后一些,落后的相差与膜的厚度有
关。在膜上某些厚度的地方,两列反射波是同相的,形成相互
加强的亮纹;在另一些厚度的地方两列反射波是反相的,形成
相互削弱的暗纹。教材没有深入分析计算光程差和薄膜厚度
的关系,也不涉及光在光疏媒质中达到光密媒质的界面反射
时的半波损失以及在液膜内的光波波长小于空气中的波长等
问题,教学中应注意掌握讲授的深度,以免加重学生的负担。

\begin{figure}[htp]
    \centering
    \includegraphics[scale=.6]{fig/6-1.png}
    \caption{}
\end{figure}


薄膜干涉在科学技术上有重要的应用,除了教材中
介绍的“用干涉法检验光学元件表面加工的质量”之外,还
可适当增加介绍在检验钢球的直径、透镜表面的曲率半径,测
量长度的微小变化等方面的应用,讲授增透膜时,要注意说明
反射光和透射光的能量之和等于入射光的能量(不考虑媒质
对光的吸收)。增透膜的作用只是使反射的两列光波产生相消
干涉,反射光的能量趋于零,因增加了透射光在入射光中所
占的比例。并不是增加了光的能量,以免学生误解。增透膜
的厚度是光在薄膜媒质中传播的波长的1/4(不是真空中波长
的1/4)。由于光垂直于薄膜表面入射时,从前后表面反射的两
列光波的路程差等于薄膜厚度的2倍,如图6.1所示,当膜的
玻璃厚度是$\lambda/4$
时,路程差恰好是$\lambda/2$,
从前后表面反射的光波的相位恰好相反,便产生相消干涉。制作增透膜的材料是氟化镁
${\rm MgF_2}$, 折射率$n=1.38$, 介于空气和玻璃之间.因此在空气
和增透膜的界面上、增透膜和玻璃的界面上,反射情况相同,
不会再额外增加两列光波的光程差。增透膜只对人眼或感光
胶片最敏感的绿光起增透作用。如果白光照射到增透膜上,
由于绿光产生相消干涉,在反射光中绿光的强度几乎是零,
而其他波长的反射光并没有完全抵消,因此,增透膜呈绿光的
互补色——淡紫色。

\subsubsection{光的衍射}

光的衍射现象进一步证明了光具有波动
性,对发展光的波动理论起了重要的作用。教材讲述光的衍射
的思路是先说明一般情况下不容易观察到光的衍射现象的原
因,同时也就说明了观察衍射现象的条件,再来做衍射实验。
由于衍射现象产生的物理过程分析起来比较复杂,课本中对
于衍射现象未作理论上的分析。

做好光过小孔或单缝发生衍射现象的实验,是学
生认识光的衍射的基础。实验中要让学生观察:
\begin{enumerate}
\item 光偏离直
线传播,绕过障碍物进入阴影中,并且在屏上出现明暗相间的
衍射条纹。
\item 只有孔或狭缝较小时,光的衍射现象才比较显
著。
\item 为了认识光的直线传播和光的衍射的关系,观察屏上的
光斑变化情况时应使孔径或狭缝的宽度逐渐变小。当孔径或
缝宽较大时,屏上光斑的边缘清晰,显示出光沿直线传播;孔
径或缝宽逐渐变小时,屏上光斑的边缘逐渐变得不清晰了,衍
射现象逐渐变得显著起来,直至出现明暗相间的衍射条纹。
\end{enumerate}
上述演示可以使学生认识到,由于光是波动,遇到障碍物时发生
衍射现象是不可避免的,只是由于在一般情况下障碍物的尺
寸比光的长大得多,因此衍射现象很不明显。当衍射现象
可以忽略时,才可以认为光是沿着直线传播的。

建议采用课堂讲授和学生实验并进的方式进行教
学。用游标卡尺的测脚形成可调宽度的狭缝观察线光源的衍
射现象,用不同孔径的小孔观察点光源的衍射现象。学生自
已动手观察到光的衍射条纹,可使他们了解到缝宽和孔径多
大时能够观察到比较显著的衍射现象。

菲涅耳圆盘衍射的中心亮斑能够引起学生的极大兴
趣,但在课堂上完成这个实验比较困难。教学时可以用细金
属丝产生的衍射来代替,由于观察到在金属丝的阴影中间有
一条亮线,学生会更加信服光的衍射现象。

光栅的衍射是衍射现象在科学技术上的重要应用.
光通过衍射光栅的亮条纹随着光栅缝数的增加而变窄和变亮
的特点是从实验现象中得出的。教学中不要求从理论上加以
分析。可介绍一些衍射光栅的应用,例如利用光栅衍射条纹的
特点可以比较精确地测量光波的波长和产生均匀分布的光
谱等。

\subsubsection{光的偏振}
横波的偏振是新概念.为了从机械波入手来认识偏
振.应将课本257页的实验演示给学生看.这个实验,形象
地给出了横波是偏振的机械模型,偏振现象是横波区别于纵
波的最明显的标志,通过机械波和光波的类比,可以从光的
偏振现象使学生认识光是横波。

光的偏振现象要通过实验给出.实验的设计思想和
横波偏振的机械模型一样。电气石晶体薄片或人造偏振片对
于某一振动方向的光具有选择吸收的本领,它们的作用相当
于课本257页的演示中限制或检验振动方向的“狭缝”.实验
应先演示光通过两个偏振片时,转动其中任一个偏振片的方
位,透射光的强度出现周期性的变化,给学生以鲜明突出的印
象,再演示光通过一个偏振片的情况,使学生产生悬念,激发
他们的探索热情。分析光的偏振实验,要引导学生把光波和
机械横波相类比,建立光的波动模型,能想象出光是一种横
波,它的振动方向跟光的传播方向垂直,教学中应注意讲清起
偏器和检偏器的不同作用,自然光和偏振光的区别和联系。

通过演示反射光的偏振现象,说明光的偏振现象是
普遍的(即教材中讲的“除了从光源直接射来的,基本上都是
偏振光),也便于使学生理解光的偏振现象在科学技术上的一
些应用,教材还安排了“偏振光与立体电影”的阅读教材,如能
配合教学看一场立体电影,学生一定会有浓厚的兴趣。

\subsection{光的电磁本性}

\subsubsection{光的电磁说}

光的电磁说比光的波动说前进了一大步.课本结合物
理学史讲述了人类认识光的本性的发展过程。教学中要注意
以下四个环节:
\begin{enumerate}
    \item 从十七世纪开始到十九世纪初,光的波动说不断地
发展和完善,逐渐为人们所接受,但是人们对光的本性认识不
足,以为光也是一种机械波,这种认识在光的传播媒质等问题
上遇到了严重的困难。
\item 法拉第发现在强磁场作用下,偏振光的振动面发生
偏转的现象。它启示人们把表面上很不相同的光现象和电磁
现象联系起来。
\item 光是一种电磁波.电磁波和机械波在本质上不同,
它可以在真空中传播而不需要任何媒质。这就不仅解决了波
动说在光的传播媒质问题上遇到的困难,而且对光的本质有
了进一步的认识。
\item 历史上,光的电磁学说是麦克斯韦作为假说提出的,
赫兹实验证实了电磁波的客观存在,也证明了光是一种电磁
波,使光的电磁理论得以确立。
\end{enumerate}

通过光的电磁说的教学,不仅要使学生了解光的电磁说
的基本内容。还要通过光的电磁理论的建立和发展过程,认
识理论和实践的辩证关系,从中受到辩证唯物主义世界观的
教育。

\subsubsection{电磁波谱} 
电磁波谱的教学,应该着重使学生领会光
的电磁说把光现象和电磁现象统一起来了。

在把无线电波、红外线、可见光、紫外线、伦琴射线和γ射
线按照频率(或波长)的顺序排列成波谱,使学生对电磁波有
一个全面的了解后,还要进一步使学生认识这些电磁波既具
有共同的本质,又有各自的特性和不同的产生机理,介绍这
些内容可以为进一步学习光的波粒二象性和原子的内部结构
做准备。

在电磁波谱的教学中,还要注意介绍红外线、紫外线、伦
琴射线的一些应用,以开阔学生的视野,激发他们学习科学
技术的志趣。

通过这一单元的教学,应当使学生体会到自然现象之间
是相互联系的,有些表面上很不相同的现象却存在着共同本
质。把光现象和电磁现象统一起来,是物理学的伟大成果,现
在物理学的研究还在更加广泛的范围上进行着类似的统一
工作。

\subsection{光谱的初步知识}
\subsubsection{介绍光谱学的知识}

要让学生认识观察光谱的仪
器——分光镜。讲述分光镜的构造原理要结合实物和挂图。应
先讲述单色光通过三棱镜的光路,单色光照亮狭缝$S$, 经过凸
透镜$L_1$形成平行光,通过三棱镜$P$发生偏折,再会聚在凸透
镜$L_2$的焦平面$MN$上成实像,狭缝$S$的实像是一条亮线,颜
色和入射光相同。再讲复色光通过三棱镜的光路,使学生理
解不同颜色的光谱线实际上是照亮的狭缝$S$在焦平面$MN$上
形成的不同颜色的实像,这些实像按光的波长顺序排列形成
光谱。教学中还可以介绍标尺管的作用及其光路,使学生对分
光镜有比较全面的认识。

\subsubsection{光谱分析}

结合观察连续光谱、明线光谱(特别是
氢原子的光谱)和吸收光谱,使学生了解有关的几个概念,了
解各种光谱产生的机制,观察同一元素原子的发射光谱和吸
收光谱时,要注意观察吸收光谱的暗线位置和发射光谱的明
线位置一致,而不同元素原子有不同的光谱。从而理解光谱
代表了每种原子的特征,为介绍原子结构的内容作些准备。正
是每种原子都有自己的特征谱线,利用光谱可以鉴别物质和
确定物质的化学组成。

需要指出的是,由于中学配备的分光镜分辨本领较差,对
于太阳的吸收光谱的暗线,有的仪器不能观察到。

建议采用以学生自学为主的方式进行教学,指导学生
通过观察实验和阅读教材,理解光谱产生的机理和光谱的分
类,理解发射光谱、连续光谱、明线光谱和吸收光谱之间的关
系,认识光谱分析的原理及其方法。

\section{实验指导}
\subsection{演示实验}
本章的双缝干涉、衍射、偏振演示实验,要用J2508型光
的干涉衍射偏振演示仪。

J2508型光的干涉衍射偏振演示仪是由可转动的光具
座、滑块、观察筒、盒式光源、光学元件组成。

光具座上部是附有长80厘米标度的单导轨,导轨支撑在
光具座的底座上并可在水平面上任意转动。滑块分三种规格,
可套在光具座的单导轨上沿导轨滑动。滑块顶部的孔上可安
插各种光具。

观察筒由遮光筒、放大镜、玻璃屏和遮光板组成。

盒式光源由低压电源供电.盒内装有12V50W的卤钨
灯,盒前有聚光透镜。在出光口上装上单缝光栏,便可作线光
源使用。

主要光具有:狭缝、牛顿环、双面镜、反射器、毛玻璃屏、双
凸透镜等。其中,狭缝包括双缝(缝宽0.016毫米,缝距分别
为0.04毫米和0.08毫米两种)、单缝(缝宽0.08毫米),光栅
(1\%)。多缝缝宽0.02毫米,缝距0.08毫米。牛顿环装在圆
形胶木架中,胶木架上有三个调整螺钉。

光源、光具,观察筒均可安装在滑块顶部的小孔上,绕安
装轴转动。

图6.2的甲、乙、丙分别为光具座、观察筒和盒式光源的
结构示意图。

\begin{figure}[htp]
    \centering
    \includegraphics[scale=.6]{fig/6-2.png}
    \caption{}
\end{figure}

\subsubsection{双缝干涉}
用J2508型光的干涉衍射偏振演示仪做双缝干涉实验.
装置按图6.3配置.套在光源前的光源单缝缝宽为0.11毫
米,双缝的缝宽0.08毫米,装在光具架上,缝上的指示刻线
对齐光具架上的零刻线。

调整光源、单缝、双缝和观察筒的共轴是实验能否成功的
关键。

\begin{figure}[htp]
    \centering
    \includegraphics[scale=.6]{fig/6-3.png}
    \caption{}
\end{figure}

光具的调整步骤如下:先使单缝和双缝大致平行,相距约
5—10厘米(双缝离单缝近一些可以增加干涉亮条纹的亮度,
但光源单缝和双缝的共轴必须调得很好),观察筒的轴线和光
具座导轨平行,调整时可在观察筒前放一张白纸作为观察
屏,转动光源使得光通过单缝和双缝后落在白纸屏上的光斑
位于观察筒的中心处。再转动光源单缝,使它和双缝平行,在
白纸屏上见到清晰的干涉条纹,如果撤去白纸屏,便在毛玻
璃屏上呈现出清晰的干涉条纹。学生可以直接看毛玻璃屏上
的干涉条纹,也可以通过透镜观察干涉条纹的放大虚像。如
果在光源单缝和双缝之间加滤色片(也可用红色、绿色或紫
色的玻璃纸),可以看到单色光的明暗相间的干涉条纹。改变
观察筒与双缝的距离,可以看到干涉条纹的宽度随观察筒与
双缝间的距离增大而增大的情况。

由于光源用卤钨灯并有聚光透镜,从而使通过单缝和双
缝的光通量增加,提高了屏上亮条纹的亮度。为延长灯泡寿
命,开始用6伏电源点亮灯泡,然后再根据实际需要逐渐升高
电压,但不得超过12伏,接收干涉条纹的毛玻璃屏位于观察
筒内,遮挡了其他杂散光,提高了屏上干涉条纹的可见度,因
此可在一般亮度的教室中观察到清晰的干涉条纹。

实验时应缓慢转动光具座,使全班同学都能看到实验
现象。

\begin{figure}[htp]
    \centering
    \includegraphics[scale=.6]{fig/6-4.png}
    \caption{}
\end{figure}

由于激光的平行度好,单色性好,亮度高,是做光的干涉、
衍射实验的理想的单色光源,实验装置如图6.4所示,将激
光光束直接照射到双缝上,通过双缝后的光再投影到远处的
屏上,在屏上便可以见到明暗相间的单色光的干涉条纹,由于
激光光束很细,屏上的干涉条纹近于明暗相间的亮点,若采用
扩束装置将激光扩束后照到双缝上,可以得到双缝干涉的平
行条纹。但扩束后明条纹的亮度很低,实验需在暗室中进行。

\subsubsection{薄膜干涉}

可让学生分组做“肥皂液膜上光的干涉”实验.金属
丝圆环用黄铜丝或铁丝自制,肥皂液应清洁无杂物,浓度适
当。往酒精灯芯上撒食盐,火焰是黄色的。把肥皂液膜靠近酒
精灯,通过薄膜的反射去看黄色火焰,在薄膜上可见到明暗相
间的干涉条纹。若用白光照射,在薄膜上见到彩色的干涉
条纹。

对于一般的光源,能够见到干涉条纹时,薄膜的厚度须足
够薄。因此,肥皂液的浓度不能太浓。干涉条纹一般先出现在薄
膜的上部,往往当薄膜将要破裂时,才能见到较多的干涉条纹。

仪器所附的牛顿环,是由一块圆的平板玻璃和一个
凸透镜叠合而成的。球面与平板玻璃间形成空气膜,其厚度
由接触点向外逐渐增大。调节框上的三个调整螺钉,使干涉
图样位于中心部分。但不要拧得过紧,以免玻璃破碎。由牛
顿环的空气膜产生的干涉条纹是等厚条纹,利用光的干涉、
衍射、偏振演示仪做牛顿环投影实验装置如图6.5。将牛
顿环置于导轨的一端,把光源、凸透镜、毛玻璃屏按图中所示
位置放置,使凸透镜($f=7$厘米)距牛顿环大约8厘米。光源
的光斜射到牛顿环上,使反射光过凸透镜在毛玻璃屏上成
像,稍稍调整牛顿环的位置,在毛玻璃屏上可见到清晰的圆
环形彩色干涉条纹。条纹越向外越密,条纹的中心是暗
斑。

\begin{figure}[htp]
    \centering
    \includegraphics[scale=.6]{fig/6-5.png}
    \caption{}
\end{figure}

若将凸透镜放在牛顿环的另一侧,并移动它的位置,也可
在光屏(如白墙)上见到牛顿环的干涉条纹,这是光透过牛顿
环后经凸透镜成的像,牛顿环的透射干涉条纹的中心是亮
斑,和反射干涉条纹是互补的。拧动牛顿环边缘上的调整螺
钉,干涉条纹的形状、位置和圆环半径的大小都会发生变化。

本实验用低压电源供电,电压为6—10伏.

\subsubsection{单缝衍射}
用光的干涉、衍射、偏振演示仪(J2508型)做单缝衍
射实验.装置与图6.3相似,需将图中的双缝换成宽度为
0.08毫米的衍射单缝,如图6.6所示,调整光源单缝,使衍射
单缝和观察筒共轴。调整方法和双缝干涉实验相同。
\begin{figure}[htp]
    \centering
    \includegraphics[scale=.6]{fig/6-6.png}
    \caption{}
\end{figure}

用游标卡尺的外测脚做宽度可调的单缝代替衍射单缝,
可以演示衍射条纹随单缝宽度变化的情况。缝宽较大时,如
缝宽为2毫米,屏上为边缘清晰的亮线.当缝宽减小时,屏上
的亮线的宽度也减小。如果进一步减小单缝的宽度,可以见
到亮线的两侧出现衍射条纹。缝宽再减小,衍射条纹的宽度
反而变大,只是明条纹的亮度降低。实验应该注意:
\begin{enumerate}
\item 缝宽较
大时,屏上的亮线中似有明暗的条纹,这不是衍射条纹,而是
仪器的光源卤钨灯灯丝的像的一部分(灯丝像的其余部分
被可调狭缝屏遮挡住了)。
\item 用卡尺的测脚做狭缝,缝宽较大
时,屏上有时也出现明暗相间的条纹,这是从卡尺反射的光和
从光源单缝直接射出的光相干涉的条纹。解决的办法是稍稍
转动卡尺,使测脚平面反射的光不能达到观察筒内的毛玻璃
屏上。
\end{enumerate}


利用这套实验装置还可以做不透明的单丝衍射实验,取
直径在0.1毫米左右的细丝(如头发丝、多股绞合电线中的一
股等),用胶水竖直地粘在光具架上,替换衍射单缝。光源单缝
换用宽0.025毫米的,转动单缝的位置,使它和不透明的单丝
平行,在屏上可以看到与单丝平行的衍射条纹,特别是在单丝
的“影子”中央有一条亮线。

用激光光源做单缝衍射实验,激光光束直接照到狭
缝上,通过狭缝后再投影到远处光屏上,可以看到明暗相间的
衍射条纹,改变狭缝的宽度,可以看到狭缝越窄,衍射条纹越
远,但亮度减弱。

用激光光源还可以做圆孔衍射实验,用细针在牙膏皮上
扎出不同孔径的圆孔,激光光束照射到小孔上,通过小孔后直
接投射到远处的光屏上,可以看中心处是最亮的圆形光斑,
周围还有明暗相间的圆环形条纹,孔径越小,中心亮班及周围
的圆环形条纹的面积越大。

\subsubsection{衍射光栅}
\begin{figure}[htp]
    \centering
    \includegraphics[scale=.6]{fig/6-7.png}
    \caption{}
\end{figure}

用光的干涉、衍射、偏振演示器(J2508型)做光栅衍射的
实验,装置如图6.7。实验时将光源、光源单缝、凸透镜和毛
玻璃屏共轴放置在光具座上。调整凸透镜($f=7$厘米)的位
置(距光源单缝稍大于7厘米),使得光源单缝在毛玻璃屏中
央成清晰的放大实像,再将衍射光栅装在光具架上,缝座的指
示刻线对齐光具架的零刻线,把光具架插到凸透镜和光屏之
间,并靠近凸透镜。调整光源单缝与光栅的刻线平行,在毛玻
璃屏上就可以见到光栅的衍射条纹。

为了说明衍射明条纹随着缝数增加而变窄且亮度增强的
特点,实验时,可在光具架上顺序装置单缝、双缝、多缝和光
栅,比较它们的衍射条纹,上述特点很容易由实验得到。单缝、
双缝和多缝的衍射亮条纹的亮度较低,可用观察筒代替毛玻
璃屏,便于同学观察。

实验用6—10伏交流电.

\subsubsection{光的偏振}

演示自然光通过偏振片产生偏振光可用光的干涉、
衍射、偏振演示仪。将光源、两个偏振片(分别装在两只光具
架上)和毛玻璃屏依次装在光具座上,使它们的中心在平行于
光具座导轨的同一条直线上。当两个偏振片的偏振化方向
(用偏振片座上的指针表示)平行时,毛玻璃屏上有明亮的光
斑;固定一个偏振片(为方便演示,偏振化方向可取竖直方
向),转动另一个偏振片,当两个偏振片的偏振化方向垂直时,
屏上的光斑几乎消失。继续转动偏振片,屏上光斑的亮度出
现周期性变化。

\begin{figure}[htp]
    \centering
    \includegraphics[scale=.6]{fig/6-8.png}
    \caption{}
\end{figure}

反射光的偏振实验装置如图6.8所示.玻片反射起
偏器置于光具架上,使其框上的刻线对准入射角为$57^{\circ}$的定
位点,先将光源置于起偏器左侧,转动光源,使出射光束经玻
片反射后,光斑落在毛玻璃屏的中央。再把偏振片装入光具
架内。转动偏振片,当指示偏振化方向的指针处于竖直方向
时,屏上有明亮的光斑。指针处于水平方向时,屏上的光斑几
乎消失。

上述两个实验用6—10伏低压电源.

偏振片可以用观看立体电影偏光眼镜片代替,左
右两只眼镜片的偏振化方向互相垂直。光源可选取平行光源
(幻灯、手电筒、激光光源等),通过两个偏振片将光斑投影在
墙上,转动其中任一偏振片,很容易观察到光的偏振现象。

反射光偏振起偏器,可以用一般的平板玻璃,将其一面
涂黑,用以吸收透过玻璃的光,当光线的入射角为布儒斯特
角时,从玻璃表面反射的光是平面偏振光,其振动方向和光的
入射平面垂直。再让反射光通过偏振片,转动偏振片的方位,
可看到屏上光斑的亮度周期性的变化,利用反射偏振光可以
检查偏振片的偏振化方向。

\subsubsection{伦琴射线的获得}
实验室中的伦琴射线管由阴极、对阴极(阳极)和辅助阳
极组成,管内还有少量活性炭,实验装置如图6.9所示,实验
时,阴极接感应圈高压输出端的负极,阳极接感应圈高压输出
端的正极。接通电源后,由阴极射出的阴极射线打到对阴极
上,在对阴极激发出伦琴射线。由于伦琴射线照到荧光屏上,
能激发出荧光。所以把手(或一本书,书内夹着钥匙)紧贴在
荧光屏前,当射线照射时,在屏上可以看到手的骨骼(或书中
的钥匙)产生的阴影。
\begin{figure}[htp]
    \centering
\includegraphics[scale=.6]{fig/6-9.png}
    \caption{伦琴射线管实验装置}
\end{figure}

实验要在暗室中进行,正常工作时,一部分伦琴射线直
达管壁,在管壁上产生荧光(玻璃材料不同,荧光的颜色也不
同)。实验时要注意,伦琴射线管的正、负极接线必须正确。如
果接反,尽管能有阴极射线产生,但不能产生伦琴射线,可通
过转换感应圈的极性开关,改变接线极性,实验时,要把荧
光屏放在对阴极的前方,屏和管壁的距离以能在屏上得到清
晰的透视图象为准。涂有荧光物质(硫化锌)的面要向着观察
者。

荧光物质——硫化锌有毒,使用时应避免接触,伦琴射
线管是高真空仪器,长期放置时,会由于漏气,造成伦琴管
损坏。

\subsubsection{用分光镜观察发射光谱和吸收光谱}
中学物理中主要用棱镜分光镜观察光谱。下面先介绍分
光镜的构造。

\begin{figure}[htp]
    \centering
\includegraphics[scale=.6]{fig/6-10.png}
    \caption{}
\end{figure}

分光镜有四个主要部件,图6.10是它的示意图.图中1
是平行光管,它的前端装有可调节的狭缝$S_1$, 另一端装有消
色差透镜$L_1$; 2是三棱镜,一般用火石玻璃制成,放在平台的
中央,三棱镜上不需要透光的都是毛面;3是望远镜,由物镜
$L_3$和透镜组$L_4$组成;4是标度管,它的前端装有透明的标尺
$S_2$, 另一端装有透镜$L_2$. 这四个部件都固定在金属圆盘上,
并位于同一水平面内。圆盘上部有一胶木罩,以遮挡杂散光
线的干扰,因此观察时不需要良好的暗室.1、3、4三部分可
拉出或推进,以便进行调节,望远镜筒3还可沿圆盘转动,标
度管4可以沿圆盘下面的固定螺丝转动.调节标度时,必须
先把固定螺丝拧松,调好后再固定。

用分光镜观察发射光谱时,应先使狭缝$S_1$位于透镜$L_1$
的焦平面上,以便使光源射入狭缝的光线经过透镜$L_1$后变成
平行光束,然后投射到三棱镜上。由于三棱镜的色散作用,光
束通过三棱镜后,便成了许多方向不同的平行光束,它们在进
入望远镜的物镜$L_3$后,分别在平面MN处聚集,形成一条条
狭缝的像(图6.11),观察时,用目镜$L_4$加以放大,就可以看
到放大的光谱像.再用小电珠照亮标度管的标尺$S_2$, 经棱镜
反射后,在$MN$处同时呈现出标尺的像(图6.12)。

\begin{figure}[htp]\centering
    \begin{minipage}[t]{0.48\textwidth}
    \centering
\includegraphics[scale=.6]{fig/6-11.png}
    \caption{}
    \end{minipage}
    \begin{minipage}[t]{0.48\textwidth}
    \centering
\includegraphics[scale=.6]{fig/6-12.png}
    \caption{}
    \end{minipage}
    \end{figure}

分光镜可按下列步骤进行调整。
\begin{enumerate}
\item 取下三棱镜,将望远镜筒对准窗外远处的物体,移动
目镜(从镜筒中拉出或推进),直到从目镜中看清楚很远处的
物体。再转动望远镜筒,使它的轴线和平行光管的轴线重合。
用小灯泡照亮狭缝,调整缝宽。开始调整时,狭缝可适当取宽
一些。将狭缝转到竖直位置并移动狭缝(从镜筒中拉出或推
进),直到目镜中呈现出竖直狭缝的清晰的像为止。
\item 把三棱镜装在镜台上,使两个光学工作面向着镜管,
毛面向着旋钮架。固定好三棱镜,并盖遮光罩。
\item 用小灯泡照亮狭缝,转动望远镜筒的位置,使目镜中
出现一条水平的光带横贯目镜视野的中部,由于三棱镜大都
工作在最小偏向角附近,因此,能够观察到光谱时,望远镜筒
的轴线与三棱镜出射工作面的夹角和平行光管的轴线与三棱
镜入射工作面的夹角大致相等。
\item 目镜中观察到的光谱,实质上是成在望远镜物镜焦
平面上的狭缝的不同色光的实像的集合.因此,调整步骤1
和3可以合并进行.转动望远镜筒的位置,在目镜中见到水
平的彩色光带,再略微移动目镜和移动狭缝的位置,使彩色
光带的边缘清晰即可。
\item 用小灯泡照亮标尺,转动标尺管,并稍稍移动标尺的
位置,使在目镜中同时看到光谱和标尺的清晰的像。标尺的
像应和彩色光带平行,而且边缘重合,分光镜的三个镜管架
的接口上均有六个调整紧固螺钉,可用来调整目镜视野中光
谱和标尺像的上下位置。
\item 调整狭缝的宽度使狭缝变窄,分光镜的分辨本领提
高。这时,线状光谱变窄,带状光谱的各个颜色能够分辨开。
但是进入平行光管的光能减少,光谱的亮度减小,也可稍加
调整平行光管中狭缝和凸透镜的距离,当距离增大时(目镜的
位置也要随之调整,像才能清晰),带状光谱的宽度(或光谱线
的长度)减小。
\end{enumerate}

观察连续光谱,光源用白炽灯(应选用色温较高的灯泡,
如汽车灯泡)。灯直接照亮狭缝,在目镜中就可以看到连续光
谱。稍稍转动望远镜筒的位置,可以看到自左向右的红、橙、
黄、绿、蓝、靛、紫等各种颜色的连续光谱。

观察明线光谱,用光谱管作光源,把光谱管接到感应圈
(或静电高压电源)的高压输出端,在高电压作用下,光谱管
内低压气体电离产生辉光放电,把分光镜的狭缝对准光谱管
的狭窄管道部分(图6.13)。实验要在暗室中进行,在目镜中
可以看到暗淡的背景上出现几条不连续的明线光谱,光谱管
每组有六只,分别充以氢、氮、氧、二氧化碳、氖和氩等气体。演
示时,应使学生重点看氢原子光谱。一般能看到可见光范围
内氢的强度较强的三条谱线,$H_{\alpha}:\; 0.6562$微米;$H_{\beta}:\; 0.4861$微
米;$H_{\gamma}:\; 0.4340$微米。实验时应尽量缩短通电时间,以延长光
谱管的寿命。
\begin{figure}[htp]
    \centering
\includegraphics[scale=.6]{fig/6-13.png}
    \caption{}
\end{figure}

观察钠的吸收光谱的装置如图6.14, 光源用汽车灯泡。
把溶有NaI的酒精倒在狭长的金属槽中(也可用固体热膨胀
演示器的酒精槽)。狭长槽的走向和光源到狭缝的连线方向
一致,使光源发出的光在酒精火焰中穿过尽可能长的距离后
再进入狭缝。实验时,先接通电源,从分光镜中观察到清晰的
连续光谱。然后点燃酒精,在连续光谱的背景上就出现两条
暗线($\lambda=0.5890$微米和0.5896微米)。如果分光镜的分辨
本领较低,这两条暗线重合在一起。用硬纸板遮挡灯泡发出
的光,在目镜中可见到钠原子光谱的明线;移去纸板后,又能
看到光谱的暗线,光谱暗线的位置和亮线的位置是一致的。
\begin{figure}[htp]
    \centering
\includegraphics[scale=.6]{fig/6-14.png}
    \caption{}
\end{figure}

实验时,分光镜的狭缝应尽量调得窄些,只有在目镜中看
到钠明线光谱的亮线很细窄时,才能见到吸收光谱的暗线。有
的分光镜狭缝质量较差,见到吸收光谱的竖直暗线时,还能见
到一些水平的横线,这是由于狭缝的边缘不平整造成的。为
了使光源的色温高一些,连续光谱的背景亮一些,汽车灯泡的
供电电压应达到或稍高于它的额定电压。此外,狭缝宽度也
要取窄一些。这些是实验成功的关键,此外,还可以在长酒
精槽上加拱形的薄铁板盖,使得酒精火焰及其中的钠离子蒸
汽局限在长拱形的盖内。这样可防止因空气流动造成火焰的
摆动,光穿过拱形通道时,也会被钠离子吸收得更多些,酒精
燃烧完后,金属槽中会出现一些没有完全挥发的NaI结晶,它
还可以继续溶于酒精中,重复使用。

\subsection{学生实验}
\subsubsection{利用双缝干涉测定光的波长}
实验仪器有J2515型双缝干涉仪、光具座和学生电源等.

实验装置涉及许多元件,为了使学生能够在短时间内把
它们正确地组装调整好,需要指导学生做好实验预习,组织
学生识别各个元件,注意组装顺序和要领,正确地进行调整,
实验完毕要整理仪器,把各个元放回盒中。

仪器的组装和调整要特别注意:
\begin{enumerate}
\item 双缝装在遮光管的一
端,单缝管套在双缝座外面,都要注意定位槽孔,以保证单缝
和双缝大致平行。    
\item 把遮光管架在半圆形支架上,使它的轴
线与光具座导轨平行,单缝与导轨垂直,直丝灯泡灯丝的中心
在遮光管的轴线上,灯丝与单缝平行,并且靠近单缝。这时,
眼睛从遮光管的另一端看,可以看到通过单缝和双缝后进入
遮光管内的光最强。    
\item 再装上测量头,在分划板上可以看到
双缝干涉的条纹。若条纹不清晰,还可左右稍稍拨动拨杆,使
单缝和双缝平行。转动测量头的方位,使分划板的刻线与干
涉条纹的亮线平行。一般说来,照在分划板上的光很弱的主
要原因是灯丝与单缝和双缝、测量头或观察筒不共轴;干涉条
纹不清晰的主要原因是单缝和双缝不平行。   
\item 双缝、测量头
和接长管安装到遮光管上,都要使各部件的定位面紧密吻合,
否则$\ell$值变大,影响实验结果。
\end{enumerate}

由于干涉亮条纹有一定的宽度,而且边缘明暗界线模糊,
测量时要使干涉亮条纹嵌在分划板的两条竖直刻线中央,亮
条纹较宽时可用图6.15甲的方式,亮条纹较窄时可用图6.15
乙的方式。
\begin{figure}[htp]
    \centering
\includegraphics[scale=.6]{fig/6-15.png}
    \caption{}
\end{figure}

实验参考数据见下表。
\begin{center}
    \begin{tabular}{c|c|c|c|c|c|c}
\hline
& \multicolumn{2}{c}{仪器参数} &\multicolumn{2}{|c|}{游标尺读数
} & \multirow{2}*{$\Delta X=\dfrac{X_7-X_1}{6}$} & \multirow{2}*{$\lambda=\dfrac{d}{\ell}\cdot \Delta X$}\\
\cline{2-5}
& $d$(mm) & $\ell$(mm) & $X_1$(mm) & $X_7$(mm) &\\ 
\hline
\multirow{4}*{绿光}&\multirow{2}*{0.25} &600&3.52&11.56&1.340&$5.58\x10^{-4}$\\
&&700&2.44&11.78&1.557&$5.56\x10^{-4}$\\
&\multirow{2}*{0.20} &600&0.64&10.50&1.643&$5.48\x10^{-4}$\\
&&700&1.28&12.86&1.930&$5.51\x10^{-4}$\\
\hline\multirow{4}*{红光}&\multirow{2}*{0.25} &600&0.70&10.04&1.557&$6.49\x10^{-4}$\\
&&700&1.10&12.20&1.850&$6.61\x10^{-4}$\\
&\multirow{2}*{0.20} &600&1.64&13.36&1.953&$6.51\x10^{-4}$\\
&&700&0.52&14.42&2.283&$6.52\x10^{-4}$\\
\hline
    \end{tabular}
\end{center}

实验测量误差主要来自分划板的刻线对不准干涉条纹的
中心,如按图6.15甲或乙所示的方法进行量,干涉条纹中
心的读数偏差不超过0.1毫米,根据误差理论,波长$\lambda$的最大
相对误差为:
\[\rho_{\lambda}=\left|\frac{\Delta d}{d}\right|+\left|\frac{\Delta \ell}{\ell}\right|+\frac{|\Delta X_n|+|\Delta X_1|}{|X_n-X_1|}\]
其中前两项双缝间的距离$d$和双缝到屏的距离$\ell$的相对误差
由仪器的技术指标可知,
\[\left|\frac{\Delta d}{d}\right|+\left|\frac{\Delta \ell}{\ell}\right|\le 2\% \]
选择$n\ge 7$时,$X_n-X_1$可达10毫米,第三项相对误差
$\dfrac{|\Delta X_n|+|\Delta X_1|}{|X_n-X_1|}$
也不超
过2\%, 所以$\rho_{\lambda}\le 4\%$. 滤色片的峰值波长$\lambda_{\text{红}}=0.6600$微米,
$\lambda_{\text{绿}}=0.5350$微米,按偏差$\pm 0.0100$微米计算,因此测量值的
范围是:
\begin{itemize}
    \item 红光:$0.6600\pm (0.6600\x4\%+0.0100)$微米
    
即:$0.6240$—$0.6960$微米,
\item 绿光:$0.5350\pm (0.5350\x4\%+0.0100)$微米

即:$0.5040$—$0.5660$微米.
\end{itemize}


\subsubsection{观察光的衍射现象}
实验可用直长灯丝的白炽灯作线状光源,使灯丝处于竖
直位置,灯的背景应暗淡,避免产生强烈的反射光。也可用日
光灯,在日光灯管外面包上黑纸,留出一条直缝透出白光作
线状光源。点光源可用手电筒的小灯泡,背景也应暗淡,避免
产生强烈的反射光。

利用卡尺做宽度可调的单缝,要注意:
\begin{enumerate}
\item 卡尺的测脚必
须擦拭干净。
\item 由卡尺测脚形成的单缝要和光源的线状灯丝
平行,并且要和线光源对正,否则,由于测脚内侧平面对光的
反射,还可以看到线状光源的虚像,出现干涉条纹。如果出现
上述现象,应转动卡尺的测脚平面,直到现象消失为止。这时,
眼睛通过测脚的狭缝观察到的只是直接由线光源发出的光。
\end{enumerate}

小孔可让学生自制,在不透光的黑纸(如包胶卷、相纸等
感光材料的黑纸)或在牙膏皮上用针扎成不同直径的圆孔。要
注意圆孔的边缘应光滑平整。

比较缝宽和孔径大小对光的衍射的影响时,应注意保持
光源以及光源和单缝(或圆孔)的距离不变。一般来说,光源
的线度越小,光源距单缝或小孔越远,能够观察到比较明显的
衍射现象的单缝的宽度(或圆孔的直径)可以越大。

\subsection{课外实验活动}
\subsection*{补充课外实验活动}

\subsubsection{观察两块平玻璃板形成的空气膜的干涉条纹(课本练习一第5题)}

取两块平玻璃板(如显微镜的载物玻璃片),把它们的表
面擦拭干净,对齐并用力压紧,在白光照射下可以看到彩色干
涉条纹。做好实验的关键:
\begin{enumerate}
\item 压紧平玻璃板。只有当平玻璃
板间的空气膜的厚度足够小时,才会出现干涉条纹。   
\item 光的
入射角要小,如果入射角太大,从薄膜的上下表面反射的光
的强度相差较大,不容易观察到干涉条纹。
\end{enumerate}

\subsubsection{观察单缝衍射(课本256页练习二第3题)}
用远处的日光灯做线光源,应选用直径较小的日光灯管
(如8W或6W日光灯).对于较粗的灯管,如果观察不到衍
射条纹,可以增加日光灯和观察者之间的距离,或者在日光灯
管外面包黑纸,使光源只留一条直缝透光。

\subsubsection{自制单缝或双缝,观察衍射现象或干涉现象}
取一块平板玻璃,将它的一面用烟熏黑(也可用墨涂黑)。
用一片刮脸刀片靠着直尺在涂黑面上划,便可划出一道很细
的透明直缝,即制成了单缝。如果用两片刮脸刀片紧紧地靠
在一起,靠着直尺在涂黑面上划,可划出两道紧紧靠在一起的
透明双缝。

用制得的单缝和双缝紧靠着眼睛,看线状光源(如远处
的日光灯管),使单缝和双缝和光源平行,可以看到单缝衍射
条纹和双缝干涉条纹。

\subsubsection{用手持分光镜观察汞光谱和太阳的吸收光谱}

\begin{figure}[htp]
    \centering
     \includegraphics[scale=.8]{fig/6-16.png}
    \caption{}
\end{figure}

手持分光镜如图6.16甲所示,由于镜筒内装有分光元件
1, 可使复色光分解成单色光.分光元件(图6.16乙)是用两
块冕牌玻璃制成的三棱镜C和一块火石玻璃制成的三棱镜
F胶合而成。分光元件的作用是使光谱中部的黄光保持原来
的传播方向不偏转,筒内还有一个凸透镜2, 它和筒外的紧
固螺钉3相连.凸透镜的位置可以前后移动.光从狭缝4射
入,眼睛靠近观察孔5进行观察.进入狭缝的光通过凸透镜
后进入分光元件,经过色散,再通过观察孔进人眼睛,在视网
膜上会聚得到不同色的狭缝像,形成光谱。观察时,狭缝
的宽度要调整适当,并要转动狭缝,使它和分光元件的边棱平
行。从观察孔中可以看到矩形的彩色光带,再稍稍移动凸透
镜的位置,可使看到的彩色光带边缘更加清晰。

将手持分光镜对着日光灯管,可从观察孔中看到浅淡的
连续光谱背景上有几条亮线。这些亮线是日光灯中低压汞蒸
汽产生的。和书中插页上的彩图相比较,可知是汞光谱中强
度较大的三条亮线,波长分别为0.5791微米,0.5461微米和
0.4358微米。

将手持分光镜直接对着太阳,如果狭缝的宽度调得比较
窄,则可以看到在连续光谱的背景上有一些暗线,这是太阳的
吸收光谱。(在太阳的吸收光谱上还可以看到一些横的暗线,
这是由于分光镜的窄缝不平整造成的)


\section{习题解答}
\subsection{练习一}

\begin{enumerate}
\item 绿光的干涉条纹与红光的干涉条纹有什么不同?用
白光做干涉实验,为什么会得到彩色的干涉条纹?在彩色条纹
中最靠近中央亮纹的是哪种颜色的条纹?为什么?


\begin{solution}
干涉条纹的相邻亮纹间的距离
\[\Delta x=\frac{\ell}{d}\cdot \lambda\]
在$d$和$\ell$
相同的情况下,$\Delta x\propto \lambda$, 由于绿光的波长较红光短,因此,绿光
的干涉条纹间的距离$\Delta x$较窄,红光的较宽。

用白光做实验,由于白光包含着各种不同波长的单色光,
不同色光的干涉条纹间的距离$\Delta x$不同,因此,除了中央亮纹
是白色的,两边出现彩色条纹。第一条亮纹的位置
\[x=\pm\frac{\ell}{d}\cdot \lambda\]
由于紫光的波长最短,紫光的亮纹最靠近中央亮纹。
\end{solution}

\item 在杨氏双缝实验中,保持双缝到屏的距离不变,调节
双缝间的距离,当距离增大时,干涉条纹间的距离将变\underline{\qquad};
当距离减小时,干涉条纹间的距离将变\underline{\qquad}.


\begin{solution}
    变窄;变宽。
\end{solution}

\item 色光从真空进入媒质后,频率不变,但传播速度减小
了,波长将如何变化?


\begin{solution}
    波的公式$v=\lambda\nu$, 频率不变时,波长与波速成正
    比。色光由真空进入媒质后,传播速度减小,波长变短。
\end{solution}

\item 用单色光做双缝干涉实验,测得双缝间的距离为0.4
毫米,双缝到屏的距离为1米,干涉条纹的间距为1.5毫米,
求所用光波的波长.


\begin{solution}
    光波的波长
\[\lambda=\frac{d}{\ell}\cdot \Delta x=\frac{0.4\x 10^{-3}\x 1.5\x 10^{-3}}{1}=6.0\x 10^{-7}{\rm m}\]
\end{solution}

\item 取两块平玻璃板,用手指把它们紧紧捏在一起,会从
玻璃板面上看到许多彩色花纹;改变手指用力的大小,花纹的
颜色和形状也随着改变,做这个实验,并解释看到的现象.


\begin{solution}
把两块平玻璃板紧紧地捏在一起,两板之间形成空
气隙。光照到玻璃板上,在空气隙的上表面和下表面反射的
光形成两列相干波,发生干涉。亮条纹和暗条纹的位置决定
于空气隙各处的厚度和光的波长。白光照射时,产生彩色的干
涉条纹。改变手指用力的大小,空气隙各处的厚度发生变化,
干涉条纹的颜色和形状也就随着改变。
\end{solution}

\end{enumerate}


\subsection{练习二}
\begin{enumerate}
\item 为什么隔着墙能听到墙那边人的说话声,但看不见
人?

\begin{solution}
    只有障碍物的尺寸跟波长相差不多时,才能明显地
    观察到衍射现象。一般人发出的声音的波长约为几米到十分
    之几米,可见光的波长约十分之几微米。把墙做为障碍物,声波能够发生比较明显的衍射现象,隔着墙能够听到墙那边人
    的说话声;光波不能发生明显的衍射现象,隔着墙不能见到墙
    那边的人。
\end{solution}
\item 光的衍射现象跟光的直线传播是否矛盾?在什么情
况下光沿直线传播?

\begin{solution}
    光的衍射现象是光的波动性的表现。衍射现象是否
    明显,取决于障碍物或孔的尺寸大小,当障碍物或孔的尺寸
    与光波长相比相差不多时,光会偏离直线传播绕到障碍物的
    影子里,形成明暗相间的衍射条纹,当障碍物或孔的尺寸远
    远大于光的波长时,光的衍射现象很不明显,可以近似认为光
    是沿着直线传播的。

    可见光的波长仅十分之几微米,比一般障碍物或孔的尺
    寸小得多,在这种情况下,可以忽略衍射现象,认为光沿着直
    线传播。
\end{solution}
\item 把两支铅笔并在一起,中间留一条狭缝,放在眼前,
进过这条狭缝去看远处的日光灯,使狭缝的方向跟灯管平行,
就看到许多条平行的彩色条纹,做这个实验,并解释看到的
现象.

\begin{solution}
    两支铅笔并在一起,形成一条狭缝。远处的日光灯
    相当于线光源,它发出的光通过与它平行的狭缝发生衍射,再
    经过眼球的会聚作用,在视网膜上形成明暗相间的衍射条纹。
    由于日光灯发出白光,它产生的衍射条纹中央是白色的亮纹,
    两侧是平行的彩色条纹。
\end{solution}
\end{enumerate}



\subsection{习题}

\begin{enumerate}
    \item 你用哪些现象或实验来说明:
    \begin{enumerate}
        \item 光是一种波;
        \item 光波的波长非常短;
        \item 绿光的波长比红光的短.
    \end{enumerate}


    \begin{solution}
\begin{enumerate}
    \item 利用光的干涉现象和衍射现象说明光是一种波。
    \item 利用单缝衍射和小孔衍射实验。只有当孔或缝宽很
    小时,才有比较明显的衍射现象,说明光波的波长非常短。
    \item 在保持缝距和缝与屏间的距离不变的条件下,分别用
    绿光和红光做双缝干涉实验,在屏上观察到绿光的干涉条纹
    的宽度比红光的窄,说明绿光的波长比红光的短。
\end{enumerate} 
    \end{solution}
    
    \item 水对真空中波长为0.656微米的红光的折射率为$n_1
    =1.33$,而对真空中波长为0.405微米的紫光的折射率为
    $n_2=1.343$.求这两种波在水中的传播速度和波长.


    \begin{solution}
红光在水中的传播速度
\[v_1=\frac{c}{n_1}=\frac{3.00\x10^8}{1.33}=2.26\x10^8\ms\]
紫光在水中的传播速度
\[v_2=\frac{c}{n_2}=\frac{3.00\x10^8}{1.343}=2.23\x10^8\ms\]
又由于
\[n=\frac{c}{v}=\frac{\lambda_0 \nu}{\lambda\nu}=\frac{\lambda_0}{\lambda}\]
所以:$\lambda=\lambda_0/n$,由此得:
\begin{itemize}
    \item 红光在水中的波长
\[\lambda_1=\frac{\lambda_{01}}{n_1}=\frac{0.656}{1.33}=0.493\mu{\rm m}\]
    \item     紫光在水中的波长
\[\lambda_2=\frac{\lambda_{02}}{n_2}=\frac{0.405}{1.343}=0.302\mu{\rm m}\]
\end{itemize}

    \end{solution}
    
    \item 波长为5890埃的黄光照在一双缝上,在距双缝为1
    米的观察屏上,测得20个亮条纹的间距共宽2.4厘米,求双
    缝间的距离.


    \begin{solution}
相邻两条亮纹间的距离
\[\Delta x=\frac{a}{n-1}=\frac{2.4\x 10^{-2}}{20-1}=1.3\x 10^{-3}{\rm m}\]
双缝间的距离
\[d=\frac{\ell \lambda}{\Delta x}=\frac{(n-1)\ell\lambda}{a}=\frac{19\x 1\x 5890\x 10^{-10}}{2.4\x 10^{-2}}=4.7\x 10^{-4}{\rm m}\]
    \end{solution}
    
    \item 用波长为0.75微米的红光做双缝干涉实验,双缝间
    的距离是0.05毫米,缝到屏的距离是1米,相邻两条暗纹间
    的距离是多大?


    \begin{solution}
相邻两条暗纹间的距离
\[\Delta x=\frac{\ell }{d}\cdot\lambda=\frac{1\x 0.75\x 10^{-6}}{0.05\x 10^{-3}}=1.5\x 10^{-2}{\rm m}\]
    \end{solution}
    
\end{enumerate}

\section{参考资料}
\subsection{光源的相干性}
相干波源的必要条件是频率相同,振动方向相同,而且相
差恒定,由于机械振动的频率很低,在观察时间内,通常的机
械波是连续不断的,因而只要两个波源的频率相同,在观察时
间内的任何时刻,这两个波源的相差总是恒定的。对于光波
则不然,可见光频率的数量级都在$10^{14}$赫左右,光源的辐射
是由原子的外部电子跃迁产生的,一批原子的跃迁发生的时
间大约只有$10^{-8}$秒左右.而一个光源,由多个彼此独立的发
光原子组成,在观察时间内,光源发出的光是由多批原子跃迁
产生的,因而光辐射的初位格是没有什么关联的。因此,不同
的光源,或者同一光源的不同部分的光辐射,一般不可能做到
相差恒定。也就是说,一般光源的相干性是很差的,在通常的
条件下很难观察到光的干涉现象。为了观察到稳定的干涉现
象,必须在观察时间内,使到达观察点的两列光波的相差总是
恒定的,如果使两列光波都来自于同一批原子的辐射,经过
不同的光程到达观察点后,便可产生干涉现象。尽管这些原子
的发光是无规则地迅速地改变着的,但是任何位相的改变总
是同时发生在这两列波中,因而它们到达观察点时总保持着
恒定不变的相差。获得相干光源的方法有两类,一类叫分波阵
面法;一类叫分振幅法。杨氏双缝干涉实验是采用分波阵面
法获取相干光源的单缝$S$起着线光源的作用,双缝$S_1$和$S_2$
处在$S$的同一波阵面上。从同一波阵面上分出的两列光波
$S_1$和$S_2$可以看做位相总是相同的两个相干光源。薄膜干涉
是采用分振幅法获得相干光源的。一束光投射到薄膜表面上,
入射光的能量之中一部分在薄膜的前表面直接反射形成一列
反射光波,其余的能量进入薄膜后,再由薄膜的后表面反射
出一部分能量,形成另一列反射光波,当入射角较小时,分别
从前后表面反射的两列光波的能量是相近的,它们都来自于
同一入射光束,因此也是相干的两列光波。经多次反射得到
的光波的能量相差很多,迭加时产生的干涉条纹很不明显,--
般都不予考虑。

\begin{figure}[htp]
    \centering
     \includegraphics[scale=.7]{fig/6-17.png}
    \caption{}
\end{figure}

双缝干涉实验中光源单缝$S$的宽度必须适当。光源单缝
$S$越宽,通过单缝$S$和双缝$S_1$、$S_2$的光能量越多,因此,屏
上干涉条纹的亮度越大,这是显然的。光源单缝$S$的宽度还
直接影响了屏上干涉条纹的清晰程度。如图6.17, 光源单缝
$S$中的每一点都可以视为子波波源发出子波。由中点$S_0$发
出的光波经$S_1$和$S_2$成为两列光波,到达屏上的$O$点的光程
差恒为0, $O$点是由$S_0$发出的光形成的中央亮纹,其余的亮
纹分布在$O$点两侧.对于单缝中的共余点,如$S_0'$点,所发出
的光波,经过$S_1$和$S_2$形成两列光波到达屏上光程差为0的
中央亮纹的位置显然不在$O$点,而在$O$点以下的某一位置$O'$
点.这样,由单缝$S$中各个点发出的光经过双缝$S_1$和$S_2$到
达光屏上形成的干涉亮条纹的位置彼此是错开的,屏上观察
到的干涉条纹是这些干涉条纹的叠加的结果(由单缝$S$中
各点发出的光波是不相干的)。如果单缝$S$比较宽,单缝上各
点形成的干涉亮条纹的位置错开较多,甚至于有的点的亮条
纹的位置落到另一些点的暗条纹处,屏上的条纹的明暗对比
度大大降低,干涉条纹变得模糊不清,甚至看不到明暗相间的
干涉条纹。

使用一般的光源做薄膜干涉实验,只有当薄膜的厚度很
小时,才能观察到干涉条纹。这是由于同一批原子辐射发光
的时间是很短的,如果薄膜的厚度较大时,由后表面反射的光
波到达观察点的时间要比前表面反射的光波推迟一段较长的
时间。这样,到达观察点的两列光波只在很短时间内才是由
同一批原子辐射的,才是相干的,能够出现干涉条纹;在其余
的时间内这两列波不是来自同一批原子辐射的,是不相干的,
也就不能出现干涉条纹了。

可见,光源单缝$S$的宽度和薄膜的厚度都与光源的相干
性有关,激光光源的相干性远比一般光源好,因而利用激光光
源做双缝干涉实验,可以不用光源单缝,直接将激光光束投
射到双缝$S_1$和$S_2$上,就能在屏上观察到清晰的干涉条纹。
用激光光源做薄膜干涉实验,薄膜较厚时也可以看到干涉条
纹(用一块平玻璃板的两个表面反射的两列光波叠加起来就
能看到干涉条纹)。

\subsection{等倾干涉和等厚干涉}

\begin{figure}[htp]
    \centering
     \includegraphics[scale=.7]{fig/6-18.png}
    \caption{}
\end{figure}

薄膜干涉形成亮纹或暗纹的条件决定于由薄膜的前后表
面反射的两列光波的光程差,设薄膜的折射率为$n$, 厚度为
$d$, 放在空气中.如图6.18, 光束$A'A$射到膜上,从薄膜的
前表面直接反射的光束$AA'_1$, 从薄膜的后表面反射的光束
$CA'_2$. 这两束光的光程差为:
\[\delta =n\cdot (AB+BC)-\left(AC'+\frac{\lambda}{2}\right)\]
其中:
\[AB=BC=\frac{d}{\cos i_2}\]
\[AC'=2d\cdot \tan i_2\cdot \sin i_1=2nd \frac{\sin^2 i_2}{\cos i_2}\]
$\lambda/2$是光在光疏媒质中到达光密媒质的界面上反射的半波
损失。代入上式得
\[\delta=2nd\cdot \cos i_2-\frac{\lambda}{2}=2d\sqrt{n^2-\sin^2 i_1}-\frac{\lambda}{2}\]
可见,光程差与光束入射角度$i_1$有关,也与薄膜的厚度$d$
有关。
\begin{figure}[htp]
    \centering
     \includegraphics[scale=.7]{fig/6-19.png}
    \caption{}
\end{figure}

如果薄膜的两个表面彼此平行,即薄膜的厚度$d$均匀不
变,光程差仅取决于入射光束的入射角$i_1$, 在图6.19中,扩
展光源上的点$S_1$、$S_2$发出的光照射到薄膜上,所有入射角相
同的光线经薄膜反射后的光线彼此平行,经凸透镜会聚在其
焦平面上一点.设入射角为$i_1$的光分别从薄膜的前后表面反
射的两列光波的光程差$\delta=2d\sqrt{n^2-\sin^2 i_1}-\frac{\lambda}{2}$
都相同,若
$\delta=(2k+1)\cdot \frac{\lambda}{2}$,
在凸透镜的焦平面上的会聚点的光强是最小
值,形成暗条纹;若$\delta=k\cdot\lambda$,在凸透镜的焦平面上的会聚点的
光强是最大值,形成亮条纹,前后表面平行的薄膜形成的干
涉条纹是由相同的入射角的光形成的,叫做等倾干涉 如果
不用凸透镜观察等倾干涉条纹,这些干涉条纹成在无限远处,
通常说,等倾干涉条纹定域在无限远处。

\begin{figure}[htp]
    \centering
     \includegraphics[scale=.7]{fig/6-20.png}
    \caption{}
\end{figure}

把点光源放在凸透镜的焦点处,得到的平行光都以相同
的入射角照到厚度不等的薄膜上,如果薄膜的厚度很小,两
个表面间的夹角也很小(近于平行),从薄膜的前后表面反射
的两列光波的光程差仍可按公式
\[\delta=2d\cdot \sqrt{n^2-\sin^2 i_1}-\frac{\lambda}{2}\]
计算。由于入射角$i_1$都相同,光程差仅取决薄膜的厚度。在
图6.20中,光照在楔形薄膜上,在楔形薄膜的$C$点处,两列反
射光波的光程差$\delta$是波长$\lambda$的整数倍,则$C$点有最大光强,与$C$点等厚的各点也都有最大光强,形成一条亮线;同样的
道理,薄膜上的暗线对应着光程差是半波长的奇数倍的等厚
点,在薄膜上形成的干涉条纹决定于薄膜厚度相同点的轨
迹,这样的干涉现象叫做等厚干涉。通常观察等厚干涉条纹
时,光的入射角都不大,看到的干涉条纹定域在薄膜上。楔形
薄膜的干涉条纹是平行于楔形的棱的直线,牛顿环是由焦距
很大的球面平凸透镜与平面玻璃组成,在球面和平面之间构
成空气薄层,牛顿环的干涉条纹是等厚干涉条纹,它是一组明
暗相间的同心环。

\subsection{干涉和衍射的区别和联系}
光的干涉现象是两列或几列光波在空间相遇时,光强在
某些区域加强,在另一些区域削弱,形成稳定的光强有规律分
布的现象。光的衍射现象是光绕过障碍物偏离直线传播进入
几何阴影,并在屏幕上出现光强不均匀分布的现象。光的
干涉和衍射现象在屏幕上都得明暗相间的条纹,这些条纹的
产生,本质上都是波的相干叠加的结果。但是光的干涉强调
了两个或多个光束的叠加,对于参加叠加的几列光波都是以
直线传播的模型描写的。这样的干涉可以认为是纯干涉的问
题。光的衍射现象强调了光偏离开直线传播的现象,光在传
播过程中遇到障碍物时,一部分子波被遮蔽,其余部分的子
波叠加的结果形成了衍射条纹。尽管二者都是相干波的叠
加,但是前者是有限的几列光波的叠加,而后者是无数多个子
波的叠加。

在实际现象中,一般既有干涉的问题,又有衍射的问题。
例如,双缝干涉实验中,我们没有考虑双缝中每一个单缝的宽
度,即认为每条单缝都是很窄的(缝宽远小于光波的波长)。由
于衍射作用,每条单缝单独在光屏上形成的光强度几乎都是
相同的。由这样的两列光波相叠加形成的干涉亮条纹差不多
都有相同的强度,实际上,由于每条单缝都有一定的宽度,它
们各自独立存在时,在光屏上都要产生光强度不均匀分布的
衍射条纹。光屏上实际得到的条纹是这两组相干的衍射条纹
叠加的结果,它是被单缝衍射调制的双缝干涉条纹,如果用激
光光源做双缝干涉实验,在光屏上可以看到级数比较高的干
涉条纹,这时,可以见到光屏上的干涉亮条纹的强度不是均匀
的,随着单缝衍射条纹的光强分布而变化。

\subsection{衍射光栅产生的光谱}
课文提到光栅的缝数增多,产生的明条纹就变窄,其原因
可分析如下:

光栅是由许多宽度和间距都相等的狭缝组成的,光通过
光栅中的每一条狭缝后,都产生衍射,在某些方向上出现明条
纹。各条单缝衍射产生的明条纹比较宽,互相重叠,由于从各
缝射来的光波的互相干涉,结果又在叠加区域产生许多明暗
条纹。

\begin{figure}[htp]
    \centering
     \includegraphics[scale=.7]{fig/6-21.png}
    \caption{}
\end{figure}

光栅衍射(图6.21)产生明条纹的条件是
\begin{equation}
    (a+b)\sin\phi=k\lambda,\qquad k=0,\pm1,\pm2,\ldots
\end{equation}
式中$a$为缝宽,$b$为两缝间的距离,$a+b$叫做光栅常数。当
光栅常数$a+b$和波长$\lambda$给定时,表示从各狭缝发出的光线
的某些方向,在这些方向上光波相互加强,产生明条纹。

当$\phi$满足条件$n(a+b)\sin\phi=(2k+1)\frac{\lambda}{2},\quad 
k=0,\pm1,\pm 2,\ldots$ 时,将出现暗条纹,式中$n$为任意整数,其原因是:例
如当$n=1$时,不论单缝衍射的情况如何,相邻两狭缝所发出
的光波的光程差是半波长的奇数倍,应干涉相消,屏上跟这些
方向对应的地方总是暗条纹,当$n$为其他任一整数时,第一条
狭缝与第$n+1$条狭缝所发出的光波的光程差是半波长的奇
数倍,干涉相消;第二条狭缝与第$n+2$条狭缝所发出的光波
也干涉相消……也是暗条纹,光栅的狭缝数越多,可以取的$n$
值就越多,形成暗条纹的范围就越广。

考虑每一条狭缝,当$\phi$满足条件$a\sin\phi=k\lambda,\quad k=0,
\pm1,\pm2,\ldots$时,每一条狭缝所发出的光波都各自干涉相消,
形成暗条纹,就谈不到各缝的光波的干涉了。

从以上分析可见,在光栅衍射中,只有当$\phi$满足(6.1)式时
才能形成明条纹,而形成暗条纹的机会要多得多,因此在明条
纹之间充满大量的暗条纹,实际上形成了一片黑暗的背景,这
就是光栅的缝数越多,产生的条纹就越窄的原因。

从(6.1)式还可以看出,光栅常数一定时,衍射角ø的大小
与入射光的波长入有关,因此白光通过光栅后各种色光将
产生各自分开的条纹,而形成光谱。中央条纹也叫零级条纹
仍是白色,在中央条纹的两侧,对称地排列着第一级、第二
级……光谱,如图6.22所示。

\begin{figure}[htp]
    \centering
     \includegraphics[scale=.7]{fig/6-22.png}
    \caption{}
\end{figure}

光栅的衍射光谱和棱镜的色散光谱是不同的,其主要区
别是:
\begin{enumerate}
\item 由于角度$\phi$实际上是很小的,$\sin\phi$或$\phi$跟波长成正
比,因此光栅光谱中各谱线到零级条纹的距离跟波长正比,
这种光谱叫做匀排光谱。在棱镜光谱中,光谱线间的距离决
定于棱镜的顶角和折射率,不具有匀排的性质,光谱的短波部
分展开的较大,长波部分展开的较小。
\item 在光栅光谱中,波
长越短的光,对应的中角越小,而在棱镜光谱中,对应的偏向
角越大。光栅光谱中各谱线的排列是由紫到红,与棱镜光谱
的由红到紫正好相反。
\end{enumerate}

\subsection{偏振光与自然光}
既然光是横波,而且具有偏振性,为什么通常光源发出的
光不能直接显示偏振现象呢?我们知道,两束光在程差很大
(达$10^{8}$个波长)时,还能观察到干涉现象.这一事实明在
$10^{8}$次振动的初始时刻和终了时刻所发的光还是相干的,它
们的振动方向在这一时间间隔内还能够保持在同一平面内。
可见在相当长的一段时间($10^{-8}$秒已足够长)内,个别原子的辐
射不仅能够保持初位相不变,而且还能保持偏振的性质不变。
因此可以认为每一个辐射原子每次所发射的是一列平面偏振
的电磁波,然而在通常光源中,那些排列得毫无规律数量又
非常众多的辐射原子或分子,它们在同一时间内发出的光各
自具有不同的初位相和不同的振动面;而且每个原子在每次
发光之后,在某一时刻又将以新的初位相和新的振动面重新
发光,所以通常光源发出的许多列光波,它们的振动面可以分
布在一切可能的方位。平均来看,任何方都有相同的振动能
量,也就是说,电矢量对于光的传播方向是对称而又均匀分布
的。这种由通常光源所辐射的光波,其电矢量在垂直于光波的
传播方向上既有时间分布的均匀性,又有空间分布的均匀性。
也就是说,在各个取向上电矢量的时间平均值是相等的,这种
光就是自然光,图6.23甲就是沿$z$轴传播的自然光,

\begin{figure}[htp]
    \centering
     \includegraphics[scale=.7]{fig/6-23.png}
    \caption{}
\end{figure}

在自然光中,任何取向的电矢量$E$都可分解为相互垂直
的两个方向(例如$x$方向和$y$方向)上的分量,所有取向的电
矢量在这两个方向上的分量的时间平均值必相等,也就是说,
自然光可以用强度相等、振动方向互相垂直的两个平面偏振
光来表示,如图6.23乙所示,但是必须注意,由于自然光中
各电矢量之间无固定的位相关系,因而其中任何两个取向不
同的电矢量不能合成为一个单独的矢量。


\begin{figure}[htp]
    \centering
     \includegraphics[scale=.7]{fig/6-24.png}
    \caption{}
\end{figure}

此外,有些光的电矢量在某一确定的方向上最强,而在和
它成正交的方向上最弱,这种光称为部分偏振光。部分偏振
光可用图6.24所示的图形来表示,其中甲表示在图面内电矢
量较强的那种部分偏振光,乙表示在垂直于图面方向上电矢
量较强的那种部分偏振光,设$I_{\max}$为某一部分的偏振光沿某
一方向上所具有的能量最大值,$I_{\min}$为在其垂直方向上具有的
能量最小值,则通常用
\[P=\frac{I_{\max}-I_{\min}}{I_{\max}+I_{\min}}\]
来量度偏振的程度,并称$P$为偏振度.如果$I_{\min}=0$, 那么
$P=1$, 振动完全限制在一个平面(振动面)内,这就是平面偏
振光,$I_{\min}\ne 0$时,$P<1$, 这就是部分偏振光.

\begin{figure}[htp]
    \centering
     \includegraphics[scale=.7]{fig/6-25.png}
    \caption{}
\end{figure}

实验证明,自然光在两种媒质的分界面上反射和折射时,
反射光和折射光就能成为部分偏振光或完全偏振光。如图
6.25所示,$MN$是两种媒质的分界面,$SI$为一束自然光的入
射线,$IR$和$IR'$分别是反射线和折射线,由于自然光中在跟
光的传播方向垂直的一切方向上光的振幅都可以看作是相等
的,没有任何一个方向较其他方向占优势,我们如果把光的振
动分解为两个分振动,一个跟入射面(即纸面)垂直,叫垂直振
动,用点子表示,一个跟入射面平行,叫平行振动,用短线表
示,那么在自然光中,表示平行振动的短线和表示垂直振动的
点子是均匀配置的。但是,在反射光束中,垂直振动多于平行
振动,在折射光束中,平行振动多于垂直振动。

布儒斯特在1812年指出,反射光偏振化的程度取决于入
射角$i$. 当$i$为某一定值$i_0$时,即满足$\tan i_0=n_{21}$时,反射
光成为完全偏振光,振动面和入射面垂直,这时平行振动
完全不反射,只发生折射(折射光不是完全偏振光),式中的
$n_{21}$是折射媒质对于入射媒质的相对折射率,$i_0$叫做起偏角。

根据折射定律$\sin i_0=\dfrac{n_2}{n_1}\sin r=n_{21}\sin r$. 由
$\tan i_0=n_{21}$
可得:
\[\sin i_0=\tan i_0\sin r\]
即
\[\sin r=\frac{\sin i_0}{\tan i_0}=\cos i_0\]
所以
\[i_0+r=\frac{\pi}{2}\]
这说明,当入射角为起偏角时,反射光线和入射光线互相垂
直。
