\chapter{光的波动性}\minitoc[n]
\section{教学要求}

本章讲述光的波动性、光的电磁理论和电磁波谱,以及光
谱和光谱分析的知识。认识光的波动性及其电磁本质,在物
理学史上有重要的意义。光谱学的知识,为研究原子结构提
供了信息。光的干涉、衍射、偏振等现象,在现代科学技术中
有主要的应用。学生应该对这些知识有所了解。

光的干涉现象,根据学生的知识水平和接受能力,有条件
分析得深入一些。因此,作为本章的重点内容。衍射现象的
分析,要复杂一些,因此只介绍了现象,本章的其他知识都是
介绍性的。

本章教材分为三个单元。第一节至第五节为第一单
元,讲述光的波动性,第六、七节为第二单元,讲述光的电磁
本性,第八节为第三单元,介绍光谱的初步知识。

讲述光的微粒说与波动说的矛盾是为了使学生了解人们
对光的本性的认识经历了曲折的过程,也是讲解光的波动性
的引子,激发学生的学习兴趣。

光的干涉和衍射现象,在日常生活中极易忽略,学生不熟
悉。教学中应注意通过演示使学生观察现象,联系机械波的
干涉和衍射现象,运用波动知识进行分析。在讲干涉现象时,
不要求解释为什么不同光源发出的光不是相干光,要把教学
的重点放在用波动说解释产生明暗条纹的条件上。在推导时
要注意突出推导的思路。通过研究光的干涉,还应该让学生
知道光的颜色跟波长(频率)有关,不同色光的频率是不同的。

光的衍射现象也证明了光具有波动性。在教学中可以指
导学生用一些简单的方法进行观察。

光的偏振现象,证明了光是横波。但是学生对偏振现象
比较生疏。教学中应跟学生比较熟悉的机械波的有关现象类
比,并在充分观察实验现象的基础上进行讲述。

从表面上看,光现象和电磁现象之间似乎没有什么联系。
麦克斯韦提出的光的电磁说和赫兹对电磁波进行的实验研
究,揭示了光的电磁本质。通过光的电磁说的教学,应该使学
生体会自然现象间的相互联系,对无线电波、红外线、可见光、
紫外线、伦琴射线等不同波长的电磁波有一个统一的认识,并
对它们的性质和产生机理有所了解。

这一章的教学要求如下:
\begin{enumerate}
\item 理解光的干涉现象,理解产生明暗条纹的条件,了解
光的干涉现象的应用。
\item 了解光的衍射现象和产生衍射现象的条件。
\item 了解光的偏振现象和光是一种横波.
\item 了解光是一种电磁波;了解无线电波、红外线、可见
光、紫外线、伦琴射线等都是波长不同的电磁波。
\item 了解光谱和光谱分析的初步知识。
\end{enumerate}

\section{教学建议}
\subsection{光的波动性}
\subsubsection{人类对光的本性的两种认识}

人类对光的本性的认
识经历了一个辩证发展的过程,到十七世纪,在人类已经积累
了许多几何光学知识的基础上,形成了对光的本性的两种认
识——微粒说和波动说。

教学中要讲述两种理论在解释实验规律上各有其成功的
一面,也要讲述两种理论的不足之处。还要说明两种理论的
尖锐矛盾。例如,微粒说难以说明光在媒质界面上同时发生
反射和折射,波动说却可以解释这个事实,波动说难以说明光
的直线传播,在微粒说看来,这都是很自然的现象。还可适当
补充一些历史事实进一步说明微粒说和波动说的尖锐矛盾,
例如,笛卡儿从微粒说推导出光的折射定律,得出了光在媒质
中的速度大于真空中的光速;惠更斯从波动说也推导出光的
折射定律,但得出了光在媒质中的速度小于真空中的光速。对
于两束光相交时各自独立传播的事实,微粒说难以解释,波动
说却很容易说明,等等。通过这些事实的讲述,使学生认识到
理论和实践的矛盾,两种理论对光的本性认识的矛盾,是推动
人类认识光的本性的内在动力。

\subsubsection{双缝干涉}

光的干涉现象是光具有波动性的重要依
据。教材首先介绍了托马斯·杨在历史上第一次解决了相干
光源的问题,成功地做出了光的干涉实验,然后具体介绍了双
缝干涉实验,并用波动理论对实验现象进行了定量的分析,得
出了干涉条纹间距与波长的关系,这对于学生认识光的波动
性重要的意义,由于学生对于光的干涉现象比较生疏,难
以把光想象成是一种波动,而且运用波动理论分析光的干涉
现象,内容比较抽象,学生接受起来会有一定的困难。建议教
学中注意如下几点:

首先要复习机械波的干涉的知识,复习重点放在什
么是波的干涉现象,相干波源的条件。运用波动理论分析空间
振动加强区域和削弱区域的产生条件,还要进一步用光程差
来分析两个振动的位相关系,为讲授本节内容做准备。

要做好演示实验,使全班同学都亲眼看到光的干涉
现象,这是认识光的干涉的基础。实验内容包括单色光干涉
的明暗条纹和白光干涉的彩色条纹。要注意介绍仪器装置。教
学要结合课本图6.2进行.

运用波动理论分析双缝干涉实验要注意:
\begin{enumerate}
\item 说明如
何获得相干光源,这是得到稳定的干涉现象的关键。结合实验
装置,对照课本图6.2说明通过双缝得到的$S_1$和$S_2$
两个光源,是在任何时刻频率和位相都相同的相干光源。
\item 
类比机械波的干涉,运用波动理论分析光的干涉现象.把$S_1$
和$S_2$相干光源发出的光想象成两列光波,两列光波在屏上相
遇。两个光振动叠加产生亮、暗条纹。亮条纹处的光能量较
强,对应着合振动互相加强;暗条纹处的光能量较弱,对应着
合振动互相削弱。
\item 推导屏上亮暗条纹的位置公式要讲清思
路:屏上亮纹或暗纹的位置用距$O$点的距离$x$表示,屏上任
一点的振动加强或削弱的情况决定于该点到光源$S_1$和$S_2$的
距离之差(光程差)$\delta=r_2-r_1$. 在$\delta$等于波长整数倍的位置,
产生亮纹;在$\delta$等于半波长奇数倍的位置时产生暗纹.寻求
$\delta$与$d$, $\ell$和$x$的关系,推出$\delta\approx \dfrac{d}{\ell}\cdot x$
\item 学生对于公式
\[x=\pm k\frac{\ell }{d}\lambda, \qquad k=0,1,2,\ldots\]
和
\[x=\pm (2k-1)\frac{\ell }{d}\cdot \frac{\lambda}{2}, \qquad k=1,2,\ldots\]
的表述方法不习惯,讲述时要从具体的中央亮纹、第1条、第2
条……亮纹或暗纹入手,归纳出一般的表述。例如,亮纹的条
件是$\delta=r_2-r_1=0,\lambda,2\lambda,\ldots$和$-\lambda,-2\lambda,\ldots$(“$-$”的意义
是$r_2<r_1$, 位置在$O$点的下方)。归纳出$\delta=\dfrac{d}{\ell}x=\pm k\lambda$, 解出
\[x=\pm k\frac{\ell }{d}\lambda, \qquad k=0,\pm 1,\pm 2,\ldots\]
然后具体说明$k=0,1,
2,\ldots$所对应的
$x=0,\pm\dfrac{\ell}{d}\lambda,\pm 2\dfrac{\ell}{d}\lambda,\ldots$
的意义,使学生理
解这种数学表述的内容。
\item 双缝干涉公式的近似条件是
$\ell\gg d$和$\ell\gg x$. 双缝间的距离$d$一般仅为十分之几毫米。而
屏上偏离开中央亮纹较远处的亮纹的强度是十分弱的,几乎
无法观测到。通常能观察到亮纹范围远小于$\ell$.
\end{enumerate}


根据相邻亮纹或暗纹间的距离公式$\Delta x=\dfrac{\ell}{d}\lambda$
可以测量光波的波长。红光的干涉条纹间距最宽,紫光的最窄,由此
可认识不同颜色的光的波长不等,由红到紫,波长越来越短,
频率越来越高。进一步说明白光的彩色干涉条纹的产生原因。
学生通过测量光波的波长,对于可见光的波长和频率可以有
一个大致的认识。

说明双缝干涉的亮暗条纹反映了光源$S_1$和$S_2$发出
的光的能量在空间的分布情况。暗条纹处的光能量几乎是零,
表明两列光波叠加彼此相互抵消。这并不是光能量损耗了或
转变成了其他形式的能量,而是按照波的传播规律,没有能
量传到该处;亮条纹处的光能量比较强,光能量增加也不是光
的干涉可以产生能量,而是按照波的传播规律,到达该处的能
量比较集中。

\subsubsection{薄膜干涉 }
学生平时都见过薄膜干涉现象(雨后路面
上的油膜形成的彩色条纹,色彩绚丽的肥皂泡等)只是没有引
起注意。提出一些现象,并让学生动手做肥皂液薄膜的干涉
实验,观察单色光的明暗条纹和白光的彩色条纹,会给学生
留下深刻的印象。实验时,可引导学生细致地观察干涉条纹,
例如,可观察到干涉条纹产生在薄膜的表面上,肥皂液薄膜的
干涉条纹基本上是水平的等等。(这些都是等厚干涉的特点决
定的。)

分析薄膜干涉实验的重点应放在如何得到相干的
两列波,薄膜上是怎样出现明暗相间的干涉条纹或彩色条纹
的.课本图6.4是肥皂液薄膜干涉的示意图.图中只给
出了从楔形薄膜的前后表面反射的两列光波的位相关系和叠
加的结果。该图表示的是光波几乎垂直地入射到楔形薄膜上
后,又从薄膜的前后表面反射回来的情况,薄膜的两个表面是
近于平行的。图中薄膜的楔形被大大夸张了,应该指出,分别
从前、后表面反射回来的两列波都来自于同一入射波,因而是
相干的。由于从后表面反射的光波比前表面反射的光波通过
的路程较长,因而位相要落后一些,落后的相差与膜的厚度有
关。在膜上某些厚度的地方,两列反射波是同相的,形成相互
加强的亮纹;在另一些厚度的地方两列反射波是反相的,形成
相互削弱的暗纹。教材没有深入分析计算光程差和薄膜厚度
的关系,也不涉及光在光疏媒质中达到光密媒质的界面反射
时的半波损失以及在液膜内的光波波长小于空气中的波长等
问题,教学中应注意掌握讲授的深度,以免加重学生的负担。

\begin{figure}[htp]
    \centering
    \includegraphics[scale=.6]{fig/6-1.png}
    \caption{}
\end{figure}


薄膜干涉在科学技术上有重要的应用,除了教材中
介绍的“用干涉法检验光学元件表面加工的质量”之外,还
可适当增加介绍在检验钢球的直径、透镜表面的曲率半径,测
量长度的微小变化等方面的应用,讲授增透膜时,要注意说明
反射光和透射光的能量之和等于入射光的能量(不考虑媒质
对光的吸收)。增透膜的作用只是使反射的两列光波产生相消
干涉,反射光的能量趋于零,因增加了透射光在入射光中所
占的比例。并不是增加了光的能量,以免学生误解。增透膜
的厚度是光在薄膜媒质中传播的波长的1/4(不是真空中波长
的1/4)。由于光垂直于薄膜表面入射时,从前后表面反射的两
列光波的路程差等于薄膜厚度的2倍,如图6.1所示,当膜的
玻璃厚度是$\lambda/4$
时,路程差恰好是$\lambda/2$,
从前后表面反射的光波的相位恰好相反,便产生相消干涉。制作增透膜的材料是氟化镁
${\rm MgF_2}$, 折射率$n=1.38$, 介于空气和玻璃之间.因此在空气
和增透膜的界面上、增透膜和玻璃的界面上,反射情况相同,
不会再额外增加两列光波的光程差。增透膜只对人眼或感光
胶片最敏感的绿光起增透作用。如果白光照射到增透膜上,
由于绿光产生相消干涉,在反射光中绿光的强度几乎是零,
而其他波长的反射光并没有完全抵消,因此,增透膜呈绿光的
互补色——淡紫色。

\subsubsection{光的衍射}

光的衍射现象进一步证明了光具有波动
性,对发展光的波动理论起了重要的作用。教材讲述光的衍射
的思路是先说明一般情况下不容易观察到光的衍射现象的原
因,同时也就说明了观察衍射现象的条件,再来做衍射实验。
由于衍射现象产生的物理过程分析起来比较复杂,课本中对
于衍射现象未作理论上的分析。

做好光过小孔或单缝发生衍射现象的实验,是学
生认识光的衍射的基础。实验中要让学生观察:
\begin{enumerate}
\item 光偏离直
线传播,绕过障碍物进入阴影中,并且在屏上出现明暗相间的
衍射条纹。
\item 只有孔或狭缝较小时,光的衍射现象才比较显
著。
\item 为了认识光的直线传播和光的衍射的关系,观察屏上的
光斑变化情况时应使孔径或狭缝的宽度逐渐变小。当孔径或
缝宽较大时,屏上光斑的边缘清晰,显示出光沿直线传播;孔
径或缝宽逐渐变小时,屏上光斑的边缘逐渐变得不清晰了,衍
射现象逐渐变得显著起来,直至出现明暗相间的衍射条纹。
\end{enumerate}
上述演示可以使学生认识到,由于光是波动,遇到障碍物时发生
衍射现象是不可避免的,只是由于在一般情况下障碍物的尺
寸比光的长大得多,因此衍射现象很不明显。当衍射现象
可以忽略时,才可以认为光是沿着直线传播的。

建议采用课堂讲授和学生实验并进的方式进行教
学。用游标卡尺的测脚形成可调宽度的狭缝观察线光源的衍
射现象,用不同孔径的小孔观察点光源的衍射现象。学生自
已动手观察到光的衍射条纹,可使他们了解到缝宽和孔径多
大时能够观察到比较显著的衍射现象。

菲涅耳圆盘衍射的中心亮斑能够引起学生的极大兴
趣,但在课堂上完成这个实验比较困难。教学时可以用细金
属丝产生的衍射来代替,由于观察到在金属丝的阴影中间有
一条亮线,学生会更加信服光的衍射现象。

光栅的衍射是衍射现象在科学技术上的重要应用.
光通过衍射光栅的亮条纹随着光栅缝数的增加而变窄和变亮
的特点是从实验现象中得出的。教学中不要求从理论上加以
分析。可介绍一些衍射光栅的应用,例如利用光栅衍射条纹的
特点可以比较精确地测量光波的波长和产生均匀分布的光
谱等。

\subsubsection{光的偏振}
横波的偏振是新概念.为了从机械波入手来认识偏
振.应将课本257页的实验演示给学生看.这个实验,形象
地给出了横波是偏振的机械模型,偏振现象是横波区别于纵
波的最明显的标志,通过机械波和光波的类比,可以从光的
偏振现象使学生认识光是横波。

光的偏振现象要通过实验给出.实验的设计思想和
横波偏振的机械模型一样。电气石晶体薄片或人造偏振片对
于某一振动方向的光具有选择吸收的本领,它们的作用相当
于课本257页的演示中限制或检验振动方向的“狭缝”.实验
应先演示光通过两个偏振片时,转动其中任一个偏振片的方
位,透射光的强度出现周期性的变化,给学生以鲜明突出的印
象,再演示光通过一个偏振片的情况,使学生产生悬念,激发
他们的探索热情。分析光的偏振实验,要引导学生把光波和
机械横波相类比,建立光的波动模型,能想象出光是一种横
波,它的振动方向跟光的传播方向垂直,教学中应注意讲清起
偏器和检偏器的不同作用,自然光和偏振光的区别和联系。

通过演示反射光的偏振现象,说明光的偏振现象是
普遍的(即教材中讲的“除了从光源直接射来的,基本上都是
偏振光),也便于使学生理解光的偏振现象在科学技术上的一
些应用,教材还安排了“偏振光与立体电影”的阅读教材,如能
配合教学看一场立体电影,学生一定会有浓厚的兴趣。

\subsection{光的电磁本性}

\subsubsection{光的电磁说}

光的电磁说比光的波动说前进了一大步.课本结合物
理学史讲述了人类认识光的本性的发展过程。教学中要注意
以下四个环节:
\begin{enumerate}
    \item 从十七世纪开始到十九世纪初,光的波动说不断地
发展和完善,逐渐为人们所接受,但是人们对光的本性认识不
足,以为光也是一种机械波,这种认识在光的传播媒质等问题
上遇到了严重的困难。
\item 法拉第发现在强磁场作用下,偏振光的振动面发生
偏转的现象。它启示人们把表面上很不相同的光现象和电磁
现象联系起来。
\item 光是一种电磁波.电磁波和机械波在本质上不同,
它可以在真空中传播而不需要任何媒质。这就不仅解决了波
动说在光的传播媒质问题上遇到的困难,而且对光的本质有
了进一步的认识。
\item 历史上,光的电磁学说是麦克斯韦作为假说提出的,
赫兹实验证实了电磁波的客观存在,也证明了光是一种电磁
波,使光的电磁理论得以确立。
\end{enumerate}

通过光的电磁说的教学,不仅要使学生了解光的电磁说
的基本内容。还要通过光的电磁理论的建立和发展过程,认
识理论和实践的辩证关系,从中受到辩证唯物主义世界观的
教育。

\subsubsection{电磁波谱} 
电磁波谱的教学,应该着重使学生领会光
的电磁说把光现象和电磁现象统一起来了。

在把无线电波、红外线、可见光、紫外线、伦琴射线和γ射
线按照频率(或波长)的顺序排列成波谱,使学生对电磁波有
一个全面的了解后,还要进一步使学生认识这些电磁波既具
有共同的本质,又有各自的特性和不同的产生机理,介绍这
些内容可以为进一步学习光的波粒二象性和原子的内部结构
做准备。

在电磁波谱的教学中,还要注意介绍红外线、紫外线、伦
琴射线的一些应用,以开阔学生的视野,激发他们学习科学
技术的志趣。

通过这一单元的教学,应当使学生体会到自然现象之间
是相互联系的,有些表面上很不相同的现象却存在着共同本
质。把光现象和电磁现象统一起来,是物理学的伟大成果,现
在物理学的研究还在更加广泛的范围上进行着类似的统一
工作。

\subsection{光谱的初步知识}
\subsubsection{介绍光谱学的知识}

要让学生认识观察光谱的仪
器——分光镜。讲述分光镜的构造原理要结合实物和挂图。应
先讲述单色光通过三棱镜的光路,单色光照亮狭缝$S$, 经过凸
透镜$L_1$形成平行光,通过三棱镜$P$发生偏折,再会聚在凸透
镜$L_2$的焦平面$MN$上成实像,狭缝$S$的实像是一条亮线,颜
色和入射光相同。再讲复色光通过三棱镜的光路,使学生理
解不同颜色的光谱线实际上是照亮的狭缝$S$在焦平面$MN$上
形成的不同颜色的实像,这些实像按光的波长顺序排列形成
光谱。教学中还可以介绍标尺管的作用及其光路,使学生对分
光镜有比较全面的认识。

\subsubsection{光谱分析}

结合观察连续光谱、明线光谱(特别是
氢原子的光谱)和吸收光谱,使学生了解有关的几个概念,了
解各种光谱产生的机制,观察同一元素原子的发射光谱和吸
收光谱时,要注意观察吸收光谱的暗线位置和发射光谱的明
线位置一致,而不同元素原子有不同的光谱。从而理解光谱
代表了每种原子的特征,为介绍原子结构的内容作些准备。正
是每种原子都有自己的特征谱线,利用光谱可以鉴别物质和
确定物质的化学组成。

需要指出的是,由于中学配备的分光镜分辨本领较差,对
于太阳的吸收光谱的暗线,有的仪器不能观察到。

建议采用以学生自学为主的方式进行教学,指导学生
通过观察实验和阅读教材,理解光谱产生的机理和光谱的分
类,理解发射光谱、连续光谱、明线光谱和吸收光谱之间的关
系,认识光谱分析的原理及其方法。

\section{实验指导}
\subsection{演示实验}
本章的双缝干涉、衍射、偏振演示实验,要用J2508型光
的干涉衍射偏振演示仪。

J2508型光的干涉衍射偏振演示仪是由可转动的光具
座、滑块、观察筒、盒式光源、光学元件组成。

光具座上部是附有长80厘米标度的单导轨,导轨支撑在
光具座的底座上并可在水平面上任意转动。滑块分三种规格,
可套在光具座的单导轨上沿导轨滑动。滑块顶部的孔上可安
插各种光具。

观察筒由遮光筒、放大镜、玻璃屏和遮光板组成。

盒式光源由低压电源供电.盒内装有12V50W的卤钨
灯,盒前有聚光透镜。在出光口上装上单缝光栏,便可作线光
源使用。

主要光具有:狭缝、牛顿环、双面镜、反射器、毛玻璃屏、双
凸透镜等。其中,狭缝包括双缝(缝宽0.016毫米,缝距分别
为0.04毫米和0.08毫米两种)、单缝(缝宽0.08毫米),光栅
(1\%)。多缝缝宽0.02毫米,缝距0.08毫米。牛顿环装在圆
形胶木架中,胶木架上有三个调整螺钉。

光源、光具,观察筒均可安装在滑块顶部的小孔上,绕安
装轴转动。

图6.2的甲、乙、丙分别为光具座、观察筒和盒式光源的
结构示意图。

\begin{figure}[htp]
    \centering
    \includegraphics[scale=.6]{fig/6-2.png}
    \caption{}
\end{figure}

\subsubsection{双缝干涉}
用J2508型光的干涉衍射偏振演示仪做双缝干涉实验.
装置按图6.3配置.套在光源前的光源单缝缝宽为0.11毫
米,双缝的缝宽0.08毫米,装在光具架上,缝上的指示刻线
对齐光具架上的零刻线。

调整光源、单缝、双缝和观察筒的共轴是实验能否成功的
关键。

\begin{figure}[htp]
    \centering
    \includegraphics[scale=.6]{fig/6-3.png}
    \caption{}
\end{figure}

光具的调整步骤如下:先使单缝和双缝大致平行,相距约
5—10厘米(双缝离单缝近一些可以增加干涉亮条纹的亮度,
但光源单缝和双缝的共轴必须调得很好),观察筒的轴线和光
具座导轨平行,调整时可在观察筒前放一张白纸作为观察
屏,转动光源使得光通过单缝和双缝后落在白纸屏上的光斑
位于观察筒的中心处。再转动光源单缝,使它和双缝平行,在
白纸屏上见到清晰的干涉条纹,如果撤去白纸屏,便在毛玻
璃屏上呈现出清晰的干涉条纹。学生可以直接看毛玻璃屏上
的干涉条纹,也可以通过透镜观察干涉条纹的放大虚像。如
果在光源单缝和双缝之间加滤色片(也可用红色、绿色或紫
色的玻璃纸),可以看到单色光的明暗相间的干涉条纹。改变
观察筒与双缝的距离,可以看到干涉条纹的宽度随观察筒与
双缝间的距离增大而增大的情况。

由于光源用卤钨灯并有聚光透镜,从而使通过单缝和双
缝的光通量增加,提高了屏上亮条纹的亮度。为延长灯泡寿
命,开始用6伏电源点亮灯泡,然后再根据实际需要逐渐升高
电压,但不得超过12伏,接收干涉条纹的毛玻璃屏位于观察
筒内,遮挡了其他杂散光,提高了屏上干涉条纹的可见度,因
此可在一般亮度的教室中观察到清晰的干涉条纹。

实验时应缓慢转动光具座,使全班同学都能看到实验
现象。

\begin{figure}[htp]
    \centering
    \includegraphics[scale=.6]{fig/6-4.png}
    \caption{}
\end{figure}

由于激光的平行度好,单色性好,亮度高,是做光的干涉、
衍射实验的理想的单色光源,实验装置如图6.4所示,将激
光光束直接照射到双缝上,通过双缝后的光再投影到远处的
屏上,在屏上便可以见到明暗相间的单色光的干涉条纹,由于
激光光束很细,屏上的干涉条纹近于明暗相间的亮点,若采用
扩束装置将激光扩束后照到双缝上,可以得到双缝干涉的平
行条纹。但扩束后明条纹的亮度很低,实验需在暗室中进行。

\subsubsection{薄膜干涉}

可让学生分组做“肥皂液膜上光的干涉”实验.金属
丝圆环用黄铜丝或铁丝自制,肥皂液应清洁无杂物,浓度适
当。往酒精灯芯上撒食盐,火焰是黄色的。把肥皂液膜靠近酒
精灯,通过薄膜的反射去看黄色火焰,在薄膜上可见到明暗相
间的干涉条纹。若用白光照射,在薄膜上见到彩色的干涉
条纹。

对于一般的光源,能够见到干涉条纹时,薄膜的厚度须足
够薄。因此,肥皂液的浓度不能太浓。干涉条纹一般先出现在薄
膜的上部,往往当薄膜将要破裂时,才能见到较多的干涉条纹。

仪器所附的牛顿环,是由一块圆的平板玻璃和一个
凸透镜叠合而成的。球面与平板玻璃间形成空气膜,其厚度
由接触点向外逐渐增大。调节框上的三个调整螺钉,使干涉
图样位于中心部分。但不要拧得过紧,以免玻璃破碎。由牛
顿环的空气膜产生的干涉条纹是等厚条纹,利用光的干涉、
衍射、偏振演示仪做牛顿环投影实验装置如图6.5。将牛
顿环置于导轨的一端,把光源、凸透镜、毛玻璃屏按图中所示
位置放置,使凸透镜($f=7$厘米)距牛顿环大约8厘米。光源
的光斜射到牛顿环上,使反射光过凸透镜在毛玻璃屏上成
像,稍稍调整牛顿环的位置,在毛玻璃屏上可见到清晰的圆
环形彩色干涉条纹。条纹越向外越密,条纹的中心是暗
斑。

\begin{figure}[htp]
    \centering
    \includegraphics[scale=.6]{fig/6-5.png}
    \caption{}
\end{figure}

若将凸透镜放在牛顿环的另一侧,并移动它的位置,也可
在光屏(如白墙)上见到牛顿环的干涉条纹,这是光透过牛顿
环后经凸透镜成的像,牛顿环的透射干涉条纹的中心是亮
斑,和反射干涉条纹是互补的。拧动牛顿环边缘上的调整螺
钉,干涉条纹的形状、位置和圆环半径的大小都会发生变化。

本实验用低压电源供电,电压为6—10伏.

\subsubsection{单缝衍射}
用光的干涉、衍射、偏振演示仪(J2508型)做单缝衍
射实验.装置与图6.3相似,需将图中的双缝换成宽度为
0.08毫米的衍射单缝,如图6.6所示,调整光源单缝,使衍射
单缝和观察筒共轴。调整方法和双缝干涉实验相同。
\begin{figure}[htp]
    \centering
    \includegraphics[scale=.6]{fig/6-6.png}
    \caption{}
\end{figure}

用游标卡尺的外测脚做宽度可调的单缝代替衍射单缝,
可以演示衍射条纹随单缝宽度变化的情况。缝宽较大时,如
缝宽为2毫米,屏上为边缘清晰的亮线.当缝宽减小时,屏上
的亮线的宽度也减小。如果进一步减小单缝的宽度,可以见
到亮线的两侧出现衍射条纹。缝宽再减小,衍射条纹的宽度
反而变大,只是明条纹的亮度降低。实验应该注意:
\begin{enumerate}
\item 缝宽较
大时,屏上的亮线中似有明暗的条纹,这不是衍射条纹,而是
仪器的光源卤钨灯灯丝的像的一部分(灯丝像的其余部分
被可调狭缝屏遮挡住了)。
\item 用卡尺的测脚做狭缝,缝宽较大
时,屏上有时也出现明暗相间的条纹,这是从卡尺反射的光和
从光源单缝直接射出的光相干涉的条纹。解决的办法是稍稍
转动卡尺,使测脚平面反射的光不能达到观察筒内的毛玻璃
屏上。
\end{enumerate}


利用这套实验装置还可以做不透明的单丝衍射实验,取
直径在0.1毫米左右的细丝(如头发丝、多股绞合电线中的一
股等),用胶水竖直地粘在光具架上,替换衍射单缝。光源单缝
换用宽0.025毫米的,转动单缝的位置,使它和不透明的单丝
平行,在屏上可以看到与单丝平行的衍射条纹,特别是在单丝
的“影子”中央有一条亮线。

用激光光源做单缝衍射实验,激光光束直接照到狭
缝上,通过狭缝后再投影到远处光屏上,可以看到明暗相间的
衍射条纹,改变狭缝的宽度,可以看到狭缝越窄,衍射条纹越
远,但亮度减弱。

用激光光源还可以做圆孔衍射实验,用细针在牙膏皮上
扎出不同孔径的圆孔,激光光束照射到小孔上,通过小孔后直
接投射到远处的光屏上,可以看中心处是最亮的圆形光斑,
周围还有明暗相间的圆环形条纹,孔径越小,中心亮班及周围
的圆环形条纹的面积越大。

\subsubsection{衍射光栅}
\begin{figure}[htp]
    \centering
    \includegraphics[scale=.6]{fig/6-7.png}
    \caption{}
\end{figure}

用光的干涉、衍射、偏振演示器(J2508型)做光栅衍射的
实验,装置如图6.7。实验时将光源、光源单缝、凸透镜和毛
玻璃屏共轴放置在光具座上。调整凸透镜($f=7$厘米)的位
置(距光源单缝稍大于7厘米),使得光源单缝在毛玻璃屏中
央成清晰的放大实像,再将衍射光栅装在光具架上,缝座的指
示刻线对齐光具架的零刻线,把光具架插到凸透镜和光屏之
间,并靠近凸透镜。调整光源单缝与光栅的刻线平行,在毛玻
璃屏上就可以见到光栅的衍射条纹。

为了说明衍射明条纹随着缝数增加而变窄且亮度增强的
特点,实验时,可在光具架上顺序装置单缝、双缝、多缝和光
栅,比较它们的衍射条纹,上述特点很容易由实验得到。单缝、
双缝和多缝的衍射亮条纹的亮度较低,可用观察筒代替毛玻
璃屏,便于同学观察。

实验用6—10伏交流电.

\subsubsection{光的偏振}

演示自然光通过偏振片产生偏振光可用光的干涉、
衍射、偏振演示仪。将光源、两个偏振片(分别装在两只光具
架上)和毛玻璃屏依次装在光具座上,使它们的中心在平行于
光具座导轨的同一条直线上。当两个偏振片的偏振化方向
(用偏振片座上的指针表示)平行时,毛玻璃屏上有明亮的光
斑;固定一个偏振片(为方便演示,偏振化方向可取竖直方
向),转动另一个偏振片,当两个偏振片的偏振化方向垂直时,
屏上的光斑几乎消失。继续转动偏振片,屏上光斑的亮度出
现周期性变化。

\begin{figure}[htp]
    \centering
    \includegraphics[scale=.6]{fig/6-8.png}
    \caption{}
\end{figure}

反射光的偏振实验装置如图6.8所示.玻片反射起
偏器置于光具架上,使其框上的刻线对准入射角为$57^{\circ}$的定
位点,先将光源置于起偏器左侧,转动光源,使出射光束经玻
片反射后,光斑落在毛玻璃屏的中央。再把偏振片装入光具
架内。转动偏振片,当指示偏振化方向的指针处于竖直方向
时,屏上有明亮的光斑。指针处于水平方向时,屏上的光斑几
乎消失。

上述两个实验用6—10伏低压电源.

偏振片可以用观看立体电影偏光眼镜片代替,左
右两只眼镜片的偏振化方向互相垂直。光源可选取平行光源
(幻灯、手电筒、激光光源等),通过两个偏振片将光斑投影在
墙上,转动其中任一偏振片,很容易观察到光的偏振现象。

反射光偏振起偏器,可以用一般的平板玻璃,将其一面
涂黑,用以吸收透过玻璃的光,当光线的入射角为布儒斯特
角时,从玻璃表面反射的光是平面偏振光,其振动方向和光的
入射平面垂直。再让反射光通过偏振片,转动偏振片的方位,
可看到屏上光斑的亮度周期性的变化,利用反射偏振光可以
检查偏振片的偏振化方向。





















































